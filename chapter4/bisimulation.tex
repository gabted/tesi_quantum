Density operators represents equivalence classes over probabilistic mixtures of quantum states.
The implicit equivalence relation is $\{(\ket{\phi_i}, p_i)\}_i \cong \{(\ket{\phi_j}, p_j)\}$ iff $\sum_{i} p_i \ketbra{\phi_i}{\phi_i} = \sum_{j} p_j \ketbra{\phi_j}{\phi_j}$.
The physical justification of this equivalence is that different mixtures resulting in the same density operator cannot be distinguished since they behave the same.

The same equivalence relation is trivially extended to configurations, where the use of density operators for the quantum state allows a common representation of different configurations with an equivalent mixtures of quantum states. 
We extend here this equivalence relation to distributions of configurations.
Intuitively, we give rules for exchanging the probabilistic combination $\psum{p}$ in the state of some configurations for probabilistic combination of configurations, and vice-versa. 

\begin{definition}
	A \emph{quantum distribution} is an equivalence class $\mathbf{\Delta}$ of probabilistic distributions $\Delta$ over quantum configurations defined by the minimal equivalence relation such that:
	\begin{itemize}
		\item $(\overline{\confw{\rho, P}} \psum{p} \overline{\confw{\sigma, P}}) \equiv \overline{\confw{p \rho + (1-p)\sigma, P}}$; and
%		\item $(\overline{\confw{\rho \otimes \sigma, P}} \psum{p} \overline{\confw{\rho \otimes \sigma', Q}}) \equiv (\overline{\confw{\rho \otimes \delta, P}} \psum{p} \overline{\confw{\rho \otimes \delta', Q}})$ if $\Gamma, \Sigma \vdash P$, $\Gamma, \Sigma \vdash Q$, $\rho \in \hilbert_\Sigma$, and $p \sigma + (1 - p) \sigma' =  p \sigma + (1 - p) \sigma' = p \delta + (1 - p) \delta'$; and	
		\item $\Delta_i\ \equiv\ \Theta_i$, $i = 1, 2$, implies $\Delta_1 \psum{p} \Delta_2 \equiv\ \Theta_1 \psum{p} \Theta_2$.
	\end{itemize}
	We write $Q(Conf)$ for quantum distributions over configurations.
\end{definition}

\begin{definition}
	Given $\rel \subseteq Conf \times Conf$ be a relation over quantum configurations, let its quantum lifting be the minimal relation $\sqlift{\rel} \subseteq Q(Conf) \times Q(Conf)$ over quantum distributions such that $\Delta\ \slift{\rel}\ \Theta$ with $\Delta \in \mathbf{\Delta}$ and $\Theta \in \mathbf{\Theta}$ implies $\mathbf{\Delta}\ \sqlift{\rel}\ \mathbf{\Theta}$.
\end{definition}

%\begin{lemma}
%	For any $\rel$, $\slift{\rel}$ is left and right decomposable, i.e.,
%	\begin{itemize}
%		\item $(\Delta_1 \psum{p} \Delta_2) \slift{\rel} \Theta$ implies $\Theta = \Theta_1 \psum{p} \Theta_2$ and $\Delta_i \slift{\rel} \Theta_i$, $i = 1, 2$; and
%		\item $\Delta \slift{\rel} (\Theta_1 \psum{p} \Theta_2)$ implies $\Delta = \Delta_1 \psum{p} \Delta_2$ and $\Delta_i \slift{\rel} \Theta_i$, $i = 1, 2$.
%	\end{itemize}
%\end{lemma}
%\begin{proof}
%	We prove only the first point, the second being immediately derivable.
%\end{proof}

\begin{theorem}
	Let $\rel \subseteq Conf \times Conf$ be an equivalence relation over quantum configurations, then $\sqlift{\rel}$ is an equivalence relations over distributions of quantum configurations.
\end{theorem}
\begin{proof}
	Reflexivity and symmetry holds by definition given that $\rel$ is an equivalence relation.
	For transitivity, assume $\Delta \slift{\rel} \Theta \slift{\rel} \Xi$.
	{\color{red} decomponibilita' non vale con le definizioni che abbiamo dato...}
\end{proof}
{\color{red} TODO: capire come gestire il fatto che vogiamo che $\approx$ sia una relazione di equivalenza.}

%\begin{definition}
%	Given $\rel \subseteq Conf \times Conf$ be a relation over quantum configurations, let its quantum lifting be the minimal relation $\slift{\rel} \subseteq D(Conf) \times D(Conf)$ such that
%	\begin{itemize}
%		\item $\conf\ \rel\ \conf'$ implies $\overline{\conf}\ \slift{\rel}\ \overline{\conf'}$;
%		\item $\Delta_i\ \slift{\rel}\ \Theta_i$, $i = 1, 2$, implies $\Delta_1 \psum{p} \Delta_2 \slift{\rel}\ \Theta_1 \psum{p} \Theta_2$;
%		\item $(\overline{\confw{\rho, P}} \psum{p} \overline{\confw{\sigma, P}}) \slift{\rel} \overline{\confw{p \rho + (1-p)\sigma, P}}$;
%		\item $\overline{\confw{p \rho + (1-p)\sigma, P}} \slift{\rel} (\overline{\confw{\rho, P}} \psum{p} \overline{\confw{\sigma, P}})$.
%	\end{itemize} 
%\end{definition}

%\begin{theorem}
%	Let $\rel \subseteq Conf \times Conf$ be an equivalence relation over quantum configurations, $\slift{\rel}$ is an equivalence relations over distributions of quantum configurations.
%\end{theorem}
%\begin{proof}
%	Reflexivity and symmetry holds by definition given that $\rel$ is an equivalence relation.
%	For transitivity, assume $\Delta \slift{\rel} \Theta \slift{\rel} \Xi$.
%	{\color{red} decomponibilita' non vale con le definizioni che abbiamo dato...}
%\end{proof}


%In the following we write $qv(P)$ or $qv(\confw{\rho, P})$ for the set of free quantum variables in $P$, and $env(\confw{\rho, P})$ for $tr_{qv(P)}(\rho)$.

\begin{definition}[Barb]
	A \emph{barb} is a predicate $\downarrow_{c}$ over configurations where $\langle \rho, P \rangle \downarrow_{c}$ iff $P \equiv c!x + Q \parallel R$ for some $x$, and processes $Q$, $R$.
%	A \emph{barb} is a predicate $\downarrow_{p, c}$ over configurations where $\langle \rho, P \rangle \downarrow_{p, c}$ iff $tr(\rho) = p$ and
%	$P \equiv c!x + Q \parallel R$ for some $x$, and processes $Q$, $R$.
\end{definition}

%\begin{definition}[Barb Preserving Relation]
%	A relation $\rel \subseteq \conf \times \conf$ is \emph{barb preserving} if $\conf \rel \conf'$ implies that $\barb{p}{c}{\conf}$ iff $\barb{p}{c}{\conf'}$
%	for any $p \in [0,1]$ and any classical channel $c$.
%\end{definition}

%\begin{definition}[Probabilistic Barbed Bisimulation]
%	A symmetric relation $\rel_{\Gamma, \Sigma} \subseteq \conf \times \conf$ is \emph{probabilistic barbed bisimulation} if $\conf \rel \conf'$, with $\conf, \conf'$ well-typed under the context $\Gamma; \Sigma$, implies that 
%	\begin{itemize}
%		\item if $\conf \downarrow_{c}$ then $\conf' \downarrow_{c}$; and 
%		\item whenever $\conf \xrightarrow{\tau} \Delta$, there exists $\Delta'$ such that $\conf' \xrightarrow{\tau} \Delta'$ and $\Delta \slift{\rel} \Delta'$
%%		\item whenever $\conf' \xrightarrow{\tau} \Delta'$, there exists $\Delta$ such that $\conf \xrightarrow{\tau} \Delta$ and $\Delta \slift{\rel} \Delta'$
%	\end{itemize}
%\end{definition}

{\color{red} Non capisco dove, ma ci vuole: if $\confw{\rho, P} \approx \confw{\rho', P'}$ then $\confw{\sigma \otimes \rho, P} \approx \confw{\sigma \otimes \rho', P'}$}

\begin{definition}[Saturated Probabilistic Barbed Bisimilarity]
	A symmetric relation $\rel \subseteq \conf \times \conf$ is \emph{saturated probabilistic barbed bisimulation} if $\conf\ \rel\ \conf'$ implies that $\conf, \conf'$ are well-typed under a typing context $\Gamma; \Sigma$, and for any context $C[\_] \in Context_{\Gamma, \Sigma}$
	\begin{itemize}
		\item if $C[\conf] \downarrow_{c}$ then $C[\conf'] \downarrow_{c}$; and 
		\item whenever $C[\conf] \xrightarrow{\tau} \Delta \in \mathbf{\Delta}$, there exists $\Delta' \in \mathbf{\Delta'}$ such that $C[\conf'] \xrightarrow{\tau} \Delta'$ and $\mathbf{\Delta}\ \sqlift{\rel}\ \mathbf{\Delta'}$
		%		\item whenever $\conf' \xrightarrow{\tau} \Delta'$, there exists $\Delta$ such that $\conf \xrightarrow{\tau} \Delta$ and $\Delta \slift{\rel} \Delta'$
	\end{itemize}
	Let \emph{saturated probabilistic barbed bisimilarity} $\approx_{SPB}$ be the largest saturated probabilistic barbed bisimulation.
\end{definition}

By writing $\sop \in TSO(\hilbert_{\Sigma})$, we mean that $\sop(\rho) = \sum_{i \in I} (I_{\overline{\Sigma}} \otimes A_i) \rho (I_{\overline{\Sigma}} \otimes A_i)^{\dagger}$.

%{\color{red} Per la misura come non-trace preserving, che noi usiamo nella regola di misura della semantica, abbiamo la stessa proprieta', in particolare, rispetto alla versione trace-preserving avremmo semplicemente che la sommatoria sara' composta da meno elementi.}

\begin{lemma}\label{lemma:sop}
	For any pair of configurations $\conf, \conf'$ well-typed under $\Gamma; \Sigma$, and for any $\sop \in TSO(\hilbert_{\overline{\Sigma}})$,
	$\conf \rightarrow \Delta$ iff $\sop(\conf) \rightarrow \sop(\Delta)$.
\end{lemma}
\begin{proof}
	The only non trivial rules are {\scshape SemQOp} and {\scshape SemQMeas}.
	Assume without loss of generality that $env(\conf) \in \hilbert_{\Sigma \otimes \overline{\Sigma}}$.
	By {\scshape SemQOp}, $\conf = \langle \rho, \mathcal{E}(\widetilde{x}) . P \rangle \longrightarrow \langle \mathcal{E}_{\widetilde{x}}(\rho), P \rangle = \Delta$.
	The type system ensures that $\tilde{x} \subseteq \Sigma$.
	We assume without loss of generality that $\tilde{x} = \Sigma$.
	We can also apply {\scshape SemQOp} as follows, $\sop(\conf) = \langle \sop(\rho), \mathcal{E}(\widetilde{x}) . P \rangle  \longrightarrow \langle \mathcal{E}_{\widetilde{x}}(\sop(\rho)), P \rangle$.
	We need to prove that $\langle \mathcal{E}_{\widetilde{x}}(\sop(\rho)), P \rangle = \langle \sop(\mathcal{E}_{\widetilde{x}}(\rho)), P \rangle = \sop(\Delta)$, i.e., that $\sop_{\widetilde{x}}(\sop(\rho)) = \sop(\sop_{\widetilde{x}}(\rho))$.
	By definition, $\sop_{\widetilde{x}}(\rho) = \sum_{i = 1}^{n} (I_{\Sigma} \otimes A_i) \rho (I_{\Sigma} \otimes A_i)^{\dagger}$, and $\sop(\rho) = \sum_{j = 1}^{m} (B_j \otimes I_{\overline{\Sigma}}) \rho (B_j \otimes I_{\overline{\Sigma}})^{\dagger}$.

	By linearity 
	\begin{align*}
	&\sop_{\widetilde{x}}(\sop(\rho)) =\\ 
	&\sum_{i = 1}^{n} (I_{\Sigma} \otimes A_i) (\sum_{j = 1}^{m} (B_j \otimes I_{\overline{\Sigma}}) \rho (B_j \otimes I_{\overline{\Sigma}})^{\dagger}) (I_{\Sigma} \otimes A_i)^{\dagger} =\\
	&\sum_{i = 1}^{n} \sum_{j = 1}^{m} (I_{\Sigma} \otimes A_i) (B_j \otimes I_{\overline{\Sigma}}) \rho (B_j \otimes I_{\overline{\Sigma}})^{\dagger} (I_{\Sigma} \otimes A_i)^{\dagger},
	\end{align*}
	and
	\begin{align*}
	&\sop(\sop_{\widetilde{x}}(\rho)) =\\
	&\sum_{j = 1}^{m} (B_j \otimes I_{\overline{\Sigma}}) (\sum_{i = 1}^{n} (I_{\Sigma} \otimes A_i) \rho (I_{\Sigma} \otimes A_i)^{\dagger}) (B_j \otimes I_{\overline{\Sigma}})^{\dagger} =\\
	&\sum_{i = 1}^{n} \sum_{j = 1}^{m} (B_j \otimes I_{\overline{\Sigma}}) (I_{\Sigma} \otimes A_i) \rho (I_{\Sigma} \otimes A_i)^{\dagger} (B_j \otimes I_{\overline{\Sigma}})^{\dagger}.
	\end{align*}
	Thanks to conjugate properties, it is thus sufficient to show that
	\[
	(B_{p\times p} \otimes I_{q\times q}) (I_{p\times p} \otimes A_{q\times q}) = (I_{p\times p} \otimes A_{q\times q}) (B_{p\times p} \otimes I_{q\times q}).
	\]
	
	This is easily proven thanks to the mixed product property of the Kronecker product, telling us that 
	\[ (A \otimes B)(C \otimes D) = (AC)\otimes(BD)
	\]
	so in our case, we have 
	\[
	(B_{p\times p} \otimes I_{q\times q}) (I_{p\times p} \otimes A_{q\times q}) = B_{p\times p} \otimes A_{q \times q} = (I_{p\times p} \otimes A_{q\times q}) (B_{p\times p} \otimes I_{q\times q})
	\]	
	
	
%	Take
%	\begin{align*}
%		&(B_{p\times p} \otimes I_{q\times q}) (I_{p\times p} \otimes A_{q\times q}) =\\
%		&\left(\begin{array}{c c c}
%			B_{1,1} I_{q\times q} & \dots & B_{1,p} I_{q\times q}\\
%			\dots & & \dots\\
% 			B_{p,1} I_{q\times q} & \dots & B_{p,p} I_{q\times q}\\
%	\end{array}\right)
%\left(
%\begin{array}{c c c c c }
%	A_{q\times q} & \bigzero & \bigzero & \dots & \bigzero \\
%	\bigzero & A_{q\times q} & \bigzero & \dots & \bigzero \\
%	\dots & \dots & \dots & \dots & \dots \\
%	\bigzero & \dots & \dots & \dots & A_{q\times q} \\
%\end{array}
%\right),
%\end{align*}
%and take the element at column $i$, row $j$.
%Note that it is $B_{i,j} I_{q\times q} A_{q\times q}$.
%Take then
%\begin{align*}
%	&(I_{p\times p} \otimes A_{q\times q}) (B_{p\times p} \otimes I_{q\times q}) =\\
%	&\left(
%	\begin{array}{c c c c c }
%		A_{q\times q} & \bigzero & \bigzero & \dots & \bigzero \\
%		\bigzero & A_{q\times q} & \bigzero & \dots & \bigzero \\
%		\dots & \dots & \dots & \dots & \dots \\
%		\bigzero & \dots & \dots & \dots & A_{q\times q} \\
%	\end{array}
%	\right)
%	\left(\begin{array}{c c c}
%		B_{1,1} I_{q\times q} & \dots & B_{1,p} I_{q\times q}\\
%		\dots & & \dots\\
%		B_{p,1} I_{q\times q} & \dots & B_{p,p} I_{q\times q}\\
%	\end{array}\right),
%\end{align*}
%and take the element at column $i$, row $j$.
%Note that it is $A_{q\times q} B_{i,j} I_{q\times q}$.
%The thesis follow by linearity and commutativity of the identity.

%\begin{align*}
%&\left(
%\begin{array}{c c c c }
%	B_{1,1} I_{q\times q} A_{q\times q} & B_{1,2} I_{q\times q} A_{q\times q} & \dots & B_{1,p} I_{q\times q} A_{q\times q} \\
%	B_{2,1} I_{q\times q} A_{q\times q} & B_{2,2} I_{q\times q} A_{q\times q} & \dots & B_{2,p} I_{q\times q} A_{q\times q} \\
%	\dots & \dots & \dots & \dots\\
%	B_{p,1} I_{q\times q} A_{q\times q} & B_{p,2} I_{q\times q} A_{q\times q} & \dots & B_{p,p} I_{q\times q} A_{q\times q} \\
%\end{array}
%\right) = \\
%&\left(
%\begin{array}{c c c c }
%	A_{q\times q} B_{1,1} I_{q\times q} & A_{q\times q} B_{1,2} I_{q\times q} & \dots & A_{q\times q} B_{1,p} I_{q\times q} \\
%	A_{q\times q} B_{2,1} I_{q\times q} & A_{q\times q} B_{2,2} I_{q\times q} & \dots & A_{q\times q} B_{2,p} I_{q\times q} \\
%	\dots & \dots & \dots & \dots\\
%	A_{q\times q} B_{p,1} I_{q\times q} & A_{q\times q} B_{p,2} I_{q\times q} & \dots & A_{q\times q} B_{p,p} I_{q\times q} \\
%\end{array}
%\right) = \\
%\end{align*}

The proof for rule {\scshape SemQMeas} is the same, considering every $m \in \{0, \dots, 2^{\widetilde{x}}\}$ separately.
\end{proof}


\begin{theorem}
%	For any pair of configurations $\conf, \conf'$ well-typed under $\Gamma; \Sigma$, 
%	$\conf \approx_{SPB} \conf'$ implies that for any $\sop \in TSO(\hilbert_{\overline{\Sigma}})$, $\sop (\conf)\ \approx_{SPB}\ \sop (\conf')$.
	For any pair of configurations $\conf, \conf'$ well-typed under $\Gamma; \Sigma$, 
	$\conf \approx_{SPB} \conf'$ implies
	\begin{enumerate}
		{\item for any $\sop \in TSO(\hilbert_{\overline{\Sigma}})$, $\sop (\conf)\ \approx_{SPB}\ \sop (\conf')$; and \label{point:thmchinese1}}
		{\item $tr_{\Sigma}(\conf) = tr_{\Sigma}(\conf')$. \label{point:thmchinese2}}
	\end{enumerate}
\end{theorem}
\begin{proof}
	The proof of point \ref{point:thmchinese1} follows by transitivity and Lemma~\ref{lemma:sop}.
	\[
	 \sop (\conf)\ \approx_{SPB}\ \conf\ \approx_{SPB}\ \conf'\ \approx_{SPB}\ \sop (\conf') 
	\]
	For point \ref{point:thmchinese2} we proceed by refutation.
	Assume $\conf \approx_{SPB} \conf'$ and $tr_{\Sigma}(\conf) \neq tr_{\Sigma}(\conf')$.
	Take $env(\conf), env(\conf') \in \hilbert_{\{ q \} \otimes \Sigma}$ with $env(\conf) = \ket{0} \otimes \ket{\phi}$ and $env(\conf') = \ket{1} \otimes \ket{\phi'}$.
	Consider the context $B[\_] = M[q \triangleright x] . \textbf{ if } x = 0 \textbf{ then } c!q \textbf{ else } free(q) \parallel [\_]$.
	It is clear that $B[\conf] \not\approx_{SPB} B[\conf']$. Contradiction.
\end{proof}

\begin{example}
	Consider the following, wrong definition of quantum teleportation:
	\begin{align*}
		\proc{A} &\Coloneqq \text{in}_a?x.\text{CNOT}(q_0, x).\text{H}(q_0).M(x,q_0 \rhd n).(\text{m}_a!n \parallel out_{a_1}! q_0 \parallel out_{a_2}! x )\\
    \proc{B} &\Coloneqq \text{in}_b?x.\text{m}_a?n.
       \\ & \ite{n = 0}{\sigma_0(x).\text{out}_b!x\\&\quad}
      {\ite{n = 1}{\sigma_1(x).\text{out}_b!x\\&\qquad}
          {\ite{n = 2}{\sigma_2(x).\text{out}_b!x}{\sigma_3(x).\text{out}_b!x}}
      } \\
		\proc{S} &\Coloneqq \text{H}(q_1).\text{CNOT}(q_1, q_2).(\text{in}_a!q_1 \parallel \text{in}_b!q_2) \\
		\proc{Tel} &\Coloneqq (A \parallel B \parallel S) \setminus \Set{\text{in}_a, \text{in}_b, \text{m}_a } \\
		\proc{TelSpec} &\Coloneqq \text{SWAP}(q_0,q_2).(\text{out}_b!q_2 \parallel out_{a_1}! q_0 \parallel out_{a_2}! q_1)
	\end{align*}
	We have that $\proc{Tel} \not\approx_{SPB} \proc{TelSpec}$.
	Consider indeed a context:
  \[ B[\blank] = out_{a_1} ? x . M[x \rhd y] . \ite{y = 1}{c!y}{nil}. \]
\end{example}

An established notion of behavioural equivalence between quantum processes is the Deng-Feng bisimilarity, here we focus on the strong version of it.
\begin{definition}[Deng-Feng Bisimilarity]
	A symmetric relation $\rel \subseteq \conf \times \conf$ is \emph{Deng-Feng bisimulation} if $\conf\ \rel\ \conf'$ implies that $\conf, \conf'$ are well-typed under a typing context $\Gamma; \Sigma$, and for any context $C[\blank] \in Context_{\Gamma, \Sigma}$
	\begin{itemize}
		\item $tr_{\Sigma}(\conf) = tr_{\Sigma}(\conf')$;
		\item if $C[\sop(\conf)] \downarrow_{c}$ then $C[\sop(\conf')] \downarrow_{c}$; and
		\item whenever $C[\sop(\conf)] \xrightarrow{\tau} \Delta$, there exists $\Delta'$ such that $C[\sop(\conf')] \xrightarrow{\tau} \Delta'$ and $\Delta \slift{\rel} \Delta'$
	\end{itemize}
	Let \emph{Deng-Feng bisimilarity} $\approx_{DF}$ be the largest saturated probabilistic barbed bisimulation.
\end{definition}

\begin{theorem}
	For any pair of configurations $\conf, \conf'$, $\conf \approx_{DF} \conf'$ iff $\conf \approx_{SPB} \conf'$.
\end{theorem}


From Davidson we expect that $\confw{\beta, M_{0,1}[q_0 \triangleright x] . c ! q_0 \parallel discard(q_1)} \approx_{SPB} \confw{\beta, M_{0,1}[q_0 \triangleright x] . c ! q_0 \parallel discard(q_1)}$. To prove this result, we need some preliminary lemmas.

\newcommand{\discQ}{disc(\widetilde{q})}
\newcommand{\trQ}{tr_{\widetilde{q}}}

\note{vanno sistemate le parentesi, e comunque si potrebbe anche scrivere nel capitolo del background quantum}
\begin{lemma}\label{trace_and_sop}
Let $\sigma \in  \mathcal{D}(\calH_{\widetilde{p}} \otimes \calH_{\widetilde{q}})$, with $\widetilde{p} = p_0 \ldots p_{n-1}$ and $\widetilde{q} = q_0 \ldots q_{m-1}$. Then, for any \textit{trace non-increasing} superoperator $\sop_{\widetilde{p}} \in \mathcal{S}(\calH_{\widetilde{p}})$
\[ \trQ((\sop_{\widetilde{p}} \otimes \mathcal{I}_{\widetilde{q}})(\sigma)) = \sop_{\widetilde{p}}(\trQ(\sigma))
\]
\end{lemma}
\begin{proof}
We know that $\sigma$ is a probabilistic mixture of pure states, $\sigma = \sum_l p_l \proj{\psi_l}$, and each pure state is a linear combination of separable states 
\[\sigma = \sum_l p_l \sum_{p = 0}^{2^n-1} \sum_{q = 0}^{2^m-1} \lambda_{pq} \proj{pq}\]
and we know that $\sop_{\widetilde{p}}$ has a Kraus decomposition 
\[\sop_{\widetilde{p}}(\rho) = \sum_k A_k \rho A_k^\dagger\]
Then we can write 
\hspace*{-2cm}\begin{align*}
\trQ((\sop_{\widetilde{p}} \otimes \mathcal{I}_{\widetilde{q}})(\sigma)) &= \sop_{\widetilde{p}}(\trQ(\sigma)) 
\\ 
\trQ((\sop_{\widetilde{p}} \otimes \mathcal{I}_{\widetilde{q}})\left(\sum_l p_l \sum_{p = 0}^{2^n-1} \sum_{q = 0}^{2^m-1} \lambda_{pq} \proj{pq}\right)) &= \sop_{\widetilde{p}}(\trQ\left(\sum_l p_l \sum_{p = 0}^{2^n-1} \sum_{q = 0}^{2^m-1} \lambda_{pq} \proj{pq}\right))
\\
\sum_l p_l \trQ((\sop_{\widetilde{p}} \otimes \mathcal{I}_{\widetilde{q}})\left( \sum_{p = 0}^{2^n-1} \sum_{q = 0}^{2^m-1} \lambda_{pq} \proj{pq}\right)) 
&= \sum_l p_l \sop_{\widetilde{p}}(\trQ\left( \sum_{p = 0}^{2^n-1} \sum_{q = 0}^{2^m-1} \lambda_{pq} \proj{pq}\right))
\\
\trQ(\sum_k (A_k \otimes I_{\widetilde{q}})\left( \sum_{p = 0}^{2^n-1} \sum_{q = 0}^{2^m-1} \lambda_{pq} \proj{pq}\right)(A_k \otimes I_{\widetilde{q}})^\dagger )
&= \sum_k A_k\left( \sum_{p = 0}^{2^n-1} \sum_{q = 0}^{2^m-1} \lambda_{pq} \trQ(\proj{pq})\right) A_k^\dagger 
\\
\trQ(\sum_k  \left( \sum_{p = 0}^{2^n-1} \sum_{q = 0}^{2^m-1} \lambda_{pq} (A_k \otimes I_{\widetilde{q}}) \proj{pq} (A_k \otimes I_{\widetilde{q}})^\dagger  \right) )
&= \sum_k A_k\left( \sum_{p = 0}^{2^n-1} \sum_{q = 0}^{2^m-1} \lambda_{pq} \trQ(\proj{pq})\right) A_k^\dagger
\\
\trQ(\sum_k  \left( \sum_{p = 0}^{2^n-1} \sum_{q = 0}^{2^m-1} \lambda_{pq} (A_k\ket{p})\ket{q}(\bra{p}A_k^\dagger) \bra{q} \right) )
&= \sum_k A_k\left( \sum_{p = 0}^{2^n-1} \sum_{q = 0}^{2^m-1} \lambda_{pq} \proj{p}\braket{q | q}\right)A_k^\dagger
\\
\sum_{p = 0}^{2^n-1} \sum_{q = 0}^{2^m-1} \lambda_{pq} \sum_k \trQ( (A_k\ket{p})\ket{q}(\bra{p}A_k^\dagger) \bra{q} ) 
&= \sum_k  \sum_{p = 0}^{2^n-1} \sum_{q = 0}^{2^m-1} \lambda_{pq} \sum_k  A_k \proj{p} A_k^\dagger 
\\
\sum_{p = 0}^{2^n-1} \sum_{q = 0}^{2^m-1} \lambda_{pq} \sum_k (A_k \proj{p} A_k^\dagger) 
&=  \sum_{p = 0}^{2^n-1} \sum_{q = 0}^{2^m-1} \lambda_{pq} \sop_{\widetilde{p}} (\proj{p})
\\
\sum_{p = 0}^{2^n-1} \sum_{q = 0}^{2^m-1} \lambda_{pq} \sop_{\widetilde{p}}(\proj{p})
&=  \sum_{p = 0}^{2^n-1} \sum_{q = 0}^{2^m-1} \lambda_{pq} \sop_{\widetilde{p}} (\proj{p})
\end{align*}

Where we made use of the fact that both $\trQ$ and matrix multiplication are closed for linearity.
\end{proof}

\note{va sistemata la notazione dei pedici dei superoperatori.}
\begin{lemma}\label{lemma_transition_partial_trace}
Let $\sigma \in  \mathcal{D}(\calH_{\widetilde{p}} \otimes \calH_{\widetilde{q}})$, and $\rho = tr_{\widetilde{q}}(\sigma) \in \calH_{\widetilde{p}}$. Then, for any process 
\[ \confw{\rho, P} \rightarrow \bigoplus_i p_i \confw{\rho_i, P_i} 
\qquad\Leftrightarrow\qquad
\confw{\sigma, P\parallel \discQ} \rightarrow \bigoplus p_i \confw{\sigma_i, P_i\parallel \discQ}
\]
and $\forall i \ \rho_i = \trQ(\sigma_i)$
\end{lemma}
\begin{proof}
We proceed by induction on $\rightarrow$. For the simple base cases {\footnotesize\scshape SemTau} and {\footnotesize\scshape SemReduce}, where the quantum state remains unchanged, we have 
\begin{align*}
\confw{\rho, \tau.P} \rightarrow \confw{\rho, P}
&\Leftrightarrow
\confw{\sigma, \tau.P\parallel \discQ} \rightarrow \confw{\sigma, P\parallel \discQ}
\\
 \confw{\rho, c?x.P\parallel c!v} \rightarrow \confw{\rho, P[v / x]}
&\Leftrightarrow
\confw{\sigma, c?x.P\parallel c!v \parallel \discQ} \rightarrow \confw{\sigma, P[v / x]\parallel \discQ}
\end{align*}

and in both cases $\rho' = \rho = \trQ(\sigma)$.

For the base case {\footnotesize\scshape SemQOp} we have 
\[\confw{\rho, \sop(\widetilde{p}).P} \rightarrow \confw{\sop_{\widetilde{p}}(\rho), P}
\qquad\Leftrightarrow\qquad
\confw{\sigma, \sop(\widetilde{p}).P\parallel \discQ} \rightarrow \confw{\sop_{\widetilde{p}}(\sigma), P\parallel \discQ}\]
where $\rho' = \sop_{\widetilde{p}}(\rho) = \sop_{\widetilde{p}}(\trQ(\sigma))$ and $\sigma' = \sop_{\widetilde{p}}\otimes \mathcal{I}_{\widetilde{q}}(\sigma)$, and so we have $\rho' = \trQ(\sigma') = \trQ(\sop_{\widetilde{p}}\otimes \mathcal{I}_{\widetilde{q}}(\sigma))$ thanks to lemma \ref{trace_and_sop}.


The base case {\footnotesize\scshape SemQMeasure} is similar, we have 
\begin{gather*}
\confw{\rho, M[\widetilde{p} \rhd x].P} \rightarrow \bigoplus p_m\confw{p_m^{-1}\sop_m(\rho), P} \\
\Leftrightarrow \\
\confw{\sigma, M[\widetilde{p} \rhd x].P\parallel \discQ} \rightarrow \bigoplus p'_m \confw{p_m^{-1}\sop_m(\sigma), P\parallel \discQ}
\end{gather*}
where thanks to lemma \ref{trace_and_sop}, we have $p_m = tr(\sop_m(\rho)) = tr(\sop_m(\trQ(\sigma))) = tr(\trQ(\sop_m \otimes \mathcal{I}_{\widetilde{q}}(\sigma)))$ and $\forall i \ \rho_i = \trQ(\sigma_i)$, as in the previous case.

The inductive cases are all trivial, as none of them modifies the quantum values $\rho$ and $\sigma$.
\end{proof}

\newcommand{\relTrQ}{\rel_{\trQ}}
\begin{lemma}
Let $\sigma \in \mathcal{D}(\calH_{\widetilde{p}} \otimes \calH_{\widetilde{q}})$. If $\rel$ is a bisimulation, then $\rel_{\trQ}$ is a bisimulation, where $\rel_{\trQ}$ is defined as
\[\rel_{\trQ} = \set{\Big(\confw{\sigma, P \parallel \discQ}, \confw{\nu, Q \parallel \discQ}\Big) \quad\mid\quad \confw{\trQ(\sigma), P} \rel \confw{\trQ(\nu), Q}}\]
\end{lemma}
\begin{proof}
To show that $\rel_{\trQ}$ is a bisimulation, we need first of all to show that is symmetric, which is when $\rel$ is symmetric. The we suppose that $\confw{\sigma, P \parallel \discQ} \relTrQ \confw{\nu, Q \parallel \discQ}$ and have to show that \begin{itemize}
\item For any barb $b$, if $\confw{\sigma, P \parallel \discQ} \downarrow_b$ then $\confw{\nu, Q \parallel \discQ} \downarrow_b$. We have that $\confw{\sigma, P \parallel \discQ} \downarrow_b \Leftrightarrow \confw{\trQ(\sigma), P} \downarrow b$, $\confw{\nu, Q \parallel \discQ} \downarrow_b \Leftrightarrow \confw{\trQ(\nu), Q} \downarrow b$, and 
$\confw{\trQ(\sigma), P} \rel \confw{\trQ(\nu), Q}$, so they all express the same barbs.
\item For any $R$, if $\confw{\sigma, P \parallel R \parallel \discQ} \rightarrow \Delta$, then $\confw{\nu, Q \parallel R \parallel \discQ} \rightarrow \Theta$, and $\Delta \slift{\relTrQ} \Theta$. Notice that $\discQ$ can not evolve, and so we start from the hypothesis $\confw{\sigma, P \parallel R \parallel \discQ} \rightarrow \bigoplus p_i \confw{\sigma_i, P_i \parallel \discQ}$. Then, from lemma \ref{lemma_transition_partial_trace}, we know that $\confw{\trQ(\sigma), P\parallel R} \rightarrow \bigoplus p_i \confw{\trQ(\sigma_i), P_i}$. But since $\confw{\trQ(\sigma), P} \rel \confw{\trQ(\nu), Q}$, and $\rel$ is a saturated bisimulation, it must be that $\confw{\trQ(\nu), Q\parallel R} \rightarrow \bigoplus p_i \confw{\xi_i, Q_i}$ with $\confw{\trQ(\sigma_i), P_i} \rel \confw{\xi_i, Q_i}$ for each $i$. But using the same lemma \ref{lemma_transition_partial_trace} in the other direction, we get $\confw{\nu, Q \parallel R \parallel \discQ} \rightarrow \bigoplus p_i \confw{\nu_i, Q_i \parallel \discQ}$ with $\xi_i = \trQ(\nu_i)$. In conclusion, for each transition $\confw{\sigma, P \parallel R \parallel \discQ} \rightarrow \bigoplus p_i \confw{\sigma_i, P_i \parallel \discQ}$ exists a transition $\confw{\nu, Q \parallel R \parallel \discQ} \rightarrow \bigoplus p_i \confw{\nu_i, Q_i \parallel \discQ}$ such that $\forall i \confw{\trQ(\sigma_i), P_i} \rel \confw{\trQ(\nu_i), Q_i}$, and so from the definiton of $\relTrQ$ together with probabilistic lifting we get $\bigoplus p_i \confw{\sigma_i, P_i \parallel \discQ} \slift{\relTrQ} \bigoplus p_i \confw{\nu_i, Q_i \parallel \discQ}$.
\end{itemize}
\end{proof}

\note{è così immediato che si potrebbe anche integrare nel lemma sopra.}
\begin{theorem}
If $\confw{\trQ(\sigma), P} \sim \confw{\trQ(\nu), Q}$ then $\confw{\sigma, \P \parallel \discQ} \sim \confw{\nu, Q \parallel \discQ}$.
\end{theorem}
\begin{proof}
It easily follows from the previous theorem, if $\confw{\trQ(\sigma), P} \sim \confw{\trQ(\nu), Q}$ then there exists a bisimulation $\rel$ such that $\confw{\trQ(\sigma), P} \rel \confw{\trQ(\nu), Q}$, therefore $\confw{\sigma, \P \parallel \discQ} \relTrQ \confw{\nu, Q \parallel \discQ}$ for a bisimulation $\relTrQ$, and so $\confw{\sigma, \P \parallel \discQ}$ and $\confw{\nu, Q \parallel \discQ}$ are bisimilar.
\end{proof}

\note{con questo teorema si può quasi dimostrare che $\confw{\beta, M_{01}[q].discard(q)} \sim \confw{\beta, M_{+-}[q].discard'(q)}$. Esprimono le stesse barbe, e messi in parallelo con $R$, se eseguono la misura finiscono nelle solite distribuzioni $\proj{00} \psum{1/2} \proj{11}$ vs $\proj{++} \psum{1/2} \proj{--}$. Queste distribuzioni le possiamo riscrivere come singole configurazioni con stati misti $\confw{\proj{00}\psum{1/2}\proj{11}, R\parallel discard(q)}$ e $\confw{\proj{++}\psum{1/2}\proj{--}, R\parallel discard'(q)}$, e queste due configurazioni sono bisimili per il teorema appena dimostrato.

Manca soltanto dire cosa succede se il contesto $R$ fa qualcosa. Se non coinvolge i qubit, allora è una transizione inattiva, si può ignorare. Se coinvolge i qubit, bisognerebbe dimostrare che non distingue niente. Nello specifico, se fa delle unitarie, non cambia nulla, e se fa una misura, distrgge l'entanglement e non può più distinguere, credo.
}
