
\subsection{Quantum Saturated Bisimilarity}

Density operators represents equivalence classes over probabilistic mixtures of quantum states.
The implicit equivalence relation is $\{(\ket{\phi_i}, p_i)\}_i \cong \{(\ket{\phi_j}, p_j)\}$ iff $\sum_{i} p_i \ketbra{\phi_i}{\phi_i} = \sum_{j} p_j \ketbra{\phi_j}{\phi_j}$.
The physical justification of this equivalence is that different mixtures resulting in the same density operator cannot be distinguished since they behave the same.

The same equivalence relation is trivially extended to configurations, where the use of density operators for the quantum state allows a common representation of different configurations with an equivalent mixtures of quantum states. 
We extend here this equivalence relation to distributions of configurations.
Intuitively, we give rules for exchanging the probabilistic combination $\psum{p}$ in the state of some configurations for probabilistic combination of configurations, and vice-versa. 

\begin{definition}
	A \emph{quantum distribution} is an equivalence class $\mathbf{\Delta}$ of probabilistic distributions $\Delta$ over quantum configurations defined by the minimal equivalence relation such that:
	\begin{itemize}
		\item $(\overline{\confw{\rho, P}} \psum{p} \overline{\confw{\sigma, P}}) \equiv \overline{\confw{p \rho + (1-p)\sigma, P}}$; and
%		\item $(\overline{\confw{\rho \otimes \sigma, P}} \psum{p} \overline{\confw{\rho \otimes \sigma', Q}}) \equiv (\overline{\confw{\rho \otimes \delta, P}} \psum{p} \overline{\confw{\rho \otimes \delta', Q}})$ if $\Gamma, \Sigma \vdash P$, $\Gamma, \Sigma \vdash Q$, $\rho \in \hilbert_\Sigma$, and $p \sigma + (1 - p) \sigma' =  p \sigma + (1 - p) \sigma' = p \delta + (1 - p) \delta'$; and	
		\item $\Delta_i\ \equiv\ \Theta_i$, $i = 1, 2$, implies $\Delta_1 \psum{p} \Delta_2 \equiv\ \Theta_1 \psum{p} \Theta_2$.
	\end{itemize}
	We write $Q(Conf)$ for quantum distributions over configurations.
\end{definition}

\begin{definition}
	Given $\rel \subseteq Conf \times Conf$ be a relation over quantum configurations, let its quantum lifting be the minimal relation $\sqlift{\rel} \subseteq Q(Conf) \times Q(Conf)$ over quantum distributions such that $\Delta\ \slift{\rel}\ \Theta$ with $\Delta \in \mathbf{\Delta}$ and $\Theta \in \mathbf{\Theta}$ implies $\mathbf{\Delta}\ \sqlift{\rel}\ \mathbf{\Theta}$.
\end{definition}


\note{FALSO: 
\begin{theorem}
	Let $\rel \subseteq Conf \times Conf$ be an equivalence relation over quantum configurations, then $\sqlift{\rel}$ is an equivalence relations over distributions of quantum configurations.
\end{theorem}
\begin{proof}
	Reflexivity and symmetry holds by definition given that $\rel$ is an equivalence relation.
	For transitivity, assume $\Delta \slift{\rel} \Theta \slift{\rel} \Xi$.
\end{proof}}


%\begin{definition}[Probabilistic Barbed Bisimulation]
%	A symmetric relation $\rel_{\Gamma, \Sigma} \subseteq \conf \times \conf$ is \emph{probabilistic barbed bisimulation} if $\conf \rel \conf'$, with $\conf, \conf'$ well-typed under the context $\Gamma; \Sigma$, implies that 
%	\begin{itemize}
%		\item if $\conf \downarrow_{c}$ then $\conf' \downarrow_{c}$; and 
%		\item whenever $\conf \xrightarrow{\tau} \Delta$, there exists $\Delta'$ such that $\conf' \xrightarrow{\tau} \Delta'$ and $\Delta \slift{\rel} \Delta'$
%%		\item whenever $\conf' \xrightarrow{\tau} \Delta'$, there exists $\Delta$ such that $\conf \xrightarrow{\tau} \Delta$ and $\Delta \slift{\rel} \Delta'$
%	\end{itemize}
%\end{definition}

\begin{definition}[Quantum Saturated Bisimilarity]
	A symmetric relation $\rel \subseteq \conf \times \conf$ is \emph{quantum saturated bisimulation} if $\conf\ \rel\ \conf'$ implies that $\conf, \conf'$ are well-typed under a typing context $\Gamma; \Sigma$, and for any context $B[\_]_{\Gamma, \Sigma}$
	\begin{itemize}
		\item if $B[\conf] \downarrow_{c}$ then $B[\conf'] \downarrow_{c}$; and 
		\item whenever $B[\conf] \xrightarrow{\tau} \Delta \in \mathbf{\Delta}$, there exists $\Delta' \in \mathbf{\Delta'}$ such that $C[\conf'] \xrightarrow{\tau} \Delta'$ and $\mathbf{\Delta}\ \sqlift{\rel}\ \mathbf{\Delta'}$
		%		\item whenever $\conf' \xrightarrow{\tau} \Delta'$, there exists $\Delta$ such that $\conf \xrightarrow{\tau} \Delta$ and $\Delta \slift{\rel} \Delta'$
	\end{itemize}
	Let \emph{saturated probabilistic barbed bisimilarity} $\approx_{SPB}$ be the largest saturated probabilistic barbed bisimulation.
\end{definition}





From Davidson we expect that $\confw{\beta, M_{0,1}[q_0 \triangleright x] . c ! q_0 \parallel discard(q_1)} \approx_{SPB} \confw{\beta, M_{0,1}[q_0 \triangleright x] . c ! q_0 \parallel discard(q_1)}$. To prove this result, we need can use theorem \ref{bisim_closed_by_discard}


\note{con questo teorema si può quasi dimostrare che $\confw{\beta, M_{01}[q].discard(q)} \sim \confw{\beta, M_{+-}[q].discard'(q)}$. Esprimono le stesse barbe, e messi in parallelo con $R$, se eseguono la misura finiscono nelle solite distribuzioni $\proj{00} \psum{1/2} \proj{11}$ vs $\proj{++} \psum{1/2} \proj{--}$. Queste distribuzioni le possiamo riscrivere come singole configurazioni con stati misti $\confw{\proj{00}\psum{1/2}\proj{11}, R\parallel discard(q)}$ e $\confw{\proj{++}\psum{1/2}\proj{--}, R\parallel discard'(q)}$, e queste due configurazioni sono bisimili per il teorema appena dimostrato.

Manca soltanto dire cosa succede se il contesto $R$ fa qualcosa. Se non coinvolge i qubit, allora è una transizione inattiva, si può ignorare. Se coinvolge i qubit, bisognerebbe dimostrare che non distingue niente. Nello specifico, se fa delle unitarie, non cambia nulla, e se fa una misura, distrgge l'entanglement e non può più distinguere, credo.
}


