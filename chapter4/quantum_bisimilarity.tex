
\subsection{Quantum Saturated Bisimilarity}

We target the previousy highlighted problems by updating the proposed bisimilarity. The main idea is that density matrices represents equivalence classes of probabilistic mixtures of quantum states that are indistinguishable, and that this equivalence must be lifted from distribution of quantum states to distributions of lqCCS configurations.

The implicit equivalence relation implied by density operator is $\{(\ket{\phi_i}, p_i)\}_i \cong \{(\ket{\phi_j}, p_j)\}$ if and only if $\sum_{i} p_i \ketbra{\phi_i}{\phi_i} = \sum_{j} p_j \ketbra{\phi_j}{\phi_j}$.
The physical justification of this equivalence is that different mixtures resulting in the same density operator cannot be distinguished since they behave the same. Note that this applies both to pure quantum states, where $\kz$ and $-\kz$ are equivalent, and  mixed states, where $\kz \psum{\frac{1}{2}} \ko$ is the same as $\kpl \psum{\frac{1}{2}} \km$

The same equivalence relation is trivially extended to configurations, where the use of density operators for the quantum state allows a common representation of different configurations with an equivalent mixtures of quantum states. 
We extend here this equivalence relation to distributions of configurations.
Intuitively, we give rules for exchanging the probabilistic combination $\psum{p}$ in the state of some configurations for probabilistic combination of configurations, and vice-versa. 

\begin{definition}
	The \emph{quantum equivalence} on distributions is the smallest  symmetric relation  $\equiv \subseteq \distr(S) \times \distr(S)$ generated by the rules:
	\begin{gather*}
	(\overline{\confw{\rho, P}} \psum{p} \overline{\confw{\sigma, P}}) \equiv \overline{\confw{p \rho + (1-p)\sigma, P}}
	\\[0.3cm]
	\infer{\Delta_1 \psum{p} \Delta_2 \equiv \Theta_1 \psum \Theta_2}{\Delta_1 \equiv \Theta_1 & \Delta_2 \equiv \Theta_2}
	\end{gather*}
	\end{definition}
	
	\begin{theorem}
	$\equiv$ is an equivalence relation, since it is reflexive, symmetric and transitive.
	\end{theorem}
	\begin{proof}
	Trivial by induction on the rules of $\equiv$.
	\end{proof}


We can now present Quantum Saturated Bisimilarity, that is a relaxation of probabilistic saturated bisimilarity. 

\begin{definition}[Quantum Saturated Bisimilarity]
	A symmetric relation $\rel \subseteq \conf \times \conf$ is \emph{quantum saturated (barbed) bisimulation} if $\confw{\rho, P} \ \rel\ \confw{\sigma, Q}$ implies that $P$ and $Q$  are well-typed under the same typing context $\Gamma; \Sigma$, and for any context $B[\_]_{\Gamma, \Sigma}$
	\begin{itemize}
		\item If $P \downarrow_{c}$ then $Q \downarrow_{c}$
		\item If $\confw{\rho, B[P]} \rightarrow \Delta$, there exist $\Delta', \Theta, \Theta'$ such that $\confw{\sigma, B[Q]} \rightarrow \Theta$ and \[\Delta \equiv \Delta' \ \slift{\rel}\ \Theta' \equiv \Theta\]
	\end{itemize}
	Let \emph{quantum saturated bisimilarity} $\sim_{QS}$ be the union of all saturated probabilistic barbed bisimulation. \\
	We say that two {processes} $P$ and $Q$ are \emph{bisimilar}, written $P \sim_{QS} Q$, if for any $\rho \in \mathcal{D}(\calH_{QN})$ it holds $\confw{\rho, P} \simqs \confw{\rho, Q}$.
\end{definition}

In other words, when $\conf \rightarrow \Delta$, a bisimilar configuration $\conf'$ must perform a transition $\conf' \rightarrow \Theta$, but differently from $\simps$, $\Delta$ and $\Theta$ are not required to be themselves Larsen-Skou bisimilar, but two distributions in their equivalence classes must be Larsen-Skou bisimilar.

%%%version with equivalence classes
%\begin{definition}
%	A \emph{quantum distribution} is an equivalence class $\mathbf{\Delta}$ of probabilistic distributions $\Delta$ over quantum configurations defined by the minimal equivalence relation such that:
%	\begin{itemize}
%		\item $(\overline{\confw{\rho, P}} \psum{p} \overline{\confw{\sigma, P}}) \equiv \overline{\confw{p \rho + (1-p)\sigma, P}}$; and
%%		\item $(\overline{\confw{\rho \otimes \sigma, P}} \psum{p} \overline{\confw{\rho \otimes \sigma', Q}}) \equiv (\overline{\confw{\rho \otimes \delta, P}} \psum{p} \overline{\confw{\rho \otimes \delta', Q}})$ if $\Gamma, \Sigma \vdash P$, $\Gamma, \Sigma \vdash Q$, $\rho \in \hilbert_\Sigma$, and $p \sigma + (1 - p) \sigma' =  p \sigma + (1 - p) \sigma' = p \delta + (1 - p) \delta'$; and	
%		\item $\Delta_i\ \equiv\ \Theta_i$, $i = 1, 2$, implies $\Delta_1 \psum{p} \Delta_2 \equiv\ \Theta_1 \psum{p} \Theta_2$.
%	\end{itemize}
%	We write $Q(Conf)$ for quantum distributions over configurations.
%\end{definition}
%
%\begin{definition}
%	Given $\rel \subseteq Conf \times Conf$ be a relation over quantum configurations, let its quantum lifting be the minimal relation $\sqlift{\rel} \subseteq Q(Conf) \times Q(Conf)$ over quantum distributions such that $\Delta\ \slift{\rel}\ \Theta$ with $\Delta \in \mathbf{\Delta}$ and $\Theta \in \mathbf{\Theta}$ implies $\mathbf{\Delta}\ \sqlift{\rel}\ \mathbf{\Theta}$.
%\end{definition}
%
%
%\begin{definition}[Quantum Saturated Bisimilarity]
%	A symmetric relation $\rel \subseteq \conf \times \conf$ is \emph{quantum saturated bisimulation} if $\conf\ \rel\ \conf'$ implies that $\conf, \conf'$ are well-typed under a typing context $\Gamma; \Sigma$, and for any context $B[\_]_{\Gamma, \Sigma}$
%	\begin{itemize}
%		\item if $B[\conf] \downarrow_{c}$ then $B[\conf'] \downarrow_{c}$; and 
%		\item whenever $B[\conf] \xrightarrow{\tau} \Delta \in \mathbf{\Delta}$, there exists $\Delta' \in \mathbf{\Delta'}$ such that $C[\conf'] \xrightarrow{\tau} \Delta'$ and $\mathbf{\Delta}\ \sqlift{\rel}\ \mathbf{\Delta'}$
%		%		\item whenever $\conf' \xrightarrow{\tau} \Delta'$, there exists $\Delta$ such that $\conf \xrightarrow{\tau} \Delta$ and $\Delta \slift{\rel} \Delta'$
%	\end{itemize}
%	Let \emph{quantum saturated bisimilarity} $\sim_{QS}$ be the union of all probabilistic saturated bisimulation.
%\end{definition}


\begin{theorem}
Quantum saturated bisimilarity is strictly coarser then probabilistic saturated bisimilarity.
\end{theorem}
\begin{proof}
To prove that $\simps \subseteq \simqs$ is sufficient to observe that all probabilistic saturated bisimulations are also quantum probabilistic bisimulations, since $\equiv$ is reflexive. To prove that $\simps$ is strictly finer, we need to show two processes that are not probabilistic bisimilar but are quantum bisimilar, like
\[ P = M_{01}[q \rhd x].disc(q) \qquad Q = M_{\pm}[q \rhd x].disc(q)\]  The formal proof that these two processes are quantum bisimilar is postponed to the end of the section.
\end{proof}

To ease the bisimilarity proofs, we now prove that, like probabilistic bisimilarity, also quantum bisimilarity is closed for additional discarded qubits: 
\[\confw{\trQ(\sigma), P} \simqs \confw{\trQ(\nu), Q} \Rightarrow \confw{\sigma, P\parallel \discQ} \simqs \confw{\nu, Q\parallel \discQ}\] 
For the probabilistic case, we proved that \[\confw{\trQ(\sigma), P} \simps \confw{\trQ(\nu), Q} \Rightarrow \confw{\sigma, P\parallel \discQ} \simps \confw{\nu, Q\parallel \discQ}\] so now we have a weaker hypothesis, since $\simqs$ is coarser then $\simps$.

Note that we can use Lemmas \ref{trace_and_sop}, \ref{trace_and_sop_2} and \ref{lemma_transition_partial_trace} as they do not depend on the bisimulation in hand. To deal with the weaker hypothesis, we need only one further lemma, saying that  additional discarded qubits preserve the equivalence relation.

\begin{lemma}\label{lemma_quantum_equivalence_partial_trace}
Let $\sum_i p_i \confw{\trQ(\sigma_i), P_i}$ be a distribution of configuration, with $\sigma_i \in \mathcal{D}(\calH_{\widetilde{p}} \otimes \calH_{\widetilde{q}})$ for each $i$.  If \[\sum_i p_i \confw{\trQ(\sigma_i), P_i} \equiv \sum_j p_j \confw{\rho_j, P_j}\]
then 
\[\sum_i p_i \confw{\sigma_i, P_i\parallel \discQ} \equiv \sum_j p_j \confw{\sigma_j, P_j\parallel \discQ}\]
and $\rho_j = \trQ(\sigma_j)$ for each $j$.
\end{lemma}
\begin{proof}
We proceed by induction on the rules of $\equiv$. For the base case, suppose \[
\confw{\trQ(\sigma), P} \psum{p} \confw{\trQ(\sigma'), P} \equiv \confw{(p)\trQ(\sigma) + (1-p)\trQ(\sigma'), P}\]
We also have, by definition,  \[\confw{\sigma, P\parallel \discQ} \psum{p} \confw{\sigma', P\parallel \discQ} \equiv \confw{(p)\sigma + (1-p)\sigma', P\parallel \discQ}\]
Notice that, due to linearity of partial trace, $(p)\trQ(\sigma) + (1-p)\trQ(\sigma') = \trQ((p)\sigma + (1-p)\sigma')$, and so we have proven the final condition of the base case.
The inductive case is trivial, as it simply combines two distributions, without changing the density matrix of configurations
\end{proof}

We can now prove the desired theorem on additional discarded qubits, now in the quantum case.

\begin{theorem}[$\simqs$ is closed for additional discarded qubits]\label{bisim_closed_by_discard_quantum}
If $\confw{\trQ(\sigma), P} \simqs \confw{\trQ(\nu), Q}$ then $\confw{\sigma, P \parallel \discQ} \simqs \confw{\nu, Q \parallel \discQ}$.
\end{theorem}
\begin{proof}
Let $\sigma, \nu \in \mathcal{D}(\calH_{\widetilde{p}} \otimes \calH_{\widetilde{q}})$. As for theorem \ref{bisim_closed_by_discard}, we need to prove that  if $\rel$ is a quantum saturated bisimulation, then $\rel_{\trQ}$ is a quantum saturated bisimulation, where $\rel_{\trQ}$ is defined as
\[\rel_{\trQ} = \set{\Big(\confw{\sigma, P \parallel \discQ}, \confw{\nu, Q \parallel \discQ}\Big) \quad\mid\quad \confw{\trQ(\sigma), P} \rel \confw{\trQ(\nu), Q}}\]

The proof is similar to the probabilistic bisimulation case, so we will omit some steps.
For any process $R$ we suppose $\confw{\sigma, P \parallel R \parallel \discQ} \rightarrow \sum p_i \confw{\sigma_i, P_i \parallel \discQ}$. Then, from lemma \ref{lemma_transition_partial_trace}, we know that $\confw{\trQ(\sigma), P\parallel R} \rightarrow \sum p_i \confw{\trQ(\sigma_i), P_i}$. But since $\confw{\trQ(\sigma), P} \rel \confw{\trQ(\nu), Q}$, and $\rel$ is a quantum saturated bisimulation, it must be that 
\begin{align*}
& \confw{\trQ(\sigma), P\parallel R} \rightarrow \sum_i p_i \confw{\trQ(\sigma_i), P_i} & & \equiv  &  & \sum_j p_j \confw{\rho_j, P_j} 
\\
 & & & & & \qquad\slift{\rel} 
\\
& \confw{\trQ(\nu), Q\parallel R} \rightarrow \quad\sum_i p_i \confw{\xi_i, Q_i} & & \equiv & &  \sum_j p_j \confw{\xi'_j, Q_j} 
\end{align*} 
Then, we can apply our lemmas to each part of the above diagram: \begin{itemize}
\item From $\sum_i p_i \confw{\trQ(\sigma_i), P_i} \equiv \sum_j p_j \confw{\rho_j, P_j}$ follows, for lemma \ref{lemma_quantum_equivalence_partial_trace}, $\sum_i p_i \confw{\sigma_i, P_i\parallel \discQ} \equiv  \sum_j p_j \confw{\sigma_j, P_j\parallel \discQ}$, with $\rho_j = \trQ(\sigma_j)$
\item From $\confw{\trQ(\nu), Q\parallel R} \rightarrow \quad\sum_i p_i \confw{\xi_i, Q_i}$ follows, for lemma \ref{lemma_transition_partial_trace}, $\confw{\nu, Q\parallel R\parallel \discQ} \rightarrow \sum_i p_i \confw{\nu_i, Q_i\parallel \discQ}$, with $\xi_i = \trQ(\nu_i)$
\item From $\sum_i p_i \confw{\xi_i, Q_i} \equiv  \sum_j p_j \confw{\xi'_j, Q_j}$ follows, for lemma \ref{lemma_quantum_equivalence_partial_trace}, $\sum_i p_i \confw{\nu_i, Q_i\parallel \discQ} \equiv  \sum_j p_j \confw{\nu_j, Q_j\parallel \discQ}$, with $\xi'_j = \trQ(\nu_j)$. 
\end{itemize}
In conclusion, together with the definition of $\rel_{\trQ}$, we get
\begin{align*}
& \confw{\sigma, P\parallel R \parallel \discQ} \rightarrow \sum_i p_i \confw{\sigma_i, P_i \parallel \discQ} & & \equiv  &  & \sum_j p_j \confw{\sigma_j, P_j \parallel \discQ} 
\\
 & & & & & \qquad\quad\slift{\rel_{\trQ}} 
\\
& \confw{\nu, Q\parallel R\parallel \discQ} \rightarrow \quad\sum_i p_i \confw{\nu_i, Q_i\parallel \discQ} & & \equiv & &  \sum_j p_j \confw{\nu_j, Q_j\parallel \discQ}
\end{align*}
And so $\confw{\sigma, P\parallel R \parallel \discQ} \rel_{\trQ}\confw{\nu, Q\parallel R\parallel \discQ}$.
\end{proof}

As an example, considere the following formula, that from \cite{davidsonFormalVerificationTechniques2012} we expect to hold.
\begin{example}
We expect that $$P =  M_{01}[q_0 \rhd x].disc(q_0) \simqs Q = M_\pm[q_0 \rhd x].disc(q_0)$$
\end{example}

To prove this result, we need the above Theorem \ref{bisim_closed_by_discard_quantum}

Given that $\emptyset, {q_0} \vdash P$ and $\emptyset, {q_0} \vdash Q$, we show that
\[\rel = \big\{ \confw{\sigma, B[P]}, \confw{\sigma, B[Q]} \mid \sigma \in \mathcal{D}(\calH_{QN}), B[\blank]_{\emptyset; \set{q}} \text{ typed context}\big\}^S\ \cup \sim_{PS}
\]
is a quantum saturated bisimulation, where $\rel^S$ denotes the symmetric closure of a relation $\rel$. This is sufficient to prove our statement as it trivially follows that $\confw{\sigma, P} \sim_{QS} \confw{\sigma, Q}$ for any $\sigma$, and so $P$ and $Q$ are bisimilar processes.

$\rel$ is a \textit{saturated} relation, meaning that if $\conf \rel \conf'$, then $B[\conf] \rel B[\conf']$ for any $B$. So, to prove that $\rel$ is a quantum saturated bisimulation, we just need to show that $\rel$ is a quantum bisimulation.

Suppose that $\confw{\sigma, R \parallel P} \rel \confw{\sigma, R \parallel Q}$, and that $\confw{\sigma, R \parallel P} \rightarrow \sum_i p_i \conf_i$.\begin{itemize}
\item If the reductions happens in $R$, it must be of the form $\confw{\sigma, R \parallel P} \rightarrow \sum_i p_i \confw{\sigma_i, R_i \parallel P}$, but then there exists a transition $\confw{\sigma, R \parallel Q} \rightarrow \sum_i p_i \confw{\sigma_i, R_i \parallel Q}$, and for each $i$, $\confw{\sigma_i, R_i \parallel P} \rel \confw{\sigma_i, R_i \parallel Q}$ by definition of $\rel$.
\item If the reductions happens in $P$, it must be 
 \[\confw{\sigma, R \parallel M_{01}[q \rhd x].disc(q)} \rightarrow \confw{\frac{1}{p_0}\sop_{0, q}(\sigma), R \parallel disc(q)} \psum{p_0} \confw{\frac{1}{p_1}\sop_{1, q}(\sigma), R \parallel disc(q)}\]
 where $\sop_{0, q}(\rho) = (\proj{0}\otimes I) \rho (\proj{0} \otimes I)^\dagger$ is the trace non-increasing superoperator that projects qubit $q$ to $\kz$, $p_0$ is the probability of obtaining outcome $0$ from $\sigma$, and similarly for $\sop_{1, q }$ and $p_1$. But then  
 \[\confw{\sigma, R \parallel M_\pm[q \rhd x].disc(q)} \rightarrow \confw{\frac{1}{p_+}\sop_{+, q}(\sigma), R \parallel disc(q)} \psum{+} \confw{\frac{1}{p_-}\sop_{-, q}(\sigma), R \parallel disc(q)}
 \] 
 Noticeably these two distributions are not probabilistic bisimilar, as it is evident in the case $\confw{\Phi^+, P} \rightarrow \confw{\proj{00}, \discQ} \psum{\frac{1}{2}} \confw{\proj{11}, \discQ}$ and $\confw{\Phi^+, Q} \rightarrow \confw{\proj{++}, \discQ} \psum{\frac{1}{2}} \confw{\proj{--}, \discQ}$.  Thanks to the quantum equivalence relation, however, we have 
 \begin{gather*}
 \confw{\frac{1}{p_0}\sop_{0, q}(\sigma), R \parallel disc(q)} \psum{p_0} \confw{\frac{1}{p_1}\sop_{1, q}(\sigma), R \parallel disc(q)} \equiv \overline{\confw{\sop_{0, q}(\sigma) + \sop_{1, q}(\sigma), R \parallel disc(q)}}
 \\
 \confw{\frac{1}{p_+}\sop_{+, q}(\sigma), R \parallel disc(q)} \psum{p_+} \confw{\frac{1}{p_-}\sop_{-, q}(\sigma), R \parallel disc(q)} \equiv \overline{\confw{\sop_{+, q}(\sigma) + \sop_{-, q}(\sigma), R \parallel disc(q)}}
 \end{gather*}
 Observe that $\sop_{0, q}(\sigma) + \sop_{1, q}(\sigma)$ can be seen as $\sop_{01, q}(\sigma)$, where $\sop_{01, q}$ is the trace-preserving superoperator that measures the qubit $q$ in the computational basis and then discards the result. $\sop_{01, q}(\sigma)$ is indeed a well defined superoperator, with $\proj{0}\otimes I$ and $\proj{1}\otimes I$ one of its Kraus decomposition. The same holds also for $\sop_{+, q}(\sigma) + \sop_{-, q}(\sigma) = \sop_{\pm, q}(\sigma)$. So we can conclude
 \[ \confw{\sop_{01,q},  R \parallel disc(q)} \simqs \confw{\sop_{\pm,q},  R \parallel disc(q)}
 \]
 from Theorem \ref{bisim_closed_by_discard_quantum}, since $tr_q(\sop_{01, q}(\sigma)) = tr_q(\sop_{\pm, q}(\sigma)) = tr_q(\sigma)$.
\end{itemize}

As a general result, note that for quantum saturated bisimilarity all the following processes are equivalent 
\[	U(q).disc(q) \simqs M[q \rhd x].disc(q) \simqs \sop(q).disc(q) \simqs \tau.disc(q) \]
for any unitary $U$, measurement $M$ or superoperator $\sop$.


