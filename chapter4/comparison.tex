We summarize here the similarity and differences of lqCCS with
the related works discussed in \ref{chapter3}.
In particular, we focus on QPALg and qCCS, that share with us a similar model.
We will see that a relevant problem of QPALg is solved by our approach, that
our bisimulations share some desired properties of qCCS, and finally we will show
that starting from a qCCS-like calculus, we have establishd results similar to the
ones that are peculiar of CQP.

\subsection{QPALg}
We show here, by means of an example, that lqCCS solve a relevant problem of qbit transmission in QPALg, resulting in a too much coarse-graned bisimilarity. 

Consider the two configurations
\[ \conf = \confw{\frac{1}{2}I \otimes \frac{1}{2}I, c!q_1 \parallel c!q_2} \qquad \conf' = \confw{\proj{\Phi^+}, c!q_1 \parallel c!q_2}
\] where $\ket{\Phi^+} = \oost\ket{00} + \oost{11}$. According to the labelled bisimulation of QPAlg, the two configurations are bisimilar, as they both send two qubits with reduced density operator $\frac{1}{2}I$ on channel $c$. For our definition instead it holds $\conf \not\simps \conf'$, as they are distinguished by the context \[B[\blank] = [\blank]\parallel c?x.c?y.M[x, y \rhd z].\ite{z = 2}{d!0 \parallel disc(x, y)}{\parallel disc(x, y)}\] where $M$ is the measurement on the $4$-dimensional computational basis \[M = \set{M_0 = \proj{00}, M_1 = \proj{01}, M_2 = \proj{10}, M_3 = \proj{11}}\]
After receiving and measuring two unrelated mixed state qubits, $B[\conf]$ will evolve in the distribution  $\Delta$
\[ \frac{1}{4}\confw{\proj{00}, disc(\widetilde{q})} + \frac{1}{4}\confw{\proj{01}, disc(\widetilde{q})} + \frac{1}{4}\confw{\proj{10}, d!0 \parallel disc(\widetilde{q})} + \frac{1}{4}\confw{\proj{11}, disc(\widetilde{q})} \] where $\widetilde{q}$ is the couple $q_1, q_2$.
After receiving and measuring two entangled qubits, instead, $B[\conf']$ will evolve in the distribution $\Delta'$
\[ \frac{1}{2}\confw{\proj{00}, disc(\widetilde{q})} +  \frac{1}{2}\confw{\proj{11}, disc(\widetilde{q})} \] 
We have that $\Delta \slift{\not\sim}_{PS} \Delta'$, as there is no decomposition of $\Delta = \sum_{i\in I} p_i \conf_i$ and $\Delta' = \sum_{i\in I} p_i \conf_i'$ such that $\conf_i \simps \conf_i'$ for each $i$, because $\Delta$ contains a configuration that expresses the barb $\downarrow_d$, while $\Delta'$ contains none.

\subsection{qCCS}
We compare here lqCCS with qCCS, showing that some defining properties of CCS is recovered in lqCCS. We omit here the main difference discusse above, which is the new bisimulation ~QS, shown above to be more adequate to odel behavioural equivalence in quantum processes.

Our probabilistic saturated bisimilarity is designed to be equivalent to open bisimilarity for qCCS, except for a few intended modification:\begin{itemize}
\item Open bisimilarity requires bisimilar processes to have the same free variables, $\simps$ requires bisimilar processes to have the same typing context.
\item Open bisimilarity is superoperator-closed by definition, $\simps$ is proven to be superoperator closed.
\item Open bisimilarity requires bisimilar configurations to have the same environment, $\simps$ implies that bisimilar configurations have the same environment.
\item Open bisimilarity is contex closed as a property, $\simps$ is context closed by definition.
\end{itemize}


In the following, taken a superoperator $\sop \in \mathcal{TS}(\hilbert_{\overline{\Sigma}})$ that acts only on the qubits \textit{outside} $\Sigma$, we write $\sop(\confw{\rho, P})$ to denote the configuration $\confw{\sop\otimes\mathcal{I}_\Sigma(\rho), P}$, with $\mathcal{I}_\Sigma$ the identity superoperator on qubits $\Sigma$.

\begin{theorem}
	For any pair of configurations $\confw{\rho, P}, \confw{\sigma, Q}$ well-typed under $\Gamma; \Sigma$, 
	$\confw{\rho, P} \simps \confw{\rho, P}$ implies
	\begin{enumerate}
		{\item for any $\sop \in \mathcal{TS}(\hilbert_{\overline{\Sigma}})$, $\sop (\confw{\rho, P})\ \simps \ \sop (\confw{\rho, P})$\label{point:thmchinese1_quantum}}
		{\item $tr_{\Sigma}(\rho) = tr_{\Sigma}(\sigma)$. \label{point:thmchinese2_quantum}}
	\end{enumerate}
\end{theorem}
\begin{proof}
 To prove point \ref{point:thmchinese1_quantum}, suppose $\confw{\rho, P} \simqs \confw{\sigma, Q}$, with $\rho, \sigma \in \calH_{\widetilde{q},\widetilde{p}}$ and $\Gamma;\widetilde{p} \vdash P$, $\Gamma;\widetilde{p} \vdash Q$.  For any superoperator $\sop \in \mathcal{TS}(\hilbert_{\widetilde{q}})$ we can construct a context $$B[\blank] = [\blank] \parallel \sop({\widetilde{q}}).a!0 \parallel c!\widetilde{q}$$ where $a$ is a fresh channel. We know that $\confw{\rho, B[P]}$ and $\confw{\sigma, B[Q]}$ are bisimilar, and$\confw{\rho, B[P]}$ can evolve in $\confw{\sop\otimes\mathcal{I}_{\widetilde{p}}(\rho), a!0 \parallel c!\widetilde{q} \parallel P}$. Then $\confw{\sigma, B[Q]}$ must necessarily evolve in $\confw{\sop\otimes \mathcal{I}_{\widetilde{p}}(\sigma), a!0 \parallel c!\widetilde{q} \parallel Q}$, because it must match the $\downarrow_a$ barb, and $a$ is fresh. So we have that  
\[ \confw{\sop\otimes\mathcal{I}_{\widetilde{p}}(\rho), a!0 \parallel c!\widetilde{q} \parallel P} \simqs 
  \confw{\sop\otimes \mathcal{I}_{\widetilde{p}}(\sigma), a!0 \parallel c!\widetilde{q} \parallel Q}
\]
and from this it follows 
\[ \confw{\sop\otimes\mathcal{I}_{\widetilde{p}}(\rho), P} \simqs 
  \confw{\sop\otimes \mathcal{I}_{\widetilde{p}}(\sigma), Q}
\] simply by contradiction: if there was a context capable of distinguishing $\sop(\rho, P)$ from $\sop(\rho, Q)$ then there would be a context able to distinguish also $\sop(\rho, P \parallel a!0 \parallel c!\widetilde{q})$ from $\sop(\sigma, Q \parallel a!0 \parallel c!\widetilde{q})$

To prove point \ref{point:thmchinese2_quantum} we proceed by contradiction, supposing  $\confw{\rho, P} \simqs \confw{\sigma, Q}$  and $tr_{\widetilde{p}}(\rho) \neq tr_{\widetilde{p}}(\sigma)$, with a $\rho, \sigma \in \calH_{\widetilde{q}, \widetilde{p}}$ and $\Gamma;\widetilde{p} \vdash P$, $\Gamma;\widetilde{p} \vdash Q$.
 If $tr_{\widetilde{p}}(\rho) \neq tr_{\widetilde{p}}(\sigma)$, then there exists a measurement $M_{\widetilde{q}} = \set{M_1, \ldots ,M_{m}}$ that distinguishes them, i.e. such that $p_m(tr_{\widetilde{p}}(\rho)) = tr(M_m tr_{\widetilde{p}}(\rho) M_m^\dagger) \neq tr(M_m tr_{\widetilde{p}}(\sigma) M_m^\dagger) = p_m(tr_{\widetilde{p}}(\sigma))$ for some $m$. But for lemma \ref{trace_and_sop}, the same probabilities arise also from the measurement $M_{\widetilde{q}\widetilde{p}} = \set{M_1 \otimes I_{\widetilde{p}}, \ldots , M_m \otimes I_{\widetilde{p}}}$, that  can therefore distinguish the whole state $\rho$ from $\sigma$. So, taken the context 
\[[\blank] \parallel M[\widetilde{q} \rhd x].\big(disc(\widetilde{q}) \parallel \ite{x = 1}{c_1!0}{\ldots \ite{x = m-1}{c_{m-1}!0}{c_m!0}}\big) \]
where $c_0 \ldots c_m$ are fresh channels, we have that $\confw{\rho, B[P]}$ should be bisimilar to $\confw{\sigma, B[Q]}$. But 
\[\confw{\rho, B[P]} \rightarrow \sum_m p_m(\rho) \confw{\rho_m, c_m!0}\]
and $\confw{\sigma, B[Q]}$ can perform only the transition \[\confw{\sigma, B[Q]} \rightarrow \sum_m p_m(\sigma) \confw{\sigma_m, c_m!0}\] to match the barbs, but we know that $p_m(\rho) \neq p_m(\sigma)$ for at least one $m$.
\end{proof}

We can actually prove a stronger property, that in lqCCS the action of "external" superoperators $\sop \in \mathcal{TS}(\calH_{\overline{\Sigma}})$ does not change the observable behaviour of a quantum system.

\note{By writing $\sop \in TSO(\hilbert_{\Sigma})$, we mean that $\sop(\rho) = \sum_{i \in I} (I_{\overline{\Sigma}} \otimes A_i) \rho (I_{\overline{\Sigma}} \otimes A_i)^{\dagger}$.}

\begin{proposition}\label{lemma:sop}
	For any pair of configurations $\conf, \conf'$ well-typed under $\Gamma; \Sigma$, and for any $\sop \in TSO(\hilbert_{\overline{\Sigma}})$,
	$\conf \rightarrow \Delta$ iff $\sop(\conf) \rightarrow \sop(\Delta)$.
\end{proposition}
\begin{proof}
	The only non trivial rules are {\scshape SemQOp} and {\scshape SemQMeas}.
	Assume without loss of generality that $env(\conf) \in \hilbert_{\Sigma \otimes \overline{\Sigma}}$.
	By {\scshape SemQOp}, $\conf = \langle \rho, \mathcal{E}(\widetilde{x}) . P \rangle \longrightarrow \langle \mathcal{E}_{\widetilde{x}}(\rho), P \rangle = \Delta$.
	The type system ensures that $\tilde{x} \subseteq \Sigma$.
	We assume without loss of generality that $\tilde{x} = \Sigma$.
	We can also apply {\scshape SemQOp} as follows, $\sop(\conf) = \langle \sop(\rho), \mathcal{E}(\widetilde{x}) . P \rangle  \longrightarrow \langle \mathcal{E}_{\widetilde{x}}(\sop(\rho)), P \rangle$.
	We need to prove that $\langle \mathcal{E}_{\widetilde{x}}(\sop(\rho)), P \rangle = \langle \sop(\mathcal{E}_{\widetilde{x}}(\rho)), P \rangle = \sop(\Delta)$, i.e., that $\sop_{\widetilde{x}}(\sop(\rho)) = \sop(\sop_{\widetilde{x}}(\rho))$.
	By definition, $\sop_{\widetilde{x}}(\rho) = \sum_{i = 1}^{n} (I_{\Sigma} \otimes A_i) \rho (I_{\Sigma} \otimes A_i)^{\dagger}$, and $\sop(\rho) = \sum_{j = 1}^{m} (B_j \otimes I_{\overline{\Sigma}}) \rho (B_j \otimes I_{\overline{\Sigma}})^{\dagger}$.

	By linearity 
	\begin{align*}
	&\sop_{\widetilde{x}}(\sop(\rho)) =\\ 
	&\sum_{i = 1}^{n} (I_{\Sigma} \otimes A_i) (\sum_{j = 1}^{m} (B_j \otimes I_{\overline{\Sigma}}) \rho (B_j \otimes I_{\overline{\Sigma}})^{\dagger}) (I_{\Sigma} \otimes A_i)^{\dagger} =\\
	&\sum_{i = 1}^{n} \sum_{j = 1}^{m} (I_{\Sigma} \otimes A_i) (B_j \otimes I_{\overline{\Sigma}}) \rho (B_j \otimes I_{\overline{\Sigma}})^{\dagger} (I_{\Sigma} \otimes A_i)^{\dagger},
	\end{align*}
	and
	\begin{align*}
	&\sop(\sop_{\widetilde{x}}(\rho)) =\\
	&\sum_{j = 1}^{m} (B_j \otimes I_{\overline{\Sigma}}) (\sum_{i = 1}^{n} (I_{\Sigma} \otimes A_i) \rho (I_{\Sigma} \otimes A_i)^{\dagger}) (B_j \otimes I_{\overline{\Sigma}})^{\dagger} =\\
	&\sum_{i = 1}^{n} \sum_{j = 1}^{m} (B_j \otimes I_{\overline{\Sigma}}) (I_{\Sigma} \otimes A_i) \rho (I_{\Sigma} \otimes A_i)^{\dagger} (B_j \otimes I_{\overline{\Sigma}})^{\dagger}.
	\end{align*}
	Thanks to conjugate properties, it is thus sufficient to show that
	\[
	(B_{p\times p} \otimes I_{q\times q}) (I_{p\times p} \otimes A_{q\times q}) = (I_{p\times p} \otimes A_{q\times q}) (B_{p\times p} \otimes I_{q\times q}).
	\]
	
	This is easily proven thanks to the mixed product property of the Kronecker product, telling us that 
	\[ (A \otimes B)(C \otimes D) = (AC)\otimes(BD)
	\]
	so in our case, we have 
	\[
	(B_{p\times p} \otimes I_{q\times q}) (I_{p\times p} \otimes A_{q\times q}) = B_{p\times p} \otimes A_{q \times q} = (I_{p\times p} \otimes A_{q\times q}) (B_{p\times p} \otimes I_{q\times q})
	\]	
The proof for rule {\scshape SemQMeas} is the same, considering every $m \in \{0, \dots, 2^{\widetilde{x}}\}$ separately.
\end{proof}



\subsection{CQP}

The type system is inspired from the affine type system of CQP. To transform an affine system in a linear system, is necessary to remove the \textit{weakening} rule for $\Sigma$:
\[ \infer[\mbox{\footnotesize\scshape QWeak}]{\Gamma; \Sigma,q \vdash P}{\Gamma; \Sigma \vdash P}
\]
that allows to introduce a quantum name $q$ in $\Sigma$ even if it is not sent by $P$. In CQP, weakening is not an explicit rule of the typesystem, but is used implicitly in various rules, making it an affine type system.


noi abbiamo il ite, quindi vere distribuzioni, non solo mixed


\begin{example}
Let $\rho = \frac{1}{2} \ketbra{0}{0}$ and $\rho' = \frac{1}{2} \ketbra{1}{1}$, then
\begin{align*}
	\langle \rho, P \rangle \boxplus \langle \rho', P \rangle \equiv \langle \rho + \rho', P \rangle \sim \langle \frac{1}{2} I, P \rangle
\end{align*}
In particular, the following holds in our system but not in~\cite{Feng:2012, Deng:2012}
\begin{align*}
	Set_{\ketbra{+}{+}}(q).M(q \triangleright x).c!0 \sim Set_{\frac{1}{2} I}(q).\tau.c!0
\end{align*}
\end{example}
