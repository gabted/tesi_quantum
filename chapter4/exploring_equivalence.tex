\subsection{Exploring quantum behavioural equivalence}
We now investigate some expected properties of a bisimulation for quantum processes, and how that they are not met by the standard probabilistic bisimulation. We will see that these problems are related to the use of density operators to represent mixed states, with interesting on sequences on non-determinism.

\subsubsection*{The problem of mixed states}

We show that the usual way of defining behavioural equivalence through probabilistic lifting (section \ref{pLTS}) does not work in the quantum case.
This because probabilistic mixtures of quantum states are represented in two alternative ways, i.e., using density operators and using distributions of configurations.
Two different representations of the same mixture should be equivalent, but this is not always the case in the proposed bisimilarity (and in literature).

\begin{example}
	Consider the following configurations and distributions.
	\begin{align*}
		\conf &= \confw{\ketbra{0}{0}, Set_{\ketbra{+}{+}}(q).M_{0,1}(q \triangleright x).(c!0 \parallel discard(q))}\\
		\conf' &= \confw{\ketbra{0}{0}, Set_{\frac{1}{2} I}(q).\tau.(c!0 \parallel discard(q))}\\
		\Delta &= \overline{\confw{\ketbra{0}{0}, c!0 \parallel discard(q)}} \psum{1/2} \overline{\confw{\ketbra{1}{1}, c!0 \parallel discard(q)}}\\
		\Delta' &= \overline{\confw{1/2 I, c!0 \parallel discard(q)}}
	\end{align*}
	It is trivial that $\conf \simps \conf'$ iff $\Delta	\slift{\simps} \Delta'$.
	It seems obvious that the former should hold.
	Indeed the behaviour of $\confw{\rho, c!0 \parallel discard(q)}$ does not depend on $\rho$.
	Unfortunately, this is not the case for $\slift{\simps}$ .
\end{example}
Indeed, $\conf$ and $\conf'$ of the previous example are not bisimilar in qCCS according to \cite{fengBisimulationQuantumProcesses2012, dengOpenBisimulationQuantum2012}.

A related problem is that a single density operator may represent different probabilistic ensembles (i.e. distribution), that are indistinguishable according to quantum theory. In (l)qCCS, however, these distributions of configurations can be
distinguished in $\simps$ by some context. 
\begin{example}
	Consider the following distributions.
	\begin{align*}
		\Delta &= \overline{\confw{\ketbra{0}{0}, c!q}} \psum{1/2} \overline{\confw{\ketbra{1}{1}, c!q}}\\
		\Delta' &= \overline{\confw{\ketbra{+}{+}, c!q}} \psum{1/2} \overline{\confw{\ketbra{-}{-}, c!q}}
	\end{align*}
	The two distributions only send a qbit on the channel $c$, and the qbit may be in state $\ket{0}$ or $\ket{1}$ for $\Delta$ ($\ket{+}$ or $\ket{-}$ for $\Delta'$) with equal probability.
	It seems reasonable that $\Delta$ and $\Delta'$ cannot be distinguished by an external observer receiving the qbit.
	This because the received qbit is in the probabilistic mixture $\{(\ket{0}, 1/2), (\ket{1}, 1/2)\}$ for $\Delta$ and $\{(\ket{+}, 1/2), (\ket{-}, 1/2)\}$ respectively for $\Delta'$, but the two mixtures are both represented by the same density operator $1/2 I = 1/2 \ketbra{0}{0} + 1/2 \ketbra{1}{1} = 1/2 \ketbra{+}{+} + 1/2 \ketbra{-}{-}$.
	Unfortunately, this is not the case for $\slift{\simps}$, because the two distributions are composed by configurations that are not "point-wise" bisimilar.
\end{example}
Once more, \cite{fengBisimulationQuantumProcesses2012, dengOpenBisimulationQuantum2012} distinguish the two distributions, hence, e.g., the two processes $Set_{\frac{1}{2} I}(q).M_{0,1}[q \triangleright x].c!q$ and $Set_{\frac{1}{2} I}(q).M_{+,-}[q \triangleright x].c!q$ are not considered bisimilar.

\subsubsection*{The problem of non-determinism}

We present here a strange behaviour that common process algebras express when mixing quantum states with non-deterministic choices.
Assume a distribution obtained by tracing out one qubit of a bell pair $\ket{\Phi^+} = 1/\sqrt{2}(\ket{00} + \ket{11})$.
This can be obtained, e.g., with the following configuration $\confw{\ket{\Phi^+}, discard(q_0) \parallel \dots}$.
The qbit is in a probabilistic ensamble of pure states, represented with the density opertator $1/2 I$, which stands for both $\{(\ket{0}, 1/2), (\ket{1}, 1/2)\}$ and $\{(\ket{+}, 1/2), (\ket{-}, 1/2)\}$.
The two ensambles are indeed indistinguishable from the quantum point of view.
However, the two express a different behaviour according to a trivial semantics that just applies probabilistic reasoning to quantum processes, as shown in the following.
\begin{example}
	Let the processes $P$ and $Q$ and distributions $\Delta$, $\Theta$ be
	\begin{align*}
		P\ =\ &M_{0,1}[q \triangleright x] . \ite{x=0}{z!0}{u!0}\\
		Q\ =\ &M_{+,-}[q \triangleright x] . \ite{x=0}{p!0}{m!0}\\
		\Delta\ =\ &\confw{\ketbra{0}{0}, P + Q} \psum{1/2} \confw{\ketbra{1}{1}, P + Q}\\
		\Theta\ =\ &\confw{\ketbra{+}{+}, P + Q} \psum{1/2} \confw{\ketbra{-}{-}, P + Q}
	\end{align*}
	There is a way of performing the non deterministic choice such that $\Delta \rightarrow \confw{\ketbra{0}{0}, z!0}$ with probability $1/2$, $\Delta \rightarrow \confw{\ketbra{+}{+}, p!0}$ and $\Delta \rightarrow \confw{\ketbra{-}{-}, m!0}$ both with probability $1/4$.
	The apparent contradiction is that $\Theta$ cannot replicate this behaviour.
\end{example}

Apparently, the previous example seems to contradict the common understanding of indistinguishability of quantum states.
To recover it, either this behaviour must be expressed by both $\Delta$ and $\Theta$, or by none of them, resulting in two different notions of non deterministic choice.
Actually, something strange happens in the example, namely, the non deterministic choice is performed in the two probabilistic branches, possibly in different ways.
