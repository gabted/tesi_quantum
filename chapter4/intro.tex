We have seen that a number of different behavioural equivalences have been proposed in the literature. We have also seen that this is partially due to an ambiguity related to implicitly discarded qbits; a problem that is solved by  lqCCS. Noticeably, others  differencies on observable properties are more involved and inherently quantum related. As in classical process algebra, it is always possible to observe if a process $P$ can perform a classical input or output action, which is the foundation of both \textit{labeled} and \textit{barbed} bisimulation. The discrepancies in state-of-the-art proposals are mainly about the observable properties of quantum states. In QPAlg, the quantum state is observed only when sent, while in qCCS, the whole "environment" is visible, i.e. all the qubits that are not used by $P$ anymore. In (mixed configuration) CQP there is again a similar notion of "environment", but what is observable is the environment of the whole distribution, not the environment of the single configurations.

The main purpose of this section is to investigate which are the most natural notions of behavioural equivalence, and how some apparently minor details lead to completely diverse equivalence relations. Being interested in the purely quantum aspect of behavioural equivalence, we will only consider "strong" relations, like strong bisimilarity and strong barbed congruence. Strong relations in fact are usually the first, simpler step to develop a behevioural equivalence notion, and after establishing the most appropriate notion of observable property of quantum states, we plan to extend it to the weak case.

We will first define a \textit{probabilistic saturated bisimilarity} \cite{bonchiGeneralTheoryBarbs2014} for Linear qCCS. Saturated bisimilarity is a solid and general notion of observable equivalence, aiming to capture when two processes can or can not be distinguished by an external observer, i.e. by an arbitraty context. We will describe some example of such a bisimilarity, togoether with a useful property that helps when proving the bisimilarity of two processes.

Then we will analyze some peculiarities of these probabilistic equivalence notions, extending the ideas already discussed in \cite{davidsonFormalVerificationTechniques2012}. In particular, the bisimilarities defined for QPAlg and qCCS and lqCCS, seem to grant the external observer a greater discerning power then what is prescribed by quantum mechanics. That is, the resulting behavioural equivalence is too much  fine-grained. We identify the well established notion of Larsen-Skou bisimulation (as described in \ref{pLTS}) as the cause of these undesired behaviours, and propose a novel notion of \textit{quantum saturated bisimilarity} that better fits the observable properties of distribution of quantum configurations. We believe that this bisimilarity complies with the ideas presented by \cite{davidsonFormalVerificationTechniques2012} for mixed configuration CQP, albeit here presented in a essentially different transition system.

Finally we will rephrase some of the examples in  the previous chapter in lqCCS, comparing our bisimilarity with the other behavioural equivalence presented in the literature. We will show that our notion of probabilistic saturated bisimilarity for lqCCS is consistent with the open bisimilarity  \cite{dengOpenBisimulationQuantum2012} for qCCS, while quantum saturated bisimilarity is strictly larger.