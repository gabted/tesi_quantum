\subsection{Probabilistic Saturated Bisimilarity}

Saturated bisimilarity \cite{bonchiGeneralTheoryBarbs2014} grants the external observer, when comparing two configurations $\conf$ and $\conf'$ the power to put $\conf$ and $\conf'$ inside any context at each step of the computations. Recall that in barbed congruence (as defined in section \ref{bkg_reduction_system}), the observer can compare $\conf$ and $\conf'$ using just one arbitrary context at the start of the computation, not any context at each step like in saturated bisimilarity.

We use (probabilistic) saturated bisimilarity as it is a established, general behavioural equivalence, useful to investigate reduction system where there is no affirmed notion of observable property. Besides, as we will see, saturated bisimilarity is able to capture some properties used in \cite{dengOpenBisimulationQuantum2012} to define open bisimilarity.


We define as a barb, the atomic observable property, only the capability of performing an output on a channel. We do not consider input action observables, as it is the norm for asynchronous calculi \cite{amadioBisimulationsAsynchronousPcalculus1998}. Since the external observer is an asynchronous, non blocking context, it can not observe the process $P$ to discorver if it has a $c?x$ transition available: the context can only perform a terminal action $c!v$, but it will not know if the output action is ever received.

\begin{definition}[Barb]
	A \emph{barb} is a predicate $\downarrow_{c}$ over well typed processes, defined as follows: $P \downarrow_{c}$ if and only if $P \equiv c!e \parallel R$ for some expression $e$, and processes $R$.
\end{definition}

%\begin{definition}[Barb on configuration]
%	A \emph{barb} is a predicate $\downarrow_{c}$ over configurations, defined as follows: $\langle \rho, P \rangle \downarrow_{c}$ iff $P \equiv c!x + Q \parallel R$ for some $x$, and processes $Q$, $R$.
%\end{definition}

If $\conf = \confw{\rho, P}$ is a configuration, we will often write $C\downarrow_c$ instead of $P\downarrow_c$, and $B[\conf]$ instead of $\confw{\rho, B[P]}$.

\begin{definition}[Probabilistic Saturated Bisimilarity]
	A symmetric relation $\rel \subseteq \conf \times \conf$ is \emph{probabilistic saturated (barbed) bisimulation} if $\confw{\rho, P} \ \rel\ \confw{\sigma, Q}$ implies that $P$ and $Q$  are well-typed under the same typing context $\Gamma; \Sigma$, and for any context $B[\_]_{\Gamma, \Sigma}$
	\begin{itemize}
		\item If $P \downarrow_{c}$ then $Q \downarrow_{c}$
		\item If $\confw{\rho, B[P]} \rightarrow \Delta$, there exists $\Delta' $ such that $\confw{\sigma, B[Q]} \rightarrow \Delta'$ and $\mathbf{\Delta}\ \sqlift{\rel}\ \mathbf{\Delta'}$
	\end{itemize}
	Let \emph{probabilistic saturated bisimilarity} $\sim_{PS}$ be the union of all saturated probabilistic barbed bisimulation. \\
	We say that two {processes} $P$ and $Q$ are \emph{bisimilar}, written $P \sim_{PS} Q$, if for any $\rho \in \mathcal{D}(\calH_{QN})$ it holds $\confw{\rho, P} \simps \confw{\rho, Q}$.
\end{definition}

Note that it is not necessary to use a barb $\downarrow_{c!v}$, so to consider also which is the value that is communicated. A context, in fact, is capable to discern $c!v.P$ and $c!v'.P$ thanks to the \textbf{if-then-else} construct for classical values, or to the measure operator for quantum values. This is an innovation with respect to QPAlg, wehere the observables of $\confw{\rho, c!q.P}$ were both the channel and the value (i.e. the reduced density operator) of $q$. 

\note{Non so se questo esempio metterlo qui o metterlo in una sezione confronti} 

Consider the two configurations
\[ \conf = \confw{\frac{1}{2}I \otimes \frac{1}{2}I, c!q_1 \parallel c!q_2} \qquad \conf' = \confw{\proj{\Phi^+}, c!q_1 \parallel c!q_2}
\] where $\ket{\Phi^+} = \oost\ket{00} + \oost{11}$. According to the labelled bisimulation of QPAlg, the two configurations are bisimilar, as they both send two qubits with reduced density operator $\frac{1}{2}I$ on channel $c$. For our definition instead it holds $\conf \not\simps \conf'$, as there exists the context \[B[\blank] = [\blank]\parallel c?x.c?y.M[x, y \rhd z].\ite{z = 2}{d!0 \parallel disc(x, y)}{\parallel disc(x, y)}\] where $M$ is the measurement on the $4$-dimensional computational basis \[M = \set{M_0 = \proj{00}, M_1 = \proj{01}, M_2 = \proj{10}, M_3 = \proj{11}}\]
After receiving and measuring two unrelated mixed state qubits, $B[\conf]$ will evolve in the distribution  $\Delta$
\[ \frac{1}{4}\confw{\proj{00}, disc(\widetilde{q})} + \frac{1}{4}\confw{\proj{01}, d!0 \parallel disc(\widetilde{q})} + \frac{1}{4}\confw{\proj{10}, disc(\widetilde{q})} + \frac{1}{4}\confw{\proj{11}, disc(\widetilde{q})} \] where $\widetilde{q}$ is the couple $q_1, q_2$.
After receiving and measuring two unrelated qubits, instead, $B[\conf']$ will evolve in the distribution $\Delta'$
\[ \frac{1}{2}\confw{\proj{00}, disc(\widetilde{q})} +  \frac{1}{2}\confw{\proj{11}, disc(\widetilde{q})} \] 
We have that $\Delta \slift{\not\sim}_{PS} \Delta'$ bisimilar, as there is no decomposition of $\Delta = \sum_{i\in I} p_i \conf_i$ and $\Delta' = \sum_{i\in I} p_i \conf_i'$ such that $\conf_i \simps \conf_i'$ for each $i$, because $\Delta$ contains a configuration that expresses the barb $\downarrow_d$, while $\Delta'$ contains none.


\begin{example}
	Consider the following, wrong definition of quantum teleportation:
	\begin{align*}
		\proc{A} &\Coloneqq \text{in}_a?x.\text{CNOT}(q_0, x).\text{H}(q_0).M(x,q_0 \rhd n).(\text{m}_a!n \parallel out_{a_1}! q_0 \parallel out_{a_2}! x )\\
    \proc{B} &\Coloneqq \text{in}_b?x.\text{m}_a?n.
       \\ & \ite{n = 0}{\sigma_0(x).\text{out}_b!x\\&\quad}
      {\ite{n = 1}{\sigma_1(x).\text{out}_b!x\\&\qquad}
          {\ite{n = 2}{\sigma_2(x).\text{out}_b!x}{\sigma_3(x).\text{out}_b!x}}
      } \\
		\proc{S} &\Coloneqq \text{H}(q_1).\text{CNOT}(q_1, q_2).(\text{in}_a!q_1 \parallel \text{in}_b!q_2) \\
		\proc{Tel} &\Coloneqq (A \parallel B \parallel S) \setminus \Set{\text{in}_a, \text{in}_b, \text{m}_a } \\
		\proc{TelSpec} &\Coloneqq \text{SWAP}(q_0,q_2).(\text{out}_b!q_2 \parallel out_{a_1}! q_0 \parallel out_{a_2}! q_1)
	\end{align*}
	We have that $\proc{Tel} \not\sim_{PS} \proc{TelSpec}$.
	Consider indeed a context:
  \[ B[\blank] = out_{a_1} ? x . M[x \rhd y] . \ite{y = 1}{c!y}{nil}. \]
\end{example}

Probabilistic saturated bisimilarity is designed to be equivalent to open bisimilarity for qCCS, except for a few intended modification:\begin{itemize}
\item Open bisimilarity equates processes with the same free variables, $\simps$ equates processes with the same typing context.
\item Open bisimilarity is superoperator closed by definition, $\simps$ is superoperator closed as a property.
\item Open bisimilarity requires the environment of two configurations to be equal by definition, $\simps$ requires the environment of two configurations to be equal as a property.
\item Open bisimilarity is contex closed as a property, $\simps$ is context closed by definition.
\end{itemize}


By writing $\sop \in TSO(\hilbert_{\Sigma})$, we mean that $\sop(\rho) = \sum_{i \in I} (I_{\overline{\Sigma}} \otimes A_i) \rho (I_{\overline{\Sigma}} \otimes A_i)^{\dagger}$.

\begin{lemma}\label{lemma:sop}
	For any pair of configurations $\conf, \conf'$ well-typed under $\Gamma; \Sigma$, and for any $\sop \in TSO(\hilbert_{\overline{\Sigma}})$,
	$\conf \rightarrow \Delta$ iff $\sop(\conf) \rightarrow \sop(\Delta)$.
\end{lemma}
\begin{proof}
	The only non trivial rules are {\scshape SemQOp} and {\scshape SemQMeas}.
	Assume without loss of generality that $env(\conf) \in \hilbert_{\Sigma \otimes \overline{\Sigma}}$.
	By {\scshape SemQOp}, $\conf = \langle \rho, \mathcal{E}(\widetilde{x}) . P \rangle \longrightarrow \langle \mathcal{E}_{\widetilde{x}}(\rho), P \rangle = \Delta$.
	The type system ensures that $\tilde{x} \subseteq \Sigma$.
	We assume without loss of generality that $\tilde{x} = \Sigma$.
	We can also apply {\scshape SemQOp} as follows, $\sop(\conf) = \langle \sop(\rho), \mathcal{E}(\widetilde{x}) . P \rangle  \longrightarrow \langle \mathcal{E}_{\widetilde{x}}(\sop(\rho)), P \rangle$.
	We need to prove that $\langle \mathcal{E}_{\widetilde{x}}(\sop(\rho)), P \rangle = \langle \sop(\mathcal{E}_{\widetilde{x}}(\rho)), P \rangle = \sop(\Delta)$, i.e., that $\sop_{\widetilde{x}}(\sop(\rho)) = \sop(\sop_{\widetilde{x}}(\rho))$.
	By definition, $\sop_{\widetilde{x}}(\rho) = \sum_{i = 1}^{n} (I_{\Sigma} \otimes A_i) \rho (I_{\Sigma} \otimes A_i)^{\dagger}$, and $\sop(\rho) = \sum_{j = 1}^{m} (B_j \otimes I_{\overline{\Sigma}}) \rho (B_j \otimes I_{\overline{\Sigma}})^{\dagger}$.

	By linearity 
	\begin{align*}
	&\sop_{\widetilde{x}}(\sop(\rho)) =\\ 
	&\sum_{i = 1}^{n} (I_{\Sigma} \otimes A_i) (\sum_{j = 1}^{m} (B_j \otimes I_{\overline{\Sigma}}) \rho (B_j \otimes I_{\overline{\Sigma}})^{\dagger}) (I_{\Sigma} \otimes A_i)^{\dagger} =\\
	&\sum_{i = 1}^{n} \sum_{j = 1}^{m} (I_{\Sigma} \otimes A_i) (B_j \otimes I_{\overline{\Sigma}}) \rho (B_j \otimes I_{\overline{\Sigma}})^{\dagger} (I_{\Sigma} \otimes A_i)^{\dagger},
	\end{align*}
	and
	\begin{align*}
	&\sop(\sop_{\widetilde{x}}(\rho)) =\\
	&\sum_{j = 1}^{m} (B_j \otimes I_{\overline{\Sigma}}) (\sum_{i = 1}^{n} (I_{\Sigma} \otimes A_i) \rho (I_{\Sigma} \otimes A_i)^{\dagger}) (B_j \otimes I_{\overline{\Sigma}})^{\dagger} =\\
	&\sum_{i = 1}^{n} \sum_{j = 1}^{m} (B_j \otimes I_{\overline{\Sigma}}) (I_{\Sigma} \otimes A_i) \rho (I_{\Sigma} \otimes A_i)^{\dagger} (B_j \otimes I_{\overline{\Sigma}})^{\dagger}.
	\end{align*}
	Thanks to conjugate properties, it is thus sufficient to show that
	\[
	(B_{p\times p} \otimes I_{q\times q}) (I_{p\times p} \otimes A_{q\times q}) = (I_{p\times p} \otimes A_{q\times q}) (B_{p\times p} \otimes I_{q\times q}).
	\]
	
	This is easily proven thanks to the mixed product property of the Kronecker product, telling us that 
	\[ (A \otimes B)(C \otimes D) = (AC)\otimes(BD)
	\]
	so in our case, we have 
	\[
	(B_{p\times p} \otimes I_{q\times q}) (I_{p\times p} \otimes A_{q\times q}) = B_{p\times p} \otimes A_{q \times q} = (I_{p\times p} \otimes A_{q\times q}) (B_{p\times p} \otimes I_{q\times q})
	\]	
The proof for rule {\scshape SemQMeas} is the same, considering every $m \in \{0, \dots, 2^{\widetilde{x}}\}$ separately.
\end{proof}


\begin{theorem}
	For any pair of configurations $\conf, \conf'$ well-typed under $\Gamma; \Sigma$, 
	$\conf \simps \conf'$ implies
	\begin{enumerate}
		{\item for any $\sop \in \mathcal{TS}(\hilbert_{\overline{\Sigma}})$, $\sop (\conf)\ \simps \ \sop (\conf')$; and \label{point:thmchinese1}}
		{\item $tr_{\Sigma}(\conf) = tr_{\Sigma}(\conf')$. \label{point:thmchinese2}}
	\end{enumerate}
\end{theorem}
\begin{proof}
	The proof of point \ref{point:thmchinese1} follows by transitivity and Lemma~\ref{lemma:sop}. \note{questa dimostrazione va bene per la probabilistic bisimulation, ma non per la quantum bisimulation, quindi la cambierei direttamente con quella nuova}
	\[
	 \sop (\conf)\ \approx_{SPB}\ \conf\ \approx_{SPB}\ \conf'\ \approx_{SPB}\ \sop (\conf') 
	\]
	For point \ref{point:thmchinese2} we proceed by refutation.
	Assume $\conf \approx_{SPB} \conf'$ and $tr_{\Sigma}(\conf) \neq tr_{\Sigma}(\conf')$.
	Take $env(\conf), env(\conf') \in \hilbert_{\{ q \} \otimes \Sigma}$ with $env(\conf) = \ket{0} \otimes \ket{\phi}$ and $env(\conf') = \ket{1} \otimes \ket{\phi'}$.
	Consider the context $B[\_] = M[q \triangleright x] . \textbf{ if } x = 0 \textbf{ then } c!q \textbf{ else } free(q) \parallel [\_]$.
	It is clear that $B[\conf] \not\approx_{SPB} B[\conf']$. Contradiction.
\end{proof}

%An established notion of behavioural equivalence between quantum processes is the Deng-Feng bisimilarity, here we focus on the strong version of it.
%\begin{definition}[Deng-Feng Bisimilarity]
%	A symmetric relation $\rel \subseteq \conf \times \conf$ is \emph{Deng-Feng bisimulation} if $\conf\ \rel\ \conf'$ implies that $\conf, \conf'$ are well-typed under a typing context $\Gamma; \Sigma$, and for any context $C[\blank] \in Context_{\Gamma, \Sigma}$
%	\begin{itemize}
%		\item $tr_{\Sigma}(\conf) = tr_{\Sigma}(\conf')$;
%		\item if $C[\sop(\conf)] \downarrow_{c}$ then $C[\sop(\conf')] \downarrow_{c}$; and
%		\item whenever $C[\sop(\conf)] \xrightarrow{\tau} \Delta$, there exists $\Delta'$ such that $C[\sop(\conf')] \xrightarrow{\tau} \Delta'$ and $\Delta \slift{\rel} \Delta'$
%	\end{itemize}
%	Let \emph{Deng-Feng bisimilarity} $\approx_{DF}$ be the largest saturated probabilistic barbed bisimulation.
%\end{definition}
%
%\begin{theorem}
%	For any pair of configurations $\conf, \conf'$, $\conf \approx_{DF} \conf'$ iff $\conf \approx_{SPB} \conf'$.
%\end{theorem}

\note{altre dimostrazioni che la chiusura per superoperatori non serve a niente}

A useful proof techinque: $\sim$ is closed for additional discarded qubits. 

\newcommand{\discQ}{disc(\widetilde{q})}
\newcommand{\trQ}{tr_{\widetilde{q}}}

\note{vanno sistemate le parentesi, e comunque si potrebbe anche scrivere nel capitolo del background quantum}
\begin{lemma}\label{trace_and_sop}
Let $\sigma \in  \mathcal{D}(\calH_{\widetilde{p}} \otimes \calH_{\widetilde{q}})$, with $\widetilde{p} = p_0 \ldots p_{n-1}$ and $\widetilde{q} = q_0 \ldots q_{m-1}$. Then, for any \textit{trace non-increasing} superoperator $\sop_{\widetilde{p}} \in \mathcal{S}(\calH_{\widetilde{p}})$
\[ \trQ((\sop_{\widetilde{p}} \otimes \mathcal{I}_{\widetilde{q}})(\sigma)) = \sop_{\widetilde{p}}(\trQ(\sigma))
\]
\end{lemma}
\begin{proof}
We know that $\sigma$ is a probabilistic mixture of pure states, $\sigma = \sum_l p_l \proj{\psi_l}$, and each pure state is a linear combination of separable states 
\[\sigma = \sum_l p_l \sum_{p = 0}^{2^n-1} \sum_{q = 0}^{2^m-1} \lambda_{pq} \proj{pq}\]
and we know that $\sop_{\widetilde{p}}$ has a Kraus decomposition 
\[\sop_{\widetilde{p}}(\rho) = \sum_k A_k \rho A_k^\dagger\]
Then we can write 
\hspace*{-2cm}\begin{align*}
\trQ((\sop_{\widetilde{p}} \otimes \mathcal{I}_{\widetilde{q}})(\sigma)) &= \sop_{\widetilde{p}}(\trQ(\sigma)) 
\\ 
\trQ((\sop_{\widetilde{p}} \otimes \mathcal{I}_{\widetilde{q}})\left(\sum_l p_l \sum_{p = 0}^{2^n-1} \sum_{q = 0}^{2^m-1} \lambda_{pq} \proj{pq}\right)) &= \sop_{\widetilde{p}}(\trQ\left(\sum_l p_l \sum_{p = 0}^{2^n-1} \sum_{q = 0}^{2^m-1} \lambda_{pq} \proj{pq}\right))
\\
\sum_l p_l \trQ((\sop_{\widetilde{p}} \otimes \mathcal{I}_{\widetilde{q}})\left( \sum_{p = 0}^{2^n-1} \sum_{q = 0}^{2^m-1} \lambda_{pq} \proj{pq}\right)) 
&= \sum_l p_l \sop_{\widetilde{p}}(\trQ\left( \sum_{p = 0}^{2^n-1} \sum_{q = 0}^{2^m-1} \lambda_{pq} \proj{pq}\right))
\\
\trQ(\sum_k (A_k \otimes I_{\widetilde{q}})\left( \sum_{p = 0}^{2^n-1} \sum_{q = 0}^{2^m-1} \lambda_{pq} \proj{pq}\right)(A_k \otimes I_{\widetilde{q}})^\dagger )
&= \sum_k A_k\left( \sum_{p = 0}^{2^n-1} \sum_{q = 0}^{2^m-1} \lambda_{pq} \trQ(\proj{pq})\right) A_k^\dagger 
\\
\trQ(\sum_k  \left( \sum_{p = 0}^{2^n-1} \sum_{q = 0}^{2^m-1} \lambda_{pq} (A_k \otimes I_{\widetilde{q}}) \proj{pq} (A_k \otimes I_{\widetilde{q}})^\dagger  \right) )
&= \sum_k A_k\left( \sum_{p = 0}^{2^n-1} \sum_{q = 0}^{2^m-1} \lambda_{pq} \trQ(\proj{pq})\right) A_k^\dagger
\\
\trQ(\sum_k  \left( \sum_{p = 0}^{2^n-1} \sum_{q = 0}^{2^m-1} \lambda_{pq} (A_k\ket{p})\ket{q}(\bra{p}A_k^\dagger) \bra{q} \right) )
&= \sum_k A_k\left( \sum_{p = 0}^{2^n-1} \sum_{q = 0}^{2^m-1} \lambda_{pq} \proj{p}\braket{q | q}\right)A_k^\dagger
\\
\sum_{p = 0}^{2^n-1} \sum_{q = 0}^{2^m-1} \lambda_{pq} \sum_k \trQ( (A_k\ket{p})\ket{q}(\bra{p}A_k^\dagger) \bra{q} ) 
&= \sum_k  \sum_{p = 0}^{2^n-1} \sum_{q = 0}^{2^m-1} \lambda_{pq} \sum_k  A_k \proj{p} A_k^\dagger 
\\
\sum_{p = 0}^{2^n-1} \sum_{q = 0}^{2^m-1} \lambda_{pq} \sum_k (A_k \proj{p} A_k^\dagger) 
&=  \sum_{p = 0}^{2^n-1} \sum_{q = 0}^{2^m-1} \lambda_{pq} \sop_{\widetilde{p}} (\proj{p})
\\
\sum_{p = 0}^{2^n-1} \sum_{q = 0}^{2^m-1} \lambda_{pq} \sop_{\widetilde{p}}(\proj{p})
&=  \sum_{p = 0}^{2^n-1} \sum_{q = 0}^{2^m-1} \lambda_{pq} \sop_{\widetilde{p}} (\proj{p})
\end{align*}

Where we made use of the fact that both $\trQ$ and matrix multiplication are closed for linearity.
\end{proof}

\begin{lemma}\label{trace_and_sop_2}
Let $\sigma \in  \mathcal{D}(\calH_{\widetilde{p}} \otimes \calH_{\widetilde{q}})$, with $\widetilde{p} = p_0 \ldots p_{n-1}$ and $\widetilde{q} = q_0 \ldots q_{m-1}$. Then, for any \textit{trace non-increasing} superoperator $\sop_{\widetilde{q}} \in \mathcal{S}(\calH_{\widetilde{q}})$
\[ \trQ((\mathcal{I}_{\widetilde{p}} \otimes \sop_{\widetilde{q}})(\sigma)) = \trQ(\sigma)
\]
\end{lemma}

The proof of this lemma is extremely similar to the one before, and thus omitted. \note{si potrebbe dimostrare anche un unico lemma che li derivi entrambi, cioè che $\trQ(E_p \otimes E_q)(\sigma) = E_p\trQ(\sigma)$. Però mi pare che avere 2 lemmi separati renda le dimostrazioni di sotto più comprensibili.}


\note{va sistemata la notazione dei pedici dei superoperatori.}
\begin{lemma}\label{lemma_transition_partial_trace}
Let $\sigma \in  \mathcal{D}(\calH_{\widetilde{p}} \otimes \calH_{\widetilde{q}})$, and $\rho = tr_{\widetilde{q}}(\sigma) \in \calH_{\widetilde{p}}$. Then, for any process 
\[ \confw{\rho, P} \rightarrow \sum_i p_i \confw{\rho_i, P_i} 
\qquad\Leftrightarrow\qquad
\confw{\sigma, P\parallel \discQ} \rightarrow \sum p_i \confw{\sigma_i, P_i\parallel \discQ}
\]
and $\forall i \ \rho_i = \trQ(\sigma_i)$
\end{lemma}
\begin{proof}
We proceed by induction on $\rightarrow$. For the simple base cases {\footnotesize\scshape SemTau} and {\footnotesize\scshape SemReduce}, where the quantum state remains unchanged, we have 
\begin{align*}
\confw{\rho, \tau.P} \rightarrow \confw{\rho, P}
&\Leftrightarrow
\confw{\sigma, \tau.P\parallel \discQ} \rightarrow \confw{\sigma, P\parallel \discQ}
\\
 \confw{\rho, c?x.P\parallel c!v} \rightarrow \confw{\rho, P[v / x]}
&\Leftrightarrow
\confw{\sigma, c?x.P\parallel c!v \parallel \discQ} \rightarrow \confw{\sigma, P[v / x]\parallel \discQ}
\end{align*}

and in both cases $\rho' = \rho = \trQ(\sigma)$.

For the base case {\footnotesize\scshape SemQOp} we have 
\[\confw{\rho, \sop(\widetilde{p}).P} \rightarrow \confw{\sop_{\widetilde{p}}(\rho), P}
\qquad\Leftrightarrow\qquad
\confw{\sigma, \sop(\widetilde{p}).P\parallel \discQ} \rightarrow \confw{\sop_{\widetilde{p}}(\sigma), P\parallel \discQ}\]
where $\rho' = \sop_{\widetilde{p}}(\rho) = \sop_{\widetilde{p}}(\trQ(\sigma))$ and $\sigma' = \sop_{\widetilde{p}}\otimes \mathcal{I}_{\widetilde{q}}(\sigma)$, and so we have $\rho' = \trQ(\sigma') = \trQ(\sop_{\widetilde{p}}\otimes \mathcal{I}_{\widetilde{q}}(\sigma))$ thanks to lemma \ref{trace_and_sop}.


The base case {\footnotesize\scshape SemQMeasure} is similar, we have 
\begin{gather*}
\confw{\rho, M[\widetilde{p} \rhd x].P} \rightarrow \sum p_m\confw{p_m^{-1}\sop_m(\rho), P} \\
\Leftrightarrow \\
\confw{\sigma, M[\widetilde{p} \rhd x].P\parallel \discQ} \rightarrow \sum p'_m \confw{p_m^{-1}\sop_m(\sigma), P\parallel \discQ}
\end{gather*}
where thanks to lemma \ref{trace_and_sop}, we have $p_m = tr(\sop_m(\rho)) = tr(\sop_m(\trQ(\sigma))) = tr(\trQ(\sop_m \otimes \mathcal{I}_{\widetilde{q}}(\sigma)))$ and $\forall i \ \rho_i = \trQ(\sigma_i)$, as in the previous case.

The inductive cases are all trivial, as none of them modifies the quantum values $\rho$ and $\sigma$.
\end{proof}

\newcommand{\relTrQ}{\rel_{\trQ}}
\begin{lemma}\label{lemma_reltrq_probabilistic}
Let $\sigma \in \mathcal{D}(\calH_{\widetilde{p}} \otimes \calH_{\widetilde{q}})$. If $\rel$ is a bisimulation, then $\rel_{\trQ}$ is a bisimulation, where $\rel_{\trQ}$ is defined as
\[\rel_{\trQ} = \set{\Big(\confw{\sigma, P \parallel \discQ}, \confw{\nu, Q \parallel \discQ}\Big) \quad\mid\quad \confw{\trQ(\sigma), P} \rel \confw{\trQ(\nu), Q}}\]
\end{lemma}
\begin{proof}
To show that $\rel_{\trQ}$ is a bisimulation, we need first of all to show that is symmetric, which is when $\rel$ is symmetric. The we suppose that $\confw{\sigma, P \parallel \discQ} \relTrQ \confw{\nu, Q \parallel \discQ}$ and have to show that \begin{itemize}
\item For any barb $b$, if $\confw{\sigma, P \parallel \discQ} \downarrow_b$ then $\confw{\nu, Q \parallel \discQ} \downarrow_b$. We have that $\confw{\sigma, P \parallel \discQ} \downarrow_b \Leftrightarrow \confw{\trQ(\sigma), P} \downarrow b$, $\confw{\nu, Q \parallel \discQ} \downarrow_b \Leftrightarrow \confw{\trQ(\nu), Q} \downarrow b$, and 
$\confw{\trQ(\sigma), P} \rel \confw{\trQ(\nu), Q}$, so they all express the same barbs.
\item For any process $R$, if $\confw{\sigma, P \parallel R \parallel \discQ} \rightarrow \Delta$, then $\confw{\nu, Q \parallel R \parallel \discQ} \rightarrow \Theta$, and $\Delta \slift{\relTrQ} \Theta$. Notice that $\discQ$ can not evolve, and so we start from the hypothesis $\confw{\sigma, P \parallel R \parallel \discQ} \rightarrow \sum p_i \confw{\sigma_i, P_i \parallel \discQ}$. Then, from lemma \ref{lemma_transition_partial_trace}, we know that $\confw{\trQ(\sigma), P\parallel R} \rightarrow \sum p_i \confw{\trQ(\sigma_i), P_i}$. But since $\confw{\trQ(\sigma), P} \rel \confw{\trQ(\nu), Q}$, and $\rel$ is a saturated bisimulation, it must be that $\confw{\trQ(\nu), Q\parallel R} \rightarrow \sum p_i \confw{\xi_i, Q_i}$ with $\confw{\trQ(\sigma_i), P_i} \rel \confw{\xi_i, Q_i}$ for each $i$. But using the same lemma \ref{lemma_transition_partial_trace} in the other direction, we get $\confw{\nu, Q \parallel R \parallel \discQ} \rightarrow \sum p_i \confw{\nu_i, Q_i \parallel \discQ}$ with $\xi_i = \trQ(\nu_i)$. In conclusion, for each transition $\confw{\sigma, P \parallel R \parallel \discQ} \rightarrow \sum p_i \confw{\sigma_i, P_i \parallel \discQ}$ exists a transition $\confw{\nu, Q \parallel R \parallel \discQ} \rightarrow \sum p_i \confw{\nu_i, Q_i \parallel \discQ}$ such that $\forall i \confw{\trQ(\sigma_i), P_i} \rel \confw{\trQ(\nu_i), Q_i}$, and so from the definiton of $\relTrQ$ together with probabilistic lifting we get $\sum p_i \confw{\sigma_i, P_i \parallel \discQ} \slift{\relTrQ} \sum p_i \confw{\nu_i, Q_i \parallel \discQ}$.
\end{itemize}
\end{proof}

\note{è così immediato che si potrebbe anche integrare nel lemma sopra.}
\begin{theorem}[$\simps$ is closed for additional discarded qubits]\label{bisim_closed_by_discard}
If $\confw{\trQ(\sigma), P} \sim \confw{\trQ(\nu), Q}$ then $\confw{\sigma, P \parallel \discQ} \sim \confw{\nu, Q \parallel \discQ}$.
\end{theorem}
\begin{proof}
It easily follows from the previous lemma, considering that $\simps$ is the union of all bisimulations.
\end{proof}

Theorem \ref{bisim_closed_by_discard} is useful to prove bisimilarity for some simple processes. For example, it can be used to prove that $P = H[q].discard(q)$ and $Q = X[q].discard(q)$ are bisimilar.

Given that $\emptyset, {q} \vdash P$ and $\emptyset, {q} \vdash Q$, we show that
\[\rel = \big\{\confw{\sigma, B[P]}, \confw{\sigma, B[Q]} \mid \sigma \in \mathcal{D}(\calH_{QN}), B[\blank]_{\emptyset; \set{q}} \text{ typed context}\big\}^S\ \cup \sim_{PS}
\]
is a probabilistic saturated bisimulation, where $\rel^S$ denotes the symmetric closure of a relation $\rel$. From this follows trivially that $\confw{\sigma, P} \sim_{PS} \confw{\sigma, Q}$ for any $\sigma$, and so $P$ and $Q$ are bisimilar processes.

$\rel$ is a \textit{saturated} relations, meaning that if $\conf \rel \conf'$, then $B[\conf] \rel B[\conf']$ for any $B$. So, to prove that $\rel$ is a probabilistic saturated bisimulation, we just need to show that $\rel$ is a probabilistic bisimulation.

Suppose that $\confw{\sigma, R \parallel P} \rel \confw{\sigma, R \parallel Q}$, and that $\confw{\sigma, R \parallel P} \rightarrow \sum_i p_i \conf_i$.\begin{itemize}
\item If the reductions happens in $R$, it must be of the form $\confw{\sigma, R \parallel P} \rightarrow \sum_i p_i \confw{\sigma_i, R' \parallel P}$, but then there exists a transition $\confw{\sigma, R \parallel Q} \rightarrow \sum_i p_i \confw{\sigma_i, R_i \parallel Q}$, and for each $i$, $\confw{\sigma_i, R_i \parallel P} \rel \confw{\sigma_i, R_i \parallel Q}$ by definition of $\rel$.
\item If the reductions happens in $P$, it must be 
 $\confw{\sigma, R \parallel H[q].disc(q)} \rightarrow \confw{\sop_{H, q}(\sigma), R \parallel disc(q)}$, but then  $\confw{\sigma, R \parallel X[q].disc(q)} \rightarrow \confw{\sop_{X, q}(\sigma), R \parallel disc(q)}$, where $\sop_{H, q}$ is the superoperator that applies the H transformation only on qubit $q$, and $\sop_{Z, q}$ is the superoperator that applies the Z transformation only on qubit $q$. 
Since $tr_q(\sop_{H, q}(\sigma)) = tr_q(\sop_{Z, q}(\sigma)) = tr_q(\sigma)$ for lemma \ref{trace_and_sop_2}, we have that $\confw{\sop_{H, q}(\sigma), R \parallel disc(q)} \sim_{PS} \confw{\sop_{X, q}(\sigma), R \parallel disc(q)}$ for lemma \ref{trace_and_sop}, and so $$\confw{\sop_{H, q}(\sigma), R \parallel disc(q)} \rel \confw{\sop_{X, q}(\sigma), R \parallel disc(q)}$$ by definition of $\rel$.
 
\end{itemize}
