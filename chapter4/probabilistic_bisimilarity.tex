\subsection{Probabilistic Saturated Bisimilarity}

Saturated bisimilarity \cite{bonchiGeneralTheoryBarbs2014} grants the external observer, when comparing two configurations $\conf$ and $\conf'$, the power to put $\conf$ and $\conf'$ inside any context at each step of the computations. Recall that in barbed congruence (as defined in section \ref{bkg_reduction_system}), the observer can compare $\conf$ and $\conf'$ using just one arbitrary context at the start of the computation, not any context at each step like in saturated bisimilarity.

We use (probabilistic) saturated bisimilarity as it is an established, general behavioural equivalence, useful to investigate reduction system where there is no affirmed notion of observable property. Besides, as we will see, saturated bisimilarity is able to capture some properties used in \cite{dengOpenBisimulationQuantum2012} to define open bisimilarity.


We define "the capability of performing an output on a channel" as the only barbs, i.e. the atomic observable properties. We do not assume input actions to be observable, following the standard for asynchronous calculi \cite{amadioBisimulationsAsynchronousPcalculus1998}. Since the external observer is asynchronous, it can not observe the process $P$ to discorver if a $c?x$ transition is available: the context can only perform $c!v$ as a terminal action, hence it cannot discover if  the message is ever received.

\begin{definition}[Barb]
	A \emph{barb} is a predicate $\downarrow_{c}$ over well typed processes, defined as follows: $P \downarrow_{c}$ if and only if $P \equiv c!e \parallel R$ for some expression $e$, and processes $R$.
\end{definition}

%\begin{definition}[Barb on configuration]
%	A \emph{barb} is a predicate $\downarrow_{c}$ over configurations, defined as follows: $\langle \rho, P \rangle \downarrow_{c}$ iff $P \equiv c!x + Q \parallel R$ for some $x$, and processes $Q$, $R$.
%\end{definition}

If $\conf = \confw{\rho, P}$ is a configuration, we will often write $C\downarrow_c$ instead of $P\downarrow_c$, and $B[\conf]$ instead of $\confw{\rho, B[P]}$.

\begin{definition}[Probabilistic Saturated Bisimilarity]
	A symmetric relation $\rel \subseteq \conf \times \conf$ is \emph{probabilistic saturated (barbed) bisimulation} if $\confw{\rho, P} \ \rel\ \confw{\sigma, Q}$ implies that $P$ and $Q$  are well-typed under the same typing context $\Gamma; \Sigma$, and for any context $B[\_]_{\Gamma, \Sigma}$
	\begin{itemize}
		\item If $P \downarrow_{c}$ then $Q \downarrow_{c}$
		\item If $\confw{\rho, B[P]} \rightarrow \Delta$, there exists $\Theta$ such that $\confw{\sigma, B[Q]} \rightarrow \Theta$ and $\Delta\ \slift{\rel}\ \Theta$
	\end{itemize}
	Let \emph{probabilistic saturated bisimilarity} $\sim_{PS}$ be the union of all saturated probabilistic barbed bisimulation. \\
	We say that two {processes} $P$ and $Q$ are \emph{bisimilar}, written $P \sim_{PS} Q$, if for any $\rho \in \mathcal{D}(\calH_{QN})$ it holds $\confw{\rho, P} \simps \confw{\rho, Q}$.
\end{definition}

Note that it is not necessary to use a barb $\downarrow_{c!v}$, so to consider also which is the value that is communicated. A context, in fact, is capable to discern $c!v.P$ and $c!v'.P$ thanks to the \textbf{if-then-else} construct for classical values, or to the measure operator for quantum values. This is an innovation with respect to QPAlg, wehere the observables of $\confw{\rho, c!q.P}$ were both the channel and the value (i.e. the reduced density operator) of $q$. 

Note also that $B[\blank]$ has barb $\downarrow_c$ if and only if $B[\blank]$ has it $P$ has it. Hence it is not required to use contexts when comparing  he barbs of two configurations for checking saturated bisimilarity. 


\begin{example}
	Consider the following, wrong definition of quantum teleportation:
	\begin{align*}
		\proc{A} &\Coloneqq \text{in}_a?x.\text{CNOT}(q_0, x).\text{H}(q_0).M(x,q_0 \rhd n).(\text{m}_a!n \parallel out_{a_1}! q_0 \parallel out_{a_2}! x )\\
    \proc{B} &\Coloneqq \text{in}_b?x.\text{m}_a?n.
       \\ & \ite{n = 0}{\sigma_0(x).\text{out}_b!x\\&\quad}
      {\ite{n = 1}{\sigma_1(x).\text{out}_b!x\\&\qquad}
          {\ite{n = 2}{\sigma_2(x).\text{out}_b!x}{\sigma_3(x).\text{out}_b!x}}
      } \\
		\proc{S} &\Coloneqq \text{H}(q_1).\text{CNOT}(q_1, q_2).(\text{in}_a!q_1 \parallel \text{in}_b!q_2) \\
		\proc{Tel} &\Coloneqq (A \parallel B \parallel S) \setminus \Set{\text{in}_a, \text{in}_b, \text{m}_a } \\
		\proc{TelSpec} &\Coloneqq \text{SWAP}(q_0,q_2).(\text{out}_b!q_2 \parallel out_{a_1}! q_0 \parallel out_{a_2}! q_1)
	\end{align*}
	We have that $\proc{Tel} \not\sim_{PS} \proc{TelSpec}$.
	Consider indeed a context:
  \[ B[\blank] = out_{a_1} ? x . M[x \rhd y] . \ite{y = 1}{c!y}{nil}. \]
\end{example}


We show now some useful properties that helps in deciding bisimilarity. The first result is that \textit{$\simps$ is closed for addition of discarded qbits}.  Rougly, this means that if  $\confw{\rho, P}$ and $\confw{\sigma, Q}$ are bisimilar,  we can add a qbit $q$ and its discard $disc(q)$ to them, as in  $\confw{\rho \otimes \proj{\psi}, P \parallel disc(q)}$ and $\confw{\sigma \otimes \proj{\psi}, Q \parallel disc(q)}$, obtaining bisimilar configurations. More in details, this holds also for non separable states, as shown later.

We first need some technical lemmas that follow. 
\newcommand{\discQ}{disc(\widetilde{q})}
\newcommand{\trQ}{tr_{\widetilde{q}}}

\begin{lemma}\label{trace_and_sop}
Let $\sigma \in  \mathcal{D}(\calH_{\widetilde{p}} \otimes \calH_{\widetilde{q}})$, with $\widetilde{p} = p_0 \ldots p_{n-1}$ and $\widetilde{q} = q_0 \ldots q_{m-1}$. Then, for any \textit{trace non-increasing} superoperator $\sop_{\widetilde{p}} \in \mathcal{S}(\calH_{\widetilde{p}})$
\[ \trQ((\sop_{\widetilde{p}} \otimes \mathcal{I}_{\widetilde{q}})(\sigma)) = \sop_{\widetilde{p}}(\trQ(\sigma))
\]
\end{lemma}
\begin{proof}
We know that $\sigma$ is a probabilistic mixture of pure states, $\sigma = \sum_l p_l \proj{\psi_l}$, and each pure state is a linear combination of separable states 
\[\sigma = \sum_l p_l \sum_{p = 0}^{2^n-1} \sum_{q = 0}^{2^m-1} \lambda_{pq} \proj{pq}\]
and we know that $\sop_{\widetilde{p}}$ has a Kraus decomposition 
\[\sop_{\widetilde{p}}(\rho) = \sum_k A_k \rho A_k^\dagger\]
Then we can write 
\hspace*{-2cm}\begin{align*}
\trQ((\sop_{\widetilde{p}} \otimes \mathcal{I}_{\widetilde{q}})(\sigma)) &= \sop_{\widetilde{p}}(\trQ(\sigma)) 
\\ 
\trQ((\sop_{\widetilde{p}} \otimes \mathcal{I}_{\widetilde{q}})\left(\sum_l p_l \sum_{p = 0}^{2^n-1} \sum_{q = 0}^{2^m-1} \lambda_{pq} \proj{pq}\right)) &= \sop_{\widetilde{p}}(\trQ\left(\sum_l p_l \sum_{p = 0}^{2^n-1} \sum_{q = 0}^{2^m-1} \lambda_{pq} \proj{pq}\right))
\\
\sum_l p_l \trQ((\sop_{\widetilde{p}} \otimes \mathcal{I}_{\widetilde{q}})\left( \sum_{p = 0}^{2^n-1} \sum_{q = 0}^{2^m-1} \lambda_{pq} \proj{pq}\right)) 
&= \sum_l p_l \sop_{\widetilde{p}}(\trQ\left( \sum_{p = 0}^{2^n-1} \sum_{q = 0}^{2^m-1} \lambda_{pq} \proj{pq}\right))
\\
\trQ(\sum_k (A_k \otimes I_{\widetilde{q}})\left( \sum_{p = 0}^{2^n-1} \sum_{q = 0}^{2^m-1} \lambda_{pq} \proj{pq}\right)(A_k \otimes I_{\widetilde{q}})^\dagger )
&= \sum_k A_k\left( \sum_{p = 0}^{2^n-1} \sum_{q = 0}^{2^m-1} \lambda_{pq} \trQ(\proj{pq})\right) A_k^\dagger 
\\
\trQ(\sum_k  \left( \sum_{p = 0}^{2^n-1} \sum_{q = 0}^{2^m-1} \lambda_{pq} (A_k \otimes I_{\widetilde{q}}) \proj{pq} (A_k \otimes I_{\widetilde{q}})^\dagger  \right) )
&= \sum_k A_k\left( \sum_{p = 0}^{2^n-1} \sum_{q = 0}^{2^m-1} \lambda_{pq} \trQ(\proj{pq})\right) A_k^\dagger
\\
\trQ(\sum_k  \left( \sum_{p = 0}^{2^n-1} \sum_{q = 0}^{2^m-1} \lambda_{pq} (A_k\ket{p})\ket{q}(\bra{p}A_k^\dagger) \bra{q} \right) )
&= \sum_k A_k\left( \sum_{p = 0}^{2^n-1} \sum_{q = 0}^{2^m-1} \lambda_{pq} \proj{p}\braket{q | q}\right)A_k^\dagger
\\
\sum_{p = 0}^{2^n-1} \sum_{q = 0}^{2^m-1} \lambda_{pq} \sum_k \trQ( (A_k\ket{p})\ket{q}(\bra{p}A_k^\dagger) \bra{q} ) 
&= \sum_k  \sum_{p = 0}^{2^n-1} \sum_{q = 0}^{2^m-1} \lambda_{pq} \sum_k  A_k \proj{p} A_k^\dagger 
\\
\sum_{p = 0}^{2^n-1} \sum_{q = 0}^{2^m-1} \lambda_{pq} \sum_k (A_k \proj{p} A_k^\dagger) 
&=  \sum_{p = 0}^{2^n-1} \sum_{q = 0}^{2^m-1} \lambda_{pq} \sop_{\widetilde{p}} (\proj{p})
\\
\sum_{p = 0}^{2^n-1} \sum_{q = 0}^{2^m-1} \lambda_{pq} \sop_{\widetilde{p}}(\proj{p})
&=  \sum_{p = 0}^{2^n-1} \sum_{q = 0}^{2^m-1} \lambda_{pq} \sop_{\widetilde{p}} (\proj{p})
\end{align*}

Where we made use of the fact that both $\trQ$ and matrix multiplication are closed for linearity.
\end{proof}

\begin{lemma}\label{trace_and_sop_2}
Let $\sigma \in  \mathcal{D}(\calH_{\widetilde{p}} \otimes \calH_{\widetilde{q}})$, with $\widetilde{p} = p_0 \ldots p_{n-1}$ and $\widetilde{q} = q_0 \ldots q_{m-1}$. Then, for any \textit{trace non-increasing} superoperator $\sop_{\widetilde{q}} \in \mathcal{S}(\calH_{\widetilde{q}})$
\[ \trQ((\mathcal{I}_{\widetilde{p}} \otimes \sop_{\widetilde{q}})(\sigma)) = \trQ(\sigma)
\]
\end{lemma}

The proof of this lemma is extremely similar to the one before, and thus omitted. 

Now we prove that the behaviour of a configuration $\confw{\rho, P}$ is identical to the behaviour of a "bigger" configuration with additional qubits, if these additional qubits are discarded, and so cannot be modified nor measured.

\begin{lemma}\label{lemma_transition_partial_trace}
Let $\sigma \in  \mathcal{D}(\calH_{\widetilde{p}} \otimes \calH_{\widetilde{q}})$, and $\rho = tr_{\widetilde{q}}(\sigma) \in \calH_{\widetilde{p}}$. Then, for any process 
\[ \confw{\rho, P} \rightarrow \sum_i p_i \confw{\rho_i, P_i} 
\qquad\Leftrightarrow\qquad
\confw{\sigma, P\parallel \discQ} \rightarrow \sum p_i \confw{\sigma_i, P_i\parallel \discQ}
\]
and $\forall i \ \rho_i = \trQ(\sigma_i)$
\end{lemma}
\begin{proof}
We proceed by induction on $\rightarrow$. For the simple base cases {\footnotesize\scshape SemTau} and {\footnotesize\scshape SemReduce}, where the quantum state remains unchanged, we have 
\begin{align*}
\confw{\rho, \tau.P} \rightarrow \confw{\rho, P}
&\Leftrightarrow
\confw{\sigma, \tau.P\parallel \discQ} \rightarrow \confw{\sigma, P\parallel \discQ}
\\
 \confw{\rho, c?x.P\parallel c!v} \rightarrow \confw{\rho, P[v / x]}
&\Leftrightarrow
\confw{\sigma, c?x.P\parallel c!v \parallel \discQ} \rightarrow \confw{\sigma, P[v / x]\parallel \discQ}
\end{align*}

and in both cases $\rho' = \rho = \trQ(\sigma)$.

For the base case {\footnotesize\scshape SemQOp} we have 
\[\confw{\rho, \sop(\widetilde{p}).P} \rightarrow \confw{\sop_{\widetilde{p}}(\rho), P}
\qquad\Leftrightarrow\qquad
\confw{\sigma, \sop(\widetilde{p}).P\parallel \discQ} \rightarrow \confw{\sop_{\widetilde{p}}(\sigma), P\parallel \discQ}\]
where $\rho' = \sop_{\widetilde{p}}(\rho) = \sop_{\widetilde{p}}(\trQ(\sigma))$ and $\sigma' = \sop_{\widetilde{p}}\otimes \mathcal{I}_{\widetilde{q}}(\sigma)$, and so we have $\rho' = \trQ(\sigma') = \trQ(\sop_{\widetilde{p}}\otimes \mathcal{I}_{\widetilde{q}}(\sigma))$ thanks to lemma \ref{trace_and_sop}.


The base case {\footnotesize\scshape SemQMeasure} is similar, we have 
\begin{gather*}
\confw{\rho, M[\widetilde{p} \rhd x].P} \rightarrow \sum p_m\confw{p_m^{-1}\sop_m(\rho), P} \\
\Leftrightarrow \\
\confw{\sigma, M[\widetilde{p} \rhd x].P\parallel \discQ} \rightarrow \sum p'_m \confw{p_m^{-1}\sop_m(\sigma), P\parallel \discQ}
\end{gather*}
where thanks to lemma \ref{trace_and_sop}, we have $p_m = tr(\sop_m(\rho)) = tr(\sop_m(\trQ(\sigma))) = tr(\trQ(\sop_m \otimes \mathcal{I}_{\widetilde{q}}(\sigma)))$ and $\forall i \ \rho_i = \trQ(\sigma_i)$, as in the previous case.

The inductive cases are all trivial, as none of them modifies the quantum values $\rho$ and $\sigma$.
\end{proof}

\begin{theorem}[$\simps$ is closed for additional discarded qubits]\label{bisim_closed_by_discard}
If $\confw{\trQ(\sigma), P} \sim \confw{\trQ(\nu), Q}$ then $\confw{\sigma, P \parallel \discQ} \sim \confw{\nu, Q \parallel \discQ}$.
\end{theorem}
\newcommand{\relTrQ}{\rel_{\trQ}}
\begin{proof} 
Let $\sigma, \nu \in \mathcal{D}(\calH_{\widetilde{p}} \otimes \calH_{\widetilde{q}})$. We show that if $\rel$ is a bisimulation, then $\rel_{\trQ}$ is a bisimulation, where $\rel_{\trQ}$ is defined as
\[\rel_{\trQ} = \set{\Big(\confw{\sigma, P \parallel \discQ}, \confw{\nu, Q \parallel \discQ}\Big) \quad\mid\quad \confw{\trQ(\sigma), P} \rel \confw{\trQ(\nu), Q}}\]
From this, it easily follows that $\simps$ is closed for additional discarded qubits, becuse if $\confw{\trQ(\sigma), P} \sim \confw{\trQ(\nu), Q}$, then there exist a bisimulation $\rel$ such that $\confw{\trQ(\sigma), P} \rel \confw{\trQ(\nu), Q}$, and so there exists a bisimulation $\relTrQ \subseteq \simps$ such that $\confw{\sigma, P \parallel \discQ} \relTrQ \confw{\nu, Q \parallel \discQ}$.

To show that $\rel_{\trQ}$ is a bisimulation, we need first of all to show that is symmetric, which is when $\rel$ is symmetric. The we suppose that $\confw{\sigma, P \parallel \discQ} \relTrQ \confw{\nu, Q \parallel \discQ}$ and have to show that \begin{itemize}
\item For any barb $b$, if $\confw{\sigma, P \parallel \discQ} \downarrow_b$ then $\confw{\nu, Q \parallel \discQ} \downarrow_b$. We have that $\confw{\sigma, P \parallel \discQ} \downarrow_b \Leftrightarrow \confw{\trQ(\sigma), P} \downarrow b$, $\confw{\nu, Q \parallel \discQ} \downarrow_b \Leftrightarrow \confw{\trQ(\nu), Q} \downarrow b$, and 
$\confw{\trQ(\sigma), P} \rel \confw{\trQ(\nu), Q}$, so they all express the same barbs.
\item For any process $R$, if $\confw{\sigma, P \parallel R \parallel \discQ} \rightarrow \Delta$, then $\confw{\nu, Q \parallel R \parallel \discQ} \rightarrow \Theta$, and $\Delta \slift{\relTrQ} \Theta$. Notice that $\discQ$ can not evolve, and so we start from the hypothesis $\confw{\sigma, P \parallel R \parallel \discQ} \rightarrow \sum p_i \confw{\sigma_i, P_i \parallel \discQ}$. Then, from lemma \ref{lemma_transition_partial_trace}, we know that $\confw{\trQ(\sigma), P\parallel R} \rightarrow \sum p_i \confw{\trQ(\sigma_i), P_i}$. But since $\confw{\trQ(\sigma), P} \rel \confw{\trQ(\nu), Q}$, and $\rel$ is a saturated bisimulation, it must be that $\confw{\trQ(\nu), Q\parallel R} \rightarrow \sum p_i \confw{\xi_i, Q_i}$ with $\confw{\trQ(\sigma_i), P_i} \rel \confw{\xi_i, Q_i}$ for each $i$. But using the same lemma \ref{lemma_transition_partial_trace} in the other direction, we get $\confw{\nu, Q \parallel R \parallel \discQ} \rightarrow \sum p_i \confw{\nu_i, Q_i \parallel \discQ}$ with $\xi_i = \trQ(\nu_i)$. In conclusion, for each transition $\confw{\sigma, P \parallel R \parallel \discQ} \rightarrow \sum p_i \confw{\sigma_i, P_i \parallel \discQ}$ exists a transition $\confw{\nu, Q \parallel R \parallel \discQ} \rightarrow \sum p_i \confw{\nu_i, Q_i \parallel \discQ}$ such that $\forall i \confw{\trQ(\sigma_i), P_i} \rel \confw{\trQ(\nu_i), Q_i}$, and so from the definiton of $\relTrQ$ together with probabilistic lifting we get $\sum p_i \confw{\sigma_i, P_i \parallel \discQ} \slift{\relTrQ} \sum p_i \confw{\nu_i, Q_i \parallel \discQ}$.
\end{itemize}
\end{proof}



Theorem \ref{bisim_closed_by_discard} is useful when proving bisimilarity of processes that use discard. For example, it can be used to prove that $P = H[q].discard(q)$ and $Q = X[q].discard(q)$ are bisimilar.

Given that $\emptyset, {q} \vdash P$ and $\emptyset, {q} \vdash Q$, we show that
\[\rel = \big\{\confw{\sigma, B[P]}, \confw{\sigma, B[Q]} \mid \sigma \in \mathcal{D}(\calH_{QN}), B[\blank]_{\emptyset; \set{q}} \text{ typed context}\big\}^S\ \cup \sim_{PS}
\]
is a probabilistic saturated bisimulation, where $\rel^S$ denotes the symmetric closure of a relation $\rel$. From this follows trivially that $\confw{\sigma, P} \sim_{PS} \confw{\sigma, Q}$ for any $\sigma$, and so $P$ and $Q$ are bisimilar processes.

$\rel$ is a \textit{saturated} relation, meaning that if $\conf \rel \conf'$, then $B[\conf] \rel B[\conf']$ for any $B$. So, to prove that $\rel$ is a probabilistic saturated bisimulation, we just need to show that $\rel$ is a probabilistic bisimulation.

Suppose that $\confw{\sigma, R \parallel P} \rel \confw{\sigma, R \parallel Q}$, and that $\confw{\sigma, R \parallel P} \rightarrow \sum_i p_i \conf_i$.\begin{itemize}
\item $\confw{\sigma, P \parallel \discQ} \sim \confw{\nu, Q \parallel \discQ}$.
\item If the reductions happens in $R$, it must be of the form $\confw{\sigma, R \parallel P} \rightarrow \sum_i p_i \confw{\sigma_i, R' \parallel P}$, but then there exists a transition $\confw{\sigma, R \parallel Q} \rightarrow \sum_i p_i \confw{\sigma_i, R_i \parallel Q}$, and for each $i$, $\confw{\sigma_i, R_i \parallel P} \rel \confw{\sigma_i, R_i \parallel Q}$ by definition of $\rel$.
\item If the reductions happens in $P$, it must be 
 $\confw{\sigma, R \parallel H[q].disc(q)} \rightarrow \confw{\sop_{H, q}(\sigma), R \parallel disc(q)}$, but then  $\confw{\sigma, R \parallel X[q].disc(q)} \rightarrow \confw{\sop_{X, q}(\sigma), R \parallel disc(q)}$, where $\sop_{H, q}$ is the superoperator that applies the H transformation only on qubit $q$, and $\sop_{Z, q}$ is the superoperator that applies the Z transformation only on qubit $q$. 
Since $tr_q(\sop_{H, q}(\sigma)) = tr_q(\sop_{Z, q}(\sigma)) = tr_q(\sigma)$ for lemma \ref{trace_and_sop_2}, we have that $\confw{\sop_{H, q}(\sigma), R \parallel disc(q)} \sim_{PS} \confw{\sop_{X, q}(\sigma), R \parallel disc(q)}$ for theorem \ref{bisim_closed_by_discard}, and so $$\confw{\sop_{H, q}(\sigma), R \parallel disc(q)} \rel \confw{\sop_{X, q}(\sigma), R \parallel disc(q)}$$ by definition of $\rel$.
 
\end{itemize}
