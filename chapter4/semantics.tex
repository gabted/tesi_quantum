
Assume a fixed set  $QN = {q_1, q_2, \ldots q_n}$ of quantum names, where each name $q_i$ refers to a unique qubit with state space $\calH_i$. We denote as $\calH_{QN}$ the $2^n$-dimensional Hilbert space $\bigoplus_{i=1}^n \calH_i$.

Assume also a fixed typing contest $\Gamma_c = \set{c_i: \widehat{T_i}}_i$, containing typing assumptions for a finite set of classical and quantum channels.

\begin{definition}
Let $P$ be a process and $\rho \in \mathcal{D}(\calH_{QN})$ an arbitrary density operator. We say that a configuration $\confw{\rho, P}$ is well typed, given a set of quantum names $QN$ and a set of typed channels $\Gamma_c$, if $\Gamma_c; \Sigma \vdash P$ and $\Sigma \subseteq QN$.

%  Let $I$ be an arbitrary index set. $\Gamma; \Sigma \vdash \boxplus_{i \in I} \langle \rho_i, P_i \rangle$ iff for each $i \in I$ such that $\rho_i \neq \mathbf{0}$, then $\Gamma; \Sigma \vdash \langle \rho_i, P_i \rangle$.
\end{definition}

Notice that the context $\Gamma_c$ contains only assignment for channels, not for classical variables. This means that in well typed configuration, since $\Gamma_c; \Sigma \vdash P$, $P$ does not contain any free classical variable, and all the free quantum variables are reference to a qubit of the configuration.


\subsection{Reduction System}

We consider as a \textit{value} any expressions $n \in \mathbb{N},\  b \in \mathbb{B},\  v \in  \text{Var}$, and use $v$ as a metavariable for them.

We define a big step semantic for classical and quantum expression in the usual way. We write $e \Downarrow v$ to indicate that the expression $e$ evaluates to value $v$. Recall that the only quantum expression admitted by the type system are quantum variables.

\begin{gather*}
\infer[\mbox{\footnotesize\scshape Var}]{x \Downarrow x}{} \qquad
    \infer[\mbox{\footnotesize\scshape Nat}]{n \Downarrow n}{} \qquad
    \infer[\mbox{\footnotesize\scshape Bool}]{b \Downarrow b}{}  \qquad
\\[0.3cm]
    \infer[\mbox{\footnotesize\scshape Or}]{(e_1 \lor e_2) \Downarrow b}{e_1 \Downarrow b_1 & e_2 \Downarrow b_2 & b = b_1 \lor b_2} \qquad
    \infer[\mbox{\footnotesize\scshape Neg}]{\neg e \Downarrow b}{e \Downarrow b_1 & b = \neg b_1} \\[0.3cm]
    \infer[\mbox{\footnotesize\scshape Leq}]{(e_1 \leq e_2) \Downarrow b}{e_1 \Downarrow n_1 & e_2 \Downarrow n_2 & b = n_1 \leq n_2}
\end{gather*}


	\begin{gather*}
    \infer[\mbox{\footnotesize\scshape ParNil}]{P \parallel nil \equiv P}{} \qquad
    \infer[\mbox{\footnotesize\scshape ParComm}]{P \parallel Q \equiv Q \parallel P}{} \qquad
    \infer[\mbox{\footnotesize\scshape ParAssoc}]{P \parallel (Q \parallel R) \equiv (P \parallel Q) \parallel R}{} 
    \\[0.3cm]
    \infer[\mbox{\footnotesize\scshape SumNil}]{M + nil \equiv M}{} \qquad
    \infer[\mbox{\footnotesize\scshape SumComm}]{M + N \equiv N + M}{} \qquad
    \infer[\mbox{\footnotesize\scshape SumAssoc}]{M + (N + O) \equiv (M + N) + O}{} 
    \\[0.3cm]
    \infer[\mbox{\footnotesize\scshape RestrOrd}]{P \setminus c \setminus d \equiv P \setminus d \setminus c}{} \qquad 
    \infer[\mbox{\footnotesize\scshape RestrNil}]{nil \setminus c \equiv nil}{} \qquad 
    \infer[\mbox{\footnotesize\scshape RestrPar}]{(P \parallel Q) \setminus c \equiv P \parallel (Q \setminus c)}{c \not\in fc(P)} \\[0.3cm]
    \infer[\mbox{\footnotesize\scshape ValExpr}]{ P \equiv P[\sfrac{e'}{e}] }{e \Downarrow e'} \\[0.3cm]
    \infer[\mbox{\footnotesize\scshape TrueGuard}]{\ITE{tt}{P}{Q} \equiv P}{} \qquad
    \infer[\mbox{\footnotesize\scshape FalseGuard}]{\ITE{ff}{P}{Q} \equiv Q}{} 
	\end{gather*}	

  \begin{gather*}
    \infer[\mbox{\footnotesize\scshape SemTau}]{\langle \rho, \tau . P \rangle \longrightarrow \langle \rho, P \rangle}{} \\[0.3cm]
    \infer[\mbox{\footnotesize\scshape SemRename}]{\langle \rho, P \{f\} \rangle \longrightarrow \langle \rho', P' \{f\}\rangle}{\langle \rho, f(P) \rangle \longrightarrow \langle \rho', f(P') \rangle} \qquad
    \infer[\mbox{\footnotesize\scshape SemRestrict}]{\langle \rho, P \setminus L \rangle \longrightarrow \langle \rho, P' \setminus L \rangle}{\langle \rho, P \rangle \longrightarrow \langle \rho', P' \rangle} \\[0.3cm]
    \infer[\mbox{\footnotesize\scshape SemQOp}]{\langle \rho, \mathcal{E}(\widetilde{x}) . P \rangle \longrightarrow \langle \mathcal{E}_{\widetilde{x}}(\rho), P \rangle}{} \\[0.3cm]
    \infer[\mbox{\footnotesize\scshape SemQMeas}]{\langle \rho, M(\widetilde{x} \rhd y) . P \rangle \longrightarrow \boxplus_{m = 0}^{2^{|\widetilde{x}|}} \left\langle {M_m \rho M_m^\dag}, P[\sfrac{m}{y}] \right\rangle}{} \\[0.3cm]
    \infer[\mbox{\footnotesize\scshape SemPar}]{\langle \rho, P \parallel R \rangle \longrightarrow \langle \rho', P' \parallel R \rangle}{\langle \rho, P \rangle \longrightarrow \langle \rho', P' \rangle} \qquad
    \infer[\mbox{\footnotesize\scshape SemSum}]{\langle \rho, P + R \rangle \longrightarrow \langle \rho', P' \rangle}{\langle \rho, P \rangle \longrightarrow \langle \rho', P' \rangle} \\[0.3cm]
    \infer[\mbox{\footnotesize\scshape SemReduce}]{\langle \rho, c!v \parallel c?x . P \rangle \longrightarrow \langle \rho, P[\sfrac{v}{x}] \rangle}{} \\[0.3cm]
%    \infer[\mbox{\footnotesize\scshape SemBox}]{\boxplus_{i \in I}\langle \rho_i, P_i \rangle \longrightarrow \boxplus_{i \in I, j \in J_i}\langle \rho_{i, j},P_{i, j} \rangle}
%          {\forall i \ldotp \langle \rho_i, P_i \rangle \longrightarrow \boxplus_{j \in J_i} \langle\rho_{i, j},P_{i, j}\rangle}
  \end{gather*}
  
  
\subsection{Examples}
  \paragraph{Quantum Teleportation}

\begin{align*}
  \proc{A} &\Coloneqq \text{in}_a?x.\text{CNOT}(q_0, x).\text{H}(q_0).M(x,q_0 \rhd n).(\text{m}_a!n \parallel discard(q_0) \parallel discard(x) )\\
  % \proc{B} &\Coloneqq \text{in}_b?x.\text{m}_a?n.\left(\sum_{i = 0}^3 [n = i]\sigma_i(x).\text{out}_b!x\right) \\
  \proc{B} &\Coloneqq \text{in}_b?x.\text{m}_a?n.
     \\ & \ite{n = 0}{\sigma_0(x).\text{out}_b!x\\&\quad}
    {\ite{n = 1}{\sigma_1(x).\text{out}_b!x\\&\qquad}
    		{\ite{n = 2}{\sigma_2(x).\text{out}_b!x}{\sigma_3(x).\text{out}_b!x}}
    }
\\
  \proc{S} &\Coloneqq \text{H}(q_1).\text{CNOT}(q_1, q_2).(\text{in}_a!q_1 \parallel \text{in}_b!q_2) \\
  \proc{Tel} &\Coloneqq (A \parallel B \parallel S) \setminus \Set{\text{in}_a, \text{in}_b, \text{m}_a } \\
  \proc{TelSpec} &\Coloneqq \text{SWAP}(q_0,q_2).(\text{out}_b!q_2 \parallel discard(q_0) \parallel discard(q_1))
\end{align*}
$\Gamma = \set{\text{in}_a : \hat{\mathcal{Q}}, \text{in}_b : \hat{\mathcal{Q}}, \text{m}_a : \hat{\mathbb{N}}, \text{out}_b : \hat{\mathcal{Q}}}$, $\Sigma = \set{q_0, q_1, q_2}$.