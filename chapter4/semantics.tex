We present a reduction semantics for lqCCS, consistent with the labelled semantic for qCCS presented in \cite{fengBisimulationQuantumProcesses2012, dengOpenBisimulationQuantum2012}. A reduction semantic does not make any assumption on the observable properties of the system (like the labels of a transition), and so in better suited to explore and compare different notion of behavioural equivalence.

Besides, as explained in chapter \ref{chapter3}, in a labelled transition system made of configurations, a quantum input transition like $\xrightarrow{c?q}$ requires that the value of qubit $q$ is already present inside the transition. This is an atypical assumption, as a labeled transition usually model the communication with an unknown external environment, but in this case requires at least some partial knowledge of the environment. In a reduction system there are no such labelled transitions, so a process communicates only with other process, on which we have total information.

Our semantics defines a probabilistic reduction system $\langle S, \rightarrow \rangle$, where \begin{itemize}
\item $S$ is a set of \textit{configurations}, of the form $\confw{\rho, P}$ (like in qCCS).
\item $\rightarrow \subseteq S \times \distr(S)$ is the probabilistic transition relation, corresponding to the $\xrightarrow{\tau}$ transition in qCCS \cite{fengBisimulationQuantumProcesses2012, dengOpenBisimulationQuantum2012}.
\end{itemize}

We assume a fixed set  $QN = {q_1, q_2, \ldots q_n}$ of quantum names, where each name $q_i$ refers to a unique qubit with state space $\calH_i$. We denote as $\calH_{QN}$ the $2^n$-dimensional Hilbert space $\bigoplus_{i=1}^n \calH_i$.

We assume also a fixed typing contest $\Gamma_c = \set{c_i: \widehat{T_i}}_i$, containing typing assumptions for a finite set of classical and quantum channels.

\begin{definition}
Let $P$ be a process and $\rho \in \mathcal{D}(\calH_{QN})$ an arbitrary density operator. We say that a configuration $\confw{\rho, P}$ is well typed, given a set of quantum names $QN$ and a set of typed channels $\Gamma_c$, if $\Gamma_c; \Sigma \vdash P$ and $\Sigma \subseteq QN$.

%  Let $I$ be an arbitrary index set. $\Gamma; \Sigma \vdash \boxplus_{i \in I} \langle \rho_i, P_i \rangle$ iff for each $i \in I$ such that $\rho_i \neq \mathbf{0}$, then $\Gamma; \Sigma \vdash \langle \rho_i, P_i \rangle$.
\end{definition}

Notice that the context $\Gamma_c$ contains assignments only for channels, not for classical variables. This means that in well typed configuration, since $\Gamma_c; \Sigma \vdash P$, $P$ does not contain any free classical variable, and all the free quantum variables are references to qubits in the configuration.

From now on, we will consider only well typed configurations.

\subsection{Reduction System}
In order to define the reduction transition, we first need to introduce a semantic for expressions and a structural congruence relation on processes, like in \cite{gayCommunicatingQuantumProcesses2005}.

We consider as a \textit{value} any expressions $n \in \mathbb{N},\  b \in \mathbb{B},\  v \in  \text{Var}$, and use $v$ as a metavariable for them.

In figure \ref{big_step_exp}, we define a big step semantic for classical and quantum expression in the usual way. We write $e \Downarrow v$ to indicate that the expression $e$ evaluates to value $v$. Recall that the only quantum expression admitted by the type system are quantum variables.

\begin{figure}[h!]
\caption{Big step semantic for Linear qCCS expressions}
\label{big_step_exp}
\begin{gather*}
\infer[\mbox{\footnotesize\scshape Var}]{x \Downarrow x}{} \qquad
    \infer[\mbox{\footnotesize\scshape Nat}]{n \Downarrow n}{} \qquad
    \infer[\mbox{\footnotesize\scshape Bool}]{b \Downarrow b}{}  \qquad
\\[0.3cm]
    \infer[\mbox{\footnotesize\scshape Or}]{(e_1 \lor e_2) \Downarrow b}{e_1 \Downarrow b_1 & e_2 \Downarrow b_2 & b = b_1 \lor b_2} \qquad
    \infer[\mbox{\footnotesize\scshape Neg}]{\neg e \Downarrow b}{e \Downarrow b_1 & b = \neg b_1} \\[0.3cm]
    \infer[\mbox{\footnotesize\scshape Leq}]{(e_1 \leq e_2) \Downarrow b}{e_1 \Downarrow n_1 & e_2 \Downarrow n_2 & b = n_1 \leq n_2}
\end{gather*}
\end{figure}

We define as the \textit{structural congruence relation} $\equiv$ the smallest equivalence relation that satisfies the axioms in figure \ref{str_cong}. We emplyed the usual axioms of \cite{milnerFunctionsProcesses1990} for parallel composition, summation and reduction. We also add axioms to evaluate classical expressions and resolve \textbf{If Then Else} constructs.

\begin{figure}[h!]
\caption{Structural congruence for Linear qCCS}
\label{str_cong}
\begin{gather*}
    \infer[\mbox{\footnotesize\scshape ParNil}]{P \parallel nil \equiv P}{} \qquad
    \infer[\mbox{\footnotesize\scshape ParComm}]{P \parallel Q \equiv Q \parallel P}{} \qquad
    \infer[\mbox{\footnotesize\scshape ParAssoc}]{P \parallel (Q \parallel R) \equiv (P \parallel Q) \parallel R}{} 
    \\[0.3cm]
    \infer[\mbox{\footnotesize\scshape SumNil}]{M + nil \equiv M}{} \qquad
    \infer[\mbox{\footnotesize\scshape SumComm}]{M + N \equiv N + M}{} \qquad
    \infer[\mbox{\footnotesize\scshape SumAssoc}]{M + (N + O) \equiv (M + N) + O}{} 
    \\[0.3cm]
    \infer[\mbox{\footnotesize\scshape RestrOrd}]{P \setminus c \setminus d \equiv P \setminus d \setminus c}{} \qquad 
    \infer[\mbox{\footnotesize\scshape RestrNil}]{nil \setminus c \equiv nil}{} \qquad 
    \infer[\mbox{\footnotesize\scshape RestrPar}]{(P \parallel Q) \setminus c \equiv P \parallel (Q \setminus c)}{c \not\in fc(P)} \\[0.3cm]
    \infer[\mbox{\footnotesize\scshape TrueGuard}]{\ITE{tt}{P}{Q} \equiv P}{} \qquad
    \infer[\mbox{\footnotesize\scshape FalseGuard}]{\ITE{ff}{P}{Q} \equiv Q}{} \\[0.3cm]
    \infer[\mbox{\footnotesize\scshape ValExpr}]{ P \equiv P[\sfrac{e'}{e}] }{e \Downarrow e'} 
	\end{gather*}	
\end{figure}

We can now define the transition relation $\rightarrow$, presented in figure \ref{reduction}. As usual, we write $\confw{\rho, P} \rightarrow \Delta$ to intend $(\confw{\rho, P}, \Delta) \in \rightarrow$. To lighten the notation, we will also write $\confw{\rho, P} \rightarrow \confw{\rho', P'}$ instead of $\confw{\rho, P} \rightarrow \overline{\confw{\rho', P'}}$.

\note{lifting, serve per la reachability e le logiche temporali}

\begin{figure}[h!]
\caption{Reduction system for Linear qCCS}
\label{reduction}
  \begin{gather*}
    \infer[\mbox{\footnotesize\scshape SemTau}]{\langle \rho, \tau . P \rangle \longrightarrow \langle \rho, P \rangle}{} \\[0.3cm]
    \infer[\mbox{\footnotesize\scshape SemRename}]{\langle \rho, P \{f\} \rangle \longrightarrow \langle \rho', P' \{f\}\rangle}{\langle \rho, f(P) \rangle \longrightarrow \langle \rho', f(P') \rangle} \qquad
    \infer[\mbox{\footnotesize\scshape SemRestrict}]{\langle \rho, P \setminus L \rangle \longrightarrow \langle \rho, P' \setminus L \rangle}{\langle \rho, P \rangle \longrightarrow \langle \rho', P' \rangle} \\[0.3cm]
    \infer[\mbox{\footnotesize\scshape SemQOp}]{\langle \rho, \mathcal{E}(\widetilde{x}) . P \rangle \longrightarrow \langle \mathcal{E}_{\widetilde{x}}(\rho), P \rangle}{} \\[0.3cm]
    \infer[\mbox{\footnotesize\scshape SemQMeas}]{\langle \rho, M(\widetilde{x} \rhd y) . P \rangle \longrightarrow \boxplus_{m = 0}^{2^{|\widetilde{x}|}} \left\langle {M_m \rho M_m^\dag}, P[\sfrac{m}{y}] \right\rangle}{} \\[0.3cm]
    \infer[\mbox{\footnotesize\scshape SemPar}]{\langle \rho, P \parallel R \rangle \longrightarrow \langle \rho', P' \parallel R \rangle}{\langle \rho, P \rangle \longrightarrow \langle \rho', P' \rangle} \qquad
    \infer[\mbox{\footnotesize\scshape SemSum}]{\langle \rho, P + R \rangle \longrightarrow \langle \rho', P' \rangle}{\langle \rho, P \rangle \longrightarrow \langle \rho', P' \rangle} \\[0.3cm]
    \infer[\mbox{\footnotesize\scshape SemReduce}]{\langle \rho, c!v \parallel c?x . P \rangle \longrightarrow \langle \rho, P[\sfrac{v}{x}] \rangle}{} \\[0.3cm]
    \infer[\mbox{\footnotesize\scshape SemCongr}]{\langle \rho, P \rangle \longrightarrow \langle \rho', P' \rangle}
    {P \equiv Q & \confw{\rho, Q} \rightarrow \confw{\rho', Q'} & Q' \equiv P'}
  \end{gather*}
\end{figure}
  
\subsection{Examples}
  \paragraph{Quantum Teleportation}

\begin{align*}
  \proc{A} &\Coloneqq \text{in}_a?x.\text{CNOT}(q_0, x).\text{H}(q_0).M(x,q_0 \rhd n).(\text{m}_a!n \parallel discard(q_0) \parallel discard(x) )\\
  % \proc{B} &\Coloneqq \text{in}_b?x.\text{m}_a?n.\left(\sum_{i = 0}^3 [n = i]\sigma_i(x).\text{out}_b!x\right) \\
  \proc{B} &\Coloneqq \text{in}_b?x.\text{m}_a?n.
     \\ & \ite{n = 0}{\sigma_0(x).\text{out}_b!x\\&\quad}
    {\ite{n = 1}{\sigma_1(x).\text{out}_b!x\\&\qquad}
    		{\ite{n = 2}{\sigma_2(x).\text{out}_b!x}{\sigma_3(x).\text{out}_b!x}}
    }
\\
  \proc{S} &\Coloneqq \text{H}(q_1).\text{CNOT}(q_1, q_2).(\text{in}_a!q_1 \parallel \text{in}_b!q_2) \\
  \proc{Tel} &\Coloneqq (A \parallel B \parallel S) \setminus \Set{\text{in}_a, \text{in}_b, \text{m}_a } \\
  \proc{TelSpec} &\Coloneqq \text{SWAP}(q_0,q_2).(\text{out}_b!q_2 \parallel discard(q_0) \parallel discard(q_1))
\end{align*}
$\Gamma = \set{\text{in}_a : \hat{\mathcal{Q}}, \text{in}_b : \hat{\mathcal{Q}}, \text{m}_a : \hat{\mathbb{N}}, \text{out}_b : \hat{\mathcal{Q}}}$, $\Sigma = \set{q_0, q_1, q_2}$.