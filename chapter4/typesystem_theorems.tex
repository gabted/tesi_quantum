\subsection{Type system properties}

We now prove that reductions preserve typing. This is critical for the further developments of this thesis and is a standard property. Note that the absence of a weakening rule for quantum variables requires a different approach w.r.t. the case of CQP \cite{gayCommunicatingQuantumProcesses2005}.

We need some auxiliary results, starting from typing preservation for expression evaluation and structural congruence.
 
\begin{theorem}[Evaluation Preserves Typing]\label{thm:eval_typing}
  If $\Gamma \vdash e$ and $e \Downarrow v$, then $\Gamma \vdash v$.
\end{theorem}
\begin{proof}
  Follows by induction on the evaluation rules.
\end{proof}

\begin{theorem}[Structural Congruence Preserves Typing]
  If $\Gamma; \Sigma \vdash P$ and $P \equiv Q$, then $\Gamma; \Sigma \vdash Q$.
\end{theorem}
\begin{proof}
  By induction on the derivation of $P \equiv Q$. All rules follow trivially with the possible need of additional \textsc{CWeak} applications in the derivation
  of $\Gamma; \Sigma \vdash Q$ as the subcomponents of the if-then-else, non-deterministic sum and parallel operator may be typed by a context $\Gamma' \subseteq \Gamma$.
\end{proof}

Others technical results we need are substitution lemmas for expressions, classical and quantum substitutions.

\begin{lemma}[Expression Substitution]
  Let $\Gamma, x : T' \vdash e : T$ and let $\Gamma \vdash v : T'$, then $\Gamma \vdash e[\sfrac{v}{x}]$.
\end{lemma}
\begin{proof}
  Trivially by structural induction on the derivation of $\Gamma, x : T' \vdash e$.
\end{proof}

\begin{theorem}[Classical Substitution]
  Let $\Gamma, x : T; \Sigma \vdash P$ and let $\Gamma \vdash v : T$, then $\Gamma; \Sigma \vdash P[\sfrac{v}{x}]$.
\end{theorem}
\begin{proof}
  By structural induction on the derivation of $\Gamma, x : T; \Sigma \vdash P$.
\end{proof}

\begin{theorem}[Quantum Substitution]
  Let $\Gamma; \Sigma, x \vdash P$ and let $v \not\in \Sigma$, then $\Gamma; \Sigma, v \vdash P[\sfrac{v}{x}]$.
\end{theorem}
\begin{proof}
  By structural induction on the derivation of $\Gamma; \Sigma, x \vdash P$.
  Let us analyze the interesting cases: For the \textsc{QOp} rule it must be that $P = \mathcal{E}(\widetilde{x}).Q$ for some process $Q$.
  By induction hypothesis it holds that $\Gamma; \Sigma, v \vdash Q$, but $v \not\in \widetilde{x}$ by the hypothesis $v \not\in \Sigma$, thus we can apply the \textsc{QOp} rule.
  The same line of reasoning is valid for the \textsc{QMeas} rule.
  The \textsc{QSend} rule is also guaranteed by the $v \not\in \Sigma$ requirement.
  Finally, the derivation of \textsc{Par} imposes that only one of the components, w.l.o.g. $P_i$, contains the variable $x$ in its quantum environment $\Sigma_i$.
  Thus by induction we obtain $\Gamma_i; \Sigma_i[\sfrac{v}{x}] \vdash P_i$. While, for the other components there is no change since they do not contain the variable $x$.
  However, since $v \not\in \Sigma$, we can conclude that all smaller environments are still pairwise distinct, thus we can apply the \textsc{Par} rule again.
\end{proof}

We now prove the desired result, namely that reductions preserve typing.

\begin{theorem}[Typing Preservation]
  If $\Gamma; \Sigma \vdash P$ and $\langle \rho, P \rangle \longrightarrow \sum_{i \in I} p_i \langle \rho_i, P_i \rangle$ then
  $\forall i \in I \ldotp \Gamma; \Sigma \vdash P_i$.
\end{theorem}
\begin{proof}
  By structural induction on the transition relation $\longrightarrow$.
  Let us analyze the interesting cases: if the last step in the derivation is a \textsc{SemQMeas} rule, then $P = M(\widetilde{x} \rhd y).Q$ for some process $Q$,
  where $Q$ is typed with $\Gamma, m : \mathbb{N}; \Sigma \vdash Q$. Each component of the box sum is of the form $Q[\sfrac{m}{y}]$ with $m \in \mathbb{N}$, thus
  by the classical substitution theorem it holds that $\Gamma; \Sigma \vdash Q[\sfrac{m}{y}]$.
  If the last step is a \textsc{SemPar} rule, then $P = Q \parallel R$ for some processes $Q$ and $R$, where
  $\Gamma_1; \Sigma_1 \vdash Q$ and $\Gamma_2; \Sigma_2 \vdash R$ with $\Gamma = \Gamma_1 \cup \Gamma_2$, $\Sigma = \Sigma_1 \cup \Sigma_2$ and $\Sigma_1 \cap \Sigma_2 = \emptyset$.
  By induction $\Gamma_1; \Sigma_1 \vdash Q'$, however since the conditions on the $\Sigma$ are still true, it also holds that $\Gamma; \Sigma \vdash Q' \parallel R$, by applying \textsc{CWeak} if $\Gamma_1 \subset \Gamma$.
  The argument is similar for the \textsc{SemSum} rule.
  If the last step is a \textsc{SemReduce} rule, then $P = c!e \parallel c?x.Q$ for some process $Q$. If $c : \hat{T}$ where $T = \mathbb{N}$ or $T = \mathbb{B}$, then
  the theorem holds trivially by the classical substitution theorem.
  If $c : \hat{\mathcal{Q}}$ then it must be that $c : \hat{\mathcal{Q}}; \set{e} \vdash c!e$ and $\Gamma', c : \hat{\mathcal{Q}}; \Sigma' \vdash c?x.Q$ thus $\Gamma'; \Sigma', x \vdash Q$,
  with $e \not\in \Sigma'$ and $x \not\in \Sigma'$ respectively by the \textsc{Par} and \textsc{QRecv} rules. By the quantum substitution theorem, $\Gamma'; \Sigma', e \vdash Q[\sfrac{e}{x}]$,
  but since $\Sigma', e = \Sigma$ and by application of the \textsc{CWeak} it holds that $\Gamma; \Sigma \vdash Q[\sfrac{e}{x}]$.
\end{proof}
