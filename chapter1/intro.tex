\begin{itemize}
\item why quantum computing \\superpostition and entanglement

\item the need for process calculi \\ probabilistic systems, typed languages?
\end{itemize}

\subsection*{A lacking standard notion of behavioural equivalence}

There is a number of proposals of quantum process calculi in the literature, often with different syntax, semantics and behavioural equivalences, even if they all model the same systems and the same protocols \cite{lalireProcessAlgebraicApproach2004, gayCommunicatingQuantumProcesses2005, fengProbabilisticBisimulationsQuantum2007, yingAlgebraQuantumProcesses2010, wangProbabilisticProcessAlgebra2019}. Of all these calculi the most established and developed are \textbf{QPAlg}, \textbf{CQP} and \textbf{qCCS}. They all differ for a number of minor "classical" details, that unfortunately make the calculi difficult to compare: some are inspired by $\pi$-calculus, some by CCS, some employ strong bisimilarity while others employ weak or branching bisimilarity, some apply Larsen-Skou probabilistic bisimilarity \cite{larsenBisimulationProbabilisticTesting1991}, some apply Segala probabilistic bisimilarity \cite{segalaProbabilisticSimulationsProbabilistic1994}.

More importantly, the proposed languages treat quantum information in different ways, leading to different notions of behavioural equivalence. 
A typical example are the two processes 
\[P = c?x.H(x).\nil \qquad Q = c?x.X(x).\nil\]
written in the syntax of qCCS. Both processes receive a qubit, apply a unitary transformation on it, and then terminate. The first process applies an Hadamard transformation, while the second applies a $X$ transformation. The discriminating question is, should $P$ and $Q$ be considered bisimilar? That is, do $P$ and $Q$ express the same observable behaviour, and are therefore indistinguishable? 

The answer depends on the exact notion of \textit{observable property}, which varies between the proposed calculi. In QPAlg and CQP, a qubit is observable only when it is sent, as it is the case for "classical" value passing or name passing calculi. In this setting, $P$ and $Q$ are indeed bisimilar, because the qubit they modified is not visible anymore to an external observer. In qCCS instead, after a process has terminated, the qubits it has operated on are observable, so when $P$ and $Q$ reach the $\nil$ step, they will have left the qubit in two different states, and so are considered not bisimilar. This discrepancy is due to a sort of "ambiguity", present in the syntax of all the calculi proposed up to now, on what happens to the modified qubit after the termination of the process. 
 
 An other crucial notion when defining behavioural equivalence is how to confront qubit values. In a classical process calculus, we can say that $c!0.\nil$ is not bisimilar to $c!1.\nil$ because they send two different \textit{values}. In a quantum setting, in the presence of entanglement, it is difficult to talk about the value of a single qubit.  Take for example the two processes 
 \[R = Entangle_{\Phi}(q_1, q_2).c!q_1.c!q_2 \qquad 
   S = Entangle_{\Psi}(q_1, q_2).c!q_1.c!q_2\]
where $R$ and $S$ create entanglement between the two qubits  $q_1$, $q_2$ and then send them. $R$ sets the two qubits in the state $\Phi = \oost\ket{00} + \oost\ket{11}$, while $S$ sets them in the state $\Psi = \oost\ket{01} + \oost\ket{10}$. These two states can be described (and distinguished) only when taken as a whole, as a compound 2-qubit system, where in $\Phi$ the two qubits always present the same value, and in $\Psi$ the two qubits only present different values. In the presence of entanglement it is impossible to describe the state of just qubit $q_1$ without loosing some information.
 
Different calculi deals with this unique quantum feature in different ways. In QPAlg the value of a qubit is defined to ignore entanglement, and so the two processes $R$ and $S$ are deemed bisimilar. In qCCS, the "environment", i.e. the whole state of all visible qubits, is considered when comparing two processes, and so $R$ and $S$ are not bisimilar. In CQP there is a similar notion of environment, but it takes into account also the limited knowledge of an external observer and the probabilistic behaviour of quantum measurement.
 
\subsection*{Linear qCCS: a unifying approach}
The main objective of this thesis is to introduce Linear qCCS, and use it to investigate the different notions of behavioural equivalence of quantum systems. Linear qCCS (lqCCS) is designed to resolve the ambiguities and discrepancies of the previous proposals, and to identify the most general behavioural equivalence relation, without relying on any preconceived notion of observable property.

lqCCS is a (asynchronous) version on qCCS, equipped with a linear type system to regulate exclusively quantum communication. The type system is inspired by the affine one in CQP, but it enforces a stricter policy on qubit input and outputs. In lqCCS each qubit used by a process must be sent \textit{exactly} once, while in CQP it must be sent at most one. This means that the two processes $P$ and $Q$ seen before are not well typed terms in lqCCS, while $P', Q', P'', Q''$ are.
\begin{align*}
P &= c?x.H(x).\nil &  Q &= c?x.X(x).\nil \\
P' &= c?x.H(x).c!x &  Q' &= c?x.X(x).c!x \\
P'' &= c?x.H(x).discard(x) &  Q'' &= c?x.X(x).discard(x)
\end{align*}
The $discard(q)$ action is typed the same as a send action $c!q$, but when it comes to bisimilarity, $q$ is not considered a visible qubit. For example, in (standard) qCCS $discard(q)$ can be implemented with channel restrictions, like $c!q\setminus c$.

Thanks to its linear type system, lqCCS forces the designer to make an explicit choice on what happens to the qubits a t the end of a computation. Writing processes like $P'$ or $Q'$, the qubit is visible, and the two processes are not bisimilar, for qCCS and QPAlg and CQP as well. Writing processes like $P''$ and $Q''$, the qubit is not visible, the two processes are bisimilar, and one obtains the intended semantics of QPAlg/CQP, reformulated in qCCS-like syntax.

Linear qCCS makes possible to compare directly the different bisimilarity relations presented in literature, and determine which are the crucial  quantum-related details that makes a difference when defining a bisimilarity. Thanks to linearity, there is no ambiguity on when a qubit should be visible or not, and so the only relevant detail is how describe the quantum state. 

Since there  is no unique and accepted notion of "observable property" of quantum processes, each definition of labeled bisimilarity makessome assumption on which are the properties to be checked when comparing two processes. Instead we opted for a different notion of bisimilarity, called (probabilistic) \textit{saturated bisimilarity}, based on an Unlabeled Transition System. Saturated bisimilarity is a synthesis of bisimilarity and contextual equivalence, where two processes $P$ and $Q$ are saturated bisimilar if they express the same atomic observable properties (called \textit{barbs}), and when put inside the same context they evolve to bisimilar processes. As barbs we chosen the simple, classical notion of "being able to send \textit{some} value on a specific channel", without any assumption on how the quantum value should be represented. In saturated bisimilarity, in fact, it is the job of the context to distinguish the two processes $P'$ and $Q'$ seen before, because there exists a context that receives the modified qubit, measures it and is able to tell the difference.

The advantage of such saturated bisimilarity is that, instead of adding additional requirement to make the labelled bisimilarity a congruence, we start from a relation that is a congruence by definition, and explore which are the properties that can be observed by an arbitrary context.


\subsection*{The unexpected inadequacy of probabilistic bisimilarity}

With probabilistic saturated bisimilarity we can recover the same (labelled) bisimilarity notion of qCCS, but not the one presented in the latest works on CQP. The main difference between qCCS and CQP bisimilarity is in the treatment of measurement operators, that exhibit a probabilistic behaviour.

Consider the two processes 
\[P = c?x.M_{01}(x).c!x \qquad Q = c?x.M_\pm(x).c!x\]
Both processes receive a qubit, measure it and then send the measured qubit, that will have changed, or "decayed" due to the measurement. $P$ measures the qubit in the $01$ or computational basis, so will send two possible qubits, $\kz$ or $\ko$, each with a certain probability. $Q$ measures in the $\pm$ or diagonal basis, so will send two possible qubits, $+$ and $-$, each with a certain probability. There is an input for which $P$ will evolve in a symmetric \textit{distribution} of states, i.e. will send a qubit with state $\kz$ with probability $0.5$, or a qubit with state $\ko$ with probability $0.5$, and similarly also $Q$ will evolve in a symmetric distribution of sending $\kpl$ and sending $\km$.

When dealing with distribution, the usual notion of  probabilistic bisimilarity says that two distribution are bisimilar when they assign the same probability to bisimilar processes. So, according to this definition, the processes $P$ and $Q$ are not considered bisimilar in qCCS, because they end up in distributions that are not composed by bisimilar processes. The same holds for probabilistic saturated bisimilarity in lqCCS, indicating that this behaviour does not depend on any particular assumption on the observable properties of quantum states.

In quantum computing, however, it is not possible to distinguish a source $P$ sending half of the time $\kz$ and half of the time $\ko$ from a source $Q$ sending half of the time $\kpl$ and half of the time $\km$. This is because an external observer, trying to distinguish the two sources, can only receive a qubit a measure it in some basis, like the computational basis or the diagonal basis. But when measuring a qubit coming from $P$ in the computational basis, the result will be half of the time $\kz$ and half of the time $\ko$, as the measurement in the computational basis does nothing on a qubit already in the computational basis. When measuring a qubit coming from $Q$ in the computational basis, a $\kpl$ qubit would decay in $\kz$ or $\ko$ with half of the probability each, and the same happens for $\km$. So, when measuring a qubit from either of these sources, the outcome is always half of the time $\kz$ or half of the time $\ko$\footnote{The same happens for a measurement in the diagonal basis, and in any other basis }, and the two sources are indistinguishable. 

Following from this peculiar quantum feature, in CQP the processes $P$ and $Q$ are considered bisimilar, because the bisimilarity of distributions is not (entirely) derived from the bisimilarity of their states, but there are instead some \textit{observable property of the distribution} as a whole, based on the density matrix representation of quantum states.

We identify the well established notion of Larsen-Skou probabilistic bisimulation  as the cause of these undesired behaviour of qCCS, and propose a novel notion of \textit{quantum saturated bisimilarity} that represents more correctly the observable properties of distribution of quantum states. We define a \textit{quantum equivalence} relation between distribution of processes, in the same way as density matrices define an equivalence between distribution of quantum states. Thanks to this equivalence relation, we relax the conditions of saturated bisimilarity, asking that when two distribution $\Delta$ and $\Theta$ are bisimilar, they are not necessarily Larsen-Skou bisimilar, but there are some equivalent distribution $\Delta'$ and $\Theta'$ that are Larsen-Skou bisimilar. Using quantum saturated bisimilarity we can recover many results obtained for CQP bisimilarity, although in a different and arguably more standard transition system.
 
 \subsection*{A minimal quantum process calculus}
 
The usual notion of probabilistic bisimilarity seems not well suited to treat distributions of configurations containing quantum values. On the other end, quantum system have some obvious probabilistic features, and it is correct to talk about their probabilistic classical properties. So we propose a different model, that does not treat distributions of configurations. Instead in our model we abstract from the quantum value in the initial configuration: each state is a \textit{quantum distribution}, i.e. a function from the Hilbert space to a distribution of classical states. In a certain way, the probabilistic behaviour of a quantum system is in fact determined by its initial quantum values, and the observable property of a quantum system are always (probabilistic) classical values, since a qubit can not be directly observed.

Since the behavioural properties of quantum system are still not clear and explored, it is necessary to pursue a foundational approach, defining a Minimal Quantum Process Algebra (MQPA). MQPA is designed  with the smallest possible set of operator, discarding all the irrelevant classical details. It feature only unitaries, communication, non-determistic sum and a novel notion of \textit{quantum sum} \note{projective sum}. If in a probabilistic process algebra there is a probabilistic sum operator, that produces a distribution of states, in a quantum setting there is the measurement operator, that produces a quantum distribution of states, i.e. a distribution that depends on the quantum value to be measured.
 