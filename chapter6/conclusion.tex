\subsection*{Conclusions}
We have explored the main  quantum process calculi proposed in the literature, ignoring their classical differences and underlining the quantum-related design choice, which lead to fairly different notions of behavioural equivalence.

One of the discriminating factor between calculi, namely the visibility of qubits, stems from the intrinsic ambiguity of the proposed syntax, and on how that syntax should model the real system. Thanks to the linear type system of lqCCS, is it possible to eliminate this ambiguity, and to construct processes that are interpreted in the same way by all the proposed calculi. Probabilistic saturated bisimilarity has been introduced, to provide a behavioural equivalence notion that cpatures exactly what can  be or cannot be distinguished by an external observer.

The other discriminating factor, i.e. how to compare quantum values, has instead proven to be a significant detail, and it is in a sense bound to some foundational considerations on observable properties of quantum system. Perhaps surprisingly, the crucial detail that separates qCCS and CQP it is not in how they specify quantum properties of a configuration, but in how these quantum properties are lifted to properties of a distributions of configurations. Since the peculiar characteristics of quantum computing allows a different distributions (i.e. ensembles) to have the same observable properties (i.e. mixed states), it was necessary to relax the conditions of Larsen-Skou bisimilarity, introducing quantum equivalence between distributions. Quantum saturated bisimilarity makes use of these equivalence, and allows to prove some expected result that were absent in qCCS, for example that measurements and superoperator could have different semantics, but yields bisimilar transition systems. 
\subsection*{Future Work}
Linear qCCS and probabilistic/quantum saturated bisimilarity have raised a number of interesting question and challenges, on both practical and foundational aspects of quantum process calculus.

Our saturated bisimilarity are strong relation, while the other calculi in the literature propose also weak and branching bisimilarity. To define a weak bisimilarity, and so the transitive closure of the $\rightarrow$ transition relation, lqCCS could adopt the distribution transformer approach of qCCS, yielding to a Segala-style probabilistic weak saturated bisimilarity. Such a relation could be formulated as a bisimilarity between distribution, like in \cite{hennessyExploringProbabilisticBisimulations2012}, instead of a bisimilarity between configurations. Distribution bisimilarity can be defined without the Larsen-Skou lifting, but the problematic interaction between non-determinism and quantum probabilistic behaviour would be still present in such a transition system, and so we expect that the difference between probabilistic and quantum saturated bisimilarity will still be relevant.

One disadvantage of quantum bisimilarity over probabilistic bisimilarity is that it is not an equivalence relation, as bisimulations are not closed for composition. One possible solution would be to define both the transition system and the bisimulation as relations on \textit{equivalence classes} of distributions, just like \cite{hennessyExploringProbabilisticBisimulations2012} defines transitions and bisimulation between distributions. In such a system, quantum saturated bisimilarity is an equivalence relation, and it should be possible to recover the same results obtained in this work.

Supposing that two configurations are saturated bisimilar is a strong hypothesys, in principle stronger than labeled bisimlarity. This means that from saturated bisimilarity descend a lot of properties, but it is also more cumbersome to prove the bisimilairty between two arbitrary configuration. One of the most successful technique to help in this task revolve around defining a Context-LTS, a labelled transition system with context as labels \cite{bonchiGeneralTheoryBarbs2014}. Due to the inherently stateful nature of quantum computation, there are two different ways in which a process $P$ can communicate with a context $B[\blank]$: when they synchronize on some channel, as in classical process algebra, but also when the context manipulates some qubits that are entangled with the qubits in $P$. It would be interesting to explore which proof technique for saturated bisimilarity would still be sound for a probabilistic quantum process calculus.