We have explored the main  quantum process calculi proposed in the literature, focusing on the quantum-related design choices underlying them, which lead to fairly different notions of behavioural equivalence.

One of the discriminating factor between calculi, namely the visibility of qubits, stems from the intrinsic ambiguity of the proposed syntaxes, and on how it affects the modelling of real systems. We have enriched lqCCS with a linear type system, that eliminates this ambiguity. Thanks to this result, lqCCS processes are interpreted in the same way by all the considered proposed calculi, allowing us to compare the behavioural equivalences they propose. Probabilistic saturated bisimilarity has been introduced, to capture what can or cannot be distinguished by an external observer.

Another discriminating factor, i.e. how to compare quantum values, has instead proven to be a significant quantum related detail that cannot be reduced to a syntactical ambiguity, as it reflects foundational assumptions on observable properties of quantum systems. Perhaps surprisingly, the crucial detail that separates qCCS and CQP is not in how they specify quantum properties of a configuration, but in how these quantum properties are lifted to properties of probabilistic distributions. Indeed the peculiar characteristics of quantum computing allows different distributions (i.e. ensembles) to have the same observable properties (i.e. mixed states). To model these defining properties of quantum theory, we have relaxed the conditions of Larsen-Skou bisimilarity, introducing a quantum equivalence relation between distributions. Thanks to this novelty, Quantum Saturated Bisimilarity satisfies some expected properties that were absent in qCCS. For example, measurements and superoperator have different semantics, but may yield bisimilar transition systems. 

We have also proposed a minimal process algebra that abstracts away from most classical details and only focus on quantum behaviour. In this simplified setting, we give an entirely new, purely quantum-based notion of semantics and bisimilarity, and we prove it behaves well in some previously discussed problematic cases. 

\subsection*{Future Work}
Linear qCCS and probabilistic/quantum saturated bisimilarity have raised a number of interesting questions and challenges, on both practical and foundational aspects of quantum process calculus.

We only focused on strong saturated bisimilarity, while the other calculi in the literature propose also weak and branching bisimilarity. We leave the extension of our work to the non strong case as a future work. To define the needed transitive closure of the $\rightarrow$ transition relation, we could adopt the distribution transformer approach of qCCS, yielding to a Segala-style probabilistic weak saturated bisimilarity. The resulting bisimilarity could be formulated as a bisimilarity between distributions, like in \cite{hennessyExploringProbabilisticBisimulations2012}, instead of a bisimilarity between configurations. Distribution bisimilarity will allow us to avoid basing our future extension on the supposed
problematic Larsen-Skou lifting, but the tricky interaction between non-determinism and quantum probabilistic behaviour would still be relevant, and so we still expect probabilistic and quantum saturated bisimilarity not to coincide.

One disadvantage of quantum bisimilarity over probabilistic bisimilarity is that it is not an equivalence relation, as bisimulations are not closed for composition. One possible solution would be to define both the transition system and the bisimulation as relations on \textit{equivalence classes} of distributions, just like \cite{hennessyExploringProbabilisticBisimulations2012} defines transitions and bisimulation between distributions. In such a system, quantum saturated bisimilarity is an equivalence relation, and it should be possible to recover the same results obtained in this thesis.

Supposing that two configurations are saturated bisimilar is a strong hypothesis, not always easy to compare with labeled bisimlarity. Actually, lots of properties descend from saturated bisimilarity, but as a drawback it is also cumbersome to prove the bisimilairty between two arbitrary configurations. One of the most successful technique to help in this task is based on the definition of an equivalent Context-LTS, a labeled transition system with contexts as labels \cite{bonchiGeneralTheoryBarbs2014}. An intricacy is that, due to the inherently stateful nature of quantum computations, there are two different ways in which a process $P$ can interact with a context $B[\blank]$: they can synchronize on some channel, as in a classical process algebra, or one of the two can manipulates some (possibly entangled) qubits in the shared  underlying quantum state. Hence the results of \cite{bonchiGeneralTheoryBarbs2014} need to be adapted for the quantum case. In general, it would be interesting to explore which proof techniques for saturated bisimilarity would still be sound for a probabilistic quantum process calculus.