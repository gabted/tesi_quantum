\documentclass[10pt,a4paper, titlepage]{report}
\usepackage[utf8]{inputenc}
\usepackage{amsmath}
\usepackage{amssymb}
\usepackage{amsthm}
\usepackage{amsfonts}
\usepackage{amssymb}
\usepackage{multicol}
\usepackage{mathrsfs}
\usepackage{tikz}
\usepackage[bookmarks]{hyperref}
\usepackage[english,italian]{babel}
\usepackage [autostyle]{csquotes}
\MakeInnerQuote{"}
%\usepackage[nowrite, norules, swapnames]{frontespizio}
\usepackage{braket}
\usepackage{stmaryrd}
\usepackage{mathtools}
\usepackage{xfrac}
\usepackage{enumerate}
\usepackage[
backend=bibtex,
style=numeric,
sorting=nyt
]{biblatex}
\addbibresource{bibliography.bib}

\theoremstyle{definition}
\newtheorem{definition}{Definition}[section]


\newcommand{\send}[1]{\overline{#1}}
\newcommand{\kp}{\ket{\psi}}
\newcommand{\kf}{\ket{\phi}}


\title{QCCS reduction semantics }
\author{Gabriele Tedeschi}
\begin{document}

\maketitle

\tableofcontents

\chapter{Introduction}

\chapter{Background}
In this chapter, we review some fundamentals concepts in quantum computing and formal methods.

\section{Quantum Computing}

The laws on Quantum Mechanics, as we understand them, are elegantly formalized in a mathematical framework, built upon simple linear algebra. This framework is based on a few postulates that describe the behaviour of quantum systems. Since quantum computing is just the technique of manipulating quantum systems to perform some computation, it must follow the same postulates. 

First we recall some basic definition from linear algebra, formulated in the Dirac's "bra-ket" notation, and then we present the postulates that apply to quantum computing. For further reading, the standard textbook on the subject is \cite{nielsen_chuang_2010}.


\begin{definition}
A column vector is written $\kp$, and it's called "ket of $\psi$"
\[ \kp = \begin{pmatrix}
		\alpha_1\\
		\ldots\\
		\alpha_n
\end{pmatrix}
\]
while its conjugate transposed is writted $\bra{\psi}$
	\[
		\bra{\psi} = \kp^\dagger = (\alpha_1^*, \ldots, \alpha_n^*)
	\]
\end{definition}

A Hilbert space, often denoted as $\cal{H}$, is a complex inner product space, i.e. etc etc


orthogonal vectors

tensor product on hilbert spaces

\subsection{State vector}
\subsection{Unitary Transformation}
\subsection{Measurement}
\subsection{Density matrix formalism}

\section{Process Calculi}	

\chapter{chapter 3}
\chapter{chapter 4}

\chapter{Conclusions}
\printbibliography[
heading=bibintoc,
title={Bibliography}
]
\end{document}
