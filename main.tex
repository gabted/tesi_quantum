\documentclass[10pt,a4paper, titlepage]{report}
\usepackage[margin=1.5in]{geometry}
\usepackage[utf8]{inputenc}
\usepackage{amsmath}
\usepackage{amssymb}
\usepackage{amsthm}
\usepackage{amsfonts}
\usepackage{amssymb}
\usepackage{multicol}
\usepackage{mathrsfs}
\usepackage{tikz}
\usepackage[bookmarks]{hyperref}
\usepackage[english]{babel}
\usepackage [autostyle]{csquotes}
\MakeInnerQuote{"}
\usepackage{braket}	
\usepackage{stmaryrd}
\usepackage{mathtools}
\usepackage{xfrac}
\usepackage{proof}
\usepackage{xfrac}
\usepackage{tensor}
\usepackage{enumerate}	
\usepackage[nowrite, norules, swapnames]{frontespizio}
\usepackage[
backend=bibtex,
style=numeric,
sorting=nyt
]{biblatex} 
\addbibresource{bibliography.bib}
\addbibresource{subbib.bib}


\newcommand{\note}[1]{{\color{red} #1}}
\newcommand{\mono}[1]{\texttt{#1}}

%braket notation
\newcommand{\ketbra}[2]{\ket{#1}\!\!\bra{#2}}
\newcommand{\proj}[1]{\ketbra{#1}{#1}}
\newcommand{\kp}{\ket{\psi}}
\newcommand{\kf}{\ket{\phi}}
\newcommand{\kz}{\ket{0}}
\newcommand{\ko}{\ket{1}}
\newcommand{\kpl}{\ket{+}}
\newcommand{\km}{\ket{-}}
\newcommand{\oost}{\frac{1}{\sqrt{2}}}

%hilbert spaces, density matrices and sops
\newcommand{\calH}{\mathcal{H}}
\newcommand{\hilbert}{\calH}
\newcommand{\calHt}{\mathcal{H}^2}
\newcommand{\Hto}[1]{\mathcal{H}^{\otimes #1}}
\newcommand{\calDH}{\mathcal{D}(\mathcal{H})}
\newcommand{\calSH}{\mathcal{S}(\mathcal{H})}
\newcommand{\sop}{\mathcal{E}}
\newcommand{\cnot}{\text{CNOT}}

%process algebras
\newcommand{\ITE}[3]{\textbf{If }{#1}\textbf{ Then } {#2} \textbf{ Else } {#3}}
\newcommand{\bigzero}{\mbox{\normalfont\bfseries 0}}
\newcommand{\qsum}[1]{\tensor[_{#1}]{\boxplus}{}}
\newcommand{\psum}[1]{\tensor[_{#1}]{\oplus}{}}
\newcommand{\nil}{\mathbf{0}}
\newcommand{\sem}[1]{\llbracket#1\rrbracket}
\newcommand{\sema}[1]{\llparenthesis\,#1\,\rrparenthesis}
\newcommand{\rel}{\mathcal{R}}
\newcommand{\lrel}{\mathring{\rel}}
\newcommand{\conf}{\mathcal{C}}
\newcommand{\confw}[1]{\langle#1\rangle}
\newcommand{\blank}{{-}}
\newcommand{\qlift}[1]{\stackrel{\scriptscriptstyle \boxplus}{#1}}
\newcommand{\slift}[1]{\stackrel{\scriptscriptstyle \circ}{#1}}
\newcommand{\sqlift}[1]{\hat{#1}}
\newcommand{\barb}[3]{#3\downarrow_{#1, #2}}
\newcommand{\proc}[1]{\text{\textbf{#1}}}
\newcommand{\ite}[3]{\textbf{if } #1 \textbf{ then } #2 \textbf{ else } #3}

%bisimulation symbols
\newcommand{\simps}{\sim_{PS}}
\newcommand{\simqs}{\sim_{QS}}


\newcommand{\distr}{\mathfrak{D}}



\newtheorem{theorem}{Theorem}[section]
\newtheorem{corollary}{Corollary}[section]
\newtheorem{lemma}{Lemma}[section]
\newtheorem{definition}{Definition}[section]
\newtheorem{proposition}{Proposition}[section]
\theoremstyle{remark}
\newtheorem{remark}{Remark}[section]
\newtheorem{example}{Example}[section]
\newtheorem{assumption}{Assumption}[section]

\title{Exploring Quantum Process Calculi via barbs and contexts }
\author{Gabriele Tedeschi}
\begin{document}

\begin{frontespizio}
\Margini{2cm}{2cm}{2cm}{2cm}
\Rientro{1cm}
\Logo[5cm]{cherubino_black}
\Istituzione{University of Pisa}
\Divisione{Department of Computer Science}
\Scuola{Master Degree in Computer Science}
\Titolo{Exploring Quantum Process Calculi\\ via barbs and contexts}
\NCandidato{Candidate}
\Candidato{Gabriele Tedeschi}
\NRelatore{Thesis supervisor}{}
\Relatore{Prof. Fabio Gadducci}
\NCorrelatore{Thesis co-supervisor}{}
\Correlatore{Dr. Lorenzo Ceragioli}
%\Correlatore{B. L. User}
\Piede{Academic Year 2021-2022}

%\Dipartimento{Informatica}
%\Corso[laurea triennale]{Informatica}
%\Annoaccademico{2021-2022}
\end{frontespizio}

\begin{abstract}
With the development of quantum communication protocols, numerous \textit{quantum process calculi} have been proposed, but none as emerged has an accepted standard. Moreover, an established notion of behavioural equivalence is still missing. In this work we present a new asynchronous calculus, \textit{Linear qCCS}, featuring a linear type system and two novel bisimilarity relations.
Our language allows to directly compare the semantics of previous proposals, and investigate which notion of behavioural equivalence is most consistent with the quantum mechanical rules. One of the main result is that the usual and well-accepted definition of probabilistic bisimilarity is not appropriate for the quantum setting. Hence we introduce quantum saturated bisimilarity, which adequately formalize the observable properties of quantum systems. Finally, we also explore a foundational approach, presenting a \textit{Minimal Quantum Process Algebra}, in order to investigate, in a minimal framework, the essential characteristics of communicating quantum systems. For this model we define a symbolic semantics, where the probabilistic behaviour of each process is "parametric" with respect to its input quantum values.

 
\end{abstract}

\tableofcontents

\chapter{Introduction}

Quantum computing exploits non-classical phenomena described by quantum
mechanics, such as entanglement and superposition, to perform computations,
 often with considerable speedup with respect to classical computers.

Both theory and practical implementations have attracted considerable research
efforts in the last fourty years. Different physical systems may be used to represent
quantum information, ranging from the state of
particles or molecules, e.g. trapped ions~\cite{pogorelov_compact_2021}, to
small superconducting circuits~\cite{clarke_superconducting_2008}. Despite the
variety of solutions, realizability of large scale quantum machines is still an
open question, since the impact of negative phenomena, such as decoherence,
increase exponentially with the increases in state size, thus requiring more resources allocated to
perform error-correction. However, private enterprises started
either selling machines or providing cloud access to them, proving it is actually 
an emergent technology, with increasing interest among the scientific community.

The theory of quantum computing started with the development of the quantum
turing machine model~\cite{benioff_quantum_1982}. However the most common model
in use today is the quantum circuit model based on qubits, quantum gates and
measurement operations. A qubit is the basic unit of quantum information, a
two-state quantum mechanical system. The analogy with the bit is obvious, as a qubit can be in states $\kz$ and $k1$. However the state of a qubit can also be a linear combination of $\kz$ and $\ko$, also called a superposition of $\kz$ and $\ko$. For example, a qubit can be in state $\kpl$ or $\km$, that are both combination of $\kz$ and $\ko$. Interestingly, also $\kz$ and $\ko$ can be expressed as a linear combination of $\kpl$ and $\km$, and in fact  $\kz$ and $\ko$ are usually called the computational basis, and $\kpl$ and $\km$ are known as the diagonal basis of the state space. When measured in a basis, a qubit 
in superposition collapses to one of the two basis states,
with probability dependent on the linear combination. A quantum system may also evolve through unitary transformations, that are physically implemented as quantum gates on one or more qubits (potentially controlled by classical bits).

Quantum computing theory is mainly developed along two axis: algorithms and
protocols. Even though quantum computers obey the Church-Turing
thesis~\cite{nielsen_quantum_2010}, algorithms have been found that have more
than polynomial speedup over classical counterparts, e.g. Shor's algorithm for
factorization~\cite{shor_algorithms_1994} and the HHL algorithm for solving
linear systems of equations~\cite{harrow_quantum_2009}.

Quantum protocols, in particular cryptography-related protocols, have been  one of the first application 
of quantum theory to computer science. Due to lower physical
requirements with respect to algorithms, various protocols have been implemented
in real world contexts, also with large-scale commercially available physical implementations.
The prototipical case is the BB84 Quantum Key Distribution
protocol \note{\cite{shor_simple_2000}}, that has been proven unconditionally secure. As for algorithms, most protocols only require
classical control and communication. However some protocols such as Quantum Key
Distribution~\cite{poppe_practical_2004} and Quantum Leader
Election~\cite{tani_exact_2012} also require quantum communication (i.e. sending and receiving qubits).


Correctness proof are available for some algorithm, but there is no systematic approach for verifying protocols ad their application in larger systems.
As it is well known, correctness of classical communication protocols is already hard to prove at design stage, with notorious examples of vulnerabilities discovered for protocols previously considered to be secure.
be incorrect later in time. This is more-so true for quantum protocols, which must resist both quantum and classical attackers.


Process calculi have been successful in modeling classical concurrent systems with non-determinism, probability and communication. Process calculi provide systematic solutions for analyzing and verifying the correctness of new systems and protocols. We expect the same will hold also in the quantum case, which however turns out to be a challenging setting.


To model quantum protocols, a process calculus needs to handle both classical and quantum communication at once, and take into account the peculiar rules of quantum computations. Due to the so called \textit{No-cloning Theorem}, quantum information can not be duplicated, so can not be treated in the usual functional, "pass-by-value" fashion of process calculi. Besides, qubit in superposition can not be directly observed, as they "decay" in a probabilistic way when measured. Finally, a quantum process calculus must also deal with \textit{entangled qubits}, i.e. qubits that can not be described separately, one at a time, but must be considered as a whole compound system, even when they are stored in physically distant locations.

Of all the proposed model, none has emerged as universally accepted standard. Our objective is to address the extension of classical process algebra to the quantum world. For this we will compare the different proposals, with a foundational approach aiming at identifying the best notion of behavioural equivalence. We propose a novel calculus named Linear qCCS, for which we consider different definitions of bisimilarity.
Defining an appropriate bisimilarity is in fact the first step towards temporal logics, model checking and all the other tools available in the classical models.


\subsection*{A lacking standard notion of behavioural equivalence}

There is a number of proposals of quantum process calculi in the literature, often with different syntax, semantics and behavioural equivalences, even if they all model the same systems and the same protocols \cite{lalireProcessAlgebraicApproach2004, gayCommunicatingQuantumProcesses2005, fengProbabilisticBisimulationsQuantum2007, yingAlgebraQuantumProcesses2010, wangProbabilisticProcessAlgebra2019}. Of all these, the most established and developed are \textbf{QPAlg} \cite{lalireProcessAlgebraicApproach2004}, \textbf{CQP} \cite{gayCommunicatingQuantumProcesses2005} and \textbf{qCCS} \cite{fengProbabilisticBisimulationsQuantum2007}. They all differ in  a number of minor "classical" details, that unfortunately make the calculi difficult to compare: some are inspired by $\pi$-calculus, some by CCS, some employ strong bisimilarity while others employ weak or branching bisimilarity, some apply Larsen-Skou probabilistic bisimilarity \cite{larsenBisimulationProbabilisticTesting1991}, some apply Segala probabilistic bisimilarity \cite{segalaProbabilisticSimulationsProbabilistic1994}.

More importantly, the proposed languages treat quantum information in different ways, leading to different notions of behavioural equivalence. 
A typical example are the two processes 
\[P = c?x.H(x).\nil \qquad Q = c?x.X(x).\nil\]
(here written in the syntax of qCCS). Both processes receive a qubit, apply a unitary transformation on it, and then terminate. The first process applies an Hadamard transformation, while the second applies a $X$ transformation, resulting in a different quantum state. The discriminating question is, should $P$ and $Q$ be considered bisimilar? That is, do $P$ and $Q$ express the same observable behaviour, and are therefore indistinguishable? 

The answer depends on the exact notion of \textit{observable property}, which varies between the proposed calculi. In QPAlg and CQP, a qubit is observable only when it is sent, as it is the case for "classical" value passing or name passing calculi. In this setting, $P$ and $Q$ are indeed bisimilar, because the qubit they modified is not visible to an external observer. In qCCS instead, after a process has terminated, the qubits it has operated on become observable, so when $P$ and $Q$ reach $\nil$, they will have left the qubit in two different states, and this sufficies to refutate bisimilarity in qCCS. This discrepancy is due to a sort of "ambiguity", present in the syntax of all the calculi proposed up to now, on what happens to the qubits after the process termination.
 
 Another crucial notion when defining behavioural equivalence is how to confront qubit values. In a classical process calculus, we can say that $c!0.\nil$ is not bisimilar to $c!1.\nil$ because they send two different \textit{values}. In a quantum setting it is difficult to talk about the value of a single qubit, because the qubit may be entangled.  Take for example the two processes 
 \[R = Set_{\Phi}(q_1, q_2).c!q_1.c!q_2 \qquad 
   S = Set_{\Psi}(q_1, q_2).c!q_1.c!q_2\]
where $R$ and $S$ set the value of the two qubits to different entangled states, and send them.
Quantum theory tells us that entangled states are distinguishable only when taken as a whole, as a compound 2-qubit system (for example, we could take $\Phi = \oost\ket{00} + \oost\ket{11}$ and $\Psi = \oost\ket{01} + \oost\ket{10}$, as it will be explained in the next chapter). In the presence of entanglement, one cannot describe the state of just qubit $q_1$ without loosing some information. For this reason, if the values of qbits are considered only one at the time, we lose some discriminating power allowed by quantum theory.

On the opposite side, quantum theory comes also with negative results about distinguishability of probabilistic mixtures (i.e. distributions) of quantum states. Also in this case, the capability of the observer to discriminate between different entities must be considered carefully, and is addressed or ignored in different calculi.
 
Summarizing, different calculi deals with quantum features in different ways. In QPAlg the value of a qubit is defined to ignore entanglement, and so the two processes $R$ and $S$ above are deemed bisimilar. In qCCS, the "environment", i.e. the state of all visible qubits, is considered when comparing two processes, thus $R$ and $S$ are not bisimilar. In CQP there is a similar notion of environment, but takes into account also the limited discriminating capability of an external observer for probabilistic mixtures of quantum states, naturally arising upon measurements.
 
\subsection*{Linear qCCS: a unifying approach}
The main objective of this thesis is to introduce Linear qCCS, and use it to investigate the different notions of behavioural equivalence of quantum systems. Linear qCCS (lqCCS) is designed to resolve the ambiguities and discrepancies of the previous proposals, and to identify the most adequate behavioural equivalence relation.
%without relying on any preconceived notion of observable property.

lqCCS is a (asynchronous) version on qCCS, equipped with a linear type system to regulate exclusively quantum communication. Indeed, due to the No-Cloning Theorem, once a qubit is sent on a channel, the sender process cannot use it anymore. The type system is inspired by the affine one in CQP, but it enforces a stricter policy on qubit input and outputs. In lqCCS each qubit used by a process must be sent \textit{exactly} once, while in CQP it must be sent at most one. This means that the two processes $P$ and $Q$ seen before are not well typed in lqCCS, while $P', Q', P'', Q''$ are.
\begin{align*}
P' &= c?x.H(x).c!x &  Q' &= c?x.X(x).c!x \\
P'' &= c?x.H(x).discard(x) &  Q'' &= c?x.X(x).discard(x)
\end{align*}
The $discard(q)$ action is typed the same as a send action $c!q$, but when it comes to bisimilarity, $q$ is not considered a visible qubit. For example, in (standard) qCCS $discard(q)$ can be implemented with channel restrictions, like $c!q\setminus c$.

Thanks to its linear type system, lqCCS forces the designer to make an explicit choice on what happens to the qubits at the end of a computation. By writing processes like $P'$ or $Q'$, the qubit are made visible, thus the two processes are not bisimilar. When writing processes like $P''$ and $Q''$, the qubit is not visible, and the processes are bisimilar. Note that when deciding to use a discard or a send action, we can switch between a behaviour a-la QPAlg/CQP (where $P$ is bisimilar to $Q$), and a-la qCCS (where they are not).

Having solved the ambiguity on when a qubit should be visible or not, the only relevant detail is how describe the quantum state. This allows us to compare directly the different bisimilarity relations presented in the literature, and to determine which are the crucial  quantum-related details that makes a difference when defining a bisimilarity.  

Since there  is no unique and accepted notion of "observable property" of quantum processes, each definition of labeled bisimilarity makes some assumption on which are the properties to be checked when comparing two processes. Instead we opted for a (probabilistic) \textit{saturated bisimilarity}, which in turn is based on an Unlabeled Transition System, and embodies both bisimilarity and contextual equivalence. Two processes $P$ and $Q$ are saturated bisimilar if they express the same atomic observable properties (called \textit{barbs}), and when put inside the same context they evolve to bisimilar processes. As barbs we take the property of "being able to send on a specific channel", which  is a standard choice for barbs. Notice that in this way we made no a-priori choice on quantum values representation. In saturated bisimilarity, in fact, it is charge of the the context to distinguish. For example, $P'$ and $Q'$ seen before are distinguished by a context that receives the modified qubit, measures it and is able to tell the difference.

A technical advantage of such saturated bisimilarity is that, instead of adding requirements to guarantee that the labelled bisimilarity is a congruence, we start from a relation that is a congruence by definition, and explore which are the properties that can be observed by an arbitrary context.


\subsection*{The unexpected inadequacy of probabilistic bisimilarity}

With probabilistic saturated bisimilarity we can recover the same (labelled) bisimilarity notion of qCCS, but not the one presented in the latest works on CQP. The main difference between qCCS and CQP bisimilarity is in the treatment of measurement operatorsand the resulting probabilistic behaviour.

Consider the two processes 
\[P = c?x.M_{01}(x).c!x \qquad Q = c?x.M_\pm(x).c!x\]
Both processes receive a qubit, measure it and then send the measured qubit. Note that measurement are of two different kind. As it will be clear later, this causes the resulting state of the sent qubits to differ. $P$ measures the qubit in the $01$ or computational basis, so will send two possible qubits, $\kz$ or $\ko$, each with a certain probability. $Q$ measures in the $\pm$ or diagonal basis, so will send two possible qubits, $+$ and $-$, each with a certain probability. There is an input for which $P$ will evolve in a symmetric \textit{distribution} of states, i.e. will send a qubit with state $\kz$ with probability $0.5$, or a qubit with state $\ko$ with probability $0.5$. Similarly, with the same input state, $Q$ will evolve in a symmetric distribution of sending $\kpl$ and $\km$.

When dealing with distributions the usual notion of  probabilistic bisimilarity says that two distribution are bisimilar when they assign the same probability to bisimilar processes. So, according to this definition, the processes $P$ and $Q$ are not bisimilar in qCCS and the same holds for probabilistic saturated bisimilarity in lqCCS.

Quantum theory, however, tells that it is not possible to distinguish a source $S_P$ sending half of the time $\kz$ and half of the time $\ko$ from a source $S_Q$ sending half of the time $\kpl$ and half of the time $\km$. This interesting property is due to the inherent limitations of the observer, that to discriminate the two sources can only try to perform measurements, and the obtained probabilistic distributions coincides for every choice of measurement. An an example, assume we measure the qubits from $S_P$ in the computational basis, the result will be half of the time $\kz$ and half of the time $\ko$, as the measurement in the computational basis does nothing on a qubit already in the computational basis. Similarly, when measuring a qubit coming from $S_Q$, a $\kpl$ qubit would decay in $\kz$ or $\ko$, each with the same probability, and the same happens for $\km$. Hence, the two sources appear the same for an external observer performing the measurement on the computational basis, and the same happens for all the basis. 
In accordance with this peculiar quantum feature, in CQP the processes $P$ and $Q$ are considered bisimilar. This is obtained by
defining observable properties directly on distributions, instead of simply lifting bisimilarty of processes.

We identify the well established notion of Larsen-Skou \cite{larsenBisimulationProbabilisticTesting1991} probabilistic bisimulation  as the cause of this undesired behaviour of qCCS, and propose a novel notion of \textit{quantum saturated bisimilarity} that represents more correctly the observable properties of probabilistic distribution of quantum states. To obtain this result, we model the undishtingishability results of quantum theory as an equivalence relation between distributions of states and processes.  
%Thanks to this equivalence relation, we relax the conditions of saturated bisimilarity, asking that when two distribution $\Delta$ and $\Theta$ are bisimilar, they are not necessarily Larsen-Skou bisimilar, but there are some equivalent distribution $\Delta'$ and $\Theta'$ that are Larsen-Skou bisimilar.
Thanks to this equivalence relation, we relax the conditions of saturated bisimilarity, allowing to change the "actual" distribution with an equivalent one to match the behaviour of a bisimilar distribution.
 Using quantum saturated bisimilarity we can recover many results obtained for CQP bisimilarity, although in a different and arguably more standard transition system.
 
 \subsection*{A minimal quantum process calculus}
 
The usual notion of probabilistic bisimilarity seems not well suited to treat distributions of configurations containing quantum values, even if quantum systems have some obvious probabilistic features. Given that a well established notion of behavioural equivalence for quantum processes has not emerged yet, we have decided to pursue a foundational approach, investigating the topic on a Minimal Quantum Process Algebra (mQPA).
Working with a minimal process algebra allows us to omit all the irrelevant classical details, and to target the quantum aspects in isolation.
In addition to a minimal set of classical features, mQPA has unitary transformations (representing the evolution of isolated quantum states),
and a novel operator for combining processes, called projective sum.
Similarly to what happens in a probabilistic process algebra, where a probabilistic sum operator produces a distribution
of states, the projective sum produces a quantum distribution of states, i.e. a probabilistic distribution that depends on the quantum value
to be measured.
The state of the qubits is not kept along the process, and is not updated in a small-step fashon as the process evolves.
On the countrary, the operational semantics of a mQPA is defined as implicitly parametric on the quantum state.
This approach is peculiar, as it resamble none of the current proposals (to the best of our knowledge).
A bisimilarity then is defined in a fairly standard way, and is shown to work well with the previous problematic examples
 

\chapter{Background}

\section{Quantum Computing}

The laws on Quantum Mechanics, as we understand them, are elegantly formalized in a mathematical framework, built upon simple linear algebra. This framework is based on a few \textit{postulates} that describe the nature and evolution of quantum systems. Since quantum computing is just the technique of manipulating quantum systems to perform some computation, it will necessarily follows the same postulates. Before presenting each postulate, we will recall the necessary basic definition from linear algebra, formulated in the Dirac's "bra-ket" notation. For further reading, the standard textbook on the subject is \cite{nielsen_chuang_2010}.


\subsection{State space}

A \textit{column vector} in a complex vector space is written $\kp$, and it's called a "ket",
\[ \kp = \begin{pmatrix}
		\alpha_1\\
		\vdots\\
		\alpha_n
\end{pmatrix}
\]
where $\alpha_1, \ldots,  \alpha_n \in \mathbb{C}$. Its \textit{conjugate transpose} is written $\bra{\psi}$, and its called a "bra".
	\[
		\bra{\psi} = \kp^\dagger = (\alpha_1^* \ldots \alpha_n^*)
	\]

A (finite-dimensional) \textit{Hilbert space}, often denoted as $\cal{H}$, is a complex inner product space, i.e. a complex vector space equipped with a binary operator $\braket{  \,\_\, |\, \_\, }: \calH \times \calH \rightarrow \mathbb{C}$ called \textit{inner product}, dot product, or simply "braket".
\[
	\braket{\psi | \phi} = 
	\begin{pmatrix}
	\alpha_1^* \ldots \alpha_n^*
	\end{pmatrix}\begin{pmatrix}
	\beta_1 \\
	\vdots \\
	\beta_n
	\end{pmatrix} = 
	\sum_i \alpha_i^*\beta_i
\]

The inner product satisfies the following properties:
\begin{align*}
\text{Conjugate symmetry} & &  \braket{\psi | \phi} &= \braket{\phi | \psi}^* \\
\text{Linearity \quad\quad} & & \bra{\psi} (\alpha\ket{\phi} + \beta\ket{\varphi}) &= \alpha\braket{\psi | \phi} + \beta\braket{\psi | \varphi}\\
\text{Positive definiteness} & & \braket{\psi | \psi} &\geq 0
\end{align*}

Notice that $\braket{\psi|\psi} = 0$ if and only if $\kp$ is the $\mathbf{0}$ vector. Besides, thanks to conjugate symmetry, we have $\braket{\psi | \psi} = \braket{\psi | \psi} ^ *$, so $\braket{\psi | \psi}$ it's always a real, non-negative number, when $\kp \neq \mathbf{0}$.

Two vectors $\kp$ and $\kf$ are \textit{orthogonal} if
 \[\braket{\psi | \phi} = 0\]


The \textit{norm} of $\kp$ is defined as: 
\[ \|\kp\| = \sqrt{\braket{\psi|\psi}}\]


A \textit{unit vector} is a vector $\kp$ such that \[\|\kp\| = 1\].

A set of vectors $\{\kp_i\}_i$ is an \textit{orthonormal basis} of $\calH$ if \begin{itemize}
	\item each vector $\kf \in \calH$ can be expressed as a \textit{linear combination} of the vector in the basis, $\kf = \sum_i \alpha_i\kp_i$.
	\item All the vector in the basis are orthogonal
	\item All the vector in the basis are unit vector 
\end{itemize}

We're now ready to present the postulates of Quantum Mechanics, in the form more convenient for quantum computing.

\begin{quote} 
\textbf{Postulate I}: The state of an isolated physical system is represented, at a fixed time $t$, by a unit vector $\kp$, called the \textit{state vector}, belonging to a Hilbert space $\calH$, called the \textit{state space}. 
\end{quote}

 When describing the state oh a quantum system, we ignore the \textit{global phase factor}\footnote{An equivalent formulation, in fact, says that a quantum system is described not by a vector but by a \textit{ray}, a one-dimensional subspace of $\calH$}, i.e. 
 \[
 	\kp = \begin{pmatrix}
	\alpha \\
	\beta
	\end{pmatrix} = - \begin{pmatrix}
	\alpha \\
	\beta
	\end{pmatrix} = \lambda \begin{pmatrix}
	\alpha \\
	\beta
	\end{pmatrix} 
	\text{ for each $\lambda \in \mathbb{C}$  such that $|\lambda| = 1$}
 \]

The simplest, prototypical example of a quantum physical system is a \textit{qubit}: a qubit is a physical system with associated a two-dimensional Hilbert Space $\calH^2$. Such systems comprehend an electron in the ground or excited state, a vertically or horizontally polarized photon, or a spin up or spin down particle.

Taken for example a photon, we could say that the photon is in state $\kz$ when vertically polarized, and in state $\ko$ when is horizontally polarized, where $\kz$ and $\ko$ are the two unit vector of the Hilbert space defined as
\[
	\kz = \begin{pmatrix}
	1 \\
	0
	\end{pmatrix} \qquad
	\ko = \begin{pmatrix}
	0 \\
	1
	\end{pmatrix}
\]

The vectors $\{\kz, \ko\}$ form an orthonormal basis of $\ $, called the \textit{computational basis}. Since they form a basis, each vector $\kp \in \calH^2$ can be expressed as 
	\[\kp = \begin{pmatrix}
	\alpha \\
	\beta
	\end{pmatrix} = 
	\alpha\kz + \beta\ko
	\]
where $\alpha, \beta \in \mathbb{C}$ and $|\alpha|^2 + |\beta|^2 = 1$.

So, the state of any qubit can mathematically be described as $\kp = \alpha\kz +\beta\ko$, a linear combination of $\kz$ and $\ko$. From the physical point of view, this means that the qubit is in a \textit{quantum superposition} of state $\kz$ and $\ko$, like a photon being diagonally-polarized, or an electron being at the same time in the excited and in the ground state.

Other important vectors in the $\calHt$ state space are $\kpl$ and $\km$,
\begin{align*}
	\kpl = \oost &\begin{pmatrix}
	1 \\
	1
	\end{pmatrix}  = \oost\kz + \oost\ko   \\
	\km  = \oost &\begin{pmatrix}
	1 \\
	-1
	\end{pmatrix}  = \oost\kz - \oost\ko 
\end{align*}
that form the so called \textit{hadamard basis} of $\calHt$. As we will see, $\kpl$ and $\km$ are both an equal superposition of $\kz$ and $\ko$. What differs in the two is the \textit{relative phase}, i.e. the phase between the $\kz$ and $\ko$ component. The states $\kz + \ko$, $-\kz -\ko$, $i\kz + i\ko$ are all equal to $\kpl$ times a certain global phase, and are considered the same state. The states $\kz + \ko$, $\kz - \ko$, $\kz +i\ko$, instead, all differs for a relative phase factor, and have a different behaviors when applied to the same computation. 
\subsection{Unitary Transformations}

For each linear operator $A$ acting on a Hilbert space $\calH$, we denote as $A^\dagger$ the \textit{adjoint} of $A$, i.e. the unique linear operator such that
\[
	\braket{\psi|A\phi} = \braket{A^\dagger\psi|\phi}
\]

A linear operator $A$ acting on a $n$-dimensional Hilbert space $\calH^n$ can be represented as a $n\times n$ matrix, and its adjoint is calculated as 
\[
	A^\dagger = (A^*)^T
\] the conjugate transpose of the matrix $A$.


If it holds that $A = A^\dagger$, we say that $A$ is self-adjoint, or \textit{Hermitian}.

A linear operator $U$ is said to be \textit{unitary} when $U^\dagger = U^{-1}$, which implies 
\[
	UU^\dagger = U^\dagger U = I
\]

Unitary matrices enjoy many useful properties, first of all that they have a spectral decomposition. An other defining characteristic is that they preserve the inner product, $\braket{\psi | \phi} = \braket{U\psi | U\phi}$
\[
	\braket{U\psi | U\phi} = \bra{\psi}U^\dagger U\ket{\psi} = \bra{\psi}I\ket{\psi} = \braket{\psi|\phi}
\]

A corollary of this property is that applying a unitary operator to a unit vector gives a unit vector
\[
	\braket{U\psi | U\psi} = \braket{\psi|\psi} = 1
\]

The following postulate makes obvious why we are interested in unitary transformation.
\begin{quote}
\textbf{Postulate II}: The evolution of a closed quantum system is described by a unitary transformation. That is, the state $\kp$ of the system at time $t_0$ is related to the state $\ket{\psi^\prime}$ of the system at time $t_2$ by a unitary operator $U$ which depends only on the times $t_0$ and $t_1$.
\[
	\ket{\psi^\prime} = U\kp
\]
\end{quote}

According to what we said before, if a physical system starts in a unit state, it will always remain in a unit state.

In quantum computing, the the programmer can manipulate the state of a qubit, applying unitary transformations to it. Some of the most frequent transformation, implemented in every quantum computer, are:
\begin{gather*}
	X = \sigma_X = \begin{bmatrix}
	0 & 1 \\
	1 & 0
	\end{bmatrix} \qquad
	Y = \sigma_Y = \begin{bmatrix}
	0 & -i \\
	i & 0
	\end{bmatrix} \qquad
	Z = \sigma_Z = \begin{bmatrix}
	1 & 0 \\
	0 & -1
	\end{bmatrix} \\ 
	H = \oost\begin{bmatrix}
	1 & 1 \\
	1 & -1
	\end{bmatrix} \qquad
 	S = \begin{bmatrix}
	1 & 0 \\
	0 & i
	\end{bmatrix}\qquad
	T = \begin{bmatrix}
	1 & 0 \\
	0 & e^{i\frac{\pi}{4}}
	\end{bmatrix} 
\end{gather*}

For example, the $H$ operator, called the Hadamard operator, or Hadamard gate, is used to \textit{create superposition}, as it transforms a vector from the computational basis to the Hadamard basis:
\begin{gather*}
	H\kz = \kpl \qquad H\kpl = \kz  \\
	H\ko = \km \qquad H\km = \ko \\	 
\end{gather*}
\subsection{Measurement}

The second postulate describes only the evolution of isolated systems. Such system do not exchange energy nor information with the environment, and all their computations are always reversible (and are in fact formalized with invertible, unitary matrices). To extract classical information from the system, to \textit{measure} the output of a quantum computation, it is needed an interaction between the quantum system and the environment. As we will see, this measurement operation is first of all non-invertible, as different state could produce the same outcome when measured, but is also fundamentally probabilistic: a generic state $\kp$ could produce different measurement outcomes $m_1, m_2,\ldots$, each with a certain probability that depends on $\kp$.

From a physical point of view, if a system is in a superposition of states, measuring it can  cause the wavefunction to collapse to a single state, in a purely probabilistic way. This means that, even if we compute a state that contains the desired information, this information is often difficult to recover, because directly measuring  it can destroy the information and produce a trivial outcome.

\begin{quote} \textbf{Postulate III}: Quantum measurements are described by a set $\{M_m\}_m$ of measurement operators, where the index $m$ refers to the measurement outcomes that may occur in the experiment. The set of measurement operators must be \textit{complete}, i.e.:
\[\sum_m M_m^\dagger M_m = I\]
If the state of the quantum system is $\kp$ before the measurement, then the probability that result $m$ occurs is 
\[p_m = \bra{\psi}M_m^\dagger M_m\kp\]
and the state after the measurement will be \[\frac{1}{\sqrt{p_m}}M_m\kp\]
\end{quote}

The most common class of quantum measurements is composed of \textit{projective measurements}. Such measurements are described by a set of \textit{orthogonal projectors}, i.e. Hermitian operators such that \[ M_mM_{m'} = \begin{cases}\mathbf{0}\quad\text{if } m \neq m' \\ M_m\quad\text{if }m = m'\end{cases}\]


The simplest example of (projective) measurement is simply measuring a state in the computational basis, i.e. projecting it in its $0$-$1$ component. The measurement in the computational basis is defined as \[ M_0 = \begin{pmatrix} 1 & 0 \\ 0 & 0\end{pmatrix} \qquad
M_1 = \begin{pmatrix}0 & 0 \\ 0 & 1\end{pmatrix} \]
And its effect of the state $\kp = \begin{pmatrix}\alpha \\ \beta \end{pmatrix}$ is:
\begin{align*}
\frac{1}{\sqrt{p_0}}M_0\kp = \frac{1}{|\alpha|}\begin{pmatrix} 1 & 0 \\ 0 & 0\end{pmatrix}
\begin{pmatrix}\alpha \\ \beta \end{pmatrix} = \begin{pmatrix} 1 \\ 0 \end{pmatrix} = \kz 
\qquad \text{with probability } \bra{\psi}M_0^\dagger M_0\kp = |\alpha^2| \\
\\
\frac{1}{\sqrt{p_1}}M_1\kp = \frac{1}{|\beta|}\begin{pmatrix} 0 & 0 \\ 0 & 1\end{pmatrix}
\begin{pmatrix}\alpha \\ \beta \end{pmatrix} = \begin{pmatrix} 0 \\ 1 \end{pmatrix} = \ko 
\qquad \text{with probability } \bra{\psi}M_0^\dagger M_1\kp = |\beta^2|
\end{align*}

Notice that, when measuring the $\kz$ state in the computational basis, the outcome will always be $\kz$ with probability $|\alpha|^2 = 1$, in a completely deterministic behavior. 
When instead measuring the $\kpl$ or the $\km$ state, we get either $\kz$ or $\ko$, with equal probability $|\alpha|^2 = |\beta|^2 = \left(\oost\right)^2 = \frac{1}{2}$.
\subsection{Composite quantum systems}
In the previous sections we characterized 1-qubit systems and how they evolve, describing a computation as a series of unitaries and measurements acting on a 2-dimensional Hilbert space. What said before applies easily to larger quantum systems and higher dimensional Hilbert spaces, once again thanks to an elegant mathematical formulation.

If we have two photons, each described by a (2-dimensional) Hilbert space $\calH$, the most natural way to describe the system composed of both photons is as the \textit{tensor product} of Hilbert spaces.	

If $\calH_n$ is a $n$-dimensional Hilbert space, and $\calH_m$ is a $m$ dimensional Hilbert space, their tensor product $\calH_n \otimes \calH_m$ is a $nm$ Hilbert space. If $\{\ket{\psi_1}, \ldots \ket{\psi_n}\}$ is a basis of $\calH_n$, and $\{\ket{\phi_1}, \ldots\ket{\phi_m}\}$ is a base of $\calH_m$, then a basis of $\calH_n \otimes \calH_m$ is\[\big\{\ket{\psi_i}\otimes\ket{\phi_j} \mid i \in [1, \ldots n], j \in [1, \ldots m] \big\}\]
 where $\kp \otimes \kf$ denotes the Kronecker product. We will often omit the tensor symbol, writing $\ket{\psi}\ket{\phi}$ or also $\ket{\psi\phi}$ instead of $\kp\otimes\kf$. We can now state the last postulate we need:

\begin{quote}
\textbf{Postulate IV}: The state space of a composite physical system is the tensor product of the state spaces of the component physical systems. 
\end{quote}

If a single qubit is described by a $2$-dimensional space $\calH$, we will write $\calH^{\otimes n}$ to intend the otimes product $n$ copies of $\calH$, of dimension $2^n$. So, a compound system composed of two qubits has a state space $\calH^{\otimes 2}$, its canonical basis is 
\[\{\ket{00}, \ket{01}, \ket{10}, \ket{11}\}\]
and all its vector can be expressed as a linear combination:
\[\kp \in \Hto{2} = \begin{pmatrix}
\alpha \\ \beta \\ \gamma \\ \delta
\end{pmatrix} = \alpha\ket{00} + \beta\ket{01} + \gamma\ket{10} + \delta\ket{11}\]

A quantum state in $\calH_1 \otimes \calH_2$ is said \textit{separable} when can be expressed as the product of two vectors, one in $\calH_1$ and the other in $\calH_2$. From the definition of the Kronecker product, all separable states of $\Hto{2}$ are of the form:
\[
 \kp = \begin{pmatrix}
 \alpha\gamma \\ \alpha\delta \\ \beta\gamma \\ \beta\delta
 \end{pmatrix} = 
 \begin{pmatrix}
 \alpha \\ \beta
 \end{pmatrix} \otimes 
 \begin{pmatrix}
 \gamma \\ \delta
 \end{pmatrix}
\]

One of the defining characteristics of quantum systems is that not all states in $\Hto{2}$ are separable. The existence of such states, called \textit{entangled} states, implies that a composite system can not always be described as simply the juxtaposition of two smaller states. When a qubit $q_1$ is entangled with an other qubit $q_2$, its evolution depends not only on the transformations applied to $q_1$, but also on the transformations applied on $q_2$, that could be even light-years away. This surprising result does not allow faster then light communication, as we will see in the next section.

The classical example of an entangled state is the so called $\ket{\Phi^+}$ \textit{Bell state}:
\[\ket{\Phi^+} = \oost\ket{00} + \oost\ket{11} = \oost\begin{pmatrix}
1 \\ 0 \\ 0\\ 1
\end{pmatrix}
\]

The fourth postulate tells us that the state space of a composite system is simply the tensor product of the state spaces of the smaller systems. The tensor product of Hilbert spaces is still an Hilbert space, the composition of unit vector is still a unit vector, and the composition of unitary transformation is still unitary. For example, the (Kronecker) composition of $H$ and $I$ matrices is defined as a block matrix:
\[	H\otimes I = \oost\begin{pmatrix}
	I & I \\ I & -I
	\end{pmatrix}\]
And when applied to a two-qubit system, it applies $H$ on the first qubit, and leaves the second one unaltered:
\[ (H\otimes I) \ket{00} = H\kz \otimes I\kz = \oost \begin{pmatrix}
	I & I \\ I & -I
	\end{pmatrix} \begin{pmatrix}
	1 \\ 0 \\0 \\0
	\end{pmatrix} = \oost \begin{pmatrix}
	1 \\ 0 \\ 1 \\ 0
	\end{pmatrix} = \kpl\otimes\kz
\]

One of the most common two-qubit unitary that is not just the composition of one-qubit transformations is the CNOT matrix:
\[\text{CNOT} = 
\begin{pmatrix}
1 & 0 & 0 & 0 \\
0 & 1 & 0 & 0 \\
0 & 0 & 0 & 1 \\
0 & 0 & 1 & 0 \\
\end{pmatrix}
\]

This matrix applies a $X$ transformation on the second qubit only if the first one is $\ko$. This means that if the first bit is in a superposition of $\kz$ and $\ko$, after applying this transformation the whole system will be in a superposition: or the two bits are left equal, or the second one has been flipped. As an example, we can show how the CNOT transformation is used to create entanglement.
\begin{gather*}
\cnot \ket{00} = \ket{00} \qquad \cnot \ket{10} = \ket{11} \\
\cnot \ket{+0} = \oost\big(\cnot\ket{00} + \cnot\ket{10}\big) = \oost
\big(\ket{00} + \ket{11}\big) = \ket{\Phi^+}\\
\cnot \ket{-0} = \oost\big(\cnot\ket{00} - \cnot\ket{10}\big) = \oost
\big(\ket{00} - \ket{11}\big) = \ket{\Phi^-}
\end{gather*}

.\subsection{Density operator formalism}

The formalism presented so far describes quantum system in terms of unit vectors and unitary transformations. There is an alternative, more general formulation, the \textit{density operator formalism}, in which states are represented as positive operators, and transformations as linear maps from operators to operators, i.e. superoperators.

The main advantage of this formulation is that represents also \textit{partial information} about a quantum system. When describing open systems, that are systems which interact with the external environment, it is often impossible to have complete knowledge on the state of our systems. Instead, one could know that the open system is either in state $\kp$, with a certain probability $p$, or in state $\kf$, with probability $1-p$. In other words, we know that the system is in a \textit{probabilistic mixture of states}, called an \textit{ensemble} of states, or also a \textit{mixed state}.

In general, given an $n$-dimensional Hilbert space $\calH$, an \textit{ensemble} of quantum states is a set:
\[\{(\ket{\psi_i}, p_i)\} \]
of quantum states in $\calH$, each with a different probability, such that $\forall i \, p_i > 0$ and $\sum_i p_i \leq 1$. Notice that when $\sum_i p_i = 1$, we have a probability distribution of states, when $\sum_i p_i < 1$, we have a so called subprobability distribution.

Each ensemble defines a density operator, that is a matrix in $\mathbb{C}^{n \times n}$, i.e. an operator $\calH \rightarrow \calH$. The ensemble $\{(\ket{\psi_i}, p_i)\}$, with $\ket{\psi_i} \in \calH$ defines the density operator: 
\[
	\rho = \sum_i p_i \proj{\psi_i}
\]
where $\ketbra{\psi}{\phi}$ denotes the matrix multiplication between the column vector $\kp$ and the row vector $\bra{\phi}$, known as the \textit{outer product}. Notice that this construction is not injective, as there are different ensembles that correspond to the same density operator. We indicate with $\calDH$ the set of density operators of $\calH$.

Density operators enjoy two useful properties (see \cite{nielsen_chuang_2010}): \begin{enumerate}
\item The trace of $\rho$ is the sum of the probabilities of the ensemble $tr(\rho) = \sum_i p_i \leq 1$.
\item $\rho$ is \textit{positive semidefinite}, i.e. 
\[\forall \kp \in \calH \ \bra{\psi}\rho\kp \geq 0\]
\end{enumerate}

Positive semidefinite operators are always diagonalizable with eigenvalues real and positives. So, each positive semidefinite operator with trace $\leq 1$ represents at least one ensemble, with the eigenvectors as states and the corresponding eigenvalues as probabilities.


One of the main application of density operators is to describe the state of a subsystem of a composite quantum system.

Suppose a composite system, made of two subsystem $A$ and $B$, with state space  $\calH = \calH_A \otimes \calH_B$.  Given a generic $\rho^{AB} \in \calH$, that describes the state of the whole system, the operator $\rho^A$ that describes the subsystem $A$ is obtained as 
\[
	\rho^A = tr_B(\rho^{AB})
\]
where the $tr_B$ is called the \textit{partial trace over $B$}, and is defined by 
\[
 tr_B\big(\proj{\psi} \otimes \proj{\phi}\big) = \proj{\psi}tr\big(\proj{\phi}\big)
\]
together with linearity.

When applied to a separable state $\rho^A \otimes \rho^B$, the partial trace operator $tr_B$ leaves $\rho^A$ unaltered. When applied to an entangled state, instead, it produces a probabilistic mixture of states, because "forgetting" the information on the state of subsystem $B$ leaves us with only partial information on subsystem $A$. The canonical example is the bell state $\rho = \frac{1}{2} \proj{00} + \frac{1}{2}\proj{11} \in \calH_A \otimes \calH_B$:

\begin{align*}
tr_B(\rho) &= \frac{1}{2} tr_B\big(\proj{00}\big) + \frac{1}{2}tr_B\big(\proj{11}\big)  \\
	&= \frac{1}{2} tr_B\big(\proj{0}\otimes\proj{0}\big) + \frac{1}{2}tr_B\big(\proj{1}\otimes\proj{1}\big) \\
	&= \frac{1}{2} \proj{0}tr\big(\proj{0}\big) + \frac{1}{2}\proj{1}tr\big(\proj{1}\big) \\
	&= \frac{1}{2} \big(\proj{0} + \proj{1}\big) = \frac{1}{2}I
\end{align*}

\note{superoperators definitions}

Let $\calH_A$ and $\calH_B$ be Hilbert spaces. Given $\mathcal{E}: \mathcal{D}(\calH_A) \rightarrow \mathcal{D}(\calH_A)$ and $\mathcal{F}: \mathcal{D}(\calH_B) \rightarrow \mathcal{D}(\calH_B)$,  their tensor product $\sop\otimes \mathcal{F}: \mathcal{D}(\calH_A \otimes \calH_B)\rightarrow \mathcal{D}(\calH_A \otimes \calH_B)$ is defined as:

\[
	(\sop \otimes \mathcal{F})(\rho_A \otimes \rho_B) = \sop(\rho_A)\otimes \mathcal{F}(\rho_B)
\]

The superoperator $\mathcal{E} \otimes \mathcal{F}$ can be applied to arbitrary density operators $\rho$, expressed as sum of separable states, by linearity.


Superoperators on $\calH$: all the maps $\sop: \calDH \rightarrow \calDH$ satifying:
\begin{itemize}
\item $\sop$ is \textit{convex linear}: for any set of probabilities	$\{p_i\}_i$,
\[\sop\left(\sum_i p_i \rho_i \right) = \sum_i p_i \sop(\rho_i)\]
\item $\sop$ is \textit{completely positive}: for any extra Hilbert space $\calH_R$, for any positive $\rho \in \calH_R \otimes \calH$, $(\mathcal{I}_R\otimes \sop)(\rho)$ is also positive, where $\mathcal{I}_R$ is the identity operator on $\mathcal{D}(\calH_R)$.
\item $\sop$ is \textit{trace non-increasing}: for any density operator $\rho \in \calDH$, 
\[tr\big(\sop(\rho)\big) \leq tr(\rho) \leq 1\].
\end{itemize}

We call $\calSH$ the set of all superoperators on $\calH$.

 	
\textit{Löwner order}: we define the partial order $\sqsubseteq$ on $\calDH$ as :
\[ \rho \sqsubseteq \sigma \text{ if and only if } \sigma-\rho \text{ is positive}\]

\textit{Kraus operator sum}: For any superoperator $\sop \in \calSH$, there exists a (non-unique) finite set of operators $\{E_i\}$, with $1 \leq i \leq dim(\calH)$ such that
\begin{gather*}
\sop(\rho) = \sum_i E_i\rho E_i^\dagger \\
\sum_i E_i^\dagger E_i \sqsubseteq I_\calH
\end{gather*}

Although there are possibly multiple Kraus form for any superoperator $\sop \in \calSH$, they are related by a unitary transformation.
Formally, for any two set of decompositions $\{C_i\}_i$ and $\{B_j\}_j$, there is a unitary matrix $U = \{u_{ij}\}_{ij}$ such that $C_i = \sum_j u_{ij} B_j$.

Given $\mathcal{E}: \mathcal{D}(\calH_A) \rightarrow \mathcal{D}(\calH_A)$ and $\mathcal{F}: \mathcal{D}(\calH_B) \rightarrow \mathcal{D}(\calH_B)$,
respectively expressed in Kraus form as $\mathcal{E}(\rho) = \sum_i E_i \rho E_i^\dag$ and $\mathcal{F}(\rho) = \sum_j F_j \rho E_j^\dag$, the thensor product $\mathcal{E} \otimes \mathcal{F}$
can be expressed in Kraus form as 

\[
  (\mathcal{E} \otimes \mathcal{F})(\rho) = \sum_i \sum_j (E_i \otimes F_j)\,\rho\:(E_i \otimes F_j)^\dag
\]

Superoperators can describe both unitary transformations and measurements. Given unitary operator $U$ on $\calH$, we can define  the (trace preserving) superoperator $\sop_U \in \calSH$ as:
\[ \sop_U(\rho) = U \rho U^\dagger\]
Given a measurement $\{M_m\}_m$, we can define the (trace non-increasing) superoperator $\sop_m \in \calSH$ as 
\[\sop_m(\rho) = M_m\rho M_m^\dagger\]
Notice that $\sop_m(\rho)$ is equal to $p_m\rho_m$, where $p_m$ is the probability of the outcome $m$ when measuring the state $\rho$, and $\rho_m$ is the state after outcome $m$ has occurred. 


\section{Process Calculus}
We review here some results on
process calculi and bisimilarity, that will
be used in the following.
Given the probabilistic nature of quantum
mechanics, we include probabilistic systems
in our discussion.
	
\subsection{Process Calculi and LTS}


Process Calculus, also known as Process Algebra, is an algebraic approach to model concurrent computation, often focus on the communication between different agents. Each agent is formalized as a \textit{process}, a syntactic element that describes its capabilities. Processes can be composed in different ways, and so form an \textit{algebra of processes}, with various operators for parallel composition, sequential compositions, probabilistic and non-deterministic sum, and so on.

All the foundational and most successful process calculi \cite{milnerCalculusCommunicatingSystems1980, bergstraAlgebraCommunicatingProcesses1985, 
hoareCommunicatingSequentialProcesses1978, milnerCommunicatingMobileSystems1999},  have a number of key features in common \begin{itemize}
\item \textbf{Nil Process}: Each calculus has a \textit{constant process}, usually denoted as $nil$ or $\nil$, that is in a terminal state, can not perform any action.
\item \textbf{Visible Actions}: Each calculus has \textit{action prefixes}, usually denotes as $\alpha, \beta, \ldots, \overline{\alpha}, \overline{\beta}, \ldots$. If $P$ is a process, $\alpha.P$ is a process that can perform an action $\alpha$, and then behaves as $P$. The two actions $\alpha$ and $\overline{\alpha}$ are called \textit{coactions}, and denotes two dual vies of the same comunication event. If $\alpha$ is the action of sending a qubit through a channel, $\overline{\alpha}$ is the dual action of receiving that qubit. 
\item \textbf{Internal Actions} Each calculus defines also a $\tau$ action, called \textit{internal action} or silent action. Process calculi are designed to model the interaction between system, and abstract away from the low level details. A process $\tau.P$ can then performs a silent action, indicating internal operations, not known to an external observer, outside the scope of the system being modelled. 
%For example, when describing a TCP handshake protocol, only the IP packets sent on the net are represented as visible actions, while while writing data on the buffer of the socket is an internal action.
\item \textbf{Non-determinism}: Each calculus has a \textit{non-deterministic sum} of processes, denoted as $P + Q$. Such a process can non-deterministically "choose" to behave like $P$ or like $Q$. 
\item \textbf{Parallel Execution}: Each calculus has \textit{parallel composition} of processes, denoted as $P \parallel Q$. Such a process represent the concurrent execution of both $P$ and $Q$, and can perform all the actions of $P$ as well as all the action of $Q$, in a interleaving manner. Differently from the nondeterministic sum, after $P\parallel Q$ performs an action of $P$, it can still perform the actions of $Q$.
\item \textbf{Synchronization}: Each calculus has inter-process communication. Two processes executing in parallel can \textit{synchronize}, performing at the same time an action and a coaction. A parallel process $\alpha.P \parallel \overline{\alpha}.Q$ can evolve in $P \parallel Q$, as the two processes had performed a synchronous communication. Such synchronization transition is considered as a invisible action, as it can be understood as a internal action of the (composite) process $\alpha.P \parallel \overline{\alpha}.Q$ 

\end{itemize}

To give a running example of a calculus with these features, we will define syntax and semantics of \textit{value-passing CCS} \cite{hennessyTheoryCommunicatingProcesses1993}.

\begin{align*}
  P \Coloneqq \nil \mid c!v. P \mid c?x.P \mid \tau.P \mid P + P \mid P \parallel P
\end{align*}	

In value passing CCS, $c!v$ is the action of sending a value $v$ through a channel $c$, and its coaction is $c?v$, of receiving a value from channel $c$. In the process $c?x.P$, we say that $x$ is a bound variable, otherwise is free.

Usually, the semamtic of a process calculus is given in a SOS-fashion, as a \textit{Labelled Transition System}. Given a process $P$, its LTS can be understood as a rooted directed graph, where the nodes are processes, i.e. states of the computation, and the outgoing edges are the actions that a process can perform.

Formally, a Labelled Transition System is a triple $\langle S , Act, \rightarrow \rangle$ where \begin{itemize}
\item $S$ is a set of states
\item $Act$ is a set of transition labels
\item $\rightarrow 	\subseteq S\times Act_\tau \times S$ is the transition relation, with $Act_\tau = Act \cup \{\tau\}$
\end{itemize} 

An element $(s, \alpha, t) \in \rightarrow$ is called a \textit{transition}, and is often written as $s \xrightarrow{\alpha} t$. We denote with $\Rightarrow$ the reflexive and transitive closure of $\xrightarrow{\tau}$, and use $s \xRightarrow{\alpha} t$ as an abbreviation for $s \Rightarrow s' \xrightarrow{\alpha} t' \Rightarrow t$ for some $s', t' \in S$. When dealing with process calculi, the set $S$ of states is the set of all processes, and a transition $P \xrightarrow{\alpha} P'$ means that the process $P$ performs an action $\alpha$ and evolves in $P'$.

The LTS of a process in value passing CCS is given by the following set of derivation rules: 

\begin{gather*}
c!v.P \xrightarrow{c!v} P \qquad c?x.P \xrightarrow{c?v} P[v/x] \qquad \	tau.P \xrightarrow{\tau} P
\\
\infer{P + Q \xrightarrow{\alpha} P'}{P \xrightarrow{\alpha} P'} \qquad \infer{P + Q \xrightarrow{\alpha} Q'}{Q \xrightarrow{\alpha} Q'} 
\\
\infer{P \parallel Q \xrightarrow{\alpha} P'\parallel Q}{P \xrightarrow{\alpha} P'} \qquad \infer{P \parallel Q \xrightarrow{\alpha} P\parallel Q'}{Q \xrightarrow{\alpha} Q'} 
\\
\infer{P \parallel Q \xrightarrow{\tau} P'\parallel Q'}{P \xrightarrow{c!v} P' & Q \xrightarrow{c?v} Q' } \qquad
\infer{P \parallel Q \xrightarrow{\tau} P'\parallel Q'}{P \xrightarrow{c?v} P' & Q \xrightarrow{c!v} Q' } 
\\
\end{gather*}

\subsection{Bisimulation}

One of the most important notion in the theory of process calculi is the bisimilarity relation. Two processes are bisimilar when they are behaviourally equivalent, i.e. express the same behaviour. Different ways to compare the behaviour of two processes yields different definitions of bisimilarity.

The simplest and more natural notion of bisimilarity is \textit{strong bisimilarity}, where two bisimilar processes must express exactly the same behaviour, i.e. must perform the same action, both visible and silent.
Let $\langle S , Act, \rightarrow \rangle$ be a LTS. Then a symmetric relation $\rel \subseteq S \times S$ is a \textit{strong bisimulation} if and only if, whenever $s \rel t$, then 
\begin{center}
if $s \xrightarrow{\alpha} s'$ then $t \xrightarrow{\alpha} t'$ for some $t'$ such that $s' \rel t'$
\end{center}
Two states $s, t \in S$ are said to be \textit{strongly bisimilar}, written $s \sim t$, if exists a strong bisimulation $\rel$ such that $s \rel t$. This means that bisimilarity is the largest bisimulation, the union of all the bisimulation relations.
Strong bisimulation is an equivalence relation, meaning that is reflexive, symmetrical and transitive, and it is also a \textit{congruence}, meaning that, if $P$ and $Q$ are bisimilar, so must be $P\parallel R$ and $Q\parallel R$, or $P + R$ and $Q + R$, or any other couple of processes that can be constructed starting from $P$ and $Q$.



A weaker notion of behavioural equivalence is \textit{weak bisimilarity}, which requires that two bisimilar processes exhibit the same visible actions, but allows internal action to be matched by zero or more internal action.
Let $\langle S , Act, \rightarrow \rangle$ be a LTS. Then a symmetric relation $\rel \subseteq S \times S$ is a \textit{weak bisimulation} if and only if, whenever $s \rel t$, then 
\begin{center}
if $s \xrightarrow{\alpha} s'$ then $t \xRightarrow{\alpha} t'$ for some $t'$ such that $s' \rel t'$
\end{center}
Two states $s, t \in S$ are said to be \textit{weakly bisimilar}, written $s \approx t$, if a weak bisimulation $\rel$ exists such that $s \rel t$.  Weak bisimilarity is not a congruence, as $P = \alpha.\nil$ and $Q = \tau.\alpha.\nil$ are bisimilar, but $P + \beta.\nil$ and $Q + \beta.\nil$ are not.


Halfway between strong and weak bisimilarity there is the notion of \textit{branching bisimilarity}, that is coarser than strong bisimilarity, as it ignores internal actions, but is finer than weak bisimilarity, as it matches the braching structure more accurately.
Let $\langle S , Act, \rightarrow \rangle$ be a LTS. Then a symmetric relation $\rel \subseteq S \times S$ is a \textit{branching bisimulation} if and only if, whenever $s \rel t$, then 
\begin{center}
if $s \xrightarrow{\alpha} s'$ then $t \xRightarrow t' \xrightarrow{\alpha} t''$ for some $t', t''$ such that $s \rel t'$ and $s' \rel t''$
\end{center}
Two states $s, t \in S$ are said to be \textit{branching bisimilar}, written $s \simeq t$, if a branching bisimulation $\rel$ exists such that $s \rel t$.

The two processes \[P = \alpha.\nil + \tau.\beta.\nil \qquad Q = \alpha.\nil + \beta.\nil + \tau.\beta.\nil\] are weakly bisimilar, because obviously if $P$ performs an action $Q$ can replicate it, and in $Q$ performs its additional $\beta$ action, $P$ can replicate it as $\tau.\beta$. These two processes are instead not branching bisimilar, because if $Q$ performs its $\beta$ action, $P$ can't replicate it, as after a $\tau$ transition it evolves in $\beta.\nil$, that it is not bisimilar to $Q$.

\subsection{Probabilistic LTS} \label{pLTS}

The usual concepts of process calculus has been successfully extended to model probabilistic systems. We will present in this section the most common definitions of probabilistic LTS and probabilistic bisimulation, following the comprehensive analysis of \cite{hennessyExploringProbabilisticBisimulations2012, dengLogicalMetricAlgorithmic2011}.

\subsubsection{Probabilistic Lifting}
First of all, we give some preliminary mathematical definitions about probabilistic distributions, and about the very general concept of \textit{lifting} a relation between object to a relation between distribution of objects.

Given a set $S$, a (discrete) \textit{probability distribution on $S$} is a mapping $\Delta: S \rightarrow [0, 1]$ such that $\sum_{s\in S} \Delta(s) = 1$. We indicate with $\distr(S)$ the set of all probability distribution on $S$.
The \textit{support} of $\Delta \in \distr(S)$ is defined as $\lceil\Delta\rceil = \{s \in S \mid \Delta(s) > 0\}$.

We use $\overline{s}$ to denote the \textit{point distribution} on $s$ (also known as Dirac distribution, in the continuous case):
\[
	\overline{s}(t) = 
	\begin{cases} 1 \text{ if }t = s \\
	0 \text{ if } t\neq s
	\end{cases}
\]

Given a set $\{p_i\}$ of probabilities (i.e. $\sum_i p_i = 1$ and $p_i > 0$ for each $i$), we define the \textit{convex combination} of distributions $\left(\sum_i p_i \Delta_i\right)$ as the only 
distribution such that
\[
\left(\sum_i p_i \Delta_i\right)(s) = \sum_i p_i \Delta_i(s)
\]
We often abbreviate $p \Delta + (1-p) \Theta$ as $\Delta \psum{p} \Theta$.

For any set $D \subseteq \distr(S)$ of distributions, we denote with $CC(D)$ the \textit{convex closure} of $D$, i.e. the least subset of $\distr(S)$ that contains $D$ and is closed un the operations $ - \psum{p} - $ for any $p$ with $0 \leq p \leq 1$.


In order to extend the notions from to "classical" process algebraic approach to a probabilistic setting, it is useful to define a \textit{probabilistic lifting}, based on the concept of linearity.

A relation $\mathcal{R} \subseteq \distr(S) \times \distr(S)$  between distributions is said to be \textit{linear} if $\Delta_1 \rel \Theta_1$ and $\Delta_2 \rel \Theta_2$ implies $(\Delta_1 \psum{p} \Delta_2) \rel (\Theta_1 \psum{p} \Theta_2)$ of any $0 \leq p \leq 1$.

Given a relation $\rel \subseteq S \times S$, we define its \textit{lifting} $\mathring{\rel} \subseteq \distr(S) \times \distr(S)$ as the smaller linear relation such that $s \rel t$ implies $\overline{s} \mathring{\rel} \overline{t}$.

With abuse of notation, we denote with the same symbol also the lifting of relations $\rel \subseteq S \times \distr(S)$. Given a relation $\rel \subseteq S \times \distr(S)$, we define its \textit{lifting} $\mathring{\rel} \subseteq \distr(S) \times \distr(S)$ as the smaller linear relation such that $s \rel \Delta$ implies $\overline{s} \mathring{\rel} \Delta$.


These lifted relations enjoys two useful properties. Interestingly, both this property are equivalent to the given definition, and are indeed used as the definition in various works on probabilistic bisimulations. \begin{itemize}
\item Given $\rel \subseteq S \times S$, then $\Delta \lrel \Theta$ if ans only if $\Delta$ and $\Theta$ can be decomposed as follows: \begin{enumerate}
\item $\Delta = \sum_{i \in I} p_i \overline{s_i}$, where $I$ is a finite index set and $\sum_{i \in I}p_i = 1$
\item For each $i \in I$ there is a state $t_i$ such that $s_i \rel t_i$
\item $\Theta = \sum_{i\in I}p_i\overline{t_i}$ 
\end{enumerate}
\item Given an equivalence $\rel \subseteq S \times S$, then $\Delta \lrel \Theta$ if and only if, for all equivalence classes $C \in S/R$
\[\sum_{s\in C} \Delta(s) = \sum_{s\in C} \Theta(s)\]
\end{itemize}

\subsubsection{Probabilistic Labelled Transition Systems}

We are now ready to introduce the main features of probabilistic process calculi. In addition to the usual operators of action prefixes, parallel composition and nondeterministic sum, such calculi often include a \textit{probabilistic choice} operator, like $P \qsum{p} Q$, where $p$ is any probability $0 \leq p \leq 1$. A process $P \qsum{p} Q$ makes a probabilistic choice between $P$ and $Q$, that is, it evolves in a \textit{distribution of processes}.

Formally, we can define the operational semantic of a probabilistic process calculus as a \textit{pLTS}.
	
A \textit{Probabilistic Labelled Transition System} (pLTS) is a triple $\langle S , Act, \rightarrow \rangle$ where \begin{itemize}
\item $S$ is a set of states
\item $Act$ is a set of transition labels
\item $\rightarrow 	\subseteq S\times Act_\tau \times \mathcal{D}(S)$ is the transition relation, with $Act_\tau = Act \cup \{\tau\}$
\end{itemize} 

\note{ho introdotto il $\qsum{}$ per differenziare l'operatore sintattico dell'algebra  dalla notazione per le distribuzioni, ma comunque non sono sicuro che l'esempio serva a chiarire e non a confondere le idee} For example, we could write 
\[P \qsum{\frac{2}{3}} Q \xlongrightarrow{\tau} \overline{P} \psum{\frac{2}{3}} \overline{Q}\]
where $P \qsum{\frac{2}{3}} Q$ is a single state, a syntactic element, and $\overline{P} \psum{\frac{2}{3}} \overline{Q}$ is a distribution of states, that assign to state $P$ probability $\frac{2}{3}$, and to state $Q$ probability $\frac{1}{3}$


On this definition of pLTS, there are actually two separate notion of (strong) probabilistic bisimilarity, that differ for how they treat non-determinism.

The first goes under the name of \textit{Larsen-Skou bisimulation}. Let $\langle S , Act, \rightarrow \rangle$ be a pLTS. Then a symmetric relation $\rel \subseteq S \times S$ is a \textit{Larsen-Skou bisimultion} if and only if, whenever $s \rel t$, then 
\begin{center}
if $s \xrightarrow{\alpha} \Delta$ then $t \xrightarrow{\alpha} \Theta$ for some $\Theta$ such that $\Delta \lrel \Theta$
\end{center}
The usual definition of Larsen-Skou bisimulation does not use the lifting operation, and requires that $\Theta$ assigns the same probability as $\Delta$ to each equivalence class of $S/R$, which is a completely equivalent formulation, as seen in the previous section. Two states $s, t \in S$ are said to be \textit{Larsen-Skou bisimilar}, written $s \sim_{LS} t$, if a Larsen-Skou bisimulation $\rel$ exists such that $s \rel t$.

The second, strictly coarser notion of probabilistic bisimilarity is \textit{Segala bisimilarity}, which consider also the probabilistic behaviour that could happen in presence of an \textit{adversary}. When there is a non-deterministic choice, like in $\alpha.P + \alpha.Q$, it is common to assume the existance of an external agent, an adversary, that resolves such nondeterministic choices in an arbitrary way. In the example made before, one could suppose an adversary that always chooses the left transition, or one that always chooses the right transition. Segala bisimilarity consider the cases when the adversary can randomize, and for example choose the left transition with probability $p$ and the right transition with probability $1-p$.

To define Segala bisimilarity, is necessary to introduce \textit{combined transitions}, i.e. a transition relation $\longrightarrow_{cc} \subseteq S \times Act_t \times \distr(S)$ such that 
\[ s \xrightarrow{\alpha} \Delta \text{ if and only if } \Delta \in CC(\big\{\Theta \mid s \xrightarrow{\alpha} \Theta \big\})
\]
that is, $\Delta$ is reachable from $s$, or is a convex combination of distributions reachable from $s$.
Let $\langle S , Act, \rightarrow \rangle$ be a pLTS. Then a symmetric relation $\rel \subseteq S \times S$ is a \textit{Segala bisimultion} if and only if, whenever $s \rel t$, then 
\begin{center}
if $s \xrightarrow{\alpha} \Delta$ then $t \xrightarrow{\alpha}_{cc} \Theta$ for some $\Theta$ such that $\Delta \lrel \Theta$
\end{center}
Two states $s, t \in S$ are said to be \textit{Segala bisimilar}, written $s \sim_S t$, if a Segala bisimulation $\rel$ exists such that $s \rel t$.
\subsubsection{From pLTS to LTS}

The given definition of pLTS can be seen as a disconnected, bipartite graph, where each node is either an element of $S$, i.e. a process, or an element of $\distr(S)$, i.e. a distribution of processes, and all the edges go from $S$ to $\distr(S)$. This is not a problem to define probabilistic bisimilarity, but there are other settings where a connected LTS is preferred, for example when defining weak bisimilarity, temporal logics or model checking algorithm.


To "complete" a pLTS it is necessary to define how a distribution "evolves", which are its outgoing transitions. There are two alternative approaches, that we could call "probabilistic branching" and "distribution transformer".

The first, arguably more standard approach is \textit{probabilistic branching}, that proposes a derivation rule like
\[ \sum_i p_i \overline{s_i} \quad
\substack{  p_i  \\  \rightsquigarrow } 
\quad s_i \]
Where a distribution of processes "picks" just one process, evolving with a \textit{probabilistic transition}. This approach, used also in probabilistic model checking \cite{kwiatkowskaPRISMVerificationProbabilistic2011}, gives a probability to the single transition, and so allows to define a probabilistic measure for a whole path of computation.

With such a rule, a pLTS $\langle S, Act, \rightarrow\rangle$ can be "completed" to a bipartite LTS, where the states are either processer or distribution of processes, the $\rightarrow$ transition goes from states to distribution, with a label in $Act_\tau$, and the $\rightsquigarrow$ transition goes from distributions to states, with $0 \leq p \leq 1$ as a label. 

The second, most recent approach, is known as the \textit{distribution transformer} semantic, also called belief-state transformer semantic or labelled Markov process semantic. It specifies that a distribution of processes must evolve in a distribution of processes, with a derivation rule like 
\[ \infer{ \sum_i p_i \overline{s_i} \xrightarrow{\alpha} \sum_i p_i \Delta_i}
  { p_i \xrightarrow{\alpha} \Delta_i & \text{ for each  }i }
\]
So, for a distribution to evolve, it is necessary that all the precesses in its support can perform the same action $\alpha$. A distribution like $\alpha.P \psum{\frac{1}{2}} \beta.Q$, for example, has no outgoing transition.

Notice that the addition of this rule is equivalent to lifting the transition relation $\rightarrow \subseteq S \times Act \times \distr(S)$ to a relation $\mathring{\rightarrow} \subseteq \distr(S) \times Act \times \distr(S)$. Since each state $s \in S$ can be also seen as a distribution $\overline{s} \in \distr(S)$, this lifting relation de facto defines a "complete" LTS of distributions $\langle \distr(S), Act, \mathring{\rightarrow}\rangle$. Observe that $\rightarrow_{cc} \subset \mathring{\rightarrow}$, and in fat this distribution transformation semantic is more often than not associated to a Segala bisimilarity.



\subsection{Reduction systems}\label{bkg_reduction_system}

In the early works on process calculus, bisimilarity in its various form was adopted as the mainstream notion of behavioural equivalence. Starting from \cite{milnerBarbedBisimulation1992}, a different notion of equivalence was considered, conceptually simpler and more general, under the name
of \textbf{barbed congruence}, or barbed congruence. According to this notion, two processes are equivalent if they can not be distinguished by an external observer. That is, two processes $P$ and $Q$ are equivalent if, under any context $B[]$, $B[P]$ and $B[Q]$ express the same observable behaviour, based on a general concept of "observable", called \textit{barb}.

We will first introduce the \textit{reduction semantic} for process calculi, and then show how it can be used to define barbed equivalence. 

A reduction semantic for a process calculus \cite{milnerFunctionsProcesses1990, berryChemicalAbstractMachine1989}, is an alternative way to define the dynamics of a process, using a \textit{Reduction system} instead of a Labelled Transition System. 

A \textit{Reduction System} (RS), or unlabelled transition system, is a couple $\langle S,  \rightarrow \rangle$ where \begin{itemize}
\item $S$ is a set of states
\item $\rightarrow 	\subseteq S\times S$ is the transition relation.
\end{itemize} 

In a reduction system, like lambda calculus and all other term-rewriting, a reduction is possible only when the two subtemrs that interacts are syntactically contiguous, i.e. they form a redex. 

It is possible to define a reduction system for a process calculus like value-passing CCS, for example identifying the redexes:
\[ \tau.P \rightarrow P \qquad c!v.P \parallel c?x.Q \rightarrow P \parallel Q[v/x]\]
together with the usual rules
\[ \infer{P \parallel Q \rightarrow P'\parallel Q}{P \rightarrow P'}
\qquad
\infer{P + Q \rightarrow P'}{P \rightarrow P'}\]

In process calculi, thanks to the labelled semantic, subprocesses are allowed to interact also when they are syntactically "distant", like $c!v.\nil \parallel d?x.\nil \parallel c?x.P$. So in order to define a reduction system as the one above, it is necessary to introduce a way to ignore the syntactic arrangement and "reorder" the subprocess as needed.  This is achieved considering the reduction system modulo a \textit{structural congruence relation}. In the case of value-passing CCS, this relaction could be the smallest equivalence relation that is closed for $\alpha$-conversion and satisfies  
\begin{gather*}
P\parallel\nil \equiv P \qquad P\parallel Q \equiv Q \parallel P \qquad P\parallel(Q\parallel R) \equiv (P \parallel Q) \parallel R \\
P + \nil \equiv P \qquad P + Q \equiv Q + P \qquad P + (Q + R) \equiv (P + Q) + R
\end{gather*}

Notice that the reduction system presented above, together with the structural congruence relation, determines exactly the $\xrightarrow{\tau}$ transition of the LTS semantics presented in the previous section. Reduction systems are in fact often used to describe the dynamics of calculi with tho interaction with an "outside environment", so no input and output transitions, only inter-process communication.


Obiously, a bisimularity relation defined on a reduction system is a very coarse relation, that simply consider the number of computational steps a process can make. To recover the notion of strong LTS-bisimilarity in the reduction semantics setting, it is necessary to recover some of the "observational power" of labelled bisimulations, through the concept of \textit{barbs}.

We call \textit{barb} a predicate on states, often used to capture a certain notion of "observable property". Given a barb $b$, we write $s\downarrow_b$ to say that $s$ statisfies the predicate $b$, i.e. expresses that property. For value-passing CCS, a suitable observable property is "$P$ is capable to send any value con channel $c$". That is, we define the barb $c$ as the predicate 
\[\{P \mid \exists v, P' \ P \xrightarrow{c!v} P'\}\]
%or equivalently, 
%\[P \downarrow_c \text{ if and only if } P \xrightarrow{c!v} P'\]
It is important to remark that in this case we defined, for simplicity, a barb as a property based on a preexisting labelled semantic. In all the most recent calculus the semantic of a process can be formulated directly as a reduction system, and the barbs are usually syntactic in nature.

Given a set of barbs, i.e. a set of observable properties, is possible to define a \textit{barbed bisimulation}.
Let $\langle S , \rightarrow \rangle$ be a RS, and $B$ a set of barbs. Then a symmetric relation $\rel \subseteq S \times S$ is a \textit{barbed bisimulation} if and only if, whenever $s \rel t$, then 
\begin{itemize}
\item If $s \downarrow_b$ for some barb $b \in B$, then $t \downarrow_b$
\item If $s \rightarrow s'$ then $t \rightarrow t'$ for some $t'$ such that $s' \rel t'$
\end{itemize}
Two states $s, t \in S$ are said to be \textit{barbed bisimilar}, written $s \sim_b t$, if a barbed bisimulation $\rel$ exists such that $s \rel t$.

Barbed bisimilarity is often not useful per se, as for example $c!0.a!0.\nil \sim_b c!0.b!0.\nil$, since they both express the barb $\downarrow_c$, and have no outgoing transition. Barbed bisimulation is commonly used as the discriminating property of a contextual equivalence, called \textit{barbed equivalence}.

A context is a "process with a hole", for example $B[] = [] \parallel R$ or $B[] = [] + R$, where $R$ is any process. Given a context $B[]$, we define as $B[P]$ as the process obtained "filling" the hole with $P$, i.e. substituting $P$ in place of $[]$.

Given a set of contexts, two processes $P$ and $Q$ are said to be \textit{barbed equivalent}, or barbed congruent, written $P \simeq_b Q$, if for any context $C[]$, it holds that $C[P] \sim_b C[Q]$. With the previously defined barb $\downarrow_c$, and choosing just parallel context of the form $[] \parallel R$, it is possible to prove that
\[P \simeq_b Q \text{ if and only if } P \sim Q\]
that is, barbed equivalence is exactly the same as labelled bisimilarity.

To sum up, a reduction semantic allows to define a barbed equivalence relation, that: \begin{itemize}
\item Has the same power of labelled bisimilarity, but is defined in terms of a simpler transition system, inspired by term rewriting system.
\item Is "parametric" with respect to different barbs and different contexts, allowing for different observational power for the same calculus.
\item Is based on the very general concept of contextual equivalence, and it is used as the prototype of "natural, standard behavioural equivalence" for a lot of new calculi.
\item Is more difficult to prove, as it involves a universal quantification on all possible context, whereas proving bisimilarity requires as usual only to provide a bisimulation.
\end{itemize}




\chapter{Comparison of Quantum Process Calculi}\label{chapter3}

There is a number of proposals of quantum process calculi in the literature, often with different syntax, semantics and behavioural equivalences, even if they all model the same systems and the same protocols. 
There are three main lines of research that developed in recent years. The first, started with QPAlg and then developed with CQP, is inspired by the $\pi$-calculus. 
The second approach, developed simultaneously but independently, is centered around qCCS, that is a quantum extension of value-passing CCS. 
This thesis will focus on analyzing similarities and differences of these three calculi, QPAlg, CQP and qCCS. 
The third proposal, qACP, is less directly related and comparable with the first two, in the same way as its classical counterpart ACP is designed in a different fashion with respect to CCS/$\pi$-calculus. We postopone its comparison to future work.


\section{LTS and quantum states}
\subsection{QPAlg}

In classical process calculi, the operational semantics is given in terms of transitions $P \xrightarrow{\alpha} P'$ between processes. In [qpalg2004], the authors describe how to "quantumize" a simple classical process caulculus, adding to it quantum variables and quantum actions. To do so, processes manipulating and communicating quantum data are always coupled with a state vector, describing the current value of the quantum variables. The operational semantic of QPAlg describes in fact an LTS composed of \textit{configurations}, i.e. states of the form \footnote{The actual configurations described in QPAlg are more complex than this, containing also a stack to manage variable delcaration. We will omit these non-quantum construct to simplify the presentation and the comparisons.}
\[
	\langle q_0, \ldots, q_n = \rho, P\rangle
\] 
in which $P$ is a process containing $q_0 \ldots q_n$ as free quantum variables, and $\rho \in \mathcal{D}(\Hto{n})$  describes the state of the qubits manipulated by $P$. The same approach is used in (almost) all the other calculi proposed so far, and has become somewhat standard.

This idea captures the imperative, stateful nature of quantum computation, in which a sequence of transformations are applied to the quantum variables (the qubits), treated as a mutable data structure.
In QPAlg, processes can be contructed with the usual actions of vaule passing calculi, $c?x$, $c!v$, but also with \textit{quantum action}, like $X, Z, H, M_{01}$, that correspond to applying unitaries or measurement to the underlying quantum state: \[ \langle q = \rho, H[q].P \rangle \xrightarrow{\tau} \langle q = H\rho H^\dagger, P \rangle\]

%\note{varibale instantiation}
Separating the syntactic, control part of the configuration (the process $P$) from the underlying quantum data (the statevector $\kp$) allows also for a "pass-by-reference" way of communication, opposed to the "pass-by-value" of classical value passing process algebras.
The usual rule for reception, in fact, would be
\[ c?x.P \xrightarrow{c?v} P[v/x]  \qquad \text{ for any value } v
\]
In this way, if $Q$ contains two occurrences of $x$, each of them gets instantiated with a different, independent copy of the value $v$. But if the value $v$ was a quantum state $\rho$, this would require duplicating the quantum information, which is impossible due to the no-cloning theorem. The solution proposed in various quantum calculi is substituting just the name of the variable, duplicating only the "pointers" to the same value.
\[ \langle q_1, \ldots q_n = \rho, c?x.P \rangle \xrightarrow{c?q} \confw{q_1, \ldots q_n = \rho, P[q/x]} \qquad \text{ for any } q \in q_0 \ldots q_n
\]

Notice that this idea requires the peculiar assumption that the \textit{value} recived from the external environment is already represented in the the configuration. In QPAlg and in other early works [qccs2006] there was a different rule, where the state vector is extended with a new value received from the environment. This approach was abandoned in more recent calculi, because extending the state vector when a qubit is received (or shrinking it when a qubit is sent) works poorly when exchanging entangled qubits. 
Suppose that a process $P$ receives just one qubit of a Bell pair, i.e. a qubit that is entangled with something on which $P$ has no control. The only way to represent the new qubit in the configuration is with its reduced density operator $\frac{1}{2}I$. But what happens if the process then receives also the second qubit of the Bell pair? Since we stored the reduced density operator in the configuration, is now impossible to reconstruct the original Bell pair.

Besides, the assumption that the qubit to be received is already represented in the configuration is indeed correct when there is a synchronization between two processes. In that situation, it would be inaccurate to extend the configuration with a new density operator. The systems that allocate a new qubit when receiving end up with de facto two different behaviors, one for communication between processes and one for communication with and unknown environment. This difference is reasonable, but is in contrast with the compositional design of process algebras, and has been abandoned in later systems.


%\note{measurements and probabilities}
Another key feature present in QPAlg and in all other calculi is the coexistence of \textit{nondeterminism}, arising from sums and parallel composition, and probabilistic behaviour, arising from the probabilistic nature of quantum measurements. When the process P, with state $\rho$ performs a measurement, there is a distribution of possible configurations in which it could evolve. So in all quantum process calculi, the semantics of a process can be defined by a pLTS $\langle Conf, Act, \rightarrow \rangle$, where $Conf$ is the set of all possible configurations, and $\rightarrow$ goes from $Conf$ to $\distr(Conf)$
\[
	\confw{q = \proj{+}, M_{01}[q].P} 
	\xlongrightarrow{\tau} 
	\confw{q = \proj{0}, P'} \psum{1/2} \confw{q = \proj{1}, P''}
\]
QPAlg follows the "probabilistic-transition" approach, so the distribution state branches probabilistically in one of the possible outcomes.



%\note{parallel composition}

The last relevant detail of the rules of QPAlg, shared also by the other calculi, is the treatment of parallel composition. Due to entangled state, the parallel operator is not entirely compositional. That is, the behaviour of $P\parallel Q$ cannot be described simply as the interleaving of $P$ and $Q$.

Only if two processes $P$ and $Q$ act on separable qubits, in fact, it would possible to treat them independently
\[ \infer{\langle p,q=\rho\otimes\sigma, P\parallel Q\rangle \xrightarrow{\alpha} \langle p,q=\rho'\otimes\sigma, P'\parallel Q\rangle}{\langle p=\rho, P\rangle \xrightarrow{\alpha} \langle p=\rho', P'\rangle}\]


But when $p, q$ are entangled, the quantum actions of $P$ cannot always be considered "local" to the process, as they effect also the value of $q$ (despite being labelled as internal transition). So in QPAlg a more general rule is used:
\[\infer{\langle p,q=\nu, P\parallel Q\rangle \xrightarrow{\alpha} \langle p,q=\nu', P'\parallel Q\rangle}{\langle p, q=\nu, P\rangle \xrightarrow{\alpha} \langle p, q=\nu', P'\rangle}\]
where quantum actions of $P$ must be intended as transformations on the whole Hilbert space $\calH_p \otimes \calH_q$.



\subsection{CQP}
\note{Ha senso mettere prima il contributo principale, e poi i dettagli minori (e il confronto con qpalg), o viceversa?}


\note{contributo principale}
In [CQP2005], the authors presented their calculus Communicating Quantum Process. The main contribution of their work is the introduction of affine type system, used to restrict the set of possible processes of the algebra to the "admissible" ones. Under the assumption that Alice, Bob and Charlie are in three different physical location, in fact, the process \[Alice = b!q.c!q.P\] should not be well typed, because Bob could read from the $b$ channel, Charlie from the $c$ channel, and there will be duplication of quantum information. 


Variables and expressions in CQP can have types \textbf{Int}, \textbf{Qbit} and \textbf{Unit}, and channels have the corresponding types $\widehat{\ }\textbf{Int}$, $\widehat{\ }\textbf{Qbit}$, $\widehat{\ }\textbf{Unit}$.
The typing judgements in CQP have the form \[\Gamma \vdash P\] meaning that $P$ is well typed under the context $\Gamma$. $\Gamma$ contains both classical variables and quantum variables: the former are treated following the usual typing rules, the latter are subject to affine typing rules. Affine rules guarantee that each quantum variable will be sent at most once, and can't be used after it is sent, thanks to how the typing contexts $\Gamma$ are constructed.


From the practical point of view, this means that, \begin{itemize} 
\item if $c!q.P$ is well typed, where $q$ is a quantum variable, then $P$ cannot contain any other occurence of $q$.
\item if $P \parallel Q$ is well typed, then $P$ and $Q$ cannot have occurrences of the same free quantum variables. The authors call this property \textit{unique ownership of qubits}.
\end{itemize}

Notice that the typing rules are not \textit{linear}, as that weakening holds both for classical and quantum variables. This means that the quantum variables must be sent at most once, not exactly once. So a process like 
\[P = c?q.H[q].\nil\]
is well typed, and the qubit denoted by $q$ becomes inaccessible for all other processes.

An important remark is that the typing rules are intended to model processes in different physical location, where sending a qubit means physically moving the quantum system. In other settings this type system could be not necessary. For example different processes in a quantum computer, manipulating the same set of quantum registers, situations like 
\[ H[q].P \parallel Y[q].Q\]
are reasonable, and their behaviour would be described as usual as race condition between the to processes acting on the same shared data.

\note{dettagli minori}
All the characteristic of QPAlg, in terms of configurations, probabilistic branching and parallel composition are also present in CQP. Differently from QPAlg, CQP  describes \textit{closed quantum systems}, without an unknown environment. That is, there are no transitions $\xrightarrow{c?x}$ or $\xrightarrow{c!v}$, all the communication happens as synchronization between processes. According to this, CQP semantics is described as a reduction system, where the transitions corresponds either to internal actions or infra-process communication.  


Without interaction with an external environment, all the information is completely represented in the initial configuration, and then there is no need to use mixed quantum states. The state of the quantum variables is thus described as a state vector.


CQP is based on the typed pi calculus, with constructs to create new (typed) channel names, new classical variables and new quantum variables. When declaring  new qubit, the state vector is extended with the default value $\kz$
\[ \confw{q_1, \ldots q_n = \kp, \mono{qbit }x.P} \rightarrow \confw{x,q_1, \ldots q_n = \kz\otimes\kp , P}\]

Thanks to its type system, CQP supports integers and arithmetic expressions, and also \textit{quantum expressions} like $q_1\ldots q_n *= U$ and $M[q_1 \ldots q_n]$. There expressions that take the place of quantum action prefixes of QPAlg, changing the underlying quantum state and causing probabilistic branching.
 

In [Davidson], the author introduces a labelled semantics  to CQP, so to define a bisimilarity relation on CQP processes. On top of internal actions and synchronization, the possible actions of a process are as usual reception and sending, in the form $c?x$ and $c!v$. A process $\confw{q_1 \ldots q_n = \kp, c?x.P}$ can evolve receiving a qubit $q_i$ from the environment, but its name and value must already be present in the configuration. Reception and sending in CQP do not modify the quantum state, simply move the quantum names around. As already said, this can be seen as physically moving qubits between different location, but also as concurrent processes exchanging lock on some shared variable, to guarantee mutual exclusion.


As for the previous work, all the information is contained in the initial configuration and is never lost, so quantum variables state is expressed as a state vector, instead of as a density matrix.



In order to define a bisimilarity relation that is also a congruence, in [Davidson] a different semantic is proposed, featuring so called \textit{mixed configuration}. These mixed configuration can be considered as probabilistic distributions of configurations, but must be treated as a single state. In other words this semantics mixes the "probabilistic transition" and "probabilistic state" approches of \ref{pLTS}: after a measurement, a configuration evolves in a mixed configuration (that is, a distribution of configurations), and this mixed configuration do not perform a probabilistic transition, so does not decays in a single configuration. 

Mixed configurations represent partial knowledge of the classical variables in a configuration, just like mixed state represent partial knowledge of quantum state. So, after a measurement, a configuration must evolve in a mixed configuration, because to an external observer, the outcome of the measurement in unknown. Only when the mixed configuration performs an output transition, where it communicates the output of the measurement, a mixed configuration can branch probabilistically in one of its possible configuration. As we will see, this difference is crucial, because changes the observable properties of the defined system: in the previous semantic, after a measurement there would be always probabilistic branching, i.e. the outcome probabilities are immediately observable; in the mixed configuration semantic, after a measurement there are no probabilities to observe (as that outcome is still local to the process), and the probabilities appear only when and if the outcome is comunicated.

\note{esempio esplicativo, in tikzit? fatto a mano?}

\subsection{qCCS}

In [feng2006], the authors present qCCS, applying the ideas present in QPAlg and CQP to a simpler CCS-inspired calculus, without new name declaration and without expressions. A qCCS process id defined assuming a fixed set con classical channel, of quantum channels, and of quantum names. qCCS enforces the same conditions of "unique ownership of qubits" present in CQP, but without the use of a type system. The terms in qCCS are in fact inductively constructed in a way that preserves this property, for example specificating that $c!q.P$ is a process only if $q$ is not a free quantum variable in $P$.


In the first work on qCCS where presented two different rules for quantum input, one adding a qubit to the configuration, the other assuming that the qubit is already present in the configuration. As already said, only the second rule was used in subsequent works.


All the characteristic of QPAlg and CQP, in terms of configurations, parallel composition and communication are also present in CQP. The main difference between qCCS and the previous calculi is its "probabilistic state" approach to treat probabilities and nondeterminism.


Like in QPAlg, a process can be constructed with \textit{quantum action}, that in qCC  are unitaries, superoperators and measurements. A measurement introduces a new classical variable to represent the outcome, and evolves in a distribution of configurations
\[ \confw{q = \proj{+}, M_{01}[q \rhd x].P} \xrightarrow{\tau} \confw{q = \proj{0}, P[0/x]} \psum{1/2} \confw{q = \proj{0}, P[1/x]} 
\]

In [feng2006], a distribution of configuration cannot evolve in any manner, but in [bisimulation for quantum], [open bisimulation] the transition relations $\longrightarrow \subseteq Conf \times Act_\tau \times \distr(Conf)$ is lifted to a relation between distribution $\longrightarrow \subseteq \distr(Conf) \times Act_\tau \times \distr(Conf)$, i.e. following the "probabilistic state" approach described in section \ref{pLTS}.


Among the various semantics developed for qCCS, in order to find a bisimilarity equation that was also a congruence, in [algebra for quantum] the authors proposed a simplified fragment of qCCS, without measurements and classical data. The semantic is purely non-deterministic, without probabilistic behaviour, and the superoperators applied are treated as \textit{observable actions}, as possible labels of the transition relation. This calculus is intended to model quantum computation, more than quantum protocols, and we will not treat it inthis chapter.


%In [symbolic] the authors propose a symbolic semantic for qCCS, that abstracts away from the actual quantum state in the initial configuration. Instead of probabilistic distribution, the concept of \textit{superoperator-valued distribution} is used, i.e. functions that associate to each process a trace non-increasing superoperator, representing both the probability that the process is reached, and the transformation that were applied to the quantum state to reach that process.

\section{Bisimulation}\label{stateOfTheAr_Bisimulation}
We now focus on the various bisimilarity notion for quantum processes presented in the literature. All of these bisimilarities employ the probabilistic lifting seen in section \ref{pLTS}, some defining a Larsen-Skou bisimilarity, some others a Segala bisimilarity.  What really tells apart the different notions are their quantum related details, reguarding which are the observable properties of quantum values, and when these values are considered visible.
 
\subsection{QPAlg/CQP}\label{lalire_bisimulation}

In \cite{lalireRelationsQuantumProcesses2006}, the authors propose a notion of bisimilarity for QPAlg, that was adapted in \cite{davidsonFormalVerificationTechniques2012} for the labelled semantic of CQP. This bisimilarity is based on a probabilistic branching bisimulation, and deals with quantum communication equating the reduced density matrix of sent qubits. In other words, when a process $P$ performs an output transition $c!v$, for a classical value $v$, a bisimilar process $Q$ should perform an output transition on the same channel with the same value. When a process $P$ performs an output transition $c!x$, for a \textit{quantum name} $x$, $Q$ is not required to output the same \textit{name} $x$, it should instead send a qubit with the same \textit{state} of $x$. The chosen way to define the state of qubit $q$ in the configuration with (global) state $q, q_1 \ldots q_n = \rho$ is the reduced density matrix of $q$, i.e. $tr_{q_1 \ldots q_n}(\rho)$.

%\note{non so se metterla
%\begin{definition}{QPAlg bisimulation}
%A symmetric relation $\rel$ between configurations $\confw{\widetilde{q} = \rho, P}$, $\confw{\widetilde{q} = \sigma, Q}$ is a probabilistic branching bisimulaton if \begin{itemize}
%\item $\confw{\widetilde{q} = \rho, P} \xrightarrow{c!v} \Delta$ for a classical value $v$, then $\confw{\widetilde{q} = \sigma, Q} \Rightarrow \confw{\widetilde{q} = \sigma', Q'} \xrightarrow{c!v} \Theta$, with $\confw{\widetilde{q} = \rho, P} \rel \confw{\widetilde{q} = \sigma', Q'}$ and $\Delta \lrel \Theta$.  
%\item $\confw{\widetilde{q} = \rho, P} \xrightarrow{c!x} \Delta$ for a quantum name $x$, then $\confw{\widetilde{q} = \sigma, Q} \Rightarrow \confw{\widetilde{q} = \sigma', Q'} \xrightarrow{c!y} \Theta$, with  $tr_{\widetilde{q} - x}(\rho) = tr_{\widetilde{q} - y}(\sigma')$, $\confw{\widetilde{q} = \rho, P} \rel \confw{\widetilde{q} = \sigma', Q'}$ and $\Delta \lrel \Theta$.
%\item The same holds also for classical input and quantum input.
%\end{itemize}
%\end{definition}
%}

The notion of bisimilarity between configuration is given as usual, and two processes $P$ and $Q$ are considered bisimilar, written $P \sim Q$, when $\confw{\widetilde{q} = \rho, P}$ is bisimilar to $\confw{\widetilde{q} = \rho, Q}$ for any $\rho$.

This bisimilarity consider the state of a qubit $Q$ an observable only when it is sent to the external environment. This means that the processes 
\[ P = X[q].\nil \qquad Q = Z[q].\nil\]
are indeed bisimilar, as the qubit $q$ ends up with a different quantum state, but the difference is never observed, because the qubit is never sent.

This bisimilarity is not a congruence with respect to summation and parallel composition. The former is expected with weak or branching bisimulation, and is due to the usual problem 
\begin{align*}
c!v.\nil &\sim \tau.c!v.\nil \\
c!v.\nil + d!v &\not\sim \tau.c!v.\nil + d!v
\end{align*}

The problem with parallel composition is instead purely quantum-related, as it is caused by the side effect of measurements and entanglement. For example, the two processes $P$ and $Q$ are bisimilar, but $P\parallel c!q_2$ is not bisimilar to $Q \parallel c!q2$:
\begin{align*}
 P &= M_{01}[q_1].\nil &\sim & & Q &= M_\pm[q_1].\nil \\ 
 P \parallel c!q_2 &=  M_{01}[q_1].\nil \parallel c!q_2 &\not\sim & & Q \parallel c!q_2 &=  M_\pm[q_1].\nil \parallel c!q_2 
\end{align*} 

$P$ and $Q$ are clearly bisimilar, as they do not perform any visible action. But when $P\parallel c!q_2$ operates on an entangled state, like the Bell state $\beta = \oost\ket{00} + \oost\ket{11}$, the measurement that happens on $q_1$ has effect on $q_2$, and $q_2$ will decay in state $\kz$ or $\ko$. The same happens when $Q\parallel c!q_2$ operates on the same Bell state, and $q_2$ will decay in state $\kpl$ or $\km$. Since $q_2$ is sent, this difference becomes observable, and the two processes are not bisimilar.

\subsection{qCCS}

The bisimilarity relations proposed for qCCS differ from the one for QPAlg in a number of "classical" details, not related to the quantum properties. First of all, for qCCS are defined strong and weak bisimilarity, not branching bisimilarity. QPAlg addresses probabilistic behaviour with a Larsen-Skou bisimulation, qCCS with a combined-moves bisimulation. \note{qCCS richiede l'uguaglianza dei quantum names, QPAlg solo dei valori sottostanti.}

Besides these "classical" differences in the treatment of silent transitions and probabilities, one key difference telling the two relations apart is the notion of "observable" quantum state. Where in QPAlg the state of a qubit could be observed only when it was sent, in qCCS the state of a qubit can be observed as soon as the computation has "ended", i.e. there are no more transformation to be applied to that qubit.

In  \cite{fengProbabilisticBisimulationsQuantum2007}, two configuration $\confw{\widetilde{q} = \rho, P}$ and $\confw{\widetilde{q} = \sigma, Q}$ being bisimilar required that if $\confw{\widetilde{q} = \rho, P} \not\rightarrow$, then $\confw{\widetilde{q} = \sigma, Q} \not\rightarrow$ and $\rho$ is equal to $\sigma$, up to a permutation of the quantum names in $\widetilde{q}$. 

An equivalent condition was property in \cite{yingAlgebraQuantumProcesses2010}, where the quantum actions $U[q]$, $M[q]$ and $\sop[q]$ were considered observable actions, and used as labels of the LTS. Since two bisimilar processes will perform the same observable action, when startig from the same quantum state $\rho$ they will necessarily end up in the same quantum state $\rho'$.

The condition presented in \cite{fengProbabilisticBisimulationsQuantum2007} was too strict, and didn't work well with recursive processes, that are often non-terminating. So in \cite{fengBisimulationQuantumProcesses2012} a more general condition was proposed, requiring the equality not of the whole quantum state, but only of the qubits that will not be further modified. That is, in a configuration $\confw{q_1, q_2 = \rho, H[q_2].\nil}$, the bit $q_1$ should be regarded as observable, as it does not appear as a free quantum variable in $H[q_2].\nil$. The bit $q_2$, instead, should not be considered when checking for bisimilarity, as it is not in its final state.

More formally, \cite{fengBisimulationQuantumProcesses2012} defines the notion of \textit{environment} of a configuration, namely the reduced density operator of the values of the qubits not used by the process.
\[env(\confw{\widetilde{q} = \rho, P}) = tr_{qv(P)}(\rho)\]


The bisimilarity of $\confw{\widetilde{q} = \rho, P} $ and $\confw{\widetilde{q} = 	\sigma, Q}$ then requires that 
$env(\confw{\widetilde{q} = \rho, P}) = env(\confw{\widetilde{q} = \sigma, P})$.
\note{dire che è una congruenza}



In genral, for qCCS, in the example seen before
\[ P = X[q].\nil \qquad Q = Z[q].\nil\]

$P$ and $Q$ are not bisimilar, as they end in different quantum states. $\confw{q = \proj{0}, P}$ will evolve in a configuration with quantum state $\ko$, while $\confw{q = \proj{0}, Q}$ will evolve in a configuration with state $\kz$. Since the quantum state is observable, we can say that $\ko$ is different from $\kz$, and the two processes are not bisimilar.



\note{aggiungere open}


In \cite{fengSymbolicBisimulationQuantum2014} the authors propose a symbolic semantic for qCCS, that abstracts away from the actual quantum state in the initial configuration. Instead of probabilistic distribution, the concept of \textit{superoperator-valued distribution} is used, i.e. functions that associate to each process a trace non-increasing superoperator. A configuration  representing both the probability that the process is reached, and the transformation that were applied to the quantum state to reach that process.

\subsection{Mixed Configuration CQP}


In \cite{davidsonFormalVerificationTechniques2012}, the author provides a labelled semantic and a bisimilarity notion of CQP, but that bisimilarity is not a congruence with respect to parallel composition, as seen in section \ref{lalire_bisimulation}. In \cite{davidsonFormalVerificationTechniques2012}, the author defines the mixed configuration semantic for CQP, and on that system describes a bisimilarity that is also a congruence. This bisimilarity, similar to the one in QPAlg, is a probabilistic branching bisimilarity, that observes the quantum values only when they are sent. Differently from the previous proposal, it also requires the equality of the "environment" of a configuration, like in qCCS.


Recall that in this semantics a configuration that performs a quantum  measurement evolves in a "mixed configuration", i.e. a sort of probabilistic distribution of configurations, in which the probabilities are not immediately observable.  


Both mixed configuration and distributions of configurations describe all the various outcomes with their relative probabilities. The difference is that a distribution exhibits immediately this probabilistic behaviour, choosing only  only one of the possible configuration with a probabilistic transition. A mixed configuration, instead, can perform all the usual labelled action, remaining in the state of a mixed configuration, without picking a single outcome and dropping the others.


Reguarding bisimilairity, the consequence is that the configuration $\conf$, bisimilar to the configuration $\conf'$, must exhibit the same the probabilistic moves of $\conf'$ only when the outcome is sent, and not immediately after the measurement.

Also the environment of $\conf$ must be equal to the environment of $\conf'$ only when $\conf$ and $\conf'$ perform an output transition. Note that, since mixed configurations must be treated as a single state, the notion of environment must be extended accordingly. If $\confw{\widetilde{q} = \rho, P} \psum{p} \confw{\widetilde{q} = \sigma, Q}$ is a mixed configuration, the environment is defined as:
\[env(\confw{\widetilde{q} = \rho, P} \psum{p} \confw{\widetilde{q} = \sigma, Q}) = p*env(\confw{\widetilde{q} = \rho, P}) + (1-p)*env(\confw{\widetilde{q} = \sigma, Q})\]

and the environment of a configuration is defined as the reduced density operator of the qubits not used owned by the process.

This bisimulation is coarser than the one in qCCS, and arguably captures more appropriately the expected behevioural equivalence of quantum processes. For this bisimulation, for example, it holds that 
\begin{align*}
 P &= M_{01}[q_1].\nil &\sim & & Q &= M_\pm[q_1].\nil \\ 
 P \parallel c!q_2 &=  M_{01}[q_1].\nil \parallel c!q_2 &\sim & & Q \parallel c!q_2 &=  M_\pm[q_1].\nil \parallel c!q_2 
\end{align*} 

In fact, since the qubit $q_1$ is lost, an external observer can rely only on $q_2$ to distinguish $P$ and $Q$, which is not possible because of quantum properties of mixed states. 

As a drawback, this result is obtained through a considerably not-standard approach, diverging from the usual rules of probabilistic calculi. Recovering this result in the usual framework of pLTS remains an interesting open problem.




%\note{note:}
%\note{\textbf{Inglesi}}
%\begin{itemize}
%\item \textbf{QPALg2004} configurazioni e probabilismo. complex variablescoping, with a stack in the configuration, sending shrinks rho. 
%
%\item \textbf{QPAlg2006} probabilistic branching bisimilarity, non congruenza perchè entabglement e larsen skou. sending does not shrink rho, reception enlarges rho, but syncronization doesn't use any of the two rules. 
%\[ \langle q = \rho, c?x.P \rangle \xrightarrow{c?x} \langle x,q = \nu\otimes\rho, c?x.P \rangle
%\]
%for any $\nu$, infinite branching
%
%
%\item \textbf{CQP gay nagarajan popl 05}: pi-calculus like, measurements are expressions, no probabilistic sum (can be implemented with parallel syncronization), reduction semantic con congruenza, typesystem affine per garantire il no cloning. probability-on-transitions approach: a reduction relation $\rightarrow \subseteq S \times \distr(S)$ and a probabilistic choice transition $\rightsquigarrow \subseteq \distr(S)\times [0, 1] \times S$. Configurations of the form (quantum state, channel names, P).
%
%\item \textbf{Thesis Davidson 2011}:
%
%Gives a labelled transition semantic $\langle \sigma, \omega, P\rangle \xrightarrow{alpha} \distr(\sigma, \omega, P)$ and a probabilistic transition  $\rightsquigarrow$ as before, where $\sigma$ contains the quantum state, $\omega$ the used qubits, and $P$ the process. Quantum input doesn't extend rho (here called sigma).
%
%semantics: out removes $q$ from $\omega$, in and qbit add $q$ to $\omega$ 
%typing: measure and ops don't add $q$ to $\Sigma$, but expression does. out removes, qbit adds, in should add. 
%Sigma is a subset of omega
%
%THe chinese approach equates the quantum names, and require the same final state.
%the french-english approach doesn't equates the quantum names, but the (partial trace) of the state in the moment of communication.
%Our congurence doesn't equates quantum names, it could if we add a more specific barb $\downarrow_{c!q}$, or a name-matching construct in our contexts. 
%Ora come ora, nel nostro sistema, 
%\[ P = H(q_1, q_2).c!q_1 \parallel d!q_2 \quad H(q_1, q_2).d!q_1 \parallel c!q_2\]
%sono bisimili.
%
%The example 3.2 in page 74 shows two processes that are bisimilar but not congruent. There are to solution to this problem: provide a finer bisimilation, that distinguishes P and Q, confronting the enviroment ($tr_\Sigma(\rho)$) in a larsen skou way, or a coarser bisimulation, that doen't distinguis C[P] and C[Q], confronting the environment of distribution (called in davidson mixed configuration.)
%
%
%\item \textbf{thesis Puthoor 2015}:
%provides a correct set of equational axioms, to define behavioural equivalence axiomatically. Extends CQP to Linear optical quantum computing
%\end{itemize}
%
%
%\note{\textbf{Cinesi}}
%\begin{itemize}
%\item Feng duan 2006, probabilistic bisimulation for quantum:
%
%Probabilities-on-state approach, strong and week bisimilarity, deadlock quantum state equivalence, different inputs rules for correlated and uncorrelated qubits. Uncorrelated input extends rho. Introduces conbined transitions, i.e. convex closure transitions. bisimilarities not preserved by parallel composition, and restriction 
%P = U1[q].c!0.U2[q].nil, Q = V1[q].c!0.V2[q].nil. are bisimilar, but not $P\setminus c$ and $Q\setminus c$
%\item Ying feng 2009, an algebra of quantum, no classical comunxication. Input and output don't change rho. superoperators as visible transitions, reduction (i.e. independent superoperators) bisimilarity, approximate bisimulation based on diamond distance between superoperators
%
%\item Feng duan ying bisimulation for quantum, is a congruence, requires equality of the environment, not of the total state. 
%\item Open bisimulation for quantum
%\end{itemize}



\chapter{Linear qCCS}\label{chapter4}

We present a new quantum process calculus, designed to resolve the ambiguities and differences present in the other proposals. This allows to investigate more deeply the different notions of behavioural equivalence of quantum systems. 
This novel calculus goes under the name of Linear qCCS (lqCCS), as it is an asynchronous version of qCCS, but employs a linear type system, similar to the affine type system for CQP.

After introducing lqCCS and its type system, we overview some desired properties of behavioural equivalence, in the light of which we investigates two alternative notions of bisimilarity. More in detail, we see how the type system disambiguates between two alternative notions of observability of quantum states. We show that some distinctive features of different approaches coincides when such ambiguity is removed. We define then a standard probabilistic bisimulation a là Larsen-Skou, in the spirit of \cite{dengOpenBisimulationQuantum2012}, and we give some properties that help proving it. 

Then we show some limitations of the fully-probabilistic approach, proving it is too fine-graned for the quantum case. Thus, we derive an updated versione of bisimilarity that solves this issue while maintaining the previous properties. Finally, we compare our proposal with related works.

\section{The Language}
\subsection{Syntax}

\begin{align*}
  P \Coloneqq &\ K \mid c!e 
%  \mid discard(e) 
  \mid \bigparallel_{i \in I} P_i \\[0.1cm]
  K \Coloneqq &\ nil \mid \tau . P \mid \mathcal{E}(\widetilde{x}) . P \mid M(\widetilde{x} \rhd y) . P \mid c?x . P \mid \\[0.1cm]
              &\ [e] P \mid \sum_{i \in I} K_i \mid P \{f\} \mid P \setminus L \mid A(\widetilde{x}) \\[0.1cm]
  e \Coloneqq &\ x \mid b \mid n \mid q \mid \neg e \mid e \lor e \mid e \leq e
\end{align*}	
where $b \in \mathbb{B}$, $n \in \mathbb{N}$, $x \in \text{Var}$, $q \in \text{QC}$, with $Var$ a denumerable set of variable names, and
QC a set of names with cardinality equal to the size of the chosen Hilbert space.

We use $discard(e)$ as syntactic sugar for $c!e \setminus c$.

\subsection{Type System}

Variables Types: $\set{\mathcal{Q}, \mathbb{N}, \mathbb{B}}$
Channel types: $\set{\hat{\mathcal{Q}}, \hat{\mathbb{N}}, \hat{\mathbb{B}}}$

\begin{gather*}
\infer[\mbox{\footnotesize\scshape CBool}]{\vdash b : \mathbb{B}}{b \in \mathbb{B}} \qquad
\infer[\mbox{\footnotesize\scshape CNat}]{\vdash n : \mathbb{N}}{n \in \mathbb{N}} \qquad
\infer[\mbox{\footnotesize\scshape QVar}]{\set{x} \vdash x}{} \qquad
\infer[\mbox{\footnotesize\scshape CVar}]{x : T \vdash x : T}{} \\[0.2cm]
\infer[\mbox{\footnotesize\scshape BoolOr}]{\Gamma_1 \cup \Gamma_2 \vdash e_1 \lor e_2 : \mathbb{B}}{\Gamma_1 \vdash e_1 : \mathbb{B} & \Gamma_2 \vdash e_2 : \mathbb{B}} \qquad
\infer[\mbox{\footnotesize\scshape BoolNeg}]{\Gamma \vdash \neg e : \mathbb{B}}{\Gamma \vdash e : \mathbb{B}} \qquad
\infer[\mbox{\footnotesize\scshape NatLEq}]{\Gamma_1 \cup \Gamma_2 \vdash e_1 \leq e_2 : \mathbb{B}}{\Gamma_1 \vdash e_1 : \mathbb{N} & \Gamma_2 \vdash e_2 : \mathbb{N}} \\[0.2cm]
 \infer[\mbox{\footnotesize\scshape CWeak}]{\Gamma, x : T; \Sigma \vdash P}{\Gamma; \Sigma \vdash P & \text{$x$ fresh}} \\[0.2cm]
% \infer[\mbox{\footnotesize\scshape QWeak}]{\Gamma; \Sigma, x \vdash P}{\Gamma; \Sigma \vdash P} \\[0.2cm]
\infer[\mbox{\footnotesize\scshape Nil}]{\emptyset; \emptyset \vdash nil}{} \qquad
\infer[\mbox{\footnotesize\scshape Tau}]{\Gamma; \Sigma \vdash \tau . P}{ \Gamma; \Sigma \vdash P} \qquad
  \infer[\mbox{\footnotesize\scshape ITE}]{\Gamma_1 \cup \Gamma_2 \cup \Gamma_3; \Sigma \vdash \ite{e}{P_1}{P_2}}{\Gamma_3 \vdash e : \mathbb{B} & \Gamma_2; \Sigma \vdash P_1 & \Gamma_2; \Sigma \vdash P_2} \\[0.2cm]
\infer[\mbox{\footnotesize\scshape QOp}]
{\Gamma; \Sigma \vdash \mathcal{E}(\widetilde{x}) . P}
{\mathcal{E}: Op(n) & |\widetilde{x}| = n & \forall i, j \ldotp x_i \neq x_j & \Sigma \vdash \widetilde{x} & \Gamma; \Sigma \vdash P} \\[0.2cm]
\infer[\mbox{\footnotesize\scshape QMeas}]{\Gamma; \Sigma \vdash M(\widetilde{x} \rhd y) . P}
{\forall i, j \ldotp x_i \neq x_j & \Sigma \vdash \widetilde{x} & \Gamma, y : \mathbb{N}; \Sigma \vdash P} \\[0.2cm]
\infer[\mbox{\footnotesize\scshape CRecv}]
{\Gamma, c : \hat{T}; \Sigma \vdash c?x . P}
{\Gamma, x : T; \Sigma \vdash P} 
\qquad
\infer[\mbox{\footnotesize\scshape QRecv}]
{\Gamma, c : \hat{\mathcal{Q}}; \Sigma \vdash c?x . P}
{\Gamma; \Sigma, x \vdash P} 
\\[0.2cm]
\infer[\mbox{\footnotesize\scshape CSend}]{\Gamma, c : \hat{T}; \emptyset \vdash c!e}{\Gamma \vdash e : T} \qquad
\infer[\mbox{\footnotesize\scshape QSend}]{c : \hat{\mathcal{Q}}; \set{e} \vdash c!e}{} \\[0.2cm]
%\infer[\mbox{\footnotesize\scshape QDiscard}]{e:\mathcal{Q} ; \set{e} \vdash discard(e)}{} \\[0.2cm]
\infer[\mbox{\footnotesize\scshape Sum}]{\Gamma_1 \cup \Gamma_2; \Sigma \vdash P_1 + P_2}{\Gamma_1;\Sigma \vdash P_1 & \Gamma_2;\Sigma \vdash P_2} \qquad 
\infer[\mbox{\footnotesize\scshape Par}]{\Gamma_1 \cup \Gamma_2; \Sigma_1 \cup \Sigma_2 \vdash P_1 \parallel P_2}
  {\Sigma_1 \cap \Sigma_2 = \emptyset & \Gamma_1; \Sigma_1 \vdash P_1 & \Gamma_2; \Sigma_2 \vdash P_2} \\[0.2cm]
\infer[\mbox{\footnotesize\scshape Rename}]{\Gamma; \Sigma \vdash P \{f\}}{f(\Gamma); \Sigma \vdash f(P)} \qquad
\infer[\mbox{\footnotesize\scshape Restrict}]{\Gamma; \Sigma \vdash P \setminus L}{\Gamma; \Sigma \vdash P} \\[0.3cm]
\end{gather*}

{\color{red} 
Togliere Qvar da Gamma

Rimuovere fresh da recv

Aggiornare il proofs.
}

\begin{definition}
  Let $\rho$ and $P$ be an arbitrary partial density operator and a process. $\Gamma; \Sigma \vdash \langle \rho, P \rangle$ iff $\Gamma; \Sigma \vdash P$.

  Let $I$ be an arbitrary index set. $\Gamma; \Sigma \vdash \boxplus_{i \in I} \langle \rho_i, P_i \rangle$ iff for each $i \in I$ such that $\rho_i \neq \mathbf{0}$,
  then $\Gamma; \Sigma \vdash \langle \rho_i, P_i \rangle$.
\end{definition}

\subsection*{Contexts}
A typed context $B$ is generated by the grammar
\[
B \Coloneqq \ [\_]_{\Gamma; \Sigma} \mid B \parallel P
\]
according to the typing rules:

\[
  \infer[\mbox{\footnotesize\scshape Hole}]{\Gamma; \Sigma \vdash [\blank]_{\Gamma; \Sigma}}{}
\qquad
\infer[\mbox{\footnotesize\scshape ParHole}]{\Gamma_1 \cup \Gamma_2; \Sigma_1 \cup \Sigma_2 \vdash B \parallel P}
{\Gamma_1; \Sigma_1 \vdash B & \Gamma_2; \Sigma_2 \vdash P & \Sigma_1 \cap \Sigma_2 = \emptyset}
\]


\begin{theorem}
	Given any context $B[\blank]$ of type $\Gamma; \Sigma$, with hole of type $\Gamma'; \Sigma'$, and any process $P$ s.t. $\Gamma'; \Sigma' \vdash P$, it holds that $\Gamma; \Sigma \vdash B[P]$ 
\end{theorem}
\begin{proof}
	Trivial.
\end{proof}

\section{Semantics}
We present a reduction semantics for lqCCS, consistent with the labelled semantic for qCCS presented in \cite{fengBisimulationQuantumProcesses2012, dengOpenBisimulationQuantum2012}. A reduction semantic does not make any assumption on the observable properties of the system (like the labels of a transition), and so in better suited to explore and compare different notion of behavioural equivalence.

Besides, as explained in chapter \ref{chapter3}, in a labelled transition system made of configurations, a quantum input transition like $\xrightarrow{c?q}$ requires that the value of qubit $q$ is already present inside the transition. This is an atypical assumption, as a labeled transition usually model the communication with an unknown external environment, but in this case requires at least some partial knowledge of the environment. In a reduction system there are no such labelled transitions, so a process communicates only with other process, on which we have total information.

Our semantics defines a probabilistic reduction system $\langle S, \rightarrow \rangle$, where \begin{itemize}
\item $S$ is a set of \textit{configurations}, of the form $\confw{\rho, P}$ (like in qCCS).
\item $\rightarrow \subseteq S \times \distr(S)$ is the probabilistic transition relation, corresponding to the $\xrightarrow{\tau}$ transition in qCCS \cite{fengBisimulationQuantumProcesses2012, dengOpenBisimulationQuantum2012}.
\end{itemize}

We assume a fixed set  $QN = {q_1, q_2, \ldots q_n}$ of quantum names, where each name $q_i$ refers to a unique qubit with state space $\calH_i$. We denote as $\calH_{QN}$ the $2^n$-dimensional Hilbert space $\bigoplus_{i=1}^n \calH_i$.

We assume also a fixed typing contest $\Gamma_c = \set{c_i: \widehat{T_i}}_i$, containing typing assumptions for a finite set of classical and quantum channels.

\begin{definition}
Let $P$ be a process and $\rho \in \mathcal{D}(\calH_{QN})$ an arbitrary density operator. We say that a configuration $\confw{\rho, P}$ is well typed, given a set of quantum names $QN$ and a set of typed channels $\Gamma_c$, if $\Gamma_c; \Sigma \vdash P$ and $\Sigma \subseteq QN$.

%  Let $I$ be an arbitrary index set. $\Gamma; \Sigma \vdash \boxplus_{i \in I} \langle \rho_i, P_i \rangle$ iff for each $i \in I$ such that $\rho_i \neq \mathbf{0}$, then $\Gamma; \Sigma \vdash \langle \rho_i, P_i \rangle$.
\end{definition}

Notice that the context $\Gamma_c$ contains assignments only for channels, not for classical variables. This means that in well typed configuration, since $\Gamma_c; \Sigma \vdash P$, $P$ does not contain any free classical variable, and all the free quantum variables are references to qubits in the configuration.

From now on, we will consider only well typed configurations.

\subsection{Reduction System}
In order to define the reduction transition, we first need to introduce a semantic for expressions and a structural congruence relation on processes, like in \cite{gayCommunicatingQuantumProcesses2005}.

We consider as a \textit{value} any expressions $n \in \mathbb{N},\  b \in \mathbb{B},\  v \in  \text{Var}$, and use $v$ as a metavariable for them.

In figure \ref{big_step_exp}, we define a big step semantic for classical and quantum expression in the usual way. We write $e \Downarrow v$ to indicate that the expression $e$ evaluates to value $v$. Recall that the only quantum expression admitted by the type system are quantum variables.

\begin{figure}[h!]
\begin{gather*}
\infer[\mbox{\footnotesize\scshape Var}]{x \Downarrow x}{} \qquad
    \infer[\mbox{\footnotesize\scshape Nat}]{n \Downarrow n}{} \qquad
    \infer[\mbox{\footnotesize\scshape Bool}]{b \Downarrow b}{}  \qquad
\\[0.3cm]
    \infer[\mbox{\footnotesize\scshape Or}]{(e_1 \lor e_2) \Downarrow b}{e_1 \Downarrow b_1 & e_2 \Downarrow b_2 & b = b_1 \lor b_2} \qquad
    \infer[\mbox{\footnotesize\scshape Neg}]{\neg e \Downarrow b}{e \Downarrow b_1 & b = \neg b_1} \\[0.3cm]
    \infer[\mbox{\footnotesize\scshape Leq}]{(e_1 \leq e_2) \Downarrow b}{e_1 \Downarrow n_1 & e_2 \Downarrow n_2 & b = n_1 \leq n_2}
\end{gather*}
\caption{Big step semantic for Linear qCCS expressions}
\label{big_step_exp}
\end{figure}

We define as the \textit{structural congruence relation} $\equiv$ the smallest equivalence relation that satisfies the axioms in figure \ref{str_cong}. We emplyed the usual axioms of \cite{milnerFunctionsProcesses1990} for parallel composition, summation and reduction. We also add axioms to evaluate classical expressions and resolve \textbf{If Then Else} constructs.

\begin{figure}[h!]
\begin{gather*}
    \infer[\mbox{\footnotesize\scshape ParNil}]{P \parallel nil \equiv P}{} \qquad
    \infer[\mbox{\footnotesize\scshape ParComm}]{P \parallel Q \equiv Q \parallel P}{} \qquad
    \infer[\mbox{\footnotesize\scshape ParAssoc}]{P \parallel (Q \parallel R) \equiv (P \parallel Q) \parallel R}{} 
    \\[0.3cm]
    \infer[\mbox{\footnotesize\scshape SumNil}]{M + nil \equiv M}{} \qquad
    \infer[\mbox{\footnotesize\scshape SumComm}]{M + N \equiv N + M}{} \qquad
    \infer[\mbox{\footnotesize\scshape SumAssoc}]{M + (N + O) \equiv (M + N) + O}{} 
    \\[0.3cm]
    \infer[\mbox{\footnotesize\scshape RestrOrd}]{P \setminus c \setminus d \equiv P \setminus d \setminus c}{} \qquad 
    \infer[\mbox{\footnotesize\scshape RestrNil}]{nil \setminus c \equiv nil}{} \qquad 
    \infer[\mbox{\footnotesize\scshape RestrPar}]{(P \parallel Q) \setminus c \equiv P \parallel (Q \setminus c)}{c \not\in fc(P)} \\[0.3cm]
    \infer[\mbox{\footnotesize\scshape TrueGuard}]{\ITE{tt}{P}{Q} \equiv P}{} \qquad
    \infer[\mbox{\footnotesize\scshape FalseGuard}]{\ITE{ff}{P}{Q} \equiv Q}{} \\[0.3cm]
    \infer[\mbox{\footnotesize\scshape ValExpr}]{ P \equiv P[\sfrac{v}{e}] }{e \Downarrow v} 
	\end{gather*}	
\caption{Structural congruence for Linear qCCS}
\label{str_cong}
\end{figure}

We can now define the transition relation $\rightarrow$, presented in figure \ref{reduction}. As usual, we write $\confw{\rho, P} \rightarrow \Delta$ to intend $(\confw{\rho, P}, \Delta) \in \rightarrow$. To lighten the notation, we will also write $\confw{\rho, P} \rightarrow \confw{\rho', P'}$ instead of $\confw{\rho, P} \rightarrow \overline{\confw{\rho', P'}}$.

This transition relation can not be composed with itself, since it is a relation from configurations to distributions. As in \cite{fengBisimulationQuantumProcesses2012}, we can \textit{lift} the relation $\rightarrow$ into $\slift{\rightarrow}$, that is a relation from distributions to distributions, as described in section \ref{pLTS}. This is useful to talk about reachability and temporal logics, but will not be used in the definitions of strong bisimulations in the next sections.

\begin{figure}[h!]
  \begin{gather*}
    \infer[\mbox{\footnotesize\scshape SemTau}]{\langle \rho, \tau . P \rangle \longrightarrow \langle \rho, P \rangle}{} \\[0.3cm]
    \infer[\mbox{\footnotesize\scshape SemRename}]{\langle \rho, P \{f\} \rangle \longrightarrow \langle \rho', P' \{f\}\rangle}{\langle \rho, f(P) \rangle \longrightarrow \langle \rho', f(P') \rangle} \qquad
    \infer[\mbox{\footnotesize\scshape SemRestrict}]{\langle \rho, P \setminus L \rangle \longrightarrow \langle \rho, P' \setminus L \rangle}{\langle \rho, P \rangle \longrightarrow \langle \rho', P' \rangle} \\[0.3cm]
    \infer[\mbox{\footnotesize\scshape SemQOp}]{\langle \rho, \mathcal{E}(\widetilde{x}) . P \rangle \longrightarrow \langle \mathcal{E}_{\widetilde{x}}(\rho), P \rangle}{} \\[0.3cm]
    \infer[\mbox{\footnotesize\scshape SemQMeas}]{\langle \rho, M(\widetilde{x} \rhd y) . P \rangle \longrightarrow \boxplus_{m = 0}^{2^{|\widetilde{x}|}} \left\langle {M_m \rho M_m^\dag}, P[\sfrac{m}{y}] \right\rangle}{} \\[0.3cm]
    \infer[\mbox{\footnotesize\scshape SemPar}]{\langle \rho, P \parallel R \rangle \longrightarrow \langle \rho', P' \parallel R \rangle}{\langle \rho, P \rangle \longrightarrow \langle \rho', P' \rangle} \qquad
    \infer[\mbox{\footnotesize\scshape SemSum}]{\langle \rho, P + R \rangle \longrightarrow \langle \rho', P' \rangle}{\langle \rho, P \rangle \longrightarrow \langle \rho', P' \rangle} \\[0.3cm]
    \infer[\mbox{\footnotesize\scshape SemReduce}]{\langle \rho, c!v \parallel c?x . P \rangle \longrightarrow \langle \rho, P[\sfrac{v}{x}] \rangle}{} \\[0.3cm]
    \infer[\mbox{\footnotesize\scshape SemCongr}]{\langle \rho, P \rangle \longrightarrow \langle \rho', P' \rangle}
    {P \equiv Q & \confw{\rho, Q} \rightarrow \confw{\rho', Q'} & Q' \equiv P'}
  \end{gather*}
\caption{Reduction system for Linear qCCS}
\label{reduction}
\end{figure}
  
\subsection{Examples}
  \paragraph{Quantum Teleportation}

\begin{align*}
  \proc{A} &\Coloneqq \text{in}_a?x.\text{CNOT}(q_0, x).\text{H}(q_0).M(x,q_0 \rhd n).(\text{m}_a!n \parallel discard(q_0) \parallel discard(x) )\\
  % \proc{B} &\Coloneqq \text{in}_b?x.\text{m}_a?n.\left(\sum_{i = 0}^3 [n = i]\sigma_i(x).\text{out}_b!x\right) \\
  \proc{B} &\Coloneqq \text{in}_b?x.\text{m}_a?n.
     \\ & \ite{n = 0}{\sigma_0(x).\text{out}_b!x\\&\quad}
    {\ite{n = 1}{\sigma_1(x).\text{out}_b!x\\&\qquad}
    		{\ite{n = 2}{\sigma_2(x).\text{out}_b!x}{\sigma_3(x).\text{out}_b!x}}
    }
\\
  \proc{S} &\Coloneqq \text{H}(q_1).\text{CNOT}(q_1, q_2).(\text{in}_a!q_1 \parallel \text{in}_b!q_2) \\
  \proc{Tel} &\Coloneqq (A \parallel B \parallel S) \setminus \Set{\text{in}_a, \text{in}_b, \text{m}_a } \\
  \proc{TelSpec} &\Coloneqq \text{SWAP}(q_0,q_2).(\text{out}_b!q_2 \parallel discard(q_0) \parallel discard(q_1))
\end{align*}
$\Gamma = \set{\text{in}_a : \hat{\mathcal{Q}}, \text{in}_b : \hat{\mathcal{Q}}, \text{m}_a : \hat{\mathbb{N}}, \text{out}_b : \hat{\mathcal{Q}}}$, $\Sigma = \set{q_0, q_1, q_2}$.
\subsection{Type system properties}

We now prove that reductions preserves typing. This is critical for the further development of this thesis and is a standard property. Note that the absence of a weakening rule for quantum variables requires a different approach w.r.t. the case of CQP \cite{gayCommunicatingQuantumProcesses2005}.

We need some auxiliary results, starting from typing preservations for expression evaluation and structural congruence.
 
\begin{theorem}[Evaluation Preserves Typing]\label{thm:eval_typing}
  If $\Gamma \vdash e$ and $e \Downarrow v$, then $\Gamma \vdash v$.
\end{theorem}
\begin{proof}
  Follows by induction on the evaluation rules.
\end{proof}

\begin{theorem}[Structural Congruence Preserves Typing]
  If $\Gamma; \Sigma \vdash P$ and $P \equiv Q$, then $\Gamma; \Sigma \vdash Q$.
\end{theorem}
\begin{proof}
  By induction on the derivation of $P \equiv Q$. All rules follow trivially with the possible need of additional \textsc{CWeak} applications in the derivation
  of $\Gamma; \Sigma \vdash Q$ as the subcomponents of the if-then-else, non-deterministic sum and parallel operator may be typed by a context $\Gamma' \subseteq \Gamma$.
\end{proof}

Others technical results we need are substitution lemmas for expressions, classical and quantum substitutions.

\begin{lemma}[Expression Substitution]
  Let $\Gamma, x : T' \vdash e : T$ and let $\Gamma \vdash v : T'$, then $\Gamma \vdash e[\sfrac{v}{x}]$.
\end{lemma}
\begin{proof}
  Trivially by structural induction on the derivation of $\Gamma, x : T' \vdash e$.
\end{proof}

\begin{theorem}[Classical Substitution]
  Let $\Gamma, x : T; \Sigma \vdash P$ and let $\Gamma \vdash v : T$, then $\Gamma; \Sigma \vdash P[\sfrac{v}{x}]$.
\end{theorem}
\begin{proof}
  By structural induction on the derivation of $\Gamma, x : T; \Sigma \vdash P$.
\end{proof}

\begin{theorem}[Quantum Substitution]
  Let $\Gamma; \Sigma, x \vdash P$ and let $v \not\in \Sigma$, then $\Gamma; \Sigma, v \vdash P[\sfrac{v}{x}]$.
\end{theorem}
\begin{proof}
  By structural induction on the derivation of $\Gamma; \Sigma, x \vdash P$.
  Let us analyze the interesting cases: For the \textsc{QOp} rule it must be that $P = \mathcal{E}(\widetilde{x}).Q$ for some process $Q$.
  By induction hypothesis it holds that $\Gamma; \Sigma, v \vdash Q$, but $v \not\in \widetilde{x}$ by the hypothesis $v \not\in \Sigma$, thus we can apply the \textsc{QOp} rule.
  The same line of reasoning is valid for the \textsc{QMeas} rule.
  The \textsc{QSend} rule is also guaranteed by the $v \not\in \Sigma$ requirement.
  Finally, the derivation of \textsc{Par} imposes that only one of the components, w.l.o.g. $P_i$, contains the variable $x$ in its quantum environment $\Sigma_i$.
  Thus by induction we obtain $\Gamma_i; \Sigma_i[\sfrac{v}{x}] \vdash P_i$. While, for the other components there is no change since they do not contain the variable $x$.
  However, since $v \not\in \Sigma$, we can conclude that all smaller environments are still pairwise distinct, thus we can apply the \textsc{Par} rule again.
\end{proof}

We now prove or main result, namely that reductions preserve typing.

\begin{theorem}[Typing Preservation]
  If $\Gamma; \Sigma \vdash P$ and $\langle \rho, P \rangle \longrightarrow \sum_{i \in I} p_i \langle \rho_i, P_i \rangle$ then
  $\forall i \in I \ldotp \Gamma; \Sigma \vdash P_i$.
\end{theorem}
\begin{proof}
  By structural induction on the transition relation $\longrightarrow$.
  Let us analyze the interesting cases: if the last step in the derivation is a \textsc{SemQMeas} rule, then $P = M(\widetilde{x} \rhd y).Q$ for some process $Q$,
  where $Q$ is typed with $\Gamma, m : \mathbb{N}; \Sigma \vdash Q$. Each component of the box sum is of the form $Q[\sfrac{m}{y}]$ with $m \in \mathbb{N}$, thus
  by the classical substitution theorem it holds that $\Gamma; \Sigma \vdash Q[\sfrac{m}{y}]$.
  If the last step is a \textsc{SemPar} rule, then $P = Q \parallel R$ for some processes $Q$ and $R$, where
  $\Gamma_1; \Sigma_1 \vdash Q$ and $\Gamma_2; \Sigma_2 \vdash R$ with $\Gamma = \Gamma_1 \cup \Gamma_2$, $\Sigma = \Sigma_1 \cup \Sigma_2$ and $\Sigma_1 \cap \Sigma_2 = \emptyset$.
  By induction $\Gamma_1; \Sigma_1 \vdash Q'$, however since the conditions on the $\Sigma$ are still true, it also holds that $\Gamma; \Sigma \vdash Q' \parallel R$, by applying \textsc{CWeak} if $\Gamma_1 \subset \Gamma$.
  The argument is similar for the \textsc{SemSum} rule.
  If the last step is a \textsc{SemReduce} rule, then $P = c!e \parallel c?x.Q$ for some process $Q$. If $c : \hat{T}$ where $T = \mathbb{N}$ or $T = \mathbb{B}$, then
  the theorem holds trivially by the classical substitution theorem.
  If $c : \hat{\mathcal{Q}}$ then it must be that $c : \hat{\mathcal{Q}}; \set{e} \vdash c!e$ and $\Gamma', c : \hat{\mathcal{Q}}; \Sigma' \vdash c?x.Q$ thus $\Gamma'; \Sigma', x \vdash Q$,
  with $e \not\in \Sigma'$ and $x \not\in \Sigma'$ respectively by the \textsc{Par} and \textsc{QRecv} rules. By the quantum substitution theorem, $\Gamma'; \Sigma', e \vdash Q[\sfrac{e}{x}]$,
  but since $\Sigma', e = \Sigma$ and by application of the \textsc{CWeak} it holds that $\Gamma; \Sigma \vdash Q[\sfrac{e}{x}]$.
\end{proof}




\section{Behavioural Equivalence}\label{bisimulation_chapter4}
We have seen that a number of different behavioural equivalences have been proposed in the literature. Noticeably, different works give different concepts of "observable properties" of a quantum process. As in classical process algebra, it is always possible to observe if a process $P$ can perform a classical input or output action, which is the foundation of \textit{labeled} bisimulation or \textit{barbed} bisimulation. The discrepancies in state-of-the-art proposals are mainly about the observable properties of quantum states. In QPAlg, the quantum state is observed only when sent, while in qCCS, the whole "environment" is visible, i.e. all the qubits that are not used by $P$ anymore. In (mixed configuration) CQP there is a again a similar notion of "environment", but what is observable is the environment of the whole distribution, not the environment of the single configurations.

The main purpose of this section is to investigate which are the most natural notions of behavioural equivalence, and how some apparently minor details lead to completely diverse equivalence relations. Being interested in the purely quantum aspect of behavioural equivalence, we will only consider "strong" relations, like strong bisimilarity and strong barbed congruence. The difference between strong, weak and branching relations is in fact a well know "classical" aspect, that simply makes the comparison more complex.

We will first define a \textit{probabilistic saturated bisimilarity} \cite{bonchiGeneralTheoryBarbs2014} for Linear qCCS. Saturated bisimilarity is a solid and general notion of observable equivalence, aiming to capture when two processes can or can not be distinguished by an external observer, i.e. by an arbitraty context. We will describe some example and some properties of such a bisimilarity, showing that our notion of probabilistic saturated bisimilarity for lqCCS is consistent with the open bisimilarity  \cite{dengOpenBisimulationQuantum2012} for qCCS.

Then we will analyze some peculiarities of these probabilistic equivalence notions, extending the ideas already discussed in \cite{davidsonFormalVerificationTechniques2012}. In particular, the bisimilarities defined for QPAlg and qCCS and lqCCS, seem to grant the external observer a greater discerning power then what is prescribed by quantum mechanics.
We identify the well established notion of Larsen-Skou bisimulation (as described in \ref{pLTS}) as the cause of these undesired behaviours, and propose a novel notion of \textit{quantum saturated bisimilarity} that represents more correctly the observable properties of distribution of quantum configurations. We believe that this bisimilarity complies with the ideas presented by \cite{davidsonFormalVerificationTechniques2012} for mixed configuration CQP, albeit in a essentially different transition system.
\subsection{Probabilistic Saturated Bisimilarity}

Saturated bisimilarity \cite{bonchiGeneralTheoryBarbs2014} grants the external observer, when comparing two configurations $\conf$ and $\conf'$, the power to put $\conf$ and $\conf'$ inside any context at each step of the computations. Recall that in barbed congruence (as defined in section \ref{bkg_reduction_system}), the observer can compare $\conf$ and $\conf'$ using just one arbitrary context at the start of the computation, not any context at each step like in saturated bisimilarity.

We use (probabilistic) saturated bisimilarity as it is an established, general behavioural equivalence, useful to investigate reduction system where there is no affirmed notion of observable property. Besides, as we will see, saturated bisimilarity is able to capture some properties used in \cite{dengOpenBisimulationQuantum2012} to define open bisimilarity.


We define "the capability of performing an output on a channel" as the only barbs, i.e. the atomic observable properties. We do not assume input actions to be observable, following the standard for asynchronous calculi \cite{amadioBisimulationsAsynchronousPcalculus1998}. Since the external observer is asynchronous, it can not observe the process $P$ to discorver if a $c?x$ transition is available: the context can only perform $c!v$ as a terminal action, hence it cannot discover if  the message is ever received.

\begin{definition}[Barb]
	A \emph{barb} is a predicate $\downarrow_{c}$ over well typed processes, defined as follows: $P \downarrow_{c}$ if and only if $P \equiv c!e \parallel R$ for some expression $e$, and processes $R$.
\end{definition}

%\begin{definition}[Barb on configuration]
%	A \emph{barb} is a predicate $\downarrow_{c}$ over configurations, defined as follows: $\langle \rho, P \rangle \downarrow_{c}$ iff $P \equiv c!x + Q \parallel R$ for some $x$, and processes $Q$, $R$.
%\end{definition}

If $\conf = \confw{\rho, P}$ is a configuration, we will often write $C\downarrow_c$ instead of $P\downarrow_c$, and $B[\conf]$ instead of $\confw{\rho, B[P]}$.

\begin{definition}[Probabilistic Saturated Bisimilarity]
	A symmetric relation $\rel \subseteq \conf \times \conf$ is \emph{probabilistic saturated (barbed) bisimulation} if $\confw{\rho, P} \ \rel\ \confw{\sigma, Q}$ implies that $P$ and $Q$  are well-typed under the same typing context $\Gamma; \Sigma$, and for any context $B[\_]_{\Gamma, \Sigma}$
	\begin{itemize}
		\item If $P \downarrow_{c}$ then $Q \downarrow_{c}$
		\item If $\confw{\rho, B[P]} \rightarrow \Delta$, there exists $\Theta$ such that $\confw{\sigma, B[Q]} \rightarrow \Theta$ and $\Delta\ \slift{\rel}\ \Theta$
	\end{itemize}
	Let \emph{probabilistic saturated bisimilarity} $\sim_{PS}$ be the union of all saturated probabilistic barbed bisimulation. \\
	We say that two {processes} $P$ and $Q$ are \emph{bisimilar}, written $P \sim_{PS} Q$, if for any $\rho \in \mathcal{D}(\calH_{QN})$ it holds $\confw{\rho, P} \simps \confw{\rho, Q}$.
\end{definition}

Note that it is not necessary to use a barb $\downarrow_{c!v}$, so to consider also which is the value that is communicated. A context, in fact, is capable to discern $c!v.P$ and $c!v'.P$ thanks to the \textbf{if-then-else} construct for classical values, or to the measure operator for quantum values. This is an innovation with respect to QPAlg, wehere the observables of $\confw{\rho, c!q.P}$ were both the channel and the value (i.e. the reduced density operator) of $q$. 

Note also that $B[\blank]$ has barb $\downarrow_c$ if and only if $B[\blank]$ has it $P$ has it. Hence it is not required to use contexts when comparing  he barbs of two configurations for checking saturated bisimilarity. 


\begin{example}
	Consider the following, wrong definition of quantum teleportation:
	\begin{align*}
		\proc{A} &\Coloneqq \text{in}_a?x.\text{CNOT}(q_0, x).\text{H}(q_0).M(x,q_0 \rhd n).(\text{m}_a!n \parallel out_{a_1}! q_0 \parallel out_{a_2}! x )\\
    \proc{B} &\Coloneqq \text{in}_b?x.\text{m}_a?n.
       \\ & \ite{n = 0}{\sigma_0(x).\text{out}_b!x\\&\quad}
      {\ite{n = 1}{\sigma_1(x).\text{out}_b!x\\&\qquad}
          {\ite{n = 2}{\sigma_2(x).\text{out}_b!x}{\sigma_3(x).\text{out}_b!x}}
      } \\
		\proc{S} &\Coloneqq \text{H}(q_1).\text{CNOT}(q_1, q_2).(\text{in}_a!q_1 \parallel \text{in}_b!q_2) \\
		\proc{Tel} &\Coloneqq (A \parallel B \parallel S) \setminus \Set{\text{in}_a, \text{in}_b, \text{m}_a } \\
		\proc{TelSpec} &\Coloneqq \text{SWAP}(q_0,q_2).(\text{out}_b!q_2 \parallel out_{a_1}! q_0 \parallel out_{a_2}! q_1)
	\end{align*}
	We have that $\proc{Tel} \not\sim_{PS} \proc{TelSpec}$.
	Consider indeed a context:
  \[ B[\blank] = out_{a_1} ? x . M[x \rhd y] . \ite{y = 1}{c!y}{nil}. \]
\end{example}


We show now some useful properties that helps in deciding bisimilarity. The first result is that \textit{$\simps$ is closed for addition of discarded qbits}.  Rougly, this means that if  $\confw{\rho, P}$ and $\confw{\sigma, Q}$ are bisimilar,  we can add a qbit $q$ and its discard $disc(q)$ to them, as in  $\confw{\rho \otimes \proj{\psi}, P \parallel disc(q)}$ and $\confw{\sigma \otimes \proj{\psi}, Q \parallel disc(q)}$, obtaining bisimilar configurations. More in details, this holds also for non separable states, as shown later.

We first need some technical lemmas that follow. 
\newcommand{\discQ}{disc(\widetilde{q})}
\newcommand{\trQ}{tr_{\widetilde{q}}}

\begin{lemma}\label{trace_and_sop}
Let $\sigma \in  \mathcal{D}(\calH_{\widetilde{p}} \otimes \calH_{\widetilde{q}})$, with $\widetilde{p} = p_0 \ldots p_{n-1}$ and $\widetilde{q} = q_0 \ldots q_{m-1}$. Then, for any \textit{trace non-increasing} superoperator $\sop_{\widetilde{p}} \in \mathcal{S}(\calH_{\widetilde{p}})$
\[ \trQ((\sop_{\widetilde{p}} \otimes \mathcal{I}_{\widetilde{q}})(\sigma)) = \sop_{\widetilde{p}}(\trQ(\sigma))
\]
\end{lemma}
\begin{proof}
We know that $\sigma$ is a probabilistic mixture of pure states, $\sigma = \sum_l p_l \proj{\psi_l}$, and each pure state is a linear combination of separable states 
\[\sigma = \sum_l p_l \sum_{p = 0}^{2^n-1} \sum_{q = 0}^{2^m-1} \lambda_{pq} \proj{pq}\]
and we know that $\sop_{\widetilde{p}}$ has a Kraus decomposition 
\[\sop_{\widetilde{p}}(\rho) = \sum_k A_k \rho A_k^\dagger\]
Then we can write 
\hspace*{-2cm}\begin{align*}
\trQ((\sop_{\widetilde{p}} \otimes \mathcal{I}_{\widetilde{q}})(\sigma)) &= \sop_{\widetilde{p}}(\trQ(\sigma)) 
\\ 
\trQ((\sop_{\widetilde{p}} \otimes \mathcal{I}_{\widetilde{q}})\left(\sum_l p_l \sum_{p = 0}^{2^n-1} \sum_{q = 0}^{2^m-1} \lambda_{pq} \proj{pq}\right)) &= \sop_{\widetilde{p}}(\trQ\left(\sum_l p_l \sum_{p = 0}^{2^n-1} \sum_{q = 0}^{2^m-1} \lambda_{pq} \proj{pq}\right))
\\
\sum_l p_l \trQ((\sop_{\widetilde{p}} \otimes \mathcal{I}_{\widetilde{q}})\left( \sum_{p = 0}^{2^n-1} \sum_{q = 0}^{2^m-1} \lambda_{pq} \proj{pq}\right)) 
&= \sum_l p_l \sop_{\widetilde{p}}(\trQ\left( \sum_{p = 0}^{2^n-1} \sum_{q = 0}^{2^m-1} \lambda_{pq} \proj{pq}\right))
\\
\trQ(\sum_k (A_k \otimes I_{\widetilde{q}})\left( \sum_{p = 0}^{2^n-1} \sum_{q = 0}^{2^m-1} \lambda_{pq} \proj{pq}\right)(A_k \otimes I_{\widetilde{q}})^\dagger )
&= \sum_k A_k\left( \sum_{p = 0}^{2^n-1} \sum_{q = 0}^{2^m-1} \lambda_{pq} \trQ(\proj{pq})\right) A_k^\dagger 
\\
\trQ(\sum_k  \left( \sum_{p = 0}^{2^n-1} \sum_{q = 0}^{2^m-1} \lambda_{pq} (A_k \otimes I_{\widetilde{q}}) \proj{pq} (A_k \otimes I_{\widetilde{q}})^\dagger  \right) )
&= \sum_k A_k\left( \sum_{p = 0}^{2^n-1} \sum_{q = 0}^{2^m-1} \lambda_{pq} \trQ(\proj{pq})\right) A_k^\dagger
\\
\trQ(\sum_k  \left( \sum_{p = 0}^{2^n-1} \sum_{q = 0}^{2^m-1} \lambda_{pq} (A_k\ket{p})\ket{q}(\bra{p}A_k^\dagger) \bra{q} \right) )
&= \sum_k A_k\left( \sum_{p = 0}^{2^n-1} \sum_{q = 0}^{2^m-1} \lambda_{pq} \proj{p}\braket{q | q}\right)A_k^\dagger
\\
\sum_{p = 0}^{2^n-1} \sum_{q = 0}^{2^m-1} \lambda_{pq} \sum_k \trQ( (A_k\ket{p})\ket{q}(\bra{p}A_k^\dagger) \bra{q} ) 
&= \sum_k  \sum_{p = 0}^{2^n-1} \sum_{q = 0}^{2^m-1} \lambda_{pq} \sum_k  A_k \proj{p} A_k^\dagger 
\\
\sum_{p = 0}^{2^n-1} \sum_{q = 0}^{2^m-1} \lambda_{pq} \sum_k (A_k \proj{p} A_k^\dagger) 
&=  \sum_{p = 0}^{2^n-1} \sum_{q = 0}^{2^m-1} \lambda_{pq} \sop_{\widetilde{p}} (\proj{p})
\\
\sum_{p = 0}^{2^n-1} \sum_{q = 0}^{2^m-1} \lambda_{pq} \sop_{\widetilde{p}}(\proj{p})
&=  \sum_{p = 0}^{2^n-1} \sum_{q = 0}^{2^m-1} \lambda_{pq} \sop_{\widetilde{p}} (\proj{p})
\end{align*}

Where we made use of the fact that both $\trQ$ and matrix multiplication are closed for linearity.
\end{proof}

\begin{lemma}\label{trace_and_sop_2}
Let $\sigma \in  \mathcal{D}(\calH_{\widetilde{p}} \otimes \calH_{\widetilde{q}})$, with $\widetilde{p} = p_0 \ldots p_{n-1}$ and $\widetilde{q} = q_0 \ldots q_{m-1}$. Then, for any \textit{trace non-increasing} superoperator $\sop_{\widetilde{q}} \in \mathcal{S}(\calH_{\widetilde{q}})$
\[ \trQ((\mathcal{I}_{\widetilde{p}} \otimes \sop_{\widetilde{q}})(\sigma)) = \trQ(\sigma)
\]
\end{lemma}

The proof of this lemma is extremely similar to the one before, and thus omitted. 

Now we prove that the behaviour of a configuration $\confw{\rho, P}$ is identical to the behaviour of a "bigger" configuration with additional qubits, if these additional qubits are discarded, and so cannot be modified nor measured.

\begin{lemma}\label{lemma_transition_partial_trace}
Let $\sigma \in  \mathcal{D}(\calH_{\widetilde{p}} \otimes \calH_{\widetilde{q}})$, and $\rho = tr_{\widetilde{q}}(\sigma) \in \calH_{\widetilde{p}}$. Then, for any process 
\[ \confw{\rho, P} \rightarrow \sum_i p_i \confw{\rho_i, P_i} 
\qquad\Leftrightarrow\qquad
\confw{\sigma, P\parallel \discQ} \rightarrow \sum p_i \confw{\sigma_i, P_i\parallel \discQ}
\]
and $\forall i \ \rho_i = \trQ(\sigma_i)$
\end{lemma}
\begin{proof}
We proceed by induction on $\rightarrow$. For the simple base cases {\footnotesize\scshape SemTau} and {\footnotesize\scshape SemReduce}, where the quantum state remains unchanged, we have 
\begin{align*}
\confw{\rho, \tau.P} \rightarrow \confw{\rho, P}
&\Leftrightarrow
\confw{\sigma, \tau.P\parallel \discQ} \rightarrow \confw{\sigma, P\parallel \discQ}
\\
 \confw{\rho, c?x.P\parallel c!v} \rightarrow \confw{\rho, P[v / x]}
&\Leftrightarrow
\confw{\sigma, c?x.P\parallel c!v \parallel \discQ} \rightarrow \confw{\sigma, P[v / x]\parallel \discQ}
\end{align*}

and in both cases $\rho' = \rho = \trQ(\sigma)$.

For the base case {\footnotesize\scshape SemQOp} we have 
\[\confw{\rho, \sop(\widetilde{p}).P} \rightarrow \confw{\sop_{\widetilde{p}}(\rho), P}
\qquad\Leftrightarrow\qquad
\confw{\sigma, \sop(\widetilde{p}).P\parallel \discQ} \rightarrow \confw{\sop_{\widetilde{p}}(\sigma), P\parallel \discQ}\]
where $\rho' = \sop_{\widetilde{p}}(\rho) = \sop_{\widetilde{p}}(\trQ(\sigma))$ and $\sigma' = \sop_{\widetilde{p}}\otimes \mathcal{I}_{\widetilde{q}}(\sigma)$, and so we have $\rho' = \trQ(\sigma') = \trQ(\sop_{\widetilde{p}}\otimes \mathcal{I}_{\widetilde{q}}(\sigma))$ thanks to lemma \ref{trace_and_sop}.


The base case {\footnotesize\scshape SemQMeasure} is similar, we have 
\begin{gather*}
\confw{\rho, M[\widetilde{p} \rhd x].P} \rightarrow \sum p_m\confw{p_m^{-1}\sop_m(\rho), P} \\
\Leftrightarrow \\
\confw{\sigma, M[\widetilde{p} \rhd x].P\parallel \discQ} \rightarrow \sum p'_m \confw{p_m^{-1}\sop_m(\sigma), P\parallel \discQ}
\end{gather*}
where thanks to lemma \ref{trace_and_sop}, we have $p_m = tr(\sop_m(\rho)) = tr(\sop_m(\trQ(\sigma))) = tr(\trQ(\sop_m \otimes \mathcal{I}_{\widetilde{q}}(\sigma)))$ and $\forall i \ \rho_i = \trQ(\sigma_i)$, as in the previous case.

The inductive cases are all trivial, as none of them modifies the quantum values $\rho$ and $\sigma$.
\end{proof}

\begin{theorem}[$\simps$ is closed for additional discarded qubits]\label{bisim_closed_by_discard}
If $\confw{\trQ(\sigma), P} \sim \confw{\trQ(\nu), Q}$ then $\confw{\sigma, P \parallel \discQ} \sim \confw{\nu, Q \parallel \discQ}$.
\end{theorem}
\newcommand{\relTrQ}{\rel_{\trQ}}
\begin{proof} 
Let $\sigma, \nu \in \mathcal{D}(\calH_{\widetilde{p}} \otimes \calH_{\widetilde{q}})$. We show that if $\rel$ is a bisimulation, then $\rel_{\trQ}$ is a bisimulation, where $\rel_{\trQ}$ is defined as
\[\rel_{\trQ} = \set{\Big(\confw{\sigma, P \parallel \discQ}, \confw{\nu, Q \parallel \discQ}\Big) \quad\mid\quad \confw{\trQ(\sigma), P} \rel \confw{\trQ(\nu), Q}}\]
From this, it easily follows that $\simps$ is closed for additional discarded qubits, becuse if $\confw{\trQ(\sigma), P} \sim \confw{\trQ(\nu), Q}$, then there exist a bisimulation $\rel$ such that $\confw{\trQ(\sigma), P} \rel \confw{\trQ(\nu), Q}$, and so there exists a bisimulation $\relTrQ \subseteq \simps$ such that $\confw{\sigma, P \parallel \discQ} \relTrQ \confw{\nu, Q \parallel \discQ}$.

To show that $\rel_{\trQ}$ is a bisimulation, we need first of all to show that is symmetric, which is when $\rel$ is symmetric. The we suppose that $\confw{\sigma, P \parallel \discQ} \relTrQ \confw{\nu, Q \parallel \discQ}$ and have to show that \begin{itemize}
\item For any barb $b$, if $\confw{\sigma, P \parallel \discQ} \downarrow_b$ then $\confw{\nu, Q \parallel \discQ} \downarrow_b$. We have that $\confw{\sigma, P \parallel \discQ} \downarrow_b \Leftrightarrow \confw{\trQ(\sigma), P} \downarrow b$, $\confw{\nu, Q \parallel \discQ} \downarrow_b \Leftrightarrow \confw{\trQ(\nu), Q} \downarrow b$, and 
$\confw{\trQ(\sigma), P} \rel \confw{\trQ(\nu), Q}$, so they all express the same barbs.
\item For any process $R$, if $\confw{\sigma, P \parallel R \parallel \discQ} \rightarrow \Delta$, then $\confw{\nu, Q \parallel R \parallel \discQ} \rightarrow \Theta$, and $\Delta \slift{\relTrQ} \Theta$. Notice that $\discQ$ can not evolve, and so we start from the hypothesis $\confw{\sigma, P \parallel R \parallel \discQ} \rightarrow \sum p_i \confw{\sigma_i, P_i \parallel \discQ}$. Then, from lemma \ref{lemma_transition_partial_trace}, we know that $\confw{\trQ(\sigma), P\parallel R} \rightarrow \sum p_i \confw{\trQ(\sigma_i), P_i}$. But since $\confw{\trQ(\sigma), P} \rel \confw{\trQ(\nu), Q}$, and $\rel$ is a saturated bisimulation, it must be that $\confw{\trQ(\nu), Q\parallel R} \rightarrow \sum p_i \confw{\xi_i, Q_i}$ with $\confw{\trQ(\sigma_i), P_i} \rel \confw{\xi_i, Q_i}$ for each $i$. But using the same lemma \ref{lemma_transition_partial_trace} in the other direction, we get $\confw{\nu, Q \parallel R \parallel \discQ} \rightarrow \sum p_i \confw{\nu_i, Q_i \parallel \discQ}$ with $\xi_i = \trQ(\nu_i)$. In conclusion, for each transition $\confw{\sigma, P \parallel R \parallel \discQ} \rightarrow \sum p_i \confw{\sigma_i, P_i \parallel \discQ}$ exists a transition $\confw{\nu, Q \parallel R \parallel \discQ} \rightarrow \sum p_i \confw{\nu_i, Q_i \parallel \discQ}$ such that $\forall i \confw{\trQ(\sigma_i), P_i} \rel \confw{\trQ(\nu_i), Q_i}$, and so from the definiton of $\relTrQ$ together with probabilistic lifting we get $\sum p_i \confw{\sigma_i, P_i \parallel \discQ} \slift{\relTrQ} \sum p_i \confw{\nu_i, Q_i \parallel \discQ}$.
\end{itemize}
\end{proof}



Theorem \ref{bisim_closed_by_discard} is useful when proving bisimilarity of processes that use discard. For example, it can be used to prove that $P = H[q].discard(q)$ and $Q = X[q].discard(q)$ are bisimilar.

Given that $\emptyset, {q} \vdash P$ and $\emptyset, {q} \vdash Q$, we show that
\[\rel = \big\{\confw{\sigma, B[P]}, \confw{\sigma, B[Q]} \mid \sigma \in \mathcal{D}(\calH_{QN}), B[\blank]_{\emptyset; \set{q}} \text{ typed context}\big\}^S\ \cup \sim_{PS}
\]
is a probabilistic saturated bisimulation, where $\rel^S$ denotes the symmetric closure of a relation $\rel$. From this follows trivially that $\confw{\sigma, P} \sim_{PS} \confw{\sigma, Q}$ for any $\sigma$, and so $P$ and $Q$ are bisimilar processes.

$\rel$ is a \textit{saturated} relation, meaning that if $\conf \rel \conf'$, then $B[\conf] \rel B[\conf']$ for any $B$. So, to prove that $\rel$ is a probabilistic saturated bisimulation, we just need to show that $\rel$ is a probabilistic bisimulation.

Suppose that $\confw{\sigma, R \parallel P} \rel \confw{\sigma, R \parallel Q}$, and that $\confw{\sigma, R \parallel P} \rightarrow \sum_i p_i \conf_i$.\begin{itemize}
\item $\confw{\sigma, P \parallel \discQ} \sim \confw{\nu, Q \parallel \discQ}$.
\item If the reductions happens in $R$, it must be of the form $\confw{\sigma, R \parallel P} \rightarrow \sum_i p_i \confw{\sigma_i, R' \parallel P}$, but then there exists a transition $\confw{\sigma, R \parallel Q} \rightarrow \sum_i p_i \confw{\sigma_i, R_i \parallel Q}$, and for each $i$, $\confw{\sigma_i, R_i \parallel P} \rel \confw{\sigma_i, R_i \parallel Q}$ by definition of $\rel$.
\item If the reductions happens in $P$, it must be 
 $\confw{\sigma, R \parallel H[q].disc(q)} \rightarrow \confw{\sop_{H, q}(\sigma), R \parallel disc(q)}$, but then  $\confw{\sigma, R \parallel X[q].disc(q)} \rightarrow \confw{\sop_{X, q}(\sigma), R \parallel disc(q)}$, where $\sop_{H, q}$ is the superoperator that applies the H transformation only on qubit $q$, and $\sop_{Z, q}$ is the superoperator that applies the Z transformation only on qubit $q$. 
Since $tr_q(\sop_{H, q}(\sigma)) = tr_q(\sop_{Z, q}(\sigma)) = tr_q(\sigma)$ for lemma \ref{trace_and_sop_2}, we have that $\confw{\sop_{H, q}(\sigma), R \parallel disc(q)} \sim_{PS} \confw{\sop_{X, q}(\sigma), R \parallel disc(q)}$ for theorem \ref{bisim_closed_by_discard}, and so $$\confw{\sop_{H, q}(\sigma), R \parallel disc(q)} \rel \confw{\sop_{X, q}(\sigma), R \parallel disc(q)}$$ by definition of $\rel$.
 
\end{itemize}

\subsection{The problem of mixed states}

We show that the usual way of defining behavioural equivalence through $\sim_{LC}$ and $\sim_{LC-cm}$ does not work in the quantum case.
This because probabilistic mixtures of quantum states are represented in two alternative ways, i.e., using density operators and using distributions of configurations.
Two different representations of the same mixture should be equivalent, but this is not always the case in the literature.

\begin{example}
	Consider the following configurations and distributions.
	\begin{align*}
		\conf &= \confw{\ketbra{0}{0}, Set_{\ketbra{+}{+}}(q).M_{0,1}(q \triangleright x).(c!0 \parallel discard(q))}\\
		\conf' &= \confw{\ketbra{0}{0}, Set_{\frac{1}{2} I}(q).\tau.(c!0 \parallel discard(q))}\\
		\Delta &= \overline{\confw{\ketbra{0}{0}, c!0 \parallel discard(q)}} \psum{1/2} \overline{\confw{\ketbra{1}{1}, c!0 \parallel discard(q)}}\\
		\Delta' &= \overline{\confw{1/2 I, c!0 \parallel discard(q)}}
	\end{align*}
	It is trivial that $\conf \sim_{LC} \conf'$ iff $\Delta	\slift{\sim}_{LC} \Delta'$.
	It seems obvious that the former should hold.
	Indeed the behaviour of $\confw{\rho, c!0 \parallel discard(q)}$ does not depend on $\rho$.
	Unfortunately, this is not the case for $\slift{\sim}_{LC}$ (and similarly, neither for $\slift{\sim}_{LC-cm}$).
\end{example}
Indeed, $\conf$ and $\conf'$ of the previous example are not bisimilar according to~\cite{Feng:2012, Deng:2012}.

A related problem is that a density operator may represent different probabilistic ensambles, that are indistinguishable according to quantum theory.
\begin{example}
	Consider the following distributions.
	\begin{align*}
		\Delta &= \overline{\confw{\ketbra{0}{0}, c!q}} \psum{1/2} \overline{\confw{\ketbra{1}{1}, c!q}}\\
		\Delta' &= \overline{\confw{\ketbra{+}{+}, c!q}} \psum{1/2} \overline{\confw{\ketbra{-}{-}, c!q}}
	\end{align*}
	The two distributions only send a qbit on the channel $c$, and the qbit may be in state $\ket{0}$ or $\ket{1}$ for $\Delta$ ($\ket{+}$ or $\ket{-}$ for $\Delta'$) with equal probability.
	It seems reasonable that $\Delta$ and $\Delta'$ cannot be distinguished by an external observer receiving the qbit.
	This because the received qbit is in the probabilistic mixture $\{(\ket{0}, 1/2), (\ket{1}, 1/2)\}$ for $\Delta$ and $\{(\ket{+}, 1/2), (\ket{-}, 1/2)\}$ respectively for $\Delta'$, but the two mixtures are both represented by the same density operator $1/2 I = 1/2 \ketbra{0}{0} + 1/2 \ketbra{1}{1} = 1/2 \ketbra{+}{+} + 1/2 \ketbra{-}{-}$.
	Unfortunately, this is not the case for $\slift{\sim}_{LC}$ (and similarly, neither for $\slift{\sim}_{LC-cm}$).
\end{example}
Once more, \cite{Feng:2012, Deng:2012} distinguish the two distributions, hence, e.g., the two processes $Set_{\frac{1}{2} I}(q).M_{0,1}[q \triangleright x].c!q$ and $Set_{\frac{1}{2} I}(q).M_{+,-}[q \triangleright x].c!q$ are not considered bisimilar.

\subsection{The problem of non-determinism}

We present here a strange behaviour that common process algebras express when mixing quantum states and non-deterministic choice.
Assume a distribution obtained by tracing out one qubit of a bell pair $\ket{\Phi^+} = 1/\sqrt{2}(\ket{00} + \ket{11})$.
This can be obtained, e.g., with the following configuration $\confw{\ket{\Phi^+}, discard(q_0) \parallel \dots}$.
The qbit is in a probabilistic ensamble of pure states, represented with the density opertator $1/2 I$, which stands for both $\{(\ket{0}, 1/2), (\ket{1}, 1/2)\}$ and $\{(\ket{+}, 1/2), (\ket{-}, 1/2)\}$.
The two ensambles are indeed indistinguishable from the quantum point of view.
However, the two express a different behaviour according to a trivial semantics that just applies probabilistic reasoning to quantum processes, as shown in the following.
\begin{example}
	Let the processes $P$ and $Q$ and distributions $\Delta$, $\Theta$ be
	\begin{align*}
		P\ =\ &M_{0,1}[q \triangleright x] . \ite{x=0}{z!0}{u!0}\\
		Q\ =\ &M_{+,-}[q \triangleright x] . \ite{x=0}{p!0}{m!0}\\
		\Delta\ =\ &\confw{\ketbra{0}{0}, P + Q} \psum{1/2} \confw{\ketbra{1}{1}, P + Q}\\
		\Theta\ =\ &\confw{\ketbra{+}{+}, P + Q} \psum{1/2} \confw{\ketbra{-}{-}, P + Q}
	\end{align*}
	There is a way of performing the non deterministic choice such that $\Delta \rightarrow \confw{\ketbra{0}{0}, z!0}$ with probability $1/2$, $\Delta \rightarrow \confw{\ketbra{+}{+}, p!0}$ and $\Delta \rightarrow \confw{\ketbra{-}{-}, m!0}$ both with probability $1/4$.
	The apparent contradiction is that $\Theta$ cannot replicate this behaviour.
\end{example}

Apparently, the previous example seems to contradict the common understanding of indistinguishability of quantum states.
To recover it, either this behaviour must be expressed by both $\Delta$ and $\Theta$, or by none of them, resulting in two different notions of non deterministic choice.
Actually, something strange happens in the example, namely, the non deterministic choice is performed in the two probabilistic branches.


\subsection{Quantum Saturated Bisimilarity}

Density operators represents equivalence classes over probabilistic mixtures of quantum states.
The implicit equivalence relation is $\{(\ket{\phi_i}, p_i)\}_i \cong \{(\ket{\phi_j}, p_j)\}$ iff $\sum_{i} p_i \ketbra{\phi_i}{\phi_i} = \sum_{j} p_j \ketbra{\phi_j}{\phi_j}$.
The physical justification of this equivalence is that different mixtures resulting in the same density operator cannot be distinguished since they behave the same.

The same equivalence relation is trivially extended to configurations, where the use of density operators for the quantum state allows a common representation of different configurations with an equivalent mixtures of quantum states. 
We extend here this equivalence relation to distributions of configurations.
Intuitively, we give rules for exchanging the probabilistic combination $\psum{p}$ in the state of some configurations for probabilistic combination of configurations, and vice-versa. 

\begin{definition}
	A \emph{quantum distribution} is an equivalence class $\mathbf{\Delta}$ of probabilistic distributions $\Delta$ over quantum configurations defined by the minimal equivalence relation such that:
	\begin{itemize}
		\item $(\overline{\confw{\rho, P}} \psum{p} \overline{\confw{\sigma, P}}) \equiv \overline{\confw{p \rho + (1-p)\sigma, P}}$; and
%		\item $(\overline{\confw{\rho \otimes \sigma, P}} \psum{p} \overline{\confw{\rho \otimes \sigma', Q}}) \equiv (\overline{\confw{\rho \otimes \delta, P}} \psum{p} \overline{\confw{\rho \otimes \delta', Q}})$ if $\Gamma, \Sigma \vdash P$, $\Gamma, \Sigma \vdash Q$, $\rho \in \hilbert_\Sigma$, and $p \sigma + (1 - p) \sigma' =  p \sigma + (1 - p) \sigma' = p \delta + (1 - p) \delta'$; and	
		\item $\Delta_i\ \equiv\ \Theta_i$, $i = 1, 2$, implies $\Delta_1 \psum{p} \Delta_2 \equiv\ \Theta_1 \psum{p} \Theta_2$.
	\end{itemize}
	We write $Q(Conf)$ for quantum distributions over configurations.
\end{definition}

\begin{definition}
	Given $\rel \subseteq Conf \times Conf$ be a relation over quantum configurations, let its quantum lifting be the minimal relation $\sqlift{\rel} \subseteq Q(Conf) \times Q(Conf)$ over quantum distributions such that $\Delta\ \slift{\rel}\ \Theta$ with $\Delta \in \mathbf{\Delta}$ and $\Theta \in \mathbf{\Theta}$ implies $\mathbf{\Delta}\ \sqlift{\rel}\ \mathbf{\Theta}$.
\end{definition}


\begin{definition}[Quantum Saturated Bisimilarity]
	A symmetric relation $\rel \subseteq \conf \times \conf$ is \emph{quantum saturated bisimulation} if $\conf\ \rel\ \conf'$ implies that $\conf, \conf'$ are well-typed under a typing context $\Gamma; \Sigma$, and for any context $B[\_]_{\Gamma, \Sigma}$
	\begin{itemize}
		\item if $B[\conf] \downarrow_{c}$ then $B[\conf'] \downarrow_{c}$; and 
		\item whenever $B[\conf] \xrightarrow{\tau} \Delta \in \mathbf{\Delta}$, there exists $\Delta' \in \mathbf{\Delta'}$ such that $C[\conf'] \xrightarrow{\tau} \Delta'$ and $\mathbf{\Delta}\ \sqlift{\rel}\ \mathbf{\Delta'}$
		%		\item whenever $\conf' \xrightarrow{\tau} \Delta'$, there exists $\Delta$ such that $\conf \xrightarrow{\tau} \Delta$ and $\Delta \slift{\rel} \Delta'$
	\end{itemize}
	Let \emph{quantum saturated bisimilarity} $\sim_{QS}$ be the union of all probabilistic saturated bisimulation.
\end{definition}


\begin{theorem}
	For any pair of configurations $\confw{\rho, P}, \confw{\sigma, Q}$ well-typed under $\Gamma; \Sigma$, 
	$\confw{\rho, P} \simqs \confw{\rho, P}$ implies
	\begin{enumerate}
		{\item for any $\sop \in \mathcal{TS}(\hilbert_{\overline{\Sigma}})$, $\sop (\confw{\rho, P})\ \simqs \ \sop (\confw{\rho, P})$, where $\sop(\confw{\rho, P})$ is defined as $\confw{\sop\otimes\mathcal{I}_\Sigma(\rho), P}$, and similarly for $\sop(\confw{\sigma, Q})$, with $\mathcal{I}_\Sigma$ the identity superoperator on qubits $\Sigma$ \label{point:thmchinese1_quantum}}
		{\item $tr_{\Sigma}(\rho) = tr_{\Sigma}(\sigma)$. \label{point:thmchinese2_quantum}}
	\end{enumerate}
\end{theorem}
\begin{proof}
To prove point \ref{point:thmchinese1_quantum}, suppose $\confw{\rho, P} \simqs \confw{\rho', P'}$, with a set of quantum names $\widetilde{q}, \widetilde{p}$ and $\Gamma;\widetilde{p} \vdash P$ and $\Gamma;\widetilde{p} \vdash P'$. For any superoperator $\sop \in \mathcal{TS}(\hilbert_{\widetilde{q}})$ we can construct a context $B[\blank] = [\blank] \parallel \sop({\widetilde{q}}).a!0 \parallel c!\widetilde{q}$, where $a$ is a fresh channel. We know that $\confw{\rho, B[P]}$ and $\confw{\sigma, B[q]}$ are bisimilar, and$\confw{\rho, B[P]}$ can evolve in $\confw{\sop\otimes\mathcal{I}_{\widetilde{p}}(\rho), a!0 \parallel c!\widetilde{q} \parallel P}$. Then $\confw{\sigma, B[Q]}$ must necessarily evolve in $\confw{\sop\otimes \mathcal{I}_{\widetilde{p}}(\sigma), a!0 \parallel c!\widetilde{q} \parallel Q}$, because it must match the $\downarrow_a$ barb, and $a$ is fresh. So we have that  
\[ \confw{\sop\otimes\mathcal{I}_{\widetilde{p}}(\rho), a!0 \parallel c!\widetilde{q} \parallel P} \simqs 
  \confw{\sop\otimes \mathcal{I}_{\widetilde{p}}(\sigma), a!0 \parallel c!\widetilde{q} \parallel Q}
\]
and from this it follows 
\[ \confw{\sop\otimes\mathcal{I}_{\widetilde{p}}(\rho), P} \simqs 
  \confw{\sop\otimes \mathcal{I}_{\widetilde{p}}(\sigma), Q}
\] simply by contradiction: if there was a context capable of distinguishing $\sop(\rho, P)$ from $\sop(\rho, Q)$ then there would be a context able to distinguish also $\sop(\rho, P \parallel a!0 \parallel c!\widetilde{q})$ from $\sop(\sigma, Q \parallel a!0 \parallel c!\widetilde{q})$

To prove point \ref{point:thmchinese2_quantum},  suppose $\confw{\rho, P} \simqs \confw{\sigma, Q}$, with a set of quantum names $\widetilde{q}, \widetilde{p}$ and $\Gamma;\widetilde{p} \vdash P$ and $\Gamma;\widetilde{p} \vdash Q$. Consider the set of contexts 
\[\mathcal{B} = \set{[\blank] \parallel M_{\kp}[\widetilde{q} \rhd x].\big(\ite{x = 0}{z!0}{o!0} \parallel disc(\widetilde{q})\big)} \mid \kp \in \mathcal{H}_widetilde{q}\]
where $M_{\kp}$ is the projective measurement $\set{M_0 = \proj{\psi}, M_1 = I - \proj{\psi}}$. Since $\confw{\rho, P}$ is bisimilar to $\confw{\sigma, Q}$, $\confw{\rho, B[P]}$ must be bisimilar to $\confw{\sigma, B[Q]}$ for each $B \in \mathcal{B}$.
\end{proof}


We now prove that, like probabilistic bisimilarity, also quantum bisimilarity is closed for additional discarded qubits: 
\[\confw{\trQ(\sigma), P} \simqs \confw{\trQ(\nu), Q} \Rightarrow \confw{\sigma, P\parallel \discQ} \simqs \confw{\nu, Q\parallel \discQ}\] 
For the probabilistic case, we proved that \[\confw{\trQ(\sigma), P} \simps \confw{\trQ(\nu), Q} \Rightarrow \confw{\sigma, P\parallel \discQ} \simps \confw{\nu, Q\parallel \discQ}\] so now we have a weaker hypothesis, since $\simqs$ is coarser then $\simps$.

Note that Lemmas \ref{trace_and_sop}, \ref{trace_and_sop_2} and \ref{lemma_transition_partial_trace} descend only from lqCCS semantics, and so are still true. To deal with the weaker hypothesis, we need only one additional lemma, saying that  additional discarded qubits preserve the equivalence relation.

\begin{lemma}\label{lemma_quantum_equivalence_partial_trace}
Let $\sum_i p_i \confw{\trQ(\sigma_i), P_i}$ be a distribution of configuration, with $\sigma_i \in \mathcal{D}(\calH_{\widetilde{p}} \otimes \calH_{\widetilde{q}})$ for each $i$.  If \[\sum_i p_i \confw{\trQ(\sigma_i), P_i} \equiv \sum_j p_j \confw{\rho_j, P_j}\]
then 
\[\sum_i p_i \confw{\sigma_i, P_i\parallel \discQ} \equiv \sum_j p_j \confw{\sigma_j, P_j\parallel \discQ}\]
and $\rho_j = \trQ(\sigma_j)$ for each $j$.
\end{lemma}
\begin{proof}
We proceed by induction on the rules of $\equiv$. For the base case, suppose \[
\confw{\trQ(\sigma), P} \psum{p} \confw{\trQ(\sigma'), P} \equiv \confw{(p)\trQ(\sigma) + (1-p)\trQ(\sigma'), P}\]
We also have, by definition,  \[\confw{\sigma, P\parallel \discQ} \psum{p} \confw{\sigma', P\parallel \discQ} \equiv \confw{(p)\sigma + (1-p)\sigma', P\parallel \discQ}\]
Notice that, due to linearity of partial trace, $(p)\trQ(\sigma) + (1-p)\trQ(\sigma') = \trQ((p)\sigma + (1-p)\sigma')$, and so we have proven the final condition of the base case.
The inductive case is trivial, as it simply combines two distributions, without changing the density matrix of configurations
\end{proof}

We can now prove the same lemma about $\rel_{\trQ}$ for the quantum bisimulation case.

\begin{lemma}\label{lemma_reltrq_quantum}
Let $\sigma \in \mathcal{D}(\calH_{\widetilde{p}} \otimes \calH_{\widetilde{q}})$. If $\rel$ is a quantum saturated bisimulation, then $\rel_{\trQ}$ is a quantum saturated bisimulation, where $\rel_{\trQ}$ is defined as
\[\rel_{\trQ} = \set{\Big(\confw{\sigma, P \parallel \discQ}, \confw{\nu, Q \parallel \discQ}\Big) \quad\mid\quad \confw{\trQ(\sigma), P} \rel \confw{\trQ(\nu), Q}}\]
\end{lemma}
\begin{proof}
The proof is similar to the probabilistic bisimulation case, so we will omit some steps.
For any process $R$ we suppose $\confw{\sigma, P \parallel R \parallel \discQ} \rightarrow \sum p_i \confw{\sigma_i, P_i \parallel \discQ}$. Then, from lemma \ref{lemma_transition_partial_trace}, we know that $\confw{\trQ(\sigma), P\parallel R} \rightarrow \sum p_i \confw{\trQ(\sigma_i), P_i}$. But since $\confw{\trQ(\sigma), P} \rel \confw{\trQ(\nu), Q}$, and $\rel$ is a quantum saturated bisimulation, it must be that 
\begin{align*}
& \confw{\trQ(\sigma), P\parallel R} \rightarrow \sum_i p_i \confw{\trQ(\sigma_i), P_i} & & \equiv  &  & \sum_j p_j \confw{\rho_j, P_j} 
\\
 & & & & & \qquad\slift{\rel} 
\\
& \confw{\trQ(\nu), Q\parallel R} \rightarrow \quad\sum_i p_i \confw{\xi_i, Q_i} & & \equiv & &  \sum_j p_j \confw{\xi'_j, Q_j} 
\end{align*} 
Then, we can apply our lemmas to each part of the above diagram: \begin{itemize}
\item From $\sum_i p_i \confw{\trQ(\sigma_i), P_i} \equiv \sum_j p_j \confw{\rho_j, P_j}$ follows, for lemma \ref{lemma_quantum_equivalence_partial_trace}, $\sum_i p_i \confw{\sigma_i, P_i\parallel \discQ} \equiv  \sum_j p_j \confw{\sigma_j, P_j\parallel \discQ}$, with $\rho_j = \trQ(\sigma_j)$
\item From $\confw{\trQ(\nu), Q\parallel R} \rightarrow \quad\sum_i p_i \confw{\xi_i, Q_i}$ follows, for lemma \ref{lemma_transition_partial_trace}, $\confw{\nu, Q\parallel R\parallel \discQ} \rightarrow \sum_i p_i \confw{\nu_i, Q_i\parallel \discQ}$, with $\xi_i = \trQ(\nu_i)$
\item From $\sum_i p_i \confw{\xi_i, Q_i} \equiv  \sum_j p_j \confw{\xi'_j, Q_j}$ follows, for lemma \ref{lemma_quantum_equivalence_partial_trace}, $\sum_i p_i \confw{\nu_i, Q_i\parallel \discQ} \equiv  \sum_j p_j \confw{\nu_j, Q_j\parallel \discQ}$, with $\xi'_j = \trQ(\nu_j)$. 
\end{itemize}
In conclusion, together with the definition of $\rel_{\trQ}$, we get
\begin{align*}
& \confw{\sigma, P\parallel R \parallel \discQ} \rightarrow \sum_i p_i \confw{\sigma_i, P_i \parallel \discQ} & & \equiv  &  & \sum_j p_j \confw{\sigma_j, P_j \parallel \discQ} 
\\
 & & & & & \qquad\quad\slift{\rel_{\trQ}} 
\\
& \confw{\nu, Q\parallel R\parallel \discQ} \rightarrow \quad\sum_i p_i \confw{\nu_i, Q_i\parallel \discQ} & & \equiv & &  \sum_j p_j \confw{\nu_j, Q_j\parallel \discQ}
\end{align*}
And so $\confw{\sigma, P\parallel R \parallel \discQ} \rel_{\trQ}\confw{\nu, Q\parallel R\parallel \discQ}$.
\end{proof}

We can now restate the same theorem also for quantum saturated bisimilarity.

\begin{theorem}[$\simqs$ is closed for additional discarded qubits]\label{bisim_closed_by_discard_quantum}
If $\confw{\trQ(\sigma), P} \simqs \confw{\trQ(\nu), Q}$ then $\confw{\sigma, P \parallel \discQ} \simqs \confw{\nu, Q \parallel \discQ}$.
\end{theorem}
\begin{proof}
It easily follows from the previous lemma, considering that $\simqs$ is the union of all bisimulations.
\end{proof}

From \cite{davidsonFormalVerificationTechniques2012} we expect that $$P =  M_{01}[q_0 \rhd x].disc(q_0) \simqs Q = M_\pm[q_0 \rhd x].disc(q_0)$$To prove this result, we need the above theorem \ref{bisim_closed_by_discard_quantum}


Given that $\emptyset, {q_0} \vdash P$ and $\emptyset, {q_0} \vdash Q$, we show that
\[\rel = \big\{ \confw{\sigma, B[P]}, \confw{\sigma, B[Q]} \mid \sigma \in \mathcal{D}(\calH_{QN}), B[\blank]_{\emptyset; \set{q}} \text{ typed context}\big\}^S\ \cup \sim_{PS}
\]
is a quantum saturated bisimulation, where $\rel^S$ denotes the symmetric closure of a relation $\rel$. From this follows trivially that $\confw{\sigma, P} \sim_{QS} \confw{\sigma, Q}$ for any $\sigma$, and so $P$ and $Q$ are bisimilar processes.

$\rel$ is a \textit{saturated} relation, meaning that if $\conf \rel \conf'$, then $B[\conf] \rel B[\conf']$ for any $B$. So, to prove that $\rel$ is a quantum saturated bisimulation, we just need to show that $\rel$ is a quantum bisimulation.

Suppose that $\confw{\sigma, R \parallel P} \rel \confw{\sigma, R \parallel Q}$, and that $\confw{\sigma, R \parallel P} \rightarrow \sum_i p_i \conf_i$.\begin{itemize}
\item If the reductions happens in $R$, it must be of the form $\confw{\sigma, R \parallel P} \rightarrow \sum_i p_i \confw{\sigma_i, R_i \parallel P}$, but then there exists a transition $\confw{\sigma, R \parallel Q} \rightarrow \sum_i p_i \confw{\sigma_i, R_i \parallel Q}$, and for each $i$, $\confw{\sigma_i, R_i \parallel P} \rel \confw{\sigma_i, R_i \parallel Q}$ by definition of $\rel$.
\item If the reductions happens in $P$, it must be 
 \[\confw{\sigma, R \parallel M_{01}[q \rhd x].disc(q)} \rightarrow \confw{\frac{1}{p_0}\sop_{0, q}(\sigma), R \parallel disc(q)} \psum{p_0} \confw{\frac{1}{p_1}\sop_{1, q}(\sigma), R \parallel disc(q)}\]
 where $\sop_{0, q}(\rho) = (\proj{0}\otimes I) \rho (\proj{0} \otimes I)^\dagger$ is the trace non-increasing superoperator that projects qubit $q$ to $\kz$, $p_0$ is the probability of obtaining outcome $0$ from $\sigma$, and similarly for $\sop_{1, q }$ and $p_1$. But then  
 \[\confw{\sigma, R \parallel M_\pm[q \rhd x].disc(q)} \rightarrow \confw{\frac{1}{p_+}\sop_{+, q}(\sigma), R \parallel disc(q)} \psum{+} \confw{\frac{1}{p_-}\sop_{-, q}(\sigma), R \parallel disc(q)}
 \] 
 Noticeably these two distribution are not probabilistic bisimilar, as is evident in the case $\confw{\Phi^+, P} \rightarrow \confw{\proj{00}, \discQ} \psum{\frac{1}{2}} \confw{\proj{11}, \discQ}$ and $\confw{\Phi^+, Q} \rightarrow \confw{\proj{++}, \discQ} \psum{\frac{1}{2}} \confw{\proj{--}, \discQ}$.  Thanks to the quantum equivalence relation, however, we have 
 \begin{gather*}
 \confw{\frac{1}{p_0}\sop_{0, q}(\sigma), R \parallel disc(q)} \psum{p_0} \confw{\frac{1}{p_1}\sop_{1, q}(\sigma), R \parallel disc(q)} \equiv \confw{\sop_{0, q}(\sigma) + \sop_{1, q}(\sigma), R \parallel disc(q)}
 \\
 \confw{\frac{1}{p_+}\sop_{+, q}(\sigma), R \parallel disc(q)} \psum{p_+} \confw{\frac{1}{p_-}\sop_{-, q}(\sigma), R \parallel disc(q)} \equiv \confw{\sop_{+, q}(\sigma) + \sop_{-, q}(\sigma), R \parallel disc(q)}
 \end{gather*}
 Observe that $\sop_{0, q}(\sigma) + \sop_{1, q}(\sigma)$ can be seen as $\sop_{01, q}(\sigma)$, where $\sop_{01, q}$ is the trace-preserving superoperator that measures the qubit $q$ in the computational basis and then discards the result. $\sop_{01, q}(\sigma)$ is indeed a well defined superoperator, as $\proj{0}\otimes I$ and $\proj{1}\otimes I$ for a valid Kraus decomposition. The same holds also for $\sop_{+, q}(\sigma) + \sop_{-, q}(\sigma)$ can be seen as $\sop_{\pm, q}(\sigma)$. So we can conclude
 \[ \confw{\sop_{01,q},  R \parallel disc(q)} \simqs \confw{\sop_{\pm,q},  R \parallel disc(q)}
 \]
 from theorem \ref{bisim_closed_by_discard_quantum}, since $tr_q(\sop_{01, q}(\sigma)) = tr_q(\sop_{\pm, q}(\sigma)) = tr_q(\sigma)$.
\end{itemize}

In general, for quantum saturated bisimilarity, we have
\[	U(q).disc(q) \simqs M_[q \rhd x].disc(q) \simqs \sop(q).disc(q) \simqs \tau.disc(q) \]
for any unitary $U$, measurement $M$ or superoperator $\sop$.




\section{Comparison}

\subsection{QPALg}

Consider the two configurations
\[ \conf = \confw{\frac{1}{2}I \otimes \frac{1}{2}I, c!q_1 \parallel c!q_2} \qquad \conf' = \confw{\proj{\Phi^+}, c!q_1 \parallel c!q_2}
\] where $\ket{\Phi^+} = \oost\ket{00} + \oost{11}$. According to the labelled bisimulation of QPAlg, the two configurations are bisimilar, as they both send two qubits with reduced density operator $\frac{1}{2}I$ on channel $c$. For our definition instead it holds $\conf \not\simps \conf'$, as there exists the context \[B[\blank] = [\blank]\parallel c?x.c?y.M[x, y \rhd z].\ite{z = 2}{d!0 \parallel disc(x, y)}{\parallel disc(x, y)}\] where $M$ is the measurement on the $4$-dimensional computational basis \[M = \set{M_0 = \proj{00}, M_1 = \proj{01}, M_2 = \proj{10}, M_3 = \proj{11}}\]
After receiving and measuring two unrelated mixed state qubits, $B[\conf]$ will evolve in the distribution  $\Delta$
\[ \frac{1}{4}\confw{\proj{00}, disc(\widetilde{q})} + \frac{1}{4}\confw{\proj{01}, disc(\widetilde{q})} + \frac{1}{4}\confw{\proj{10}, d!0 \parallel disc(\widetilde{q})} + \frac{1}{4}\confw{\proj{11}, disc(\widetilde{q})} \] where $\widetilde{q}$ is the couple $q_1, q_2$.
After receiving and measuring two entangled qubits, instead, $B[\conf']$ will evolve in the distribution $\Delta'$
\[ \frac{1}{2}\confw{\proj{00}, disc(\widetilde{q})} +  \frac{1}{2}\confw{\proj{11}, disc(\widetilde{q})} \] 
We have that $\Delta \slift{\not\sim}_{PS} \Delta'$, as there is no decomposition of $\Delta = \sum_{i\in I} p_i \conf_i$ and $\Delta' = \sum_{i\in I} p_i \conf_i'$ such that $\conf_i \simps \conf_i'$ for each $i$, because $\Delta$ contains a configuration that expresses the barb $\downarrow_d$, while $\Delta'$ contains none.

\subsection{qCCS}
Probabilistic saturated bisimilarity is designed to be equivalent to open bisimilarity for qCCS, except for a few intended modification:\begin{itemize}
\item Open bisimilarity requires bisimialr processes to have the same free variables, $\simps$ requires bisimilar processes to have the same typing context.
\item Open bisimilarity is superoperator-closed by definition, $\simps$ is superoperator closed as a property.
\item Open bisimilarity requires bisimilar configurations to have the same environment, $\simps$ implies that bisimilar configurations have the same environment.
\item Open bisimilarity is contex closed as a property, $\simps$ is context closed by definition.
\end{itemize}


In the following, taken a superoperator $\sop \in \mathcal{TS}(\hilbert_{\overline{\Sigma}})$ that acts only on the qubits \textit{outside} $\Sigma$, we write $\sop(\confw{\rho, P})$ to denote the configuration $\confw{\sop\otimes\mathcal{I}_\Sigma(\rho), P}$, with $\mathcal{I}_\Sigma$ the identity superoperator on qubits $\Sigma$.

\begin{theorem}
	For any pair of configurations $\confw{\rho, P}, \confw{\sigma, Q}$ well-typed under $\Gamma; \Sigma$, 
	$\confw{\rho, P} \simqs \confw{\rho, P}$ implies
	\begin{enumerate}
		{\item for any $\sop \in \mathcal{TS}(\hilbert_{\overline{\Sigma}})$, $\sop (\confw{\rho, P})\ \simqs \ \sop (\confw{\rho, P})$\label{point:thmchinese1_quantum}}
		{\item $tr_{\Sigma}(\rho) = tr_{\Sigma}(\sigma)$. \label{point:thmchinese2_quantum}}
	\end{enumerate}
\end{theorem}
\begin{proof}
 To prove point \ref{point:thmchinese1_quantum}, suppose $\confw{\rho, P} \simqs \confw{\sigma, Q}$, with $\rho, \sigma \in \calH_{\widetilde{q},\widetilde{p}}$ and $\Gamma;\widetilde{p} \vdash P$, $\Gamma;\widetilde{p} \vdash Q$.  For any superoperator $\sop \in \mathcal{TS}(\hilbert_{\widetilde{q}})$ we can construct a context $$B[\blank] = [\blank] \parallel \sop({\widetilde{q}}).a!0 \parallel c!\widetilde{q}$$ where $a$ is a fresh channel. We know that $\confw{\rho, B[P]}$ and $\confw{\sigma, B[Q]}$ are bisimilar, and$\confw{\rho, B[P]}$ can evolve in $\confw{\sop\otimes\mathcal{I}_{\widetilde{p}}(\rho), a!0 \parallel c!\widetilde{q} \parallel P}$. Then $\confw{\sigma, B[Q]}$ must necessarily evolve in $\confw{\sop\otimes \mathcal{I}_{\widetilde{p}}(\sigma), a!0 \parallel c!\widetilde{q} \parallel Q}$, because it must match the $\downarrow_a$ barb, and $a$ is fresh. So we have that  
\[ \confw{\sop\otimes\mathcal{I}_{\widetilde{p}}(\rho), a!0 \parallel c!\widetilde{q} \parallel P} \simqs 
  \confw{\sop\otimes \mathcal{I}_{\widetilde{p}}(\sigma), a!0 \parallel c!\widetilde{q} \parallel Q}
\]
and from this it follows 
\[ \confw{\sop\otimes\mathcal{I}_{\widetilde{p}}(\rho), P} \simqs 
  \confw{\sop\otimes \mathcal{I}_{\widetilde{p}}(\sigma), Q}
\] simply by contradiction: if there was a context capable of distinguishing $\sop(\rho, P)$ from $\sop(\rho, Q)$ then there would be a context able to distinguish also $\sop(\rho, P \parallel a!0 \parallel c!\widetilde{q})$ from $\sop(\sigma, Q \parallel a!0 \parallel c!\widetilde{q})$

To prove point \ref{point:thmchinese2_quantum} we proceed by contradiction, supposing  $\confw{\rho, P} \simqs \confw{\sigma, Q}$  and $tr_{\widetilde{p}}(\rho) \neq tr_{\widetilde{p}}(\sigma)$, with a $\rho, \sigma \in \calH_{\widetilde{q}, \widetilde{p}}$ and $\Gamma;\widetilde{p} \vdash P$, $\Gamma;\widetilde{p} \vdash Q$.
 If $tr_{\widetilde{p}}(\rho) \neq tr_{\widetilde{p}}(\sigma)$, then there exists a measurement $M_{\widetilde{q}} = \set{M_1, \ldots ,M_{m}}$ that distinguishes them, i.e. such that $p_m(tr_{\widetilde{p}}(\rho)) = tr(M_m tr_{\widetilde{p}}(\rho) M_m^\dagger) \neq tr(M_m tr_{\widetilde{p}}(\sigma) M_m^\dagger) = p_m(tr_{\widetilde{p}}(\sigma))$ for some $m$. But for lemma \ref{trace_and_sop}, the same probabilities arise also from the measurement $M_{\widetilde{q}\widetilde{p}} = \set{M_1 \otimes I_{\widetilde{p}}, \ldots , M_m \otimes I_{\widetilde{p}}}$, that  can therefore distinguish the whole state $\rho$ from $\sigma$. So, taken the context 
\[[\blank] \parallel M[\widetilde{q} \rhd x].\big(disc(\widetilde{q}) \parallel \ite{x = 1}{c_1!0}{\ldots \ite{x = m-1}{c_{m-1}!0}{c_m!0}}\big) \]
where $c_0 \ldots c_m$ are fresh channels, we have that $\confw{\rho, B[P]}$ should be bisimilar to $\confw{\sigma, B[Q]}$. But 
\[\confw{\rho, B[P]} \rightarrow \sum_m p_m(\rho) \confw{\rho_m, c_m!0}\]
and $\confw{\sigma, B[Q]}$ can perform only the transition \[\confw{\sigma, B[Q]} \rightarrow \sum_m p_m(\sigma) \confw{\sigma_m, c_m!0}\] to match the barbs, but we know that $p_m(\rho) \neq p_m(\sigma)$ for at least one $m$.
\end{proof}

We can actually prove a stronger property, that in lqCCS the action of "external" superoperators $1sop \in \mathcal{TS}(\calH_{\overline{\Sigma}})$ does not change the observable behaviour of a quantum system.

By writing $\sop \in TSO(\hilbert_{\Sigma})$, we mean that $\sop(\rho) = \sum_{i \in I} (I_{\overline{\Sigma}} \otimes A_i) \rho (I_{\overline{\Sigma}} \otimes A_i)^{\dagger}$.

\begin{lemma}\label{lemma:sop}
\note{da rivedere}
	For any pair of configurations $\conf, \conf'$ well-typed under $\Gamma; \Sigma$, and for any $\sop \in TSO(\hilbert_{\overline{\Sigma}})$,
	$\conf \rightarrow \Delta$ iff $\sop(\conf) \rightarrow \sop(\Delta)$.
\end{lemma}
\begin{proof}
	The only non trivial rules are {\scshape SemQOp} and {\scshape SemQMeas}.
	Assume without loss of generality that $env(\conf) \in \hilbert_{\Sigma \otimes \overline{\Sigma}}$.
	By {\scshape SemQOp}, $\conf = \langle \rho, \mathcal{E}(\widetilde{x}) . P \rangle \longrightarrow \langle \mathcal{E}_{\widetilde{x}}(\rho), P \rangle = \Delta$.
	The type system ensures that $\tilde{x} \subseteq \Sigma$.
	We assume without loss of generality that $\tilde{x} = \Sigma$.
	We can also apply {\scshape SemQOp} as follows, $\sop(\conf) = \langle \sop(\rho), \mathcal{E}(\widetilde{x}) . P \rangle  \longrightarrow \langle \mathcal{E}_{\widetilde{x}}(\sop(\rho)), P \rangle$.
	We need to prove that $\langle \mathcal{E}_{\widetilde{x}}(\sop(\rho)), P \rangle = \langle \sop(\mathcal{E}_{\widetilde{x}}(\rho)), P \rangle = \sop(\Delta)$, i.e., that $\sop_{\widetilde{x}}(\sop(\rho)) = \sop(\sop_{\widetilde{x}}(\rho))$.
	By definition, $\sop_{\widetilde{x}}(\rho) = \sum_{i = 1}^{n} (I_{\Sigma} \otimes A_i) \rho (I_{\Sigma} \otimes A_i)^{\dagger}$, and $\sop(\rho) = \sum_{j = 1}^{m} (B_j \otimes I_{\overline{\Sigma}}) \rho (B_j \otimes I_{\overline{\Sigma}})^{\dagger}$.

	By linearity 
	\begin{align*}
	&\sop_{\widetilde{x}}(\sop(\rho)) =\\ 
	&\sum_{i = 1}^{n} (I_{\Sigma} \otimes A_i) (\sum_{j = 1}^{m} (B_j \otimes I_{\overline{\Sigma}}) \rho (B_j \otimes I_{\overline{\Sigma}})^{\dagger}) (I_{\Sigma} \otimes A_i)^{\dagger} =\\
	&\sum_{i = 1}^{n} \sum_{j = 1}^{m} (I_{\Sigma} \otimes A_i) (B_j \otimes I_{\overline{\Sigma}}) \rho (B_j \otimes I_{\overline{\Sigma}})^{\dagger} (I_{\Sigma} \otimes A_i)^{\dagger},
	\end{align*}
	and
	\begin{align*}
	&\sop(\sop_{\widetilde{x}}(\rho)) =\\
	&\sum_{j = 1}^{m} (B_j \otimes I_{\overline{\Sigma}}) (\sum_{i = 1}^{n} (I_{\Sigma} \otimes A_i) \rho (I_{\Sigma} \otimes A_i)^{\dagger}) (B_j \otimes I_{\overline{\Sigma}})^{\dagger} =\\
	&\sum_{i = 1}^{n} \sum_{j = 1}^{m} (B_j \otimes I_{\overline{\Sigma}}) (I_{\Sigma} \otimes A_i) \rho (I_{\Sigma} \otimes A_i)^{\dagger} (B_j \otimes I_{\overline{\Sigma}})^{\dagger}.
	\end{align*}
	Thanks to conjugate properties, it is thus sufficient to show that
	\[
	(B_{p\times p} \otimes I_{q\times q}) (I_{p\times p} \otimes A_{q\times q}) = (I_{p\times p} \otimes A_{q\times q}) (B_{p\times p} \otimes I_{q\times q}).
	\]
	
	This is easily proven thanks to the mixed product property of the Kronecker product, telling us that 
	\[ (A \otimes B)(C \otimes D) = (AC)\otimes(BD)
	\]
	so in our case, we have 
	\[
	(B_{p\times p} \otimes I_{q\times q}) (I_{p\times p} \otimes A_{q\times q}) = B_{p\times p} \otimes A_{q \times q} = (I_{p\times p} \otimes A_{q\times q}) (B_{p\times p} \otimes I_{q\times q})
	\]	
The proof for rule {\scshape SemQMeas} is the same, considering every $m \in \{0, \dots, 2^{\widetilde{x}}\}$ separately.
\end{proof}


\begin{example}
	Consider the two following configurations.
	\[
		\confw{\Phi^+, M_{01}[q_0 \triangleright x].discard(q_0)}\qquad \confw{\Phi^+, M_{+-}[q_0 \triangleright x].discard(q_0)}	
	\]
	In [cinesi1] these are not bisimilar since the standard lifting of the relation is used, which is decomposable.
\end{example}


\begin{example}
Let $\rho = \frac{1}{2} \ketbra{0}{0}$ and $\rho' = \frac{1}{2} \ketbra{1}{1}$, then
\begin{align*}
	\langle \rho, P \rangle \boxplus \langle \rho', P \rangle \equiv \langle \rho + \rho', P \rangle \sim \langle \frac{1}{2} I, P \rangle
\end{align*}
In particular, the following holds in our system but not in~\cite{Feng:2012, Deng:2012}
\begin{align*}
	Set_{\ketbra{+}{+}}(q).M(q \triangleright x).c!0 \sim Set_{\frac{1}{2} I}(q).\tau.c!0
\end{align*}
\end{example}

\subsection{Mixed configuration CQP}
noi abbiamo il ite, quindi vere distribuzioni, non solo mixed





\chapter{Minimal Process Calculus}
\newcommand{\quantumdst}{\mathfrak{D}}
\newcommand{\quantumdsta}{\mathfrak{T}}

Given the difficulties of finding a good notion of behavioural equivalence in quantum process algebras, we consider to address the problem on a minimal setting, only considering basic constructs.

\subsection{Mathematical Preliminaries}

A quantum distribution $\quantumdst \in D(S)^\hilbert$ over a set $S$ is a function from the finite-dimensional Hilbert space $\hilbert$ of dimension $n$ to probability distributions $\Delta$ over $S$.
In the following, we write $K_{\Delta}$ defined as the function that always returns $\Delta$ for every state $\ket{\phi} \in \hilbert$.

Let $\Set{P_i | \sum_{i = 1}^{n} P_i = I }$ be a set of quantum projectors, and let $\quantumdst_i$, $1 \leq i \leq n$, be a collection of quantum distributions.
We use $\sum_{i = 1}^{n} P_i \quantumdst_i$ to denote the distribution determined by 
\[
\left(\sum_{i = 1}^{n} P_i \quantumdst_i\right) (\ket{\phi}) = \sum_{i = 1}^{n} p_i(\ket{\phi}) \quantumdst_i \left(\frac{P_i \ket{\phi}}{p_i(\ket{\phi})}\right)
\]
where $p_i(\ket{\phi}) = \bra{\phi} P_i \ket{\phi}$.

Note that for any $\ket{\phi}$, ($\sum_{i = 1}^{n} P_i \quantumdst_i) (\ket{\phi})$ is a legal probability distribution since $\quantumdst_i(\ket{\psi})$ is a probability distribution and
\[
\sum_{i = 1}^{n} p_i (\ket{\phi}) = \sum_{i = 1}^{n} \bra{\phi} P_i \ket{\phi} = \bra{\phi} I \ket{\phi} = \braket{\phi | \phi} = 1.
\]

When $n = 2$, $P_2$ is derivable as $I - P_1$, thus we write $\quantumdst_1 \qsum{P_1} \quantumdst_2$ for $\sum_{i = 1}^{n} P_i \quantumdst_i$.
The operator $\blank \qsum{P} \blank$ can be defined from $\blank \tensor[_p]{\oplus}{} \blank$ as follows:
\[
  (\quantumdst \qsum{P_1} \quantumdsta) (\ket{\phi}) = \quantumdst \left(\frac{P_1 \ket{\phi}}{p_1(\ket{\phi})}\right) \tensor[_{p_1(\ket{\phi})}]{\oplus}{} \quantumdsta \left(\frac{P_2 \ket{\phi}}{p_2(\ket{\phi})}\right)
\]

We define the common notions of linearity and decomposability as usual.
\begin{definition}
We say that a relation $\rel \subseteq D(S)^\hilbert \times D(S)^\hilbert$ is \emph{linear} over $\blank \boxplus \blank$ if $\quantumdst_i\,\rel\,\quantumdsta_i$, $i = 1,2$, implies $(\quantumdst_1 \qsum{P} \quantumdst_2)\,\rel\,(\quantumdsta_1 \qsum{P} \quantumdsta_2)$ for any $P$.
We say that a relation $\rel \subseteq D(S)^\hilbert \times D(S)^\hilbert$ is \emph{left-decomposable} over $\blank \boxplus \blank$ if $(\quantumdst_1 \qsum{P} \quantumdst_2)\ \rel\ \quantumdsta$ implies $\quantumdsta = (\quantumdsta_1 \qsum{P} \quantumdsta_2)$ where $\quantumdst_i\ \rel\ \quantumdsta_i$, for $i = 1, 2$.
\emph{Right-decomposable} relations are defined as expected, and a relation is \emph{decomposable} if it is both left- and right-decomposable. 
\end{definition}

The quantum lifting of a relation is defined as follows.
\begin{definition}
	Let $\rel$ be a relation in $S \times S$, we define its quantum lifting $\qlift{\rel}$ as a relation in $D(S)^\hilbert \times D(S)^\hilbert$ as the minimal relation such that
	\begin{enumerate}
		\item $s\ \rel\ s'$ implies $K_{\overline{s}} \qlift{\rel} K_{\overline{s'}}$; and
		\item $\quantumdst_i \qlift{\rel} \quantumdsta_i$, i = $1, 2$, implies $\quantumdst_1 \qsum{P} \quantumdst_2 \qlift{\rel} \quantumdsta_1 \qsum{P} \quantumdsta_2$ for any projector $P$.
	\end{enumerate}
\end{definition}

%The quantum lifting of a function is linear by definition.
%We prove it is also decomposable.
%\begin{proposition}
%	Let $\rel$ be a relation in $S \times S$, its quantum lifting $\qlift{\rel}$ is decomposable.
%\end{proposition}
%\begin{proof}
%	We proceed by induction on the derivation of $\quantumdst \qlift{\rel} \quantumdsta$.
%	Case \textit{1.} is trivial, $K_{\overline{s}}$ can be decomposed only as $K_{\overline{s}} \qsum{P} K_{\overline{s}}$, or as $K_{\overline{s}} \qsum{I} \quantumdsta'$ for any $\quantumdsta'$, and the same for $K_{\overline{s'}}$.
%	In case \textit{2.}, let 
%	Assume $(\quantumdst_1 \qsum{P} \quantumdst_2) \qlift{\rel} \quantumdsta$, 
%\end{proof}
%
%\begin{proposition}
%	if $\rel$ is an equivalence relation, then $\qlift{\rel}$ is an equivalence relation.
%\end{proposition}
%\begin{proof}
%	Symmetry and reflexivity are trivial.
%	We prove transitivity by induction on the derivations of $\quantumdst \qlift{\rel} \quantumdsta$.
%	The only axiom is $K_{\overline{s}} \qlift{\rel} K_{\overline{s'}}$ if $s \rel s'$.
%	Assume 	
%\end{proof}

\begin{proposition}
	Let $\rel$ be a relation in $S \times S$, and let $\quantumdst, \quantumdsta$ be quantum distributions such that $\quantumdst \qlift{\rel} \quantumdsta$.
	Then, for any $\ket{\phi}$, $\quantumdst(\ket{\phi}) \slift{\rel} \quantumdsta(\ket{\phi})$.
\end{proposition}
\begin{proof}
	We proceed by induction on the derivation of $\quantumdst \qlift{\rel} \quantumdsta$.
	Case \textit{1.} is trivial, for any $\ket{\phi}, K_{\overline{s}}(\ket{\phi}) = \overline{s}$, and $K_{\overline{s'}}(\ket{\phi}) = \overline{s'}$, and $s\ \rel\ s'$ implies $\overline{s} \slift{\rel} \overline{s'}$.
	For case \textit{2.}, we have that
	\begin{align*}
	(\quantumdst_1 \qsum{P_1} \quantumdst_2) (\ket{\phi}) = \quantumdst_1 \left(\frac{P_1 \ket{\phi}}{p_1(\ket{\phi})}\right) \tensor[_{p_1(\ket{\phi})}]{\oplus}{} \quantumdst_2 \left(\frac{P_2 \ket{\phi}}{p_2(\ket{\phi})}\right)\\
	(\quantumdsta_1 \qsum{P_1} \quantumdsta_2) (\ket{\phi}) = \quantumdsta_1 \left(\frac{P_1 \ket{\phi}}{p_1(\ket{\phi})}\right) \tensor[_{p_1(\ket{\phi})}]{\oplus}{} \quantumdsta_2 \left(\frac{P_2 \ket{\phi}}{p_2(\ket{\phi})}\right)
	\end{align*}
	By induction hypothesis we know that $\quantumdst_i \left(\frac{P_i \ket{\phi}}{p_i(\ket{\phi})}\right) \slift{\rel} \quantumdsta_i \left(\frac{P_i \ket{\phi}}{p_i(\ket{\phi})}\right)$, for $i = 1,2$, and the thesis holds by linearity.
\end{proof}


%Given a relation $\rel \subseteq D(S) \times D(S)$, we call $\text{ext}(\rel)$ its \emph{point-wise extension} such that $\quantumdst\,\text{ext}(\rel)\,\quantumdsta$ if and only if $\forall \ket{\phi}\ldotp \quantumdst(\ket{\phi})\,\rel\,\quantumdsta(\ket{\phi})$.
%We say that a relation $\rel \subseteq D(S)^\hilbert \times D(S)^\hilbert$ is \emph{point-wise} whenever a relation $\rel_0 \subseteq D(S) \times D(S)$ exists such that $\rel = ext(\rel_0)$.  
%
%\begin{proposition}
%	Given any relation $\rel \subseteq D(S) \times D(S)$, $\text{ext}(\rel)$ is \emph{linear} over $\blank \boxplus \blank$ if and only if $\rel$ is linear over $\blank \oplus \blank$.
%\end{proposition}

%\paragraph{Physical Consistency}
%
%Not all quantum distributions are physically implementable, as a trivial example, take any $\Delta$ such that $\Delta(\ket{\phi}) \neq \Delta(-\ket{\phi})$.
%Clearly, since all the normalized vectors $\ket{\phi}, -\ket{\phi}, i \ket{\phi}, -i \ket{\phi}$ represent the same quantum state, $\Delta$ is not consistent with physical law.

%A quantum boh $\delta \in d_q(S)$ over a set $S$ is a function from the Hilbert space $\hilbert$ to $S$ such that:
%\begin{itemize}
%	\item for each $s \in S$, the constant function $\delta(\phi) = s$ is in $d_q(S)$;
%	\item for each unitary transformation $U$, if $\delta \in d_q(S)$ then $\delta'$ is also in $d_q(S)$ with $\delta'(\ket{\phi}) = U \delta(\ket{\phi})$;
%	\item for any projector $P$, if $\delta \in d_q(S)$ then $\delta'$ is also in $d_q(S)$ with $\delta'(\ket{\phi}) = \delta(\frac{P \ket{\phi}}{p(\ket{\phi}})$;
%\end{itemize}
%
%A quantum distribution $\Delta \in D_q(S)$ over a set $S$ is \emph{legal} if and only if:



\subsection{Quantum Labeled Transition Systems}

A quantum labeled transition system QLTS on an Hilbert space $\hilbert$ is a triple $(S, Act_\tau, \rightarrow)$ where
\begin{itemize}
	\item $S$ is a set of states $s, s_1, \dots$;
	\item $Act_\tau$ is a set of transition labels with $\tau$ a distinguished element;
	\item $\rightarrow\;\subseteq S \times Act_\tau \times D(S)^\hilbert$ is the transition relation. 
%	such that for each $\delta \xrightarrow{\mu} \Delta$ either:
%	\begin{enumerate}
%		\item $\Delta = \bar{\delta'}$ for some $\delta'$ and a unitary matrix $U$ exists such that $\delta'(U \ket{\phi}) = \delta(U \ket{\phi})$ for any $\ket{\phi}$;	
%	\end{enumerate}
\end{itemize}

We define a minimal quantum process algebra (mQPA) for describing quantum processes.
A quantum process $Q$ is defined as
\[
Q ::= \nil \mid \mu.Q \mid Q + Q \mid U \circ Q \mid Q \qsum{P} Q
\]
where $U$ is a unitary transformation over $\hilbert$.

We give the semantics of mQPA in terms of QLTS.
Some terms are taken as states $s \in S$, in particular the ones where unitary operators and $\blank \boxplus \blank$ are guarded.
\[
s ::= \nil \mid \mu.Q \mid s + s
\]
%We write $\sema{s}$ for the function $S^\hilbert$ they stand for, i.e., 
%\begin{align*}
%	&\sema{s}(\ket{\phi}) = (\ket{\phi}, s)
%\end{align*}

The interpretation of an arbitrary term $Q$ as quantum distribution $\sem{Q}$ over S is given by the function $\sem{\blank}$:
\begin{align*}
	&\sem{\nil}(\ket{\phi}) = \overline{\nil}\\
	&\sem{\mu.Q}(\ket{\phi}) = \overline{\mu.Q}\\
	&\sem{Q_1 + Q_2}(\ket{\phi})(s) = 
	\begin{cases}
		\sem{Q_1}(\ket{\phi})(s_1) \cdot \sem{Q_2}(\ket{\phi})(s_2) & \text{if } s = s_1 + s_2\\
		0 & \text{otherwise}
	\end{cases}\\
	&\sem{U \circ Q}(\ket{\phi}) = \sem{Q}(U \ket{\phi})\\
	&\sem{Q_1 \qsum{P} Q_2} = \sem{Q_1} \qsum{P} \sem{Q_2}
\end{align*}

The following proposition is trivially derivable by definition.
\begin{proposition}
	For any $s \in S$, $\sem{s} = K_{\overline{s}}$.
\end{proposition}

The transition relation $\to$ is defined as follows, with $s \xrightarrow{\mu} \Delta$ as notation for $(s, \mu, \Delta) \in\;\to$.
\begin{gather*}
  \infer[\mbox{\footnotesize\scshape Action}]{\mu.Q \xrightarrow{\mu} \sem{Q}}{} \qquad 
  \infer[\mbox{\footnotesize\scshape Ext.L}]{Q_1 + Q_2 \xrightarrow{\mu} \Delta}{Q_1 \xrightarrow{\mu} \Delta} \qquad
  \infer[\mbox{\footnotesize\scshape Ext.R}]{Q_1 + Q_2 \xrightarrow{\mu} \Delta}{Q_2 \xrightarrow{\mu} \Delta}
\end{gather*}

We define bisimularity on QLTS as usual.
\begin{definition}
	A relation $\rel : S \times S$ is called a QLTS \emph{bisimulation} if it's symmetric and for each pair of states $s, s' \in S$ such that $s \rel s'$,
	if $s \xrightarrow{\mu} \Delta$ then $s' \xrightarrow{\mu} \Delta'$ and $\Delta\,\qlift{\rel}\,\Delta'$, for some quantum distributions $\Delta, \Delta' \in D(S)$.
	\emph{Q-bisimilarity} $\sim_Q$ is the largest QLTS bisimulation.
\end{definition}


\subsubsection{Testing Bisimilartiy over problematic Cases}

We take the example 6, 7, 8, and show that they behave as expected in mQPA.

For examples 6 and 7 consider the following.
\begin{example}
	A qbit which is in a mixed state of $\ket{0}$ and $\ket{1}$ with equal probability behaves exactly as a qbit which is in a mixed state of $\ket{+}$ and $\ket{-}$.
	Consider the two following processes.
	\begin{align*}
		Q &= (s \qsum{\ketbra{0}{0}} s) \qsum{\ketbra{+}{+}} (s \qsum{\ketbra{0}{0}} s)\\
		Q' &= (s \qsum{\ketbra{+}{+}} s) \qsum{\ketbra{0}{0}} (s \qsum{\ketbra{+}{+}} s)
	\end{align*}
	Note that in $Q$ (in $Q'$ resp.), after the nested measure, the qbit is in state $\ket{0}$ or $\ket{1}$ ($\ket{+}$ or $\ket{-}$ resp.) with equal probability.
	Indeed, $Q$ and $Q'$ stand for the same quantum distribution: $\sem{Q} = \sem{Q'} = K_{\bar{s}}$.
\end{example}

Example 8 is addressed as follows.
\begin{example}
	Consider the bell pair $\ket{\Phi^+} = 1/\sqrt{2}(\ket{00} + \ket{11})$.
	Let $Q$, $Q'$, $P$ and $P'$ be as follows
	\begin{align*}
		Q &= z \qsum{P_{\_0}} u\\
		Q' &= p \qsum{P_{\_+}} m\\
		P_0 &= ((Q + Q') \qsum{P_{\_0}} (Q + Q')) \qsum{P_{\ketbra{\Phi^+}{\Phi^+}}} \nil\\
		P_+ &= ((Q + Q') \qsum{P_{\_+}} (Q + Q'))  \qsum{P_{\ketbra{\Phi^+}{\Phi^+}}} \nil
	\end{align*}
	Where $P_{\_0}$ and $P_{\_+}$ are defined as $\ketbra{00}{00} + \ketbra{10}{10}$ and $\ketbra{0+}{0+} + \ketbra{1+}{1+}$ respectively.
	Note that $\sem{P_0} \qlift{\sim} \sem{P_+}$ iff they behave the same on $\ket{\Phi^+}$, and they should, because the mixed states obtained by measuring a single qbit on the computational base, and on the Hadamard base are the same.

	Indeed $\sem{P_0} = \sem{P_+}$.
	Take any $\ket{\phi}$,
		\begin{align*}
		\sem{P_0}(\ket{\phi}) &= 
		\sem{((Q + Q') \qsum{P_{\_0}} (Q + Q'))}(\ket{\Phi^+}) \psum{p} \bar{\nil}\\
		&= (\sem{Q + Q'}(\ket{00}) \psum{1/2} \sem{Q + Q'}(\ket{11})) \psum{p} \bar{\nil}\\
		&= (((\bar{z} + \bar{p}) \psum{1/2} (\bar{z} + \bar{m})) \psum{1/2} ((\bar{u} + \bar{p}) \psum{1/2} (\bar{u} + \bar{m}))) \psum{p} \bar{\nil}\\
		&= (((\bar{z} + \bar{p}) \psum{1/2} (\bar{u} + \bar{p})) \psum{1/2} ((\bar{z} + \bar{m}) \psum{1/2} (\bar{u} + \bar{m}))) \psum{p} \bar{\nil}\\		
		&= (\sem{Q + Q'}(\ket{++}) \psum{1/2} \sem{Q + Q'}(\ket{--})) \psum{p} \bar{\nil}\\
		&= \sem{((Q + Q') \qsum{P_{\_+}} (Q + Q'))}(\ket{\Phi^+}) \psum{p} \bar{\nil} = \sem{P_+}(\ket{\phi}).
		\end{align*}
\end{example}


\subsubsection{Alternative}
We define an alternative characterization of a quantum transition system that is more in-line with preexisting quantum transition systems like~\cite{Feng:2012, Deng:2012}.
A quantum labeled transition system qLTS on an Hilbert space $\hilbert$ is a triple $(S, Act_\tau, \hookrightarrow)$ where
\begin{itemize}
	\item $S$ is a set of states $s, s_1, \dots$;
	\item $Act_\tau$ is a set of transition labels with $\tau$ a distinguished element;
  \item $\hookrightarrow\;\subseteq (\hilbert \times S) \times Act_\tau \times D(\hilbert \times S)$ is the transition relation. 
%	such that for each $\delta \xrightarrow{\mu} \Delta$ either:
%	\begin{enumerate}
%		\item $\Delta = \bar{\delta'}$ for some $\delta'$ and a unitary matrix $U$ exists such that $\delta'(U \ket{\phi}) = \delta(U \ket{\phi})$ for any $\ket{\phi}$;	
%	\end{enumerate}
\end{itemize}

To simplify our presentation, we reduce to $Q$ and $S$ of a specific form, namely where the branches of non-deterministic choices are always guarded by a transition, like in $(\tau. U \circ \nil) + (\alpha . U' \circ \nil)$.
Note that this is consistent with the behaviour of preexisting quantum process algebras like~\cite{Feng:2012, Deng:2012}.
We stress that a term is in this specific form by writing $\hat{Q}$ or $\hat{S}$.

We define the interpretation of a pair $(\ket{\phi}, Q)$ as a distribution as given by the function $\sema{\blank} : (\hilbert \times S) \to D(\hilbert \times S)$:
\begin{align*}
	&\sema{\ket{\phi}, s} = \overline{(\ket{\phi}, s)} \\
	&\sema{\ket{\phi}, U \circ \hat{Q}} = \sema{U\ket{\phi}, \hat{Q}} \\
	&\sema{\ket{\phi}, Q_1 \qsum{P} \hat{Q_2} } = \sema{P \ket{\phi}, \hat{Q_1}} \psum{p(P, \ket{\phi})} \sema{P^{\bot}\ket{\phi}, \hat{Q_2}} 
\end{align*}

The transition relation $\hookrightarrow$ is defined as follows, with $s \xhookrightarrow{\mu} \Delta$ as notation for $(s, \mu, \Delta) \in\;\hookrightarrow$.
\begin{gather*}
  \infer[\mbox{\footnotesize\scshape Action}]{(\ket{\phi}, \mu.Q) \xhookrightarrow{\mu} \sema{\ket{\phi}, Q}}{} \\[0.3cm]
  \infer[\mbox{\footnotesize\scshape Ext.L}]{(\ket{\phi}, Q_1 + Q_2) \xhookrightarrow{\mu} \Delta}{(\ket{\phi}, Q_1) \xhookrightarrow{\mu} \Delta} \qquad
  \infer[\mbox{\footnotesize\scshape Ext.R}]{(\ket{\phi}, Q_1 + Q_2) \xhookrightarrow{\mu} \Delta}{(\ket{\phi}, Q_2) \xhookrightarrow{\mu} \Delta}
\end{gather*}

%\subsubsection{Bisimulation}

\begin{definition}
	A relation $\rel : (\hilbert \times S) \times (\hilbert \times S)$ is a qLTS \emph{bisimulation} if it's symmetric and for each pair $(\ket{\phi}, s), (\ket{\phi'}, s)$ such that $(\ket{\phi}, s) \rel (\ket{\phi'}, s)$,
	if $(\ket{\phi}, s) \xrightarrow{\mu} \Delta$ then $(\ket{\phi}', s') \xrightarrow{\mu} C'$ and $C \slift{\rel} \Delta'$.
	\emph{q-bisimilarity} $\sim_q$ is the largest qLTS bisimulation.
\end{definition}

We now establish a common behaviour result between QLTS and qLTS.
\begin{theorem}
  $\forall s, s' \in \hat{S}$, if $s \sim_Q s'$ then $s \sim_q s'$.
\end{theorem}
\begin{proof}
  ($\Longleftarrow$) Let $\rel \subseteq S \times S$ be a relation such that $s\,\rel\,s'$ iff $\forall \ket{\phi} \ldotp (\ket{\phi}, s)\,\rel_{\ket{\phi}}\,(\ket{\phi}, s')$,
  for some set of relations $\rel_{\ket{\phi}} \subseteq (H \times S) \times (H \times S)$.
  Assume $s \xrightarrow{\mu} \Delta$ then $s \equiv \mu.Q + Q'$, without loss of generality we can assume $s = \mu.Q$,
  thus $\forall \ket{\phi} \ldotp (\ket{\phi}, s) \xrightarrow{\mu} \sema{\ket{\phi}, Q}$ and $\Delta = \sem{Q} \ldots$.
\end{proof}


\chapter{Conclusions and Future work}
We have explored the main  quantum process calculi proposed in the literature, focusing on the quantum-related design choices underlying them, which lead to fairly different notions of behavioural equivalence.

One of the discriminating factor between calculi, namely the visibility of qubits, stems from the intrinsic ambiguity of the proposed syntaxes, and on how it affects the modelling of real systems. We have enriched lqCCS with a linear type system, that eliminates this ambiguity. Thanks to this result, lqCCS processes are interpreted in the same way by all the considered proposed calculi, allowing us to compare the behavioural equivalences they propose. Probabilistic saturated bisimilarity has been introduced, to capture what can or cannot be distinguished by an external observer.

Another discriminating factor, i.e. how to compare quantum values, has instead proven to be a significant quantum related detail that cannot be reduced to a syntactical ambiguity, as it reflects foundational assumptions on observable properties of quantum systems. Perhaps surprisingly, the crucial detail that separates qCCS and CQP is not in how they specify quantum properties of a configuration, but in how these quantum properties are lifted to properties of probabilistic distributions. Indeed the peculiar characteristics of quantum computing allows different distributions (i.e. ensembles) to have the same observable properties (i.e. mixed states). To model these defining properties of quantum theory, we have relaxed the conditions of Larsen-Skou bisimilarity, introducing a quantum equivalence relation between distributions. Thanks to this novelty, Quantum Saturated Bisimilarity satisfies some expected properties that were absent in qCCS. For example, measurements and superoperator have different semantics, but may yield bisimilar transition systems. 

We have also proposed a minimal process algebra that abstracts away from most classical details and only focus on quantum behaviour. In this simplified setting, we give an entirely new, purely quantum-based notion of semantics and bisimilarity, and we prove it behaves well in some previously discussed problematic cases. 

\subsection*{Future Work}
Linear qCCS and probabilistic/quantum saturated bisimilarity have raised a number of interesting questions and challenges, on both practical and foundational aspects of quantum process calculus.

We only focused on strong saturated bisimilarity, while the other calculi in the literature propose also weak and branching bisimilarity. We leave the extension of our work to the non strong case as a future work. To define the needed transitive closure of the $\rightarrow$ transition relation, we could adopt the distribution transformer approach of qCCS, yielding to a Segala-style probabilistic weak saturated bisimilarity. The resulting bisimilarity could be formulated as a bisimilarity between distributions, like in \cite{hennessyExploringProbabilisticBisimulations2012}, instead of a bisimilarity between configurations. Distribution bisimilarity will allow us to avoid basing our future extension on the supposed
problematic Larsen-Skou lifting, but the tricky interaction between non-determinism and quantum probabilistic behaviour would still be relevant, and so we still expect probabilistic and quantum saturated bisimilarity not to coincide.

One disadvantage of quantum bisimilarity over probabilistic bisimilarity is that it is not an equivalence relation, as bisimulations are not closed for composition. One possible solution would be to define both the transition system and the bisimulation as relations on \textit{equivalence classes} of distributions, just like \cite{hennessyExploringProbabilisticBisimulations2012} defines transitions and bisimulation between distributions. In such a system, quantum saturated bisimilarity is an equivalence relation, and it should be possible to recover the same results obtained in this thesis.

Supposing that two configurations are saturated bisimilar is a strong hypothesis, not always easy to compare with labeled bisimlarity. Actually, lots of properties descend from saturated bisimilarity, but as a drawback it is also cumbersome to prove the bisimilairty between two arbitrary configurations. One of the most successful technique to help in this task is based on the definition of an equivalent Context-LTS, a labeled transition system with contexts as labels \cite{bonchiGeneralTheoryBarbs2014}. An intricacy is that, due to the inherently stateful nature of quantum computations, there are two different ways in which a process $P$ can interact with a context $B[\blank]$: they can synchronize on some channel, as in a classical process algebra, or one of the two can manipulates some (possibly entangled) qubits in the shared  underlying quantum state. Hence the results of \cite{bonchiGeneralTheoryBarbs2014} need to be adapted for the quantum case. In general, it would be interesting to explore which proof techniques for saturated bisimilarity would still be sound for a probabilistic quantum process calculus.
\printbibliography[
heading=bibintoc,
title={Bibliography}
]
\end{document}
