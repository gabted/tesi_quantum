\documentclass[10pt,a4paper, titlepage]{report}
\usepackage[margin=1.5in]{geometry}
\usepackage[utf8]{inputenc}
\usepackage{amsmath}
\usepackage{amssymb}
\usepackage{amsthm}
\usepackage{amsfonts}
\usepackage{amssymb}
\usepackage{multicol}
\usepackage{mathrsfs}
\usepackage{tikz}
\usepackage[bookmarks]{hyperref}
\usepackage[english]{babel}
\usepackage [autostyle]{csquotes}
\MakeInnerQuote{"}
\usepackage{braket}	
\usepackage{stmaryrd}
\usepackage{mathtools}
\usepackage{xfrac}
\usepackage{proof}
\usepackage{xfrac}
\usepackage{tensor}
\usepackage{enumerate}
\usepackage[
backend=bibtex,
style=numeric,
sorting=nyt
]{biblatex} 
\addbibresource{bibliography.bib}


\newcommand{\note}[1]{{\color{red} #1}}
\newcommand{\mono}[1]{\texttt{#1}}

%braket notation
\newcommand{\ketbra}[2]{\ket{#1}\!\!\bra{#2}}
\newcommand{\proj}[1]{\ketbra{#1}{#1}}
\newcommand{\kp}{\ket{\psi}}
\newcommand{\kf}{\ket{\phi}}
\newcommand{\kz}{\ket{0}}
\newcommand{\ko}{\ket{1}}
\newcommand{\kpl}{\ket{+}}
\newcommand{\km}{\ket{-}}
\newcommand{\oost}{\frac{1}{\sqrt{2}}}

%hilbert spaces, density matrices and sops
\newcommand{\calH}{\mathcal{H}}
\newcommand{\hilbert}{\calH}
\newcommand{\calHt}{\mathcal{H}^2}
\newcommand{\Hto}[1]{\mathcal{H}^{\otimes #1}}
\newcommand{\calDH}{\mathcal{D}(\mathcal{H})}
\newcommand{\calSH}{\mathcal{S}(\mathcal{H})}
\newcommand{\sop}{\mathcal{E}}
\newcommand{\cnot}{\text{CNOT}}

%process algebras
\newcommand{\ITE}[3]{\textbf{If }{#1}\textbf{ Then } {#2} \textbf{ Else } {#3}}
\newcommand{\qsum}[1]{\tensor[_{#1}]{\boxplus}{}}
\newcommand{\psum}[1]{\tensor[_{#1}]{\oplus}{}}
\newcommand{\nil}{\mathbf{0}}
\newcommand{\sem}[1]{\llbracket#1\rrbracket}
\newcommand{\sema}[1]{\llparenthesis\,#1\,\rrparenthesis}
\newcommand{\rel}{\mathcal{R}}
\newcommand{\lrel}{\mathring{\rel}}
\newcommand{\conf}{\mathcal{C}}
\newcommand{\confw}[1]{\langle#1\rangle}
\newcommand{\blank}{{-}}
\newcommand{\qlift}[1]{\stackrel{\scriptscriptstyle \boxplus}{#1}}
\newcommand{\slift}[1]{\stackrel{\scriptscriptstyle \circ}{#1}}
\newcommand{\barb}[3]{#3\downarrow_{#1, #2}}
\newcommand{\proc}[1]{\text{\textbf{#1}}}

\newcommand{\distr}{\mathcal{D}}



\newtheorem{theorem}{Theorem}%[section]
\newtheorem{corollary}{Corollary}%[section]
\newtheorem{lemma}{Lemma}%[section]
\newtheorem{definition}{Definition}%[section]
\newtheorem{proposition}{Proposition}%[section]
\newtheorem{example}{Example}%[section]
\newtheorem{assumption}{Assumption}%[section]

\title{Exploring Quantum Process Calculi via barbs and contexts }
\author{Gabriele Tedeschi}
\begin{document}

\maketitle

\tableofcontents

\chapter{Introduction}

\chapter{Background}
In this chapter, we review some fundamentals concepts in quantum computing and formal methods.

\section{Quantum Computing}

The laws on Quantum Mechanics, as we understand them, are elegantly formalized in a mathematical framework, built upon simple linear algebra. This framework is based on a few \textit{postulates} that describe the nature and evolution of quantum systems. Since quantum computing is just the technique of manipulating quantum systems to perform some computation, it necessarily follows the same postulates. Before presenting each postulate, we recall the necessary basic definition from linear algebra, formulated in the Dirac's "bra-ket" notation. For further reading, the standard textbook on the subject is \cite{nielsen_chuang_2010}.


\subsection{State space}

A \textit{column vector} in a complex vector space is written $\kp$, and it's called a "ket",
\[ \kp = \begin{pmatrix}
		\alpha_1\\
		\vdots\\
		\alpha_n
\end{pmatrix}
\]
where $\alpha_1, \ldots,  \alpha_n \in \mathbb{C}$. Its \textit{conjugate transpose} is written $\bra{\psi}$, and its called a "bra".
	\[
		\bra{\psi} = \kp^\dagger = (\alpha_1^* \ldots \alpha_n^*)
	\]

A (finite-dimensional) \textit{Hilbert space}, often denoted as $\cal{H}$, is a complex inner product space, i.e. a complex vector space equipped with a binary operator $\braket{  \,\_\, |\, \_\, }: \calH \times \calH \rightarrow \mathbb{C}$ called \textit{inner product}, dot product, or simply "braket".
\[
	\braket{\psi | \phi} = 
	\begin{pmatrix}
	\alpha_1^* \ldots \alpha_n^*
	\end{pmatrix}\begin{pmatrix}
	\beta_1 \\
	\vdots \\
	\beta_n
	\end{pmatrix} = 
	\sum_i \alpha_i^*\beta_i
\]

The inner product satisfies the following properties:
\begin{align*}
\text{Conjugate symmetry} & &  \braket{\psi | \phi} &= \braket{\phi | \psi}^* \\
\text{Linearity \quad\quad} & & \bra{\psi} (\alpha\ket{\phi} + \beta\ket{\varphi}) &= \alpha\braket{\psi | \phi} + \beta\braket{\psi | \varphi}\\
\text{Positive definiteness} & & \braket{\psi | \psi} &\geq 0
\end{align*}

Notice that $\braket{\psi|\psi} = 0$ if and only if $\kp$ is the $\mathbf{0}$ vector. Besides, thanks to conjugate symmetry, we have $\braket{\psi | \psi} = \braket{\psi | \psi} ^ *$, so $\braket{\psi | \psi}$ it's always a real, non-negative number, when $\kp \neq \mathbf{0}$.

Two vectors $\kp$ and $\kf$ are \textit{orthogonal} if
 \[\braket{\psi | \phi} = 0\]


The \textit{norm} of $\kp$ is defined as: 
\[ \|\kp\| = \sqrt{\braket{\psi|\psi}}\]


A \textit{unit vector} is a vector $\kp$ such that \[\|\kp\| = 1\].

A set of vectors $\{\kp_i\}_i$ is an \textit{orthonormal basis} of $\calH$ if \begin{itemize}
	\item each vector $\kf \in \calH$ can be expressed as a \textit{linear combination} of the vector in the basis, $\kf = \sum_i \alpha_i\kp_i$.
	\item All the vector in the basis are orthogonal
	\item All the vector in the basis are unit vector 
\end{itemize}

We're now ready to present the postulates of Quantum Mechanics, in the form more convenient for quantum computing.

\begin{quote} 
\textbf{Postulate I}: The state of an isolated physical system is represented, at a fixed time $t$, by a unit vector $\kp$, called the \textit{state vector}, belonging to a Hilbert space $\calH$, called the \textit{state space}. 
\end{quote}

 When describing the state of a quantum system, we ignore the \textit{global phase factor}\footnote{An equivalent formulation, in fact, describes a quantum system as a \textit{ray}, a one-dimensional subspace of $\calH$.}, i.e. 
 \[
 	\kp = \begin{pmatrix}
	\alpha \\
	\beta
	\end{pmatrix} = - \begin{pmatrix}
	\alpha \\
	\beta
	\end{pmatrix} = \lambda \begin{pmatrix}
	\alpha \\
	\beta
	\end{pmatrix} 
	\text{ for each $\lambda \in \mathbb{C}$  such that $|\lambda| = 1$}
 \]

The simplest, prototypical example of a quantum physical system is a \textit{qubit}: a qubit is a physical system with associated a two-dimensional Hilbert Space $\calH^2$. Such systems comprehend an electron in the ground or excited state, a vertically or horizontally polarized photon, or a spin up or spin down particle.

Taken for example a photon, we could say that the photon is in state $\kz$ when vertically polarized, and in state $\ko$ when is horizontally polarized, where $\kz$ and $\ko$ are the two unit vector of the Hilbert space defined as
\[
	\kz = \begin{pmatrix}
	1 \\
	0
	\end{pmatrix} \qquad
	\ko = \begin{pmatrix}
	0 \\
	1
	\end{pmatrix}
\]

The vectors $\{\kz, \ko\}$ form an orthonormal basis of $\ $, called the \textit{computational basis}. Since they form a basis, each vector $\kp \in \calH^2$ can be expressed as 
	\[\kp = \begin{pmatrix}
	\alpha \\
	\beta
	\end{pmatrix} = 
	\alpha\kz + \beta\ko
	\]
where $\alpha, \beta \in \mathbb{C}$ and $|\alpha|^2 + |\beta|^2 = 1$.

So, the state of any qubit can mathematically be described as $\kp = \alpha\kz +\beta\ko$, a linear combination of $\kz$ and $\ko$. From the physical point of view, this means that the qubit is in a \textit{quantum superposition} of state $\kz$ and $\ko$, like a photon being diagonally-polarized, or an electron being at the same time in the excited and in the ground state.

Other important vectors in the $\calHt$ state space are $\kpl$ and $\km$,
\begin{align*}
	\kpl = \oost &\begin{pmatrix}
	1 \\
	1
	\end{pmatrix}  = \oost\kz + \oost\ko   \\
	\km  = \oost &\begin{pmatrix}
	1 \\
	-1
	\end{pmatrix}  = \oost\kz - \oost\ko 
\end{align*}
that form the so called \textit{hadamard basis} of $\calHt$. As we will see, $\kpl$ and $\km$ are both an equal superposition of $\kz$ and $\ko$. The difference between $kpl$ and $km$ is the \textit{relative phase}, i.e. the phase between the $\kz$ and $\ko$ component. The states $\kz + \ko$, $-\kz -\ko$, $i\kz + i\ko$ are all equal to $\kpl$ times a certain global phase, and are considered the same state. The states $\kz + \ko$, $\kz - \ko$, $\kz +i\ko$, instead, all differs for a relative phase factor, and have a different behaviours when applied to the same computation. 
\subsection{Unitary Transformations}

For each linear operator $A$ acting on a Hilbert space $\calH$, we denote as $A^\dagger$ the \textit{adjoint} of $A$, i.e. the unique linear operator such that
\[
	\braket{\psi|A\phi} = \braket{A^\dagger\psi|\phi}
\]

A linear operator $A$ acting on a $n$-dimensional Hilbert space $\calH^n$ can be represented as a $n\times n$ matrix, and its adjoint is calculated as 
\[
	A^\dagger = (A^*)^T
\] the conjugate transpose of the matrix $A$.


If it holds that $A = A^\dagger$, we say that $A$ is self-adjoint, or \textit{Hermitian}.

A linear operator $U$ is said to be \textit{unitary} when $U^\dagger = U^{-1}$, which implies 
\[
	UU^\dagger = U^\dagger U = I
\]

Unitary matrices enjoy many useful properties, first of all that they have a spectral decomposition. An other defining characteristic is that they preserve the inner product, $\braket{\psi | \phi} = \braket{U\psi | U\phi}$
\[
	\braket{U\psi | U\phi} = \bra{\psi}U^\dagger U\ket{\psi} = \bra{\psi}I\ket{\psi} = \braket{\psi|\phi}
\]

A corollary of this property is that applying a unitary operator to a unit vector gives a unit vector
\[
	\braket{U\psi | U\psi} = \braket{\psi|\psi} = 1
\]

The following postulate makes obvious why we are interested in unitary transformation.
\begin{quote}
\textbf{Postulate II}: The evolution of a closed quantum system is described by a unitary transformation. That is, the state $\kp$ of the system at time $t_0$ is related to the state $\ket{\psi^\prime}$ of the system at time $t_2$ by a unitary operator $U$ which depends only on the times $t_0$ and $t_1$.
\[
	\ket{\psi^\prime} = U\kp
\]
\end{quote}

According to what we said before, if a physical system starts in a unit state, it will always remain in a unit state.

\note{No cloning theorem:}

In quantum computing, the the programmer can manipulate the state of a qubit, applying unitary transformations to it. Some of the most frequent transformation, implemented in every quantum computer, are:
\begin{gather*}
	X = \sigma_X = \begin{bmatrix}
	0 & 1 \\
	1 & 0
	\end{bmatrix} \qquad
	Y = \sigma_Y = \begin{bmatrix}
	0 & -i \\
	i & 0
	\end{bmatrix} \qquad
	Z = \sigma_Z = \begin{bmatrix}
	1 & 0 \\
	0 & -1
	\end{bmatrix} \\ 
	H = \oost\begin{bmatrix}
	1 & 1 \\
	1 & -1
	\end{bmatrix} \qquad
 	S = \begin{bmatrix}
	1 & 0 \\
	0 & i
	\end{bmatrix}\qquad
	T = \begin{bmatrix}
	1 & 0 \\
	0 & e^{i\frac{\pi}{4}}
	\end{bmatrix} 
\end{gather*}

The first three operators are known as Pauli operators. $X$ is the "bit-flip operator", the quantum equivalent of the classical not: it transforms $\kz$ in $\ko$ and $\ko$ in $\kz$. In general, the $X$ operators behaves as follows.

\[ X\kp = \begin{bmatrix}
	0 & 1 \\
	1 & 0
	\end{bmatrix} 
	\begin{pmatrix}
	\alpha \\
	\beta	
	\end{pmatrix} = 
	\begin{pmatrix}
	\beta \\
	\alpha	
	\end{pmatrix}
\] 
Notice that the $X$ operator leaves the $\kpl$ state unaltered, because it was in an equal superposition of $\kz$ and $\ko$, and it remains in a equal superposition.


$Z$ is the "phase-flip operator", as it changes the relative phases of the $kz$ and $\ko$ component. $Z \kpl$ becomes $\km$ and vice versa. When applied to $\kz$ (or $\ko$) the $Z$ operator does nothing, 

 
Finally, the $H$ operator, called the Hadamard operator, or Hadamard gate, is used to \textit{create superposition}, as it transforms a vector from the computational basis to the Hadamard basis:
\begin{gather*}
	H\kz = \kpl \qquad H\kpl = \kz  \\
	H\ko = \km \qquad H\km = \ko \\	 
\end{gather*}
\subsection{Measurement}

The second postulate describes only the evolution of isolated systems. Such system do not exchange energy nor information with the environment, and all their computations are always reversible (and are in fact formalized with invertible, unitary matrices). To extract classical information from the system, to \textit{measure} the output of a quantum computation, an interaction is needed an interaction between the quantum system and the environment. As we will see, this measurement operation is first of all non-invertible, as different state could produce the same outcome when measured, but is also fundamentally probabilistic: a generic state $\kp$ could produce different measurement outcomes $m_1, m_2,\ldots$, each with a certain probability that depends on $\kp$.

From a physical point of view, if a system is in a superposition of states, measuring it can  cause the wavefunction to collapse to a single state, in a purely probabilistic way. This means that, even if we compute a state that contains the desired information, this information is often difficult to recover, because directly measuring  it can destroy the information and produce a trivial outcome.

\begin{quote} \textbf{Postulate III}: Quantum measurements are described by a set $\{M_m\}_m$ of measurement operators, where the index $m$ refers to the measurement outcomes that may occur in the experiment. The set of measurement operators must be \textit{complete}, i.e.:
\[\sum_m M_m^\dagger M_m = I\]
If the state of the quantum system is $\kp$ before the measurement, then the probability that result $m$ occurs is 
\[p_m = \bra{\psi}M_m^\dagger M_m\kp\]
and the state after the measurement will be \[\frac{1}{\sqrt{p_m}}M_m\kp\]
\end{quote}

The most common class of quantum measurements is composed of \textit{projective measurements}. Such measurements are described by a set of \textit{orthogonal projectors}, i.e. Hermitian operators such that \[ M_mM_{m'} = \begin{cases}\mathbf{0}\quad\text{if } m \neq m' \\ M_m\quad\text{if }m = m'\end{cases}\]


The simplest example of (projective) measurement is simply measuring a state in the computational basis, i.e. projecting it in its $0$-$1$ component. The measurement in the computational basis, denoted as $M_{01}$ is defined as \[ M_0 = \begin{pmatrix} 1 & 0 \\ 0 & 0\end{pmatrix} \qquad
M_1 = \begin{pmatrix}0 & 0 \\ 0 & 1\end{pmatrix} \]
And its effect of the state $\kp = \begin{pmatrix}\alpha \\ \beta \end{pmatrix}$ is:
\begin{align*}
\frac{1}{\sqrt{p_0}}M_0\kp = \frac{1}{|\alpha|}\begin{pmatrix} 1 & 0 \\ 0 & 0\end{pmatrix}
\begin{pmatrix}\alpha \\ \beta \end{pmatrix} = \begin{pmatrix} 1 \\ 0 \end{pmatrix} = \kz 
\qquad \text{with probability } \bra{\psi}M_0^\dagger M_0\kp = |\alpha^2| \\
\\
\frac{1}{\sqrt{p_1}}M_1\kp = \frac{1}{|\beta|}\begin{pmatrix} 0 & 0 \\ 0 & 1\end{pmatrix}
\begin{pmatrix}\alpha \\ \beta \end{pmatrix} = \begin{pmatrix} 0 \\ 1 \end{pmatrix} = \ko 
\qquad \text{with probability } \bra{\psi}M_0^\dagger M_1\kp = |\beta^2|
\end{align*}

Notice that, when measuring the $\kz$ state in the computational basis, the outcome will always be $\kz$ with probability $|\alpha|^2 = 1$, in a completely deterministic behaviour. 
When instead measuring the $\kpl$ state we get either $\kz$ or $\ko$, with equal probability $|\alpha|^2 = |\beta|^2 = \left(\oost\right)^2 = \frac{1}{2}$. The same holds also for the $\km$ state. As already said, $\kpl$ and $\km$ differ only for the relative phase, and the relative phase has no influence on the outcome of a measurement in the computational basis.


An other projective measurement is the measurement in the Hadamard basis, $M_\pm$, defined as \[ M_+ = \frac{1}{2}\begin{pmatrix} 1 & 1 \\ 1 & 1\end{pmatrix} \qquad
M_- = \frac{1}{2}\begin{pmatrix}1 & -1 \\ -1 & 1\end{pmatrix} \]

From the physical point of view, this measurement can be performed composing Hadamard transformations and a measurement in the computational basis: $M_+ = HM_0H$, $M_- = H M_1 H$. Just as $M_{01}$ makes a vector decay in its $\kz$ component or in its $\ko$ component, $M_\pm$ makes a vector decay in its $\kpl$ or $\km$ component. This means that, when measuring $\kpl$ in the Hadamard basis, the outcome will always be $\kpl$ with probability $1$. Notice that a qubit in the state $\kz$ can be considered in an equal superposition of $\kpl$ and $\km$, as $\kz = \oost\kpl + \oost\km$. According to this, measuring $\kz$ in the Hadamard gives the outcomes $\kp$ or $\km$ with half the probability each.


\subsection{Composite quantum systems}
In the previous sections we characterized quantum systems and how they evolve, taking the prototypical example the two-dimensional Hilbert space of a qubit. When dealing hidher-dimensional systems, we can often describe them as \textit{composite system}, made of multiple smaller systems.

If we have for example two photons,  each described by a (2-dimensional) Hilbert space $\calH$, we can describe the system composed of both photons is as the \textit{tensor product} of the 2-dimensional Hilbert spaces.	

If $\calH_n$ is a $n$-dimensional Hilbert space, and $\calH_m$ is a $m$ dimensional Hilbert space, their tensor product $\calH_n \otimes \calH_m$ is a $nm$ Hilbert space. If $\{\ket{\psi_1}, \ldots \ket{\psi_n}\}$ is a basis of $\calH_n$, and $\{\ket{\phi_1}, \ldots\ket{\phi_m}\}$ is a base of $\calH_m$, then a basis of $\calH_n \otimes \calH_m$ is\[\big\{\ket{\psi_i}\otimes\ket{\phi_j} \mid i \in [1, \ldots n], j \in [1, \ldots m] \big\}\]
 where $\kp \otimes \kf$ denotes the Kronecker product. We will often omit the tensor symbol, writing $\ket{\psi}\ket{\phi}$ or also $\ket{\psi\phi}$ instead of $\kp\otimes\kf$. We can now state the last postulate we need:

\begin{quote}
\textbf{Postulate IV}: The state space of a composite physical system is the tensor product of the state spaces of the component physical systems. 
\end{quote}

If a single qubit is described by a $2$-dimensional space $\calH$, we will write $\calH^{\otimes n}$ to intend the otimes product $n$ copies of $\calH$, of dimension $2^n$. So, a compound system composed of two qubits has a state space $\calH^{\otimes 2}$, its canonical basis is 
\[\{\ket{00}, \ket{01}, \ket{10}, \ket{11}\}\]
and all its vector can be expressed as a linear combination:
\[\kp \in \Hto{2} = \begin{pmatrix}
\alpha \\ \beta \\ \gamma \\ \delta
\end{pmatrix} = \alpha\ket{00} + \beta\ket{01} + \gamma\ket{10} + \delta\ket{11}\]

A quantum state in $\calH_1 \otimes \calH_2$ is said \textit{separable} when can be expressed as the product of two vectors, one in $\calH_1$ and the other in $\calH_2$. From the definition of the Kronecker product, all separable states of $\Hto{2}$ are of the form:
\[
 \kp = \begin{pmatrix}
 \alpha\gamma \\ \alpha\delta \\ \beta\gamma \\ \beta\delta
 \end{pmatrix} = 
 \begin{pmatrix}
 \alpha \\ \beta
 \end{pmatrix} \otimes 
 \begin{pmatrix}
 \gamma \\ \delta
 \end{pmatrix}
\]

Some of the defining characteristics of quantum systems derive from the fact that not all states in $\Hto{2}$ are separable. The existence of such states, called \textit{entangled} states, implies that a composite system can not always be described as simply the juxtaposition of two smaller states. When a qubit $q_1$ is entangled with an other qubit $q_2$, its evolution depends not only on the transformations applied to $q_1$, but also on the transformations applied on $q_2$, that could be even light-years away. This surprising result does not allow faster then light communication, as we will see in the next section.

The classical example of an entangled state is the so called $\ket{\Phi^+}$ \textit{Bell state}:
\[\ket{\Phi^+} = \oost\ket{00} + \oost\ket{11} = \oost\begin{pmatrix}
1 \\ 0 \\ 0\\ 1
\end{pmatrix}
\]

The fourth postulate tells us that the state space of a composite system is simply the tensor product of the state spaces of the smaller systems. The tensor product of Hilbert spaces is still an Hilbert space, the composition of unit vector is still a unit vector, and the composition of unitary transformation is still unitary. For example, the (Kronecker) composition of $H$ and $I$ matrices is defined as a block matrix:
\[	H\otimes I = \oost\begin{pmatrix}
	I & I \\ I & -I
	\end{pmatrix}\]
And when applied to a two-qubit system, it applies $H$ on the first qubit, and leaves the second one unaltered:
\[ (H\otimes I) \ket{00} = H\kz \otimes I\kz = \oost \begin{pmatrix}
	I & I \\ I & -I
	\end{pmatrix} \begin{pmatrix}
	1 \\ 0 \\0 \\0
	\end{pmatrix} = \oost \begin{pmatrix}
	1 \\ 0 \\ 1 \\ 0
	\end{pmatrix} = \kpl\otimes\kz
\]

One of the most common two-qubit unitary that is not just the composition of one-qubit transformations is the CNOT matrix:
\[\text{CNOT} = 
\begin{pmatrix}
1 & 0 & 0 & 0 \\
0 & 1 & 0 & 0 \\
0 & 0 & 0 & 1 \\
0 & 0 & 1 & 0 \\
\end{pmatrix}
\]

This matrix applies a "controlled not" operator on the second qubit. That is, it leaves the second qubit unaltered if the first one is $\ko$, and applies an $X$ transformation on the second qubit if the first one is $\ko$. Together with linearity, this means that if the first bit is in a superposition of $\kz$ and $\ko$, after applying this transformation the whole system will be in a superposition of two states: one where the two bits are left unaffected, one where the second one has been flipped. As an example, we can show how the CNOT transformation is used to create entanglement.
\begin{gather*}
\cnot \ket{00} = \ket{00} \qquad \cnot \ket{10} = \ket{11} 
\\
\cnot \ket{+0} = \oost\big(\cnot\ket{00} + \cnot\ket{10}\big) = \oost
\big(\ket{00} + \ket{11}\big) = \ket{\Phi^+}
\\
\cnot \ket{-0} = \oost\big(\cnot\ket{00} - \cnot\ket{10}\big) = \oost
\big(\ket{00} - \ket{11}\big) = \ket{\Phi^-}
\\
\cnot \ket{+1} = \oost\big(\cnot\ket{01} + \cnot\ket{11}\big) = \oost
\big(\ket{01} + \ket{10}\big) = \ket{\Psi^+}
\\
\cnot \ket{-1} = \oost\big(\cnot\ket{01} - \cnot\ket{11}\big) = \oost
\big(\ket{01} - \ket{10}\big) = \ket{\Psi^-}
\end{gather*}

$\ket{\Phi^+}, \ket{\Phi^-}, \ket{\Psi^+}$ and 
$\ket{\Psi^-}$ are four different ortonormal entangled vectors, and form the so called \textit{Bell basis} of $\calH^{\otimes 2}$.

\subsection{Density operator formalism}

The formalism presented so far describes quantum system in terms of unit vectors and linear transformations. There is an alternative, more general formulation, the \textit{density operator formalism}, in which states are represented as positive operators, and transformations as linear maps from operators to operators, i.e. superoperators.

The main advantage of this formulation is that it represents also \textit{partial information} about a quantum system. When describing open systems, that are systems which interact with the external environment, it is often impossible to have complete knowledge on the state of our systems. Instead, one could know that the open system is either in state $\kp$, with a certain probability $p$, or in state $\kf$, with probability $1-p$. In other words, we know that the system is in a \textit{probabilistic mixture of states}, called an \textit{ensemble} of states, or also a \textit{mixed state}.

A typical application of mixed states is in presence of noisy channels. Suppose for example a channel that correctly transmits the sent qubit only with probability $0.9$, and with probability $0.1$ is causes a "bit-flip" error, exchanging $\kz$ with $\ko$ and $\ko$ with $\kz$ (that is, the channel applies a $X$ transformation to the qubit with probability $0.1$). If Alice sends a qubit with state $\kz$ to Bob, using this noisy channel, Bob recieves a an ensemble of states, a qubit with a $0.9$ probability of being in state $\kz$, and a $0.1$ probability of being in state $\ko$.


In general, given an $n$-dimensional Hilbert space $\calH$, an \textit{ensemble} of quantum states is a set:
\[\{(\ket{\psi_i}, p_i)\} \]
of quantum states in $\calH$, each with a different probability, such that $\forall i \, p_i > 0$ and $\sum_i p_i \leq 1$. Notice that when $\sum_i p_i = 1$, we have a probability distribution of states, when $\sum_i p_i < 1$, we have a so called subprobability distribution.

Each ensemble defines a density operator, that is a matrix in $\mathbb{C}^{n \times n}$, i.e. an operator $\calH \rightarrow \calH$. The ensemble $\{(\ket{\psi_i}, p_i)\}$, where $\ket{\psi_i} \in \calH$, defines the density operator: 
\[
	\rho = \sum_i p_i \proj{\psi_i}
\]
where $\ketbra{\psi}{\phi}$ denotes the matrix product between the column vector $\kp$ and the row vector $\bra{\phi}$, known as the \textit{outer product}. Notice that this construction is not injective, as there are different ensembles that correspond to the same density operator. We indicate with $\calDH$ the set of density operators of $\calH$.

Density operators enjoy two useful properties (see \cite{nielsen_chuang_2010}): \begin{enumerate}
\item The trace of $\rho$ is the sum of the probabilities of the ensemble $tr(\rho) = \sum_i p_i \leq 1$.
\item $\rho$ is \textit{positive semidefinite}, i.e. 
\[\forall \kp \in \calH \ \bra{\psi}\rho\kp \geq 0\]
\end{enumerate}

Positive semidefinite operators are always diagonalizable with eigenvalues real and positives. So, each positive semidefinite operator with trace $\leq 1$ represents at least one ensemble, with the eigenvectors as states and the corresponding eigenvalues as probabilities.

One of the main application of density operators is to describe the state of a subsystem of a composite quantum system. When dealing with composite system, we write $\rho \otimes \sigma \in \mathcal{D}(\calH_1 \otimes \calH_2)$, to denote the density matrix given by the Kronecker product of $\rho$ and $\sigma$. Notice that, if $\rho = \sum_{i=1}^n p_i\proj{\psi_i}$ and $\sigma = \sum_{j=1}^m p_j\proj{\phi_j}$, then 
\[\rho \otimes \sigma = \sum_{i=1}^n\sum_{j=1}^m p_ip_j \proj{\psi_i\phi_j}\]

Suppose a composite system, made of two subsystem $A$ and $B$, with state space  $\calH = \calH_A \otimes \calH_B$. A generic  Given a generic (not necessarily separable) $\rho^{AB} \in \calH$, describing the state of the whole system, the operator $\rho^A$ that describes the subsystem $A$ is obtained as 
\[
	\rho^A = tr_B(\rho^{AB})
\]
where the $tr_B$ is called the \textit{partial trace over $B$}, and is defined by 
\[
 tr_B\big(\ketbra{\psi}{\psi'} \otimes \ketbra{\phi}{\phi'}\big) = \ketbra{\psi}{\psi'}tr\big(\ketbra{\phi}{\phi'}\big)
\]
together with linearity. $\rho^A$ is called the \textit{reduced density operator} of system $A$.

When applied to a separable state $\rho^A \otimes \rho^B$, the partial trace $tr_B$ gives exactly $\rho^A$. When applied to an entangled state, instead, it produces a probabilistic mixture of states, because "forgetting" the information on the state of subsystem $B$ leaves us with only partial information on subsystem $A$. The canonical example is the bell state $\rho = \frac{1}{2} \proj{00} + \frac{1}{2}\proj{11} \in \calH_A \otimes \calH_B$: 

\begin{align*}
tr_B(\rho) &= \frac{1}{2} tr_B\big(\proj{00}\big) + \frac{1}{2}tr_B\big(\proj{11}\big)  \\
	&= \frac{1}{2} tr_B\big(\proj{0}\otimes\proj{0}\big) + \frac{1}{2}tr_B\big(\proj{1}\otimes\proj{1}\big) \\
	&= \frac{1}{2} \proj{0}tr\big(\proj{0}\big) + \frac{1}{2}\proj{1}tr\big(\proj{1}\big) \\
	&= \frac{1}{2} \big(\proj{0} + \proj{1}\big) = \frac{1}{2}I
\end{align*}


The state $\frac{1}{2}I$ is called the \textit{maximally mixed state}, as it gives us no information on the system. It can be seen as an ensemble of states $\kz$ and $\ko$ both with probability one half, but could also indicate an ensemble of $\kpl$ and $\km$ with the same probability, or even en ensemble of $\kz$, $\ko$, $\kpl$ and $\km$, and so on. 


According to the density operator formalism, the state of a quantum system is represented as a density operator. This means that a transformation applied to a quantum system can be formalized as a transformation from and operators to operators, i.e. a \textit{superoperator}.

For example, the noisy bitflip channel described before, that leaves the input qubit unaltered with $0.9$ probability, and applies $X$ to it with $0.1$ probability, can be described as the function $\sop: \calDH \rightarrow \calDH$ defines by
\[\sop(\rho) = \frac{9}{10} 
I\rho I
+
\frac{1}{10}
X \rho X^\dagger
\]
When applied to a qubit with state $\proj{0}$, the output will be 
\[\sop(\proj{0}) = \frac{9}{10} I \proj{0} I + \frac{1}{10}X \proj{0} X^\dagger = \frac{9}{10} \proj{0} + \frac{1}{10}\proj{1} = 
\begin{pmatrix} 0.9 & 0 \\ 0 & 0.1 \end{pmatrix} \]

Not all the functions on density matrices are valid superoperators. Before introducing the correct definition, we need an additional notion, the tensor product of two functions.


Let $\calH_A$ and $\calH_B$ be Hilbert spaces. Given $\mathcal{E}: \mathcal{D}(\calH_A) \rightarrow \mathcal{D}(\calH_A)$ and $\mathcal{F}: \mathcal{D}(\calH_B) \rightarrow \mathcal{D}(\calH_B)$,  their tensor product $\sop\otimes \mathcal{F}: \mathcal{D}(\calH_A \otimes \calH_B)\rightarrow \mathcal{D}(\calH_A \otimes \calH_B)$ is defined as:

\[
	(\sop \otimes \mathcal{F})(\rho_A \otimes \rho_B) = \sop(\rho_A)\otimes \mathcal{F}(\rho_B)
\]
together with linearity.


We now define the \textit{superoperators on $\calH$} as all the maps $\sop: \calDH \rightarrow \calDH$ satifying:
\begin{itemize}
\item $\sop$ is \textit{convex linear}: for any set of probabilities	$\{p_i\}_i$,
\[\sop\left(\sum_i p_i \rho_i \right) = \sum_i p_i \sop(\rho_i)\]
\item $\sop$ is \textit{completely positive}: for any extra Hilbert space $\calH_R$, for any positive $\rho \in \calH_R \otimes \calH$, $(\mathcal{I}_R\otimes \sop)(\rho)$ is also positive, where $\mathcal{I}_R$ is the identity operator on $\mathcal{D}(\calH_R)$.
\item $\sop$ is \textit{trace non-increasing}: for any density operator $\rho \in \calDH$, 
\[tr\big(\sop(\rho)\big) \leq tr(\rho) \leq 1\]
\end{itemize}
We call $\calSH$ the set of all superoperators on $\calH$.


This axiomatic definition of the superoperators on $\calH$ specifies they overall properties, but does not give any suggestion on how a superoperator can be constructed. There exists an alternative, equivalent definition, that says that a superoperator is any function $\sop: \calDH \rightarrow \calDH$ that has a \textit{Kraus operator sum decomposition}.

A function $\sop: \calDH \rightarrow \calDH$ is a superoperator on $\calH$ if and only if it has a Kraus decomposition, i.e. a finite set of operators $\{E_i\}_i$, with $1 \leq i \leq dim(\calH)$ such that
\begin{gather*}
\sop(\rho) = \sum_i E_i\rho E_i^\dagger \\
\sum_i E_i^\dagger E_i \sqsubseteq I_\calH
\end{gather*}

Where $A \sqsubseteq B$ means that $B - A$ is a positive semidefinite matrix, and $T_{calH}$ is the identity linear operator on $\calH$.

Although there are possibly multiple Kraus decompositions for any superoperator $\sop \in \calSH$, they are related by a unitary transformation.
Formally, for any two set of decompositions $\{C_i\}_i$ and $\{B_j\}_j$, there is a unitary matrix $U = \{u_{ij}\}_{ij}$ such that $C_i = \sum_j u_{ij} B_j$.

Given $\mathcal{E}: \mathcal{D}(\calH_A) \rightarrow \mathcal{D}(\calH_A)$ and $\mathcal{F}: \mathcal{D}(\calH_B) \rightarrow \mathcal{D}(\calH_B)$,
respectively expressed in Kraus form as $\mathcal{E}(\rho) = \sum_i E_i \rho E_i^\dag$ and $\mathcal{F}(\rho) = \sum_j F_j \rho E_j^\dag$, the thensor product $\mathcal{E} \otimes \mathcal{F}$
can be expressed in Kraus form as 

\[
  (\mathcal{E} \otimes \mathcal{F})(\rho) = \sum_i \sum_j (E_i \otimes F_j)\,\rho\:(E_i \otimes F_j)^\dag
\]


Using density matrices and superoperators, it is possible to restate all the quantum postulate seen in the previous sections. A physical system is represented by a density matrix $\rho \in \calDH$ with trace $1$. A composite physical  system is described by a density matrix $\rho \in \mathcal{D}(\calH_1 \otimes \calH_2)$. The second and the third postulate can be merged together: the evolution of an open quantum system can always be described by a superoperator. That is, the state $\rho$ of the system at time $t_0$ is related to the state $\rho'$ of the system at time $t_2$ by a superoperator $\sop$ which depends only on the times $t_0$ and $t_1$.
\[
	\rho' = \sop(\rho)
\]



 Superoperator can in fact describe both unitary transformations and measurements, as well as more general transformations. Given unitary operator $U$ on $\calH$, we can define  the (trace preserving) superoperator $\sop_U \in \calSH$ as:
\[ \sop_U(\rho) = U \rho U^\dagger\]
Given a measurement $\{M_m\}_m$, we can define the (trace non-increasing) superoperator $\sop_m \in \calSH$ as 
\[\sop_m(\rho) = M_m\rho M_m^\dagger\]
Notice that $\sop_m(\rho)$ is equal to $p_m\rho_m$, where $p_m$ is the probability of the outcome $m$ when measuring the state $\rho$, and $\rho_m$ is the state after outcome $m$ has occurred. 


An other class of useful superoperators are noisy channels or noisy gates, i.e. transformation that perform the desired operation only with a high probability. We already seen the noisy bitflip channel: 
\[\sop(\rho) = (p) 
I\rho I
+
(1-p)
X \rho X^\dagger
\]
with Kraus decomposition $\{\sqrt{p}I, \sqrt{1-p}X\}$.


Finally, there are also "constant" superoperators, that completely destroy the information of a quantum system, and return always the same state. For example, we will often use a $ Set$ operator like
\[Set_0(\rho) = \proj{0}\rho\proj{0} + \ketbra{0}{1}\rho\ketbra{1}{0}\]
that always returns $\proj{0}$ for any possible input.

\section{Process Calculi}	
\subsection{Process Calculi and LTS}


Process Calculus, also known as Process Algebra, is an algebraic approach to model concurrent computation, often focus on the communication between different agents. Each agent is formalized as a \textit{process}, a syntactic element that describes its capabilities. Processes can be composed in different ways, and so form an \textit{algebra of processes}, with various operators for parallel composition, sequential compositions, probabilistic and non-deterministic sum, and so on.

All the foundational and most successful process calculi \cite{milnerCalculusCommunicatingSystems1980, bergstraAlgebraCommunicatingProcesses1985, 
hoareCommunicatingSequentialProcesses1978, milnerCommunicatingMobileSystems1999},  have a number of key features in common \begin{itemize}
\item \textbf{Nil Process}: Each calculus has a \textit{constant process}, usually denoted as $nil$ or $\nil$, that is in a terminal state, i.e. it can not perform any action.
\item \textbf{Visible Actions}: Each calculus has \textit{action prefixes}, usually denotes as $\alpha, \beta, \ldots, \overline{\alpha}, \overline{\beta}, \ldots$. If $P$ is a process, $\alpha.P$ is a process that can perform an action $\alpha$, and then behaves as $P$. The two actions $\alpha$ and $\overline{\alpha}$ are called \textit{coactions}, and denotes two dual vies of the same comunication event. If $\alpha$ is the action of sending a message through a channel, $\overline{\alpha}$ is the dual action of receiving that message. 
\item \textbf{Internal Actions} Each calculus defines also a $\tau$ action, called \textit{internal action} or silent action. Process calculi are designed to model the interaction between system, and abstract away from the low level details. A process $\tau.P$ can then performs a silent action, indicating internal operations, not known to an external observer, outside the scope of the system being modelled. 
%For example, when describing a TCP handshake protocol, only the IP packets sent on the net are represented as visible actions, while while writing data on the buffer of the socket is an internal action.
\item \textbf{Non-determinism}: Each calculus has a \textit{non-deterministic sum} of processes, denoted as $P + Q$. Such a process can non-deterministically "choose" to behave like $P$ or like $Q$. 
\item \textbf{Parallel Execution}: Each calculus has \textit{parallel composition} of processes, denoted as $P \parallel Q$. Such a process represent the concurrent execution of both $P$ and $Q$, and can perform all the actions of $P$ as well as all the action of $Q$, in a interleaving manner. Differently from the nondeterministic sum, after $P\parallel Q$ performs an action of $P$, it can still perform the actions of $Q$.
\item \textbf{Synchronization}: Each calculus has inter-process communication. Two processes executing in parallel can \textit{synchronize}, performing at the same time an action and a coaction. A parallel process $\alpha.P \parallel \overline{\alpha}.Q$ can evolve in $P \parallel Q$, as the two processes had performed a synchronous communication. Such synchronization transition results in an invisible action, as it can be understood as a internal action of the (composite) process $\alpha.P \parallel \overline{\alpha}.Q$ 

\end{itemize}

To give a running example of a calculus with these features, we will define syntax and semantics of \textit{value-passing CCS} \cite{hennessyTheoryCommunicatingProcesses1993}.

\begin{align*}
  P \Coloneqq \nil \mid c!v. P \mid c?x.P \mid \tau.P \mid P + P \mid P \parallel P
\end{align*}	

In value passing CCS, $c!v$ is the action of sending a value $v$ through a channel $c$, and its coaction is $c?v$, of receiving a value from channel $c$. In the process $c?x.P$, we say that $x$ is a bound variable, otherwise it is free.

Usually, the semamtic of a process calculus is given in a SOS-fashion, as a \textit{Labelled Transition System}. Given a process $P$, its LTS can be understood as a rooted directed graph, where the nodes are processes, i.e. states of the computation, and the outgoing edges are the actions that a process can perform.

Formally, a Labelled Transition System is a triple $\langle S , Act, \rightarrow \rangle$ where \begin{itemize}
\item $S$ is a set of states
\item $Act$ is a set of transition labels
\item $\rightarrow 	\subseteq S\times Act_\tau \times S$ is the transition relation, with $Act_\tau = Act \cup \{\tau\}$
\end{itemize} 

An element $(s, \alpha, t) \in \rightarrow$ is called a \textit{transition}, and is often written as $s \xrightarrow{\alpha} t$. We denote with $\Rightarrow$ the reflexive and transitive closure of $\xrightarrow{\tau}$, and use $s \xRightarrow{\alpha} t$ as an abbreviation for $s \Rightarrow s' \xrightarrow{\alpha} t' \Rightarrow t$ for some $s', t' \in S$. When dealing with process calculi, the set $S$ of states is the set of all processes, and a transition $P \xrightarrow{\alpha} P'$ means that the process $P$ performs an action $\alpha$ and evolves in $P'$.

The LTS of a process in value passing CCS is given by the following set of derivation rules: 

\begin{gather*}
c!v.P \xrightarrow{c!v} P \qquad c?x.P \xrightarrow{c?v} P[v/x] \qquad \	\tau.P \xrightarrow{\tau} P
\\
\infer{P + Q \xrightarrow{\alpha} P'}{P \xrightarrow{\alpha} P'} \qquad \infer{P + Q \xrightarrow{\alpha} Q'}{Q \xrightarrow{\alpha} Q'} 
\\
\infer{P \parallel Q \xrightarrow{\alpha} P'\parallel Q}{P \xrightarrow{\alpha} P'} \qquad \infer{P \parallel Q \xrightarrow{\alpha} P\parallel Q'}{Q \xrightarrow{\alpha} Q'} 
\\
\infer{P \parallel Q \xrightarrow{\tau} P'\parallel Q'}{P \xrightarrow{c!v} P' & Q \xrightarrow{c?v} Q' } \qquad
\infer{P \parallel Q \xrightarrow{\tau} P'\parallel Q'}{P \xrightarrow{c?v} P' & Q \xrightarrow{c!v} Q' } 
\\
\end{gather*}

\subsection{Bisimulation}

One of the most important notion in the theory of process calculi is the bisimilarity relation. Two processes are bisimilar when they are behaviourally equivalent, i.e. express the same behaviour. Different ways to compare the behaviour of two processes yields different definitions of bisimilarity.

The simplest and more natural notion of bisimilarity is \textit{strong bisimilarity}, where two bisimilar processes must express exactly the same behaviour, i.e. must perform the same action, both visible and silent.
Let $\langle S , Act, \rightarrow \rangle$ be a LTS. Then a symmetric relation $\rel \subseteq S \times S$ is a \textit{strong bisimulation} if and only if, whenever $s \rel t$, then 
\begin{center}
if $s \xrightarrow{\alpha} s'$ then $t \xrightarrow{\alpha} t'$ for some $t'$ such that $s' \rel t'$
\end{center}
Two states $s, t \in S$ are said to be \textit{strongly bisimilar}, written $s \sim t$, if exists a strong bisimulation $\rel$ such that $s \rel t$. This means that bisimilarity is the largest bisimulation, the union of all the bisimulation relations.
Strong bisimulation is an equivalence relation, meaning that is reflexive, symmetrical and transitive, and it is also a \textit{congruence}, meaning that, if $P$ and $Q$ are bisimilar, so must be $P\parallel R$ and $Q\parallel R$, or $P + R$ and $Q + R$, or any other couple of processes that can be constructed starting from $P$ and $Q$.



A weaker notion of behavioural equivalence is \textit{weak bisimilarity}, which requires that two bisimilar processes exhibit the same visible actions, but allows internal action to be matched by zero or more internal action.
Let $\langle S , Act, \rightarrow \rangle$ be a LTS. Then a symmetric relation $\rel \subseteq S \times S$ is a \textit{weak bisimulation} if and only if, whenever $s \rel t$, then 
\begin{center}
if $s \xrightarrow{\alpha} s'$ then $t \xRightarrow{\alpha} t'$ for some $t'$ such that $s' \rel t'$
\end{center}
Two states $s, t \in S$ are said to be \textit{weakly bisimilar}, written $s \approx t$, if a weak bisimulation $\rel$ exists such that $s \rel t$.  Weak bisimilarity is not a congruence, as $P = \alpha.\nil$ and $Q = \tau.\alpha.\nil$ are bisimilar, but $P + \beta.\nil$ and $Q + \beta.\nil$ are not.


Halfway between strong and weak bisimilarity there is the notion of \textit{branching bisimilarity}, that is coarser than strong bisimilarity, as it ignores internal actions, but is finer than weak bisimilarity, as it matches the braching structure more accurately.
Let $\langle S , Act, \rightarrow \rangle$ be a LTS. Then a symmetric relation $\rel \subseteq S \times S$ is a \textit{branching bisimulation} if and only if, whenever $s \rel t$, then 
\begin{center}
if $s \xrightarrow{\alpha} s'$ then $t \xRightarrow t' \xrightarrow{\alpha} t''$ for some $t', t''$ such that $s \rel t'$ and $s' \rel t''$
\end{center}
Two states $s, t \in S$ are said to be \textit{branching bisimilar}, written $s \simeq t$, if a branching bisimulation $\rel$ exists such that $s \rel t$.

The two processes \[P = \alpha.\nil + \tau.\beta.\nil \qquad Q = \alpha.\nil + \beta.\nil + \tau.\beta.\nil\] are weakly bisimilar, because obviously if $P$ performs an action $Q$ can replicate it, and in $Q$ performs its additional $\beta$ action, $P$ can replicate it as $\tau.\beta$. These two processes are instead not branching bisimilar, because if $Q$ performs its $\beta$ action, $P$ can't replicate it, as after a $\tau$ transition it evolves in $\beta.\nil$, that it is not bisimilar to $Q$.

\subsection{Probabilistic LTS} \label{pLTS}

The usual concepts of process calculus has been successfully extended to model probabilistic systems. We will present in this section the most common definitions of probabilistic LTS and probabilistic bisimulation, following the comprehensive analysis of \cite{hennessyExploringProbabilisticBisimulations2012, dengLogicalMetricAlgorithmic2011}.

\subsubsection{Probabilistic Lifting}
First of all, we give some preliminary mathematical definitions about probabilistic distributions, and about the very general concept of \textit{lifting} a relation between object to a relation between distribution of objects.

Given a set $S$, a (discrete) \textit{probability distribution on $S$} is a mapping $\Delta: S \rightarrow [0, 1]$ such that $\sum_{s\in S} \Delta(s) = 1$. We indicate with $\distr(S)$ the set of all probability distribution on $S$.
The \textit{support} of $\Delta \in \distr(S)$ is defined as $\lceil\Delta\rceil = \{s \in S \mid \Delta(s) > 0\}$.

We use $\overline{s}$ to denote the \textit{point distribution} on $s$ (also known as Dirac distribution, in the continuous case):
\[
	\overline{s}(t) = 
	\begin{cases} 1 \text{ if }t = s \\
	0 \text{ if } t\neq s
	\end{cases}
\]

Given a set $\{p_i\}$ of probabilities (i.e. $\sum_i p_i = 1$ and $p_i > 0$ for each $i$), we define the \textit{convex combination} of distributions $\left(\sum_i p_i \Delta_i\right)$ as the only 
distribution such that
\[
\left(\sum_i p_i \Delta_i\right)(s) = \sum_i p_i \Delta_i(s)
\]
We often abbreviate $p \Delta + (1-p) \Theta$ as $\Delta \psum{p} \Theta$.

For any set $D \subseteq \distr(S)$ of distributions, we denote with $CC(D)$ the \textit{convex closure} of $D$, i.e. the least subset of $\distr(S)$ that contains $D$ and is closed un the operations $ - \psum{p} - $ for any $p$ with $0 \leq p \leq 1$.


In order to extend the notions from to "classical" process algebraic approach to a probabilistic setting, it is useful to define a \textit{probabilistic lifting}, based on the concept of linearity.

A relation $\mathcal{R} \subseteq \distr(S) \times \distr(S)$  between distributions is said to be \textit{linear} if $\Delta_1 \rel \Theta_1$ and $\Delta_2 \rel \Theta_2$ implies $(\Delta_1 \psum{p} \Delta_2) \rel (\Theta_1 \psum{p} \Theta_2)$ of any $0 \leq p \leq 1$.

Given a relation $\rel \subseteq S \times S$, we define its \textit{lifting} $\mathring{\rel} \subseteq \distr(S) \times \distr(S)$ as the smaller linear relation such that $s \rel t$ implies $\overline{s} \mathring{\rel} \overline{t}$.

With abuse of notation, we denote with the same symbol also the lifting of relations $\rel \subseteq S \times \distr(S)$. Given a relation $\rel \subseteq S \times \distr(S)$, we define its \textit{lifting} $\mathring{\rel} \subseteq \distr(S) \times \distr(S)$ as the smaller linear relation such that $s \rel \Delta$ implies $\overline{s} \mathring{\rel} \Delta$.


These lifted relations enjoys two useful properties. Interestingly, both this property are equivalent to the given definition, and are indeed used as the definition in various works on probabilistic bisimulations. \begin{itemize}
\item Given $\rel \subseteq S \times S$, then $\Delta \lrel \Theta$ if ans only if $\Delta$ and $\Theta$ can be decomposed as follows: \begin{enumerate}
\item $\Delta = \sum_{i \in I} p_i \overline{s_i}$, where $I$ is a finite index set and $\sum_{i \in I}p_i = 1$
\item For each $i \in I$ there is a state $t_i$ such that $s_i \rel t_i$
\item $\Theta = \sum_{i\in I}p_i\overline{t_i}$ 
\end{enumerate}
\item Given an equivalence $\rel \subseteq S \times S$, then $\Delta \lrel \Theta$ if and only if, for all equivalence classes $C \in S/R$
\[\sum_{s\in C} \Delta(s) = \sum_{s\in C} \Theta(s)\]
\end{itemize}

\subsubsection{Probabilistic Labelled Transition Systems}

We are now ready to introduce the main features of probabilistic process calculi. In addition to the usual operators of action prefixes, parallel composition and nondeterministic sum, such calculi often include a \textit{probabilistic choice} operator, like $P \qsum{p} Q$, where $p$ is any probability $0 \leq p \leq 1$. A process $P \qsum{p} Q$ makes a probabilistic choice between $P$ and $Q$, that is, it evolves in a \textit{distribution of processes}.

Formally, we can define the operational semantic of a probabilistic process calculus as a \textit{pLTS}.
	
A \textit{Probabilistic Labelled Transition System} (pLTS) is a triple $\langle S , Act, \rightarrow \rangle$ where \begin{itemize}
\item $S$ is a set of states
\item $Act$ is a set of transition labels
\item $\rightarrow 	\subseteq S\times Act_\tau \times \mathcal{D}(S)$ is the transition relation, with $Act_\tau = Act \cup \{\tau\}$
\end{itemize} 

\note{ho introdotto il $\qsum{}$ per differenziare l'operatore sintattico dell'algebra  dalla notazione per le distribuzioni, ma comunque non sono sicuro che l'esempio serva a chiarire e non a confondere le idee} For example, we could write 
\[P \qsum{\frac{2}{3}} Q \xlongrightarrow{\tau} \overline{P} \psum{\frac{2}{3}} \overline{Q}\]
where $P \qsum{\frac{2}{3}} Q$ is a single state, a syntactic element, and $\overline{P} \psum{\frac{2}{3}} \overline{Q}$ is a distribution of states, that assign to state $P$ probability $\frac{2}{3}$, and to state $Q$ probability $\frac{1}{3}$


On this definition of pLTS, there are actually two separate notion of (strong) probabilistic bisimilarity, that differ for how they treat non-determinism.

The first goes under the name of \textit{Larsen-Skou bisimulation}. Let $\langle S , Act, \rightarrow \rangle$ be a pLTS. Then a symmetric relation $\rel \subseteq S \times S$ is a \textit{Larsen-Skou bisimultion} if and only if, whenever $s \rel t$, then 
\begin{center}
if $s \xrightarrow{\alpha} \Delta$ then $t \xrightarrow{\alpha} \Theta$ for some $\Theta$ such that $\Delta \lrel \Theta$
\end{center}
The usual definition of Larsen-Skou bisimulation does not use the lifting operation, and requires that $\Theta$ assigns the same probability as $\Delta$ to each equivalence class of $S/R$, which is a completely equivalent formulation, as seen in the previous section. Two states $s, t \in S$ are said to be \textit{Larsen-Skou bisimilar}, written $s \sim_{LS} t$, if a Larsen-Skou bisimulation $\rel$ exists such that $s \rel t$.

The second, strictly coarser notion of probabilistic bisimilarity is \textit{Segala bisimilarity}, which consider also the probabilistic behaviour that could happen in presence of an \textit{adversary}. When there is a non-deterministic choice, like in $\alpha.P + \alpha.Q$, it is common to assume the existance of an external agent, an adversary, that resolves such nondeterministic choices in an arbitrary way. In the example made before, one could suppose an adversary that always chooses the left transition, or one that always chooses the right transition. Segala bisimilarity consider the cases when the adversary can randomize, and for example choose the left transition with probability $p$ and the right transition with probability $1-p$.

To define Segala bisimilarity, is necessary to introduce \textit{combined transitions}, i.e. a transition relation $\longrightarrow_{cc} \subseteq S \times Act_t \times \distr(S)$ such that 
\[ s \xrightarrow{\alpha} \Delta \text{ if and only if } \Delta \in CC(\big\{\Theta \mid s \xrightarrow{\alpha} \Theta \big\})
\]
that is, $\Delta$ is reachable from $s$, or is a convex combination of distributions reachable from $s$.
Let $\langle S , Act, \rightarrow \rangle$ be a pLTS. Then a symmetric relation $\rel \subseteq S \times S$ is a \textit{Segala bisimultion} if and only if, whenever $s \rel t$, then 
\begin{center}
if $s \xrightarrow{\alpha} \Delta$ then $t \xrightarrow{\alpha}_{cc} \Theta$ for some $\Theta$ such that $\Delta \lrel \Theta$
\end{center}
Two states $s, t \in S$ are said to be \textit{Segala bisimilar}, written $s \sim_S t$, if a Segala bisimulation $\rel$ exists such that $s \rel t$.
\subsubsection{From pLTS to LTS}

The given definition of pLTS can be seen as a disconnected, bipartite graph, where each node is either an element of $S$, i.e. a process, or an element of $\distr(S)$, i.e. a distribution of processes, and all the edges go from $S$ to $\distr(S)$. This is not a problem to define probabilistic bisimilarity, but there are other settings where a connected LTS is preferred, for example when defining weak bisimilarity, temporal logics or model checking algorithm.


To "complete" a pLTS it is necessary to define how a distribution "evolves", which are its outgoing transitions. There are two alternative approaches, that we could call "probabilistic branching" and "distribution transformer".

The first, arguably more standard approach is \textit{probabilistic branching}, that proposes a derivation rule like
\[ \sum_i p_i \overline{s_i} \quad
\substack{  p_i  \\  \rightsquigarrow } 
\quad s_i \]
Where a distribution of processes "picks" just one process, evolving with a \textit{probabilistic transition}. This approach, used also in probabilistic model checking \cite{kwiatkowskaPRISMVerificationProbabilistic2011}, gives a probability to the single transition, and so allows to define a probabilistic measure for a whole path of computation.

With such a rule, a pLTS $\langle S, Act, \rightarrow\rangle$ can be "completed" to a bipartite LTS, where the states are either processer or distribution of processes, the $\rightarrow$ transition goes from states to distribution, with a label in $Act_\tau$, and the $\rightsquigarrow$ transition goes from distributions to states, with $0 \leq p \leq 1$ as a label. 

The second, most recent approach, is known as the \textit{distribution transformer} semantic, also called belief-state transformer semantic or labelled Markov process semantic \note{cite}. It specifies that a distribution of processes must evolve in a distribution of processes, with a derivation rule like 
\[ \infer{ \sum_i p_i \overline{s_i} \xrightarrow{\alpha} \sum_i p_i \Delta_i}
  { p_i \xrightarrow{\alpha} \Delta_i & \text{ for each  }i }
\]
So, for a distribution to evolve, it is necessary that all the precesses in its support can perform the same action $\alpha$. A distribution like $\alpha.P \psum{\frac{1}{2}} \beta.Q$, for example, has no outgoing transition.

Notice that the addition of this rule is equivalent to lifting the transition relation $\rightarrow \subseteq S \times Act \times \distr(S)$ to a relation $\mathring{\rightarrow} \subseteq \distr(S) \times Act \times \distr(S)$. Since each state $s \in S$ can be also seen as a distribution $\overline{s} \in \distr(S)$, this lifting relation de facto defines a "complete" LTS of distributions $\langle \distr(S), Act, \mathring{\rightarrow}\rangle$. Observe that $\rightarrow_{cc} \subset \mathring{\rightarrow}$, and in fat this distribution transformation semantic is more often than not associated to a Segala bisimilarity.



\subsection{Reduction systems}\label{bkg_reduction_system}

In the early works on process calculus, bisimilarity in its various form was adopted as the mainstream notion of behavioural equivalence. Starting from \cite{milnerBarbedBisimulation1992}, a different notion of equivalence was considered, conceptually simpler and more general, under the name
of \textbf{barbed congruence}, or barbed congruence. According to this notion, two processes are equivalent if they can not be distinguished by an external observer. That is, two processes $P$ and $Q$ are equivalent if, under any context $B[]$, $B[P]$ and $B[Q]$ express the same observable behaviour, based on a general concept of "observable", called \textit{barb}.

We will first introduce the \textit{reduction semantic} for process calculi, and then show how it can be used to define barbed equivalence. 

A reduction semantic for a process calculus \cite{milnerFunctionsProcesses1990, berryChemicalAbstractMachine1989}, is an alternative way to define the dynamics of a process, using a \textit{Reduction system} instead of a Labelled Transition System. 

A \textit{Reduction System} (RS), or unlabelled transition system, is a couple $\langle S,  \rightarrow \rangle$ where \begin{itemize}
\item $S$ is a set of states
\item $\rightarrow 	\subseteq S\times S$ is the transition relation.
\end{itemize} 

In a reduction system, like lambda calculus and all other term-rewriting, a reduction is possible only when the two subtemrs that interacts are syntactically contiguous, i.e. they form a redex. 

It is possible to define a reduction system for a process calculus like value-passing CCS, for example identifying the redexes:
\[ \tau.P \rightarrow P \qquad c!v.P \parallel c?x.Q \rightarrow P \parallel Q[v/x]\]
together with the usual rules
\[ \infer{P \parallel Q \rightarrow P'\parallel Q}{P \rightarrow P'}
\qquad
\infer{P + Q \rightarrow P'}{P \rightarrow P'}\]

In process calculi, thanks to the labelled semantic, subprocesses are allowed to interact also when they are syntactically "distant", like $c!v.\nil \parallel d?x.\nil \parallel c?x.P$. So in order to define a reduction system as the one above, it is necessary to introduce a way to ignore the syntactic arrangement and "reorder" the subprocess as needed.  This is achieved considering the reduction system modulo a \textit{structural congruence relation}. In the case of value-passing CCS, this relaction could be the smallest equivalence relation that is closed for $\alpha$-conversion and satisfies  
\begin{gather*}
P\parallel\nil \equiv P \qquad P\parallel Q \equiv Q \parallel P \qquad P\parallel(Q\parallel R) \equiv (P \parallel Q) \parallel R \\
P + \nil \equiv P \qquad P + Q \equiv Q + P \qquad P + (Q + R) \equiv (P + Q) + R
\end{gather*}

Notice that the reduction system presented above, together with the structural congruence relation, determines exactly the $\xrightarrow{\tau}$ transition of the LTS semantics presented in the previous section. Reduction systems are in fact often used to describe the dynamics of calculi with tho interaction with an "outside environment", so no input and output transitions, only inter-process communication.


Obiously, a bisimularity relation defined on a reduction system is a very coarse relation, that simply consider the number of computational steps a process can make. To recover the notion of strong LTS-bisimilarity in the reduction semantics setting, it is necessary to recover some of the "observational power" of labelled bisimulations, through the concept of \textit{barbs}.

We call \textit{barb} a predicate on states, often used to capture a certain notion of "observable property". Given a barb $b$, we write $s\downarrow_b$ to say that $s$ statisfies the predicate $b$, i.e. expresses that property. For value-passing CCS, a suitable observable property is "$P$ is capable to send any value con channel $c$". That is, we define the barb $c$ as the predicate 
\[\{P \mid \exists v, P' \ P \xrightarrow{c!v} P'\}\]
%or equivalently, 
%\[P \downarrow_c \text{ if and only if } P \xrightarrow{c!v} P'\]
It is important to remark that in this case we defined, for simplicity, a barb as a property based on a preexisting labelled semantic. In all the most recent calculus the semantic of a process can be formulated directly as a reduction system, and the barbs are usually syntactic in nature.

Given a set of barbs, i.e. a set of observable properties, is possible to define a \textit{barbed bisimulation}.
Let $\langle S , \rightarrow \rangle$ be a RS, and $B$ a set of barbs. Then a symmetric relation $\rel \subseteq S \times S$ is a \textit{barbed bisimulation} if and only if, whenever $s \rel t$, then 
\begin{itemize}
\item If $s \downarrow_b$ for some barb $b \in B$, then $t \downarrow_b$
\item If $s \rightarrow s'$ then $t \rightarrow t'$ for some $t'$ such that $s' \rel t'$
\end{itemize}
Two states $s, t \in S$ are said to be \textit{barbed bisimilar}, written $s \sim_b t$, if a barbed bisimulation $\rel$ exists such that $s \rel t$.

Barbed bisimilarity is often not useful per se, as for example $c!0.a!0.\nil \sim_b c!0.b!0.\nil$, since they both express the barb $\downarrow_c$, and have no outgoing transition. Barbed bisimulation is commonly used as the discriminating property of a contextual equivalence, called \textit{barbed equivalence}.

A context is a "process with a hole", for example $B[] = [] \parallel R$ or $B[] = [] + R$, where $R$ is any process. Given a context $B[]$, we define as $B[P]$ as the process obtained "filling" the hole with $P$, i.e. substituting $P$ in place of $[]$.

Given a set of contexts, two processes $P$ and $Q$ are said to be \textit{barbed equivalent}, or barbed congruent, written $P \simeq_b Q$, if for any context $C[]$, it holds that $C[P] \sim_b C[Q]$. With the previously defined barb $\downarrow_c$, and choosing just parallel context of the form $[] \parallel R$, it is possible to prove that
\[P \simeq_b Q \text{ if and only if } P \sim Q\]
that is, barbed equivalence is exactly the same as labelled bisimilarity.

To sum up, a reduction semantic allows to define a barbed equivalence relation, that: \begin{itemize}
\item Has the same power of labelled bisimilarity, but is defined in terms of a simpler transition system, inspired by term rewriting system.
\item Is "parametric" with respect to different barbs and different contexts, allowing for different observational power for the same calculus.
\item Is based on the very general concept of contextual equivalence, and it is used as the prototype of "natural, standard behavioural equivalence" for a lot of new calculi.
\item Is more difficult to prove, as it involves a universal quantification on all possible context, whereas proving bisimilarity requires as usual only to provide a bisimulation.
\end{itemize}




\chapter{Quantum Process Calculi}

\section{LTS and quantum states} 

There is a number of proposals of quantum process calculi in the literature, often with different syntax, semantics and behavioural equivalences, even if they all model the same systems and the same protocols. 
The are three main line of research that developed in recent years. The first, started with QPAlg and then developed with CQP, is inspired by the $\pi$-calculus. 
The second approach, developed simultaneously but independently, is centered around qCCS, that is a quantum extension of value-passing CCS. 
This thesis will focus on analyzing similarities and differences of these two calculi, CQP and qCCS. 
The third proposal, exploring the quantum process algebra qACP, is less directly related and comparable with the first two, in the same way as its classical counterpart ACP is designed in a different fashion with respect to CCS/$\pi$-calculus.


Since from the first works by Lalire and Jorrand [qpalg2004], it became evident that the operational semantic of a \textit{quantum} process calculus could not of consist only of syntactic elements, like the transitions $P \rightarrow P'$ of a classical process algebra. A process manipulating and communicating quantum data should always be coupled with a state vector, describing the current state of the quantum system. In all the quantum process calculi, the LTS is always composed of \textit{configurations}, i.e. states of the form 
\[
	\langle q_0, \ldots, q_n = \kp, P\rangle
\] 
in which $P$ is a process containing $q_0 \ldots q_n$ as free variables, and $\kp \in \Hto{n}$  describes the state of the qubits manipulated by $P$. 
This approach solves two crucial problems arising from the peculiarity of quantum computation:\begin{itemize}
\item \textbf{No cloning}: Since quantum information cannot be copied, variable instantiation cannot be performed in a "pass-by-value" fashion like classical process algebras:
	\[ c?x.P \xrightarrow{c?v} P[v/x]
	\] 
In this way, if $P$ contains two occurences of $x$, each of them gets instantiated with a different, independent copy of the value $v$. But if the value $v$ was a state vector $\kp$, this would require duplicating the quantum information.
So in QPAlg and in all other quantum calculi, information is manipulated and passed in a imperative, "pass-by-reference" manner:
\[ \langle q=\kp, c?x.P \rangle \xrightarrow{c?q} \langle q=\kp, P[q/x]\rangle
\]
where $q$ is just a pointer to the quantum variable stored in a configuration.
\item \textbf{Entanglement}: Since a composite quantum system is not always separable, the semantic of two parallel processes $P$ and $Q$ cannot always be described separately, and then simply interleaved with the parallel operator. If manipulating an entangled state the semantic of process $P$ depends by the behaviour of process $Q$, and so the two must be described together, in a global configuration $\langle q_1, q_2 = \beta, P \parallel Q$ (where $\beta$ is the bell state $\oost\ket{00} + \oost\ket{11}$. 
\end{itemize}

Another key feature present in QPAlg and in all other calculi is the coexistence of \textit{nondeterminism}, arising from sums and parallel composition, and probabilistic behaviour, arising from the probabilistic nature of quantum measurements. 
So in all quantum process calculi, a process can be defined by a pLTS $\langle Conf, Act, \rightarrow \rangle$, where $Conf$ is the set of all possible configurations. 
QPAlg and CQP follow the "probabilistic-transition" approach, while qCCS follows the $probabilistic-state$ approach.


in [CQP2005], Gay and Nagarajan presented their calculus Communicating Quantum Process. CQP makes use of an (affine) type system to restrict the set of possible processes of the algebra to the "admissible" ones, the ones respecting the no-cloning theorem. Under the assumption that Alice, Bob and Charlie are in three different physical location, the process \[Alice = b!q.c!q.nil\] should not be well typed, because Bob could read from the $b$ channel, Charlie from the $c$ channel, and there will be duplication of quantum information.


Variables and expressions in CQP can have types \textbf{Int}, \textbf{Qbit} and \textbf{Unit}, and channels have the correspogint types $\widehat{\ }\textbf{Int}$, $\widehat{\ }\textbf{Qbit}$, $\widehat{\ }\textbf{Unit}$.
The typing judgements in CQP have the form \[\Gamma \vdash P\] meaning that $P$ is well typed under the context $\Gamma$. $\Gamma$ contains both classical variables and quantum variables: the former are treated following the usual typing rules, the latter are subject to affine typing rules. Affine rules guarantee that each quantum variable will be sent at most once, thanks to how the typing contexts $\Gamma$ are constructed, lacking a contraction rules for quantum variables. \note{approfondisco una spiegazione sulle regole strutturali e il type system affine?}.


From the practical point of view, this means that, \begin{itemize} 
\item if $c!q.P$ is well typed, where $q$ is a quantum variable, then $P$ cannot contain any other occurence $q$
\item if $P \parallel Q$ is well typed, then $P$ and $Q$ cannot have occurrences of the same quantum variables
\end{itemize}
\note{
CQP2005 is a probabilistic-transtition, pi calculus like reduction system, MUST REPRESENTS CLOSED SYSTEMS, WITH REDUCTION RULES, BECAUSE GLOBAL STATE INFROMATION IS REQUIRED, otherwise we can't represent input and output of entangled state. cfr with qpalg2005 and qCCS 2006.
}

\note{
QPAlg2005 introduces a probabilistic branching bisimulation


Davidson introduces a labelled semantics  to CQP. ANd two probabilistic branching bisimulation, the second is a congruence


Puthoor develops the equational theory of davidson's bisimulation, and extends CQP to LOQC
}


 


\note{note:}
\note{\textbf{Inglesi}}
\begin{itemize}
\item \textbf{Lalire} configurazioni e probabilismo. complex variablescoping, with a stack in the configuration, reception extends rho. Non congruenza perchè entabglement e larsen skou.


\item \textbf{CQP gay nagarajan popl 05}: pi-calculus like, measurements are expressions, no probabilistic sum (can be implemented with parallel syncronization), reduction semantic con congruenza, typesystem affine per garantire il no cloning. probability-on-transitions approach: a reduction relation $\rightarrow \subseteq S \times \distr(S)$ and a probabilistic choice transition $\rightsquigarrow \subseteq \distr(S)\times [0, 1] \times S$. Configurations of the form (quantum state, channel names, P).

\item \textbf{Thesis Davidson 2011}:

Gives a labelled transition semantic $\langle \sigma, \omega, P\rangle \xrightarrow{alpha} \distr(\sigma, \omega, P)$ and a probabilistic transition  $\rightsquigarrow$ as before, where $\sigma$ contains the quantum state, $\omega$ the used qubits, and $P$ the process. Quantum input doesn't extend rho (here called sigma).

semantics: out removes $q$ from $\omega$, in and qbit add $q$ to $\omega$ 
typing: measure and ops don't add $q$ to $\Sigma$, but expression does. out removes, qbit adds, in should add. 
Sigma is a subset of omega

THe chinese approach equates the quantum names, and require the same final state.
the french-english approach doesn't equates the quantum names, but the (partial trace) of the state in the moment of communication.
Our congurence doesn't equates quantum names, it could if we add a more specific barb $\downarrow_{c!q}$, or a name-matching construct in our contexts. 
Ora come ora, nel nostro sistema, 
\[ P = H(q_1, q_2).c!q_1 \parallel d!q_2 \quad H(q_1, q_2).d!q_1 \parallel c!q_2\]
sono bisimili.

The example 3.2 in page 74 shows two processes that are bisimilar but not congruent. There are to solution to this problem: provide a finer bisimilation, that distinguishes P and Q, confronting the enviroment ($tr_\Sigma(\rho)$) in a larsen skou way, or a coarser bisimulation, that doen't distinguis C[P] and C[Q], confronting the environment of distribution (called in davidson mixed configuration.)


\item \textbf{thesis Puthoor 2015}:
provides a correct set of equational axioms, to define behavioural equivalence axiomatically. Extends CQP to Linear optical quantum computing
\end{itemize}


\note{\textbf{Cinesi}}
\begin{itemize}
\item Feng duan 2006, probabilistic bisimulation for quantum:

Probabilities-on-state approach, strong and week bisimilarity, deadlock quantum state equivalence, different inputs rules for correlated and uncorrelated qubits. Uncorrelated input extends rho. Introduces conbined transitions, i.e. convex closure transitions. bisimilarities not preserved by parallel composition, and restriction 
P = U1[q].c!0.U2[q].nil, Q = V1[q].c!0.V2[q].nil. are bisimilar, but not $P\setminus c$ and $Q\setminus c$
\item Ying feng 2009, an algebra of quantum, no classical comunxication. Input and output don't change rho. superoperators as visible transitions, reduction (i.e. independent superoperators) bisimilarity, approximate bisimulation based on diamond distance between superoperators

\item Feng duan ying bisimulation for quantum, is a congruence, requires equality of the environment, not of the total state. 
\item Open bisimulation for quantum
\end{itemize}





\chapter{Linear qCCS}

\section{Definitions}
\subsection{Syntax}
\begin{align*}
  P \Coloneqq &\ K \mid c!e 
%  \mid discard(e) 
  \mid \bigparallel_{i \in I} P_i \\[0.1cm]
  K \Coloneqq &\ nil \mid \tau . P \mid \mathcal{E}(\widetilde{x}) . P \mid M(\widetilde{x} \rhd y) . P \mid c?x . P \mid \\[0.1cm]
              &\ [e] P \mid \sum_{i \in I} K_i \mid P \{f\} \mid P \setminus L \mid A(\widetilde{x}) \\[0.1cm]
  e \Coloneqq &\ x \mid b \mid n \mid q \mid \neg e \mid e \lor e \mid e \leq e
\end{align*}	
where $b \in \mathbb{B}$, $n \in \mathbb{N}$, $x \in \text{Var}$, $q \in \text{QC}$, with $Var$ a denumerable set of variable names, and
QC a set of names with cardinality equal to the size of the chosen Hilbert space.

We use $discard(e)$ as syntactic sugar for $c!e \setminus c$.

\subsection{Type System}

Variables Types: $\set{\mathcal{Q}, \mathbb{N}, \mathbb{B}}$
Channel types: $\set{\hat{\mathcal{Q}}, \hat{\mathbb{N}}, \hat{\mathbb{B}}}$

\begin{gather*}
\infer[\mbox{\footnotesize\scshape CBool}]{\vdash b : \mathbb{B}}{b \in \mathbb{B}} \qquad
\infer[\mbox{\footnotesize\scshape CNat}]{\vdash n : \mathbb{N}}{n \in \mathbb{N}} \qquad
\infer[\mbox{\footnotesize\scshape QVar}]{\set{x} \vdash x}{} \qquad
\infer[\mbox{\footnotesize\scshape CVar}]{x : T \vdash x : T}{} \\[0.2cm]
\infer[\mbox{\footnotesize\scshape BoolOr}]{\Gamma_1 \cup \Gamma_2 \vdash e_1 \lor e_2 : \mathbb{B}}{\Gamma_1 \vdash e_1 : \mathbb{B} & \Gamma_2 \vdash e_2 : \mathbb{B}} \qquad
\infer[\mbox{\footnotesize\scshape BoolNeg}]{\Gamma \vdash \neg e : \mathbb{B}}{\Gamma \vdash e : \mathbb{B}} \qquad
\infer[\mbox{\footnotesize\scshape NatLEq}]{\Gamma_1 \cup \Gamma_2 \vdash e_1 \leq e_2 : \mathbb{B}}{\Gamma_1 \vdash e_1 : \mathbb{N} & \Gamma_2 \vdash e_2 : \mathbb{N}} \\[0.2cm]
 \infer[\mbox{\footnotesize\scshape CWeak}]{\Gamma, x : T; \Sigma \vdash P}{\Gamma; \Sigma \vdash P & \text{$x$ fresh}} \\[0.2cm]
% \infer[\mbox{\footnotesize\scshape QWeak}]{\Gamma; \Sigma, x \vdash P}{\Gamma; \Sigma \vdash P} \\[0.2cm]
\infer[\mbox{\footnotesize\scshape Nil}]{\emptyset; \emptyset \vdash nil}{} \qquad
\infer[\mbox{\footnotesize\scshape Tau}]{\Gamma; \Sigma \vdash \tau . P}{ \Gamma; \Sigma \vdash P} \qquad
\infer[\mbox{\footnotesize\scshape Guard}]{\Gamma_1 \cup \Gamma_2; \Sigma \vdash [e] P}{\Gamma_1 \vdash e : \mathbb{B} & \Gamma_2; \Sigma \vdash P} \\[0.2cm]
\infer[\mbox{\footnotesize\scshape QOp}]
{\Gamma; \Sigma \vdash \mathcal{E}(\widetilde{x}) . P}
{\mathcal{E}: Op(n) & |\widetilde{x}| = n & \forall i, j \ldotp x_i \neq x_j & \Sigma \vdash \widetilde{x} & \Gamma; \Sigma \vdash P} \\[0.2cm]
\infer[\mbox{\footnotesize\scshape QMeas}]{\Gamma; \Sigma \vdash M(\widetilde{x} \rhd y) . P}
{\forall i, j \ldotp x_i \neq x_j & \text{$y$ fresh} & \Sigma \vdash \widetilde{x} & \Gamma, y : \mathbb{N}; \Sigma \vdash P} \\[0.2cm]
\infer[\mbox{\footnotesize\scshape CRecv}]
{\Gamma, c : \hat{T}; \Sigma \vdash c?x . P}
{\text{$x$ fresh} & T \neq \mathcal{Q} & \Gamma, x : T; \Sigma \vdash P} 
\qquad
\infer[\mbox{\footnotesize\scshape QRecv}]
{\Gamma, c : \hat{\mathcal{Q}}; \Sigma \vdash c?x . P}
{\text{$x$ fresh} & \Gamma, x : \mathcal{Q}; \Sigma, x \vdash P} 
\\[0.2cm]
\infer[\mbox{\footnotesize\scshape CSend}]{\Gamma, c : \hat{T}; \emptyset \vdash c!e}{T \neq \mathcal{Q} & \Gamma \vdash e : T} \qquad
\infer[\mbox{\footnotesize\scshape QSend}]{c : \hat{\mathcal{Q}}, e:\mathcal{Q} ; \set{e} \vdash c!e}{} \\[0.2cm]
%\infer[\mbox{\footnotesize\scshape QDiscard}]{e:\mathcal{Q} ; \set{e} \vdash discard(e)}{} \\[0.2cm]
\infer[\mbox{\footnotesize\scshape Sum}]{\bigcup_{i \in I}\Gamma_i; \bigcup_{i \in I}\Sigma_i \vdash \sum_{i \in I} P_i}{\forall i \ldotp \Gamma_i; \Sigma_i \vdash P_i} \qquad 
\infer[\mbox{\footnotesize\scshape Par}]{\bigcup_{i \in I}\Gamma_i; \bigcup_{i \in I}\Sigma_i \vdash \bigparallel_{i \in I} P_i}
{\forall i, j \ldotp \Sigma_i \cap \Sigma_j = \emptyset & \forall i \ldotp \Gamma_i; \Sigma_i \vdash P_i} \\[0.2cm]
\infer[\mbox{\footnotesize\scshape Rename}]{\Gamma; \Sigma \vdash P \{f\}}{f(\Gamma); \Sigma \vdash f(P)} \qquad
\infer[\mbox{\footnotesize\scshape Restrict}]{\Gamma; \Sigma \vdash P \setminus L}{\Gamma; \Sigma \vdash P} \\[0.3cm]
\end{gather*}

\begin{definition}
  Let $\rho$ and $P$ be an arbitrary partial density operator and a process. $\Gamma; \Sigma \vdash \langle \rho, P \rangle$ iff $\Gamma; \Sigma \vdash P$.

  Let $I$ be an arbitrary index set. $\Gamma; \Sigma \vdash \boxplus_{i \in I} \langle \rho_i, P_i \rangle$ iff for each $i \in I$ such that $\rho_i \neq \mathbf{0}$,
  then $\Gamma; \Sigma \vdash \langle \rho_i, P_i \rangle$.
\end{definition}




\section{Semantics}
We present a reduction semantics for lqCCS, consistent with the labelled semantic for qCCS presented in \cite{fengBisimulationQuantumProcesses2012, dengOpenBisimulationQuantum2012}. A reduction semantic does not make any assumption on the observable properties of the system (like the labels of a transition), and so is better suited to explore and compare different notion of behavioural equivalence.

Besides, as explained in Chapter \ref{chapter3}, in a labelled transition system a quantum input transition like $\xrightarrow{c?q}$ requires the value of qubit $q$ to be already present in the state. This is an atypical assumption, as a labeled transition usually model the communication with an unknown external environment, but in this case requires at least some partial knowledge of the environment. In a reduction system there are no such labelled transitions, so a process communicates only with other processes, on which we have total information.

Our semantics defines a probabilistic reduction system $\langle S, \rightarrow \rangle$, where \begin{itemize}
\item $S$ is a set of \textit{configurations}, of the form $\confw{\rho, P}$ (like in qCCS).
\item $\rightarrow \subseteq S \times \distr(S)$ is the probabilistic transition relation, corresponding to the $\xrightarrow{\tau}$ transition in qCCS \cite{fengBisimulationQuantumProcesses2012, dengOpenBisimulationQuantum2012}.
\end{itemize}

We assume a fixed set  $QN = {q_1, q_2, \ldots q_n}$ of quantum names, where each name $q_i$ refers to a unique qubit with state space $\calH_i$. We denote as $\calH_{QN}$ the $2^n$-dimensional Hilbert space $\bigotimes_{i=1}^n \calH_i$, and so any state 
$\rho \in \mathcal{D}(\calH_{QN})$ associates each name with a value.

We also assume a fixed typing contest $\Gamma_c = \set{c_i: \widehat{T_i}}_i$, containing typing assumptions for a finite set of classical and quantum channels.

\begin{definition}
Let $P$ be a process and $\rho \in \mathcal{D}(\calH_{QN})$ an arbitrary density operator. We say that a configuration $\confw{\rho, P}$ is well typed, given a set of quantum names $QN$ and a set of typed channels $\Gamma_c$, if $\Gamma_c; \Sigma \vdash P$ for some $\Sigma \subseteq QN$.

%  Let $I$ be an arbitrary index set. $\Gamma; \Sigma \vdash \boxplus_{i \in I} \langle \rho_i, P_i \rangle$ iff for each $i \in I$ such that $\rho_i \neq \mathbf{0}$, then $\Gamma; \Sigma \vdash \langle \rho_i, P_i \rangle$.
\end{definition}

Notice that the context $\Gamma_c$ contains assignments only for channels, not for classical variables. This means that in well typed configuration, $P$ does not contain any free classical variable, and all the free quantum variables are references to qubits in the configuration.

From now on, we will consider only well typed configurations.

\subsection{Reduction System}
In order to define the reduction transition, we first need to introduce a semantic for expressions and a structural congruence relation on processes, like in \cite{gayCommunicatingQuantumProcesses2005}.

We consider as a \textit{value} any expressions $n \in \mathbb{N},\  b \in \mathbb{B},\  x \in  \text{Var}$, and use $v$ as a metavariable for them.

In Figure \ref{big_step_exp}, we define a big step semantic for classical and quantum expression in the usual way. We write $e \Downarrow v$ to indicate that the expression $e$ evaluates to value $v$. Recall that the only quantum expression admitted by the type system are quantum variables.

\begin{figure}[h!]
\begin{gather*}
\infer[\mbox{\footnotesize\scshape Var}]{x \Downarrow x}{} \qquad
    \infer[\mbox{\footnotesize\scshape Nat}]{n \Downarrow n}{} \qquad
    \infer[\mbox{\footnotesize\scshape Bool}]{b \Downarrow b}{}  \qquad
\\[0.3cm]
    \infer[\mbox{\footnotesize\scshape Or}]{(e_1 \lor e_2) \Downarrow b}{e_1 \Downarrow b_1 & e_2 \Downarrow b_2 & b = b_1 \lor b_2} \qquad
    \infer[\mbox{\footnotesize\scshape Neg}]{\neg e \Downarrow b}{e \Downarrow b_1 & b = \neg b_1} \\[0.3cm]
    \infer[\mbox{\footnotesize\scshape Leq}]{(e_1 \leq e_2) \Downarrow b}{e_1 \Downarrow n_1 & e_2 \Downarrow n_2 & b = n_1 \leq n_2}
\end{gather*}
\caption{Big step semantic for Linear qCCS expressions}
\label{big_step_exp}
\end{figure}

We define as the \textit{structural congruence relation} $\equiv$ the smallest equivalence relation that satisfies the axioms in Figure \ref{str_cong}. We employed the usual axioms of \cite{milnerFunctionsProcesses1990} for parallel composition, summation and reduction. We also add axioms to evaluate classical expressions and resolve \textbf{If Then Else} constructs.

\begin{figure}[h!]
\begin{gather*}
    \infer[\mbox{\footnotesize\scshape ParNil}]{P \parallel nil \equiv P}{} \qquad
    \infer[\mbox{\footnotesize\scshape ParComm}]{P \parallel Q \equiv Q \parallel P}{} \qquad
    \infer[\mbox{\footnotesize\scshape ParAssoc}]{P \parallel (Q \parallel R) \equiv (P \parallel Q) \parallel R}{} 
    \\[0.3cm]
    \infer[\mbox{\footnotesize\scshape SumNil}]{M + nil \equiv M}{} \qquad
    \infer[\mbox{\footnotesize\scshape SumComm}]{M + N \equiv N + M}{} \qquad
    \infer[\mbox{\footnotesize\scshape SumAssoc}]{M + (N + O) \equiv (M + N) + O}{} 
    \\[0.3cm]
    \infer[\mbox{\footnotesize\scshape RestrOrd}]{P \setminus c \setminus d \equiv P \setminus d \setminus c}{} \qquad 
    \infer[\mbox{\footnotesize\scshape RestrNil}]{nil \setminus c \equiv nil}{} \qquad 
    \infer[\mbox{\footnotesize\scshape RestrPar}]{(P \parallel Q) \setminus c \equiv P \parallel (Q \setminus c)}{c \not\in fc(P)} \\[0.3cm]
    \infer[\mbox{\footnotesize\scshape TrueGuard}]{\ITE{tt}{P}{Q} \equiv P}{} \qquad
    \infer[\mbox{\footnotesize\scshape FalseGuard}]{\ITE{ff}{P}{Q} \equiv Q}{} \\[0.3cm]
    \infer[\mbox{\footnotesize\scshape ValExpr}]{ P \equiv P[\sfrac{v}{e}] }{e \Downarrow v} 
	\end{gather*}	
\caption{Structural congruence for Linear qCCS}
\label{str_cong}
\end{figure}

We can now define the transition relation $\rightarrow$, presented in Figure \ref{reduction}. As usual, we write $\confw{\rho, P} \rightarrow \Delta$ to intend $(\confw{\rho, P}, \Delta) \in \rightarrow$. To lighten the notation, we will also write $\confw{\rho, P} \rightarrow \confw{\rho', P'}$ instead of $\confw{\rho, P} \rightarrow \overline{\confw{\rho', P'}}$.

This transition relation cannot be composed with itself, since it is a relation from configurations to distributions. As in \cite{fengBisimulationQuantumProcesses2012}, we can \textit{lift} the relation $\rightarrow$ into $\slift{\rightarrow}$, that is a relation from distributions to distributions, as described in Section \ref{pLTS}. This is useful to talk about reachability and temporal logics, but will not be used in the definitions of strong bisimulations in the next sections.

\begin{figure}[h!]
  \begin{gather*}
    \infer[\mbox{\footnotesize\scshape SemTau}]{\langle \rho, \tau . P \rangle \longrightarrow \langle \rho, P \rangle}{} \\[0.3cm]
    \infer[\mbox{\footnotesize\scshape SemRename}]{\langle \rho, P \{f\} \rangle \longrightarrow \langle \rho', P' \{f\}\rangle}{\langle \rho, f(P) \rangle \longrightarrow \langle \rho', f(P') \rangle} \qquad
    \infer[\mbox{\footnotesize\scshape SemRestrict}]{\langle \rho, P \setminus L \rangle \longrightarrow \langle \rho, P' \setminus L \rangle}{\langle \rho, P \rangle \longrightarrow \langle \rho', P' \rangle} \\[0.3cm]
    \infer[\mbox{\footnotesize\scshape SemQOp}]{\langle \rho, \mathcal{E}(\widetilde{x}) . P \rangle \longrightarrow \langle \mathcal{E}_{\widetilde{x}}(\rho), P \rangle}{} \\[0.3cm]
    \infer[\mbox{\footnotesize\scshape SemQMeas}]{\langle \rho, M(\widetilde{x} \rhd y) . P \rangle \longrightarrow \sum_{m = 0}^{2^{|\widetilde{x}|}} p_m(\rho) \left\langle \frac{1}{p_m}\rho_m , P[\sfrac{m}{y}] \right\rangle}{\rho_m = M_m \rho M_m^\dag & p_m = tr(\rho_m)} \\[0.3cm]
    \infer[\mbox{\footnotesize\scshape SemPar}]{\langle \rho, P \parallel R \rangle \longrightarrow \langle \rho', P' \parallel R \rangle}{\langle \rho, P \rangle \longrightarrow \langle \rho', P' \rangle} \qquad
    \infer[\mbox{\footnotesize\scshape SemSum}]{\langle \rho, P + R \rangle \longrightarrow \langle \rho', P' \rangle}{\langle \rho, P \rangle \longrightarrow \langle \rho', P' \rangle} \\[0.3cm]
    \infer[\mbox{\footnotesize\scshape SemReduce}]{\langle \rho, c!v \parallel c?x . P \rangle \longrightarrow \langle \rho, P[\sfrac{v}{x}] \rangle}{} \\[0.3cm]
    \infer[\mbox{\footnotesize\scshape SemCongr}]{\langle \rho, P \rangle \longrightarrow \langle \rho', P' \rangle}
    {P \equiv Q & \confw{\rho, Q} \rightarrow \confw{\rho', Q'} & Q' \equiv P'}
  \end{gather*}
\caption{Reduction system for Linear qCCS}
\label{reduction}
\end{figure}
  
\begin{example}
We can formalize in lqCCS the famous \textit{quantum teleportation} algorithm:
\begin{align*}
  \proc{A} &\Coloneqq \text{in}_a?x.\text{CNOT}(q_0, x).\text{H}(q_0).M(q_0, x \rhd n).(\text{m}_ab!n \parallel discard(q_0, x)\\
  \proc{B} &\Coloneqq \text{in}_b?x.\text{m}_{ab}?n.
     \\ & \ite{n = 0}{I(x).\text{out}_b!x\\&}
    {\ite{n = 1}{X(x).\text{out}_b!x\\&}
    		{\ite{n = 2}{Z(x).\text{out}_b!x\\&}{Y(x).\text{out}_b!x}}
    }\\
  \proc{S} &\Coloneqq Set_{\Phi^+}(q_1, q_2).(\text{in}_a!q_1 \parallel \text{in}_b!q_2) \\
  \proc{Tel} &\Coloneqq (A \parallel B \parallel S) \setminus \Set{\text{in}_a, \text{in}_b, \text{m}_{ab} }
\end{align*}
Which is well typed under the context 
$\Gamma = \set{\text{out}_b : \hat{\mathcal{Q}}}$, $\Sigma = \set{q_0, q_1, q_2}$.
\end{example}
\subsection{Type system properties}

\begin{theorem}[Evaluation Preserves Typing]\label{thm:eval_typing}
  If $\Gamma \vdash e$ and $e \Downarrow v$, then $\Gamma \vdash v$.
\end{theorem}
\begin{proof}
  Follows by induction on the evaluation rules.
\end{proof}

\begin{theorem}[Structural Congruence Preserves Typing]
  If $\Gamma; \Sigma \vdash P$ and $P \equiv Q$, then $\Gamma; \Sigma \vdash Q$.
\end{theorem}
\begin{proof}
  By induction on the derivation of $P \equiv Q$. All rules follow trivially from: the rules of the operators,
  \textsc{Par}, \textsc{Sum}, which are commutative, associative, have $nil$ as unit and require
  each component to be typable; the \textsc{Nil} rule; and the \textsc{Guard} rule which requires the guarded process
  to be typable.
\end{proof}

\begin{theorem}[Quantum weakening]
  Let $\Gamma; \Sigma \vdash P$ and let $x : \mathcal{Q}$. If $x \not\in \Sigma$, then $\Gamma, x : \mathcal{Q}; \Sigma, x \vdash P$.
\end{theorem}
\begin{proof}
  By induction on the derivation of $\Gamma; \Sigma \vdash P$.
\end{proof}

\begin{theorem}[Substitution in Process]
  Let $\Gamma, x : T, \Gamma'; \Sigma \vdash P$ and let $v : T$ be a value such that:
  \begin{itemize}
    \item if $T = \mathcal{Q}$ and $v \not\in \Sigma$, then $\Gamma, \Gamma'; \Sigma[\sfrac{v}{x}] \vdash P[\sfrac{v}{x}]$.
    \item if $T \neq \mathcal{Q}$, then $\Gamma, \Gamma'; \Sigma \vdash P[\sfrac{v}{x}]$.
  \end{itemize}
\end{theorem}
\begin{proof}
  By structural induction on the deriviation of $\Gamma, x : T, \Gamma'; \Sigma \vdash P$.
  Let us analyze the interesting cases: For the \textsc{QOp} rule it must be that $P = \mathcal{E}(\widetilde{x}).Q$ for some process $Q$.
  By induction hypothesis it holds that $\Gamma, \Gamma'; \Sigma' \vdash Q[\sfrac{v}{x}]$, however we must consider two cases:
  if $v : \mathbb{N}$ or $v : \mathbb{B}$ then the conclusion follows trivially from the inductive hypothesis;
  if $v : Q$ then $x \not\in \Sigma$ and $\Sigma' = \Sigma, v$, thus $v$ is not in $\widetilde{x}$ and we can reapply the \textsc{QOp} rule to obtain
  $\Gamma, \Gamma'; \Sigma' \vdash \mathcal{E}(\widetilde{x}).Q$. The same line of reasoning is valid for the \textsc{QMeas} rule.
  The \textsc{QSend} rule is also guaranteed by the $v \not\in \Sigma$ requirement when $v : Q$.
  Finally, for the \textsc{Par} rule in the case of $v : Q$ and $x \in \Sigma$: the rule deriviation imposes that only one of the components, $P_i$,
  contains the variable $x$ in its quantum environment, $\Sigma_i$, thus by induction we obtain $\Gamma, \Gamma'; \Sigma[\sfrac{v}{x}] \vdash P_i$.
  While for the other components the substitution has no effect, thus $\Gamma, \Gamma'; \Sigma \vdash P_i$. However, since $v \not\in \Sigma$, we can
  conclude that all smaller environment are still pairwise distinct, thus we can infer $\Gamma, \Gamma'; \Sigma[\sfrac{v}{x}] \vdash \parallel_{i \in I} P_i$.
\end{proof}

\begin{theorem}[Typing Preservation]
  If $\Gamma; \Sigma \vdash P$ and $\langle \rho, P \rangle \longrightarrow \boxplus_{i \in I} \langle \rho_i, P_i \rangle$ then
  $\forall i \in I \ldotp \Gamma; \Sigma \vdash P_i$.
\end{theorem}
\begin{proof}
  By structural induction on the transition relation $\longrightarrow$.
  Let us analyze the interesting cases: if the last step in the derivation is a \textsc{SemQMeas} rule, then $P = M(\widetilde{x} \rhd y).Q$ for some process $Q$,
  where $Q$ is typed with $\Gamma, m : \mathbb{N}; \Sigma \vdash Q$. Each component of the box sum is of the form $Q[\sfrac{m}{y}]$ with $m \in \mathbb{N}$, thus
  by the substitution theorem it holds that $\Gamma; \Sigma \vdash Q[\sfrac{m}{y}]$.
  If the last step is a \textsc{SemPar} rule, then $P = Q \parallel R$ for some processes $Q$ and $R$, where
  $\Gamma; \Sigma_1 \vdash Q$ and $\Gamma; \Sigma_2 \vdash R$ with $\Sigma = \Sigma_1 \cup \Sigma_2$ and $\Sigma_1 \cap \Sigma_2 = \emptyset$.
  By induction $\Gamma; \Sigma_1 \vdash Q'$, however since the conditions on the $\Sigma$ are still true, it also holds that $\Gamma; \Sigma \vdash Q' \parallel R$.
  The argument is similar for the \textsc{SemSum} rule.
  If the last step is a \textsc{SemReduce} rule, then $P = c!e \parallel c?x.Q$ for some process $Q$. If $c : \hat{T}$ where $T = \mathbb{N}$ or $T = \mathbb{B}$, then
  the theorem holds trivially by the substitution theorem. If $c : \hat{\mathcal{Q}}$ then we can still apply the substitution theorem since by virtue of the \textsc{Par} rule,
  it must be that $\Gamma; \Sigma_1 \vdash c!e$ and $\Gamma; \Sigma_2 \vdash c?x.Q$ with $\Sigma_1 \cap \Sigma_2 = \emptyset$, but by the \textsc{QSend} rule $v$ must be in $\Sigma_1$,
  therefore $v \not\in \Sigma_2$ and thus $\Gamma; \Sigma_2[\sfrac{v}{x}] \vdash Q[\sfrac{v}{x}]$, and by weakening $\Gamma; \Sigma \vdash Q[\sfrac{v}{x}]$.
\end{proof}


\section{Contextual Equivalence}
Density operators represents equivalence classes over probabilistic mixtures of quantum states.
The implicit equivalence relation is $\{(\ket{\phi_i}, p_i)\}_i \cong \{(\ket{\phi_j}, p_j)\}$ iff $\sum_{i} p_i \ketbra{\phi_i}{\phi_i} = \sum_{j} p_j \ketbra{\phi_j}{\phi_j}$.
The physical justification of this equivalence is that different mixtures resulting in the same density operator cannot be distinguished since they behave the same.

The same equivalence relation is trivially extended to configurations, where the use of density operators for the quantum state allows a common representation of different configurations with an equivalent mixtures of quantum states. 
We extend here this equivalence relation to distributions of configurations.
Intuitively, we give rules for exchanging the probabilistic combination $\psum{p}$ in the state of some configurations for probabilistic combination of configurations, and vice-versa. 

\begin{definition}
	A \emph{quantum distribution} is an equivalence class $\mathbf{\Delta}$ of probabilistic distributions $\Delta$ over quantum configurations defined by the minimal equivalence relation such that:
	\begin{itemize}
		\item $(\overline{\confw{\rho, P}} \psum{p} \overline{\confw{\sigma, P}}) \equiv \overline{\confw{p \rho + (1-p)\sigma, P}}$; and
%		\item $(\overline{\confw{\rho \otimes \sigma, P}} \psum{p} \overline{\confw{\rho \otimes \sigma', Q}}) \equiv (\overline{\confw{\rho \otimes \delta, P}} \psum{p} \overline{\confw{\rho \otimes \delta', Q}})$ if $\Gamma, \Sigma \vdash P$, $\Gamma, \Sigma \vdash Q$, $\rho \in \hilbert_\Sigma$, and $p \sigma + (1 - p) \sigma' =  p \sigma + (1 - p) \sigma' = p \delta + (1 - p) \delta'$; and	
		\item $\Delta_i\ \equiv\ \Theta_i$, $i = 1, 2$, implies $\Delta_1 \psum{p} \Delta_2 \equiv\ \Theta_1 \psum{p} \Theta_2$.
	\end{itemize}
	We write $Q(Conf)$ for quantum distributions over configurations.
\end{definition}

\begin{definition}
	Given $\rel \subseteq Conf \times Conf$ be a relation over quantum configurations, let its quantum lifting be the minimal relation $\sqlift{\rel} \subseteq Q(Conf) \times Q(Conf)$ over quantum distributions such that $\Delta\ \slift{\rel}\ \Theta$ with $\Delta \in \mathbf{\Delta}$ and $\Theta \in \mathbf{\Theta}$ implies $\mathbf{\Delta}\ \sqlift{\rel}\ \mathbf{\Theta}$.
\end{definition}

%\begin{lemma}
%	For any $\rel$, $\slift{\rel}$ is left and right decomposable, i.e.,
%	\begin{itemize}
%		\item $(\Delta_1 \psum{p} \Delta_2) \slift{\rel} \Theta$ implies $\Theta = \Theta_1 \psum{p} \Theta_2$ and $\Delta_i \slift{\rel} \Theta_i$, $i = 1, 2$; and
%		\item $\Delta \slift{\rel} (\Theta_1 \psum{p} \Theta_2)$ implies $\Delta = \Delta_1 \psum{p} \Delta_2$ and $\Delta_i \slift{\rel} \Theta_i$, $i = 1, 2$.
%	\end{itemize}
%\end{lemma}
%\begin{proof}
%	We prove only the first point, the second being immediately derivable.
%\end{proof}

\begin{theorem}
	Let $\rel \subseteq Conf \times Conf$ be an equivalence relation over quantum configurations, then $\sqlift{\rel}$ is an equivalence relations over distributions of quantum configurations.
\end{theorem}
\begin{proof}
	Reflexivity and symmetry holds by definition given that $\rel$ is an equivalence relation.
	For transitivity, assume $\Delta \slift{\rel} \Theta \slift{\rel} \Xi$.
	{\color{red} decomponibilita' non vale con le definizioni che abbiamo dato...}
\end{proof}
{\color{red} TODO: capire come gestire il fatto che vogiamo che $\approx$ sia una relazione di equivalenza.}

%\begin{definition}
%	Given $\rel \subseteq Conf \times Conf$ be a relation over quantum configurations, let its quantum lifting be the minimal relation $\slift{\rel} \subseteq D(Conf) \times D(Conf)$ such that
%	\begin{itemize}
%		\item $\conf\ \rel\ \conf'$ implies $\overline{\conf}\ \slift{\rel}\ \overline{\conf'}$;
%		\item $\Delta_i\ \slift{\rel}\ \Theta_i$, $i = 1, 2$, implies $\Delta_1 \psum{p} \Delta_2 \slift{\rel}\ \Theta_1 \psum{p} \Theta_2$;
%		\item $(\overline{\confw{\rho, P}} \psum{p} \overline{\confw{\sigma, P}}) \slift{\rel} \overline{\confw{p \rho + (1-p)\sigma, P}}$;
%		\item $\overline{\confw{p \rho + (1-p)\sigma, P}} \slift{\rel} (\overline{\confw{\rho, P}} \psum{p} \overline{\confw{\sigma, P}})$.
%	\end{itemize} 
%\end{definition}

%\begin{theorem}
%	Let $\rel \subseteq Conf \times Conf$ be an equivalence relation over quantum configurations, $\slift{\rel}$ is an equivalence relations over distributions of quantum configurations.
%\end{theorem}
%\begin{proof}
%	Reflexivity and symmetry holds by definition given that $\rel$ is an equivalence relation.
%	For transitivity, assume $\Delta \slift{\rel} \Theta \slift{\rel} \Xi$.
%	{\color{red} decomponibilita' non vale con le definizioni che abbiamo dato...}
%\end{proof}


%In the following we write $qv(P)$ or $qv(\confw{\rho, P})$ for the set of free quantum variables in $P$, and $env(\confw{\rho, P})$ for $tr_{qv(P)}(\rho)$.

\begin{definition}[Barb]
	A \emph{barb} is a predicate $\downarrow_{c}$ over configurations where $\langle \rho, P \rangle \downarrow_{c}$ iff $P \equiv c!x + Q \parallel R$ for some $x$, and processes $Q$, $R$.
%	A \emph{barb} is a predicate $\downarrow_{p, c}$ over configurations where $\langle \rho, P \rangle \downarrow_{p, c}$ iff $tr(\rho) = p$ and
%	$P \equiv c!x + Q \parallel R$ for some $x$, and processes $Q$, $R$.
\end{definition}

%\begin{definition}[Barb Preserving Relation]
%	A relation $\rel \subseteq \conf \times \conf$ is \emph{barb preserving} if $\conf \rel \conf'$ implies that $\barb{p}{c}{\conf}$ iff $\barb{p}{c}{\conf'}$
%	for any $p \in [0,1]$ and any classical channel $c$.
%\end{definition}

%\begin{definition}[Probabilistic Barbed Bisimulation]
%	A symmetric relation $\rel_{\Gamma, \Sigma} \subseteq \conf \times \conf$ is \emph{probabilistic barbed bisimulation} if $\conf \rel \conf'$, with $\conf, \conf'$ well-typed under the context $\Gamma; \Sigma$, implies that 
%	\begin{itemize}
%		\item if $\conf \downarrow_{c}$ then $\conf' \downarrow_{c}$; and 
%		\item whenever $\conf \xrightarrow{\tau} \Delta$, there exists $\Delta'$ such that $\conf' \xrightarrow{\tau} \Delta'$ and $\Delta \slift{\rel} \Delta'$
%%		\item whenever $\conf' \xrightarrow{\tau} \Delta'$, there exists $\Delta$ such that $\conf \xrightarrow{\tau} \Delta$ and $\Delta \slift{\rel} \Delta'$
%	\end{itemize}
%\end{definition}

{\color{red} Non capisco dove, ma ci vuole: if $\confw{\rho, P} \approx \confw{\rho', P'}$ then $\confw{\sigma \otimes \rho, P} \approx \confw{\sigma \otimes \rho', P'}$}

\begin{definition}[Saturated Probabilistic Barbed Bisimilarity]
	A symmetric relation $\rel \subseteq \conf \times \conf$ is \emph{saturated probabilistic barbed bisimulation} if $\conf\ \rel\ \conf'$ implies that $\conf, \conf'$ are well-typed under a typing context $\Gamma; \Sigma$, and for any context $C[\_] \in Context_{\Gamma, \Sigma}$
	\begin{itemize}
		\item if $C[\conf] \downarrow_{c}$ then $C[\conf'] \downarrow_{c}$; and 
		\item whenever $C[\conf] \xrightarrow{\tau} \Delta \in \mathbf{\Delta}$, there exists $\Delta' \in \mathbf{\Delta'}$ such that $C[\conf'] \xrightarrow{\tau} \Delta'$ and $\mathbf{\Delta}\ \sqlift{\rel}\ \mathbf{\Delta'}$
		%		\item whenever $\conf' \xrightarrow{\tau} \Delta'$, there exists $\Delta$ such that $\conf \xrightarrow{\tau} \Delta$ and $\Delta \slift{\rel} \Delta'$
	\end{itemize}
	Let \emph{saturated probabilistic barbed bisimilarity} $\approx_{SPB}$ be the largest saturated probabilistic barbed bisimulation.
\end{definition}

\note{
Comunque, anche usando la freccia $\rightarrow_q$, mi sa che comunque serve una nozione di lifting quantum della relazione. Infatti, supponiamo una semplice bisimulazione LS con freccia quantum: $\conf \rel 	\conf'$ implica \begin{itemize}
\item $\conf\downarrow_c$ implica $\conf'\downarrow_c$
\item $B[\conf] \rightarrow_q \Delta$ implica $B[\conf'] \rightarrow_q \Delta'$ e $\Delta \lrel \Delta'$ (solo lifting prob.)
\end{itemize}
 le configurazioni $ \conf = \confw{\proj{\beta}, M_{01}[q \rhd x].disc(q)  } $ e 
 $\conf' = \confw{\proj{\beta}, M_\pm[q \rhd x].disc(q)}$ non sono bisimili. 
 Infatti, è vero che $\conf \rightarrow_q \confw{\proj{00} \psum{\frac{1}{2}} \proj{11}, disc(q)}$, e allora $\conf'$ ha la transizione $\conf' \rightarrow_q \confw{\proj{++} \psum{\frac{1}{2}} \proj{--}, disc(q)}$ e possiamo dimostrare che sono uguali, in qualche modo. Ma $\conf$ ha anche la transizione $\conf \rightarrow_q \confw{\proj{00}, disc(q)} \psum{\frac{1}{2}} \confw{\proj{00}, disc(q)}$, che va in una distribuzione, e a questo punto la relazione liftata $\lrel$ vuole uan distribuzione di configurazioni  bisimili punto a punto.

Quindi abbiamo comunque bisogno di una relazione liftata $\sqlift{\rel}$, cioè una relazione fra classi di distribuzioni, che però rende le bisimulazioni non chiuse per composizione.
}

By writing $\sop \in TSO(\hilbert_{\Sigma})$, we mean that $\sop(\rho) = \sum_{i \in I} (I_{\overline{\Sigma}} \otimes A_i) \rho (I_{\overline{\Sigma}} \otimes A_i)^{\dagger}$.

%{\color{red} Per la misura come non-trace preserving, che noi usiamo nella regola di misura della semantica, abbiamo la stessa proprieta', in particolare, rispetto alla versione trace-preserving avremmo semplicemente che la sommatoria sara' composta da meno elementi.}

\begin{lemma}\label{lemma:sop}
	For any pair of configurations $\conf, \conf'$ well-typed under $\Gamma; \Sigma$, and for any $\sop \in TSO(\hilbert_{\overline{\Sigma}})$,
	$\conf \rightarrow \Delta$ iff $\sop(\conf) \rightarrow \sop(\Delta)$.
\end{lemma}
\begin{proof}
	The only non trivial rules are {\scshape SemQOp} and {\scshape SemQMeas}.
	Assume without loss of generality that $env(\conf) \in \hilbert_{\Sigma \otimes \overline{\Sigma}}$.
	By {\scshape SemQOp}, $\conf = \langle \rho, \mathcal{E}(\widetilde{x}) . P \rangle \longrightarrow \langle \mathcal{E}_{\widetilde{x}}(\rho), P \rangle = \Delta$.
	The type system ensures that $\tilde{x} \subseteq \Sigma$.
	We assume without loss of generality that $\tilde{x} = \Sigma$.
	We can also apply {\scshape SemQOp} as follows, $\sop(\conf) = \langle \sop(\rho), \mathcal{E}(\widetilde{x}) . P \rangle  \longrightarrow \langle \mathcal{E}_{\widetilde{x}}(\sop(\rho)), P \rangle$.
	We need to prove that $\langle \mathcal{E}_{\widetilde{x}}(\sop(\rho)), P \rangle = \langle \sop(\mathcal{E}_{\widetilde{x}}(\rho)), P \rangle = \sop(\Delta)$, i.e., that $\sop_{\widetilde{x}}(\sop(\rho)) = \sop(\sop_{\widetilde{x}}(\rho))$.
	By definition, $\sop_{\widetilde{x}}(\rho) = \sum_{i = 1}^{n} (I_{\Sigma} \otimes A_i) \rho (I_{\Sigma} \otimes A_i)^{\dagger}$, and $\sop(\rho) = \sum_{j = 1}^{m} (B_j \otimes I_{\overline{\Sigma}}) \rho (B_j \otimes I_{\overline{\Sigma}})^{\dagger}$.

	By linearity 
	\begin{align*}
	&\sop_{\widetilde{x}}(\sop(\rho)) =\\ 
	&\sum_{i = 1}^{n} (I_{\Sigma} \otimes A_i) (\sum_{j = 1}^{m} (B_j \otimes I_{\overline{\Sigma}}) \rho (B_j \otimes I_{\overline{\Sigma}})^{\dagger}) (I_{\Sigma} \otimes A_i)^{\dagger} =\\
	&\sum_{i = 1}^{n} \sum_{j = 1}^{m} (I_{\Sigma} \otimes A_i) (B_j \otimes I_{\overline{\Sigma}}) \rho (B_j \otimes I_{\overline{\Sigma}})^{\dagger} (I_{\Sigma} \otimes A_i)^{\dagger},
	\end{align*}
	and
	\begin{align*}
	&\sop(\sop_{\widetilde{x}}(\rho)) =\\
	&\sum_{j = 1}^{m} (B_j \otimes I_{\overline{\Sigma}}) (\sum_{i = 1}^{n} (I_{\Sigma} \otimes A_i) \rho (I_{\Sigma} \otimes A_i)^{\dagger}) (B_j \otimes I_{\overline{\Sigma}})^{\dagger} =\\
	&\sum_{i = 1}^{n} \sum_{j = 1}^{m} (B_j \otimes I_{\overline{\Sigma}}) (I_{\Sigma} \otimes A_i) \rho (I_{\Sigma} \otimes A_i)^{\dagger} (B_j \otimes I_{\overline{\Sigma}})^{\dagger}.
	\end{align*}
	Thanks to conjugate properties, it is thus sufficient to show that
	\[
	(B_{p\times p} \otimes I_{q\times q}) (I_{p\times p} \otimes A_{q\times q}) = (I_{p\times p} \otimes A_{q\times q}) (B_{p\times p} \otimes I_{q\times q}).
	\]
	
	This is easily proven thanks to the mixed product property of the Kronecker product, telling us that 
	\[ (A \otimes B)(C \otimes D) = (AC)\otimes(BD)
	\]
	so in our case, we have 
	\[
	(B_{p\times p} \otimes I_{q\times q}) (I_{p\times p} \otimes A_{q\times q}) = B_{p\times p} \otimes A_{q \times q} = (I_{p\times p} \otimes A_{q\times q}) (B_{p\times p} \otimes I_{q\times q})
	\]	
	
	
%	Take
%	\begin{align*}
%		&(B_{p\times p} \otimes I_{q\times q}) (I_{p\times p} \otimes A_{q\times q}) =\\
%		&\left(\begin{array}{c c c}
%			B_{1,1} I_{q\times q} & \dots & B_{1,p} I_{q\times q}\\
%			\dots & & \dots\\
% 			B_{p,1} I_{q\times q} & \dots & B_{p,p} I_{q\times q}\\
%	\end{array}\right)
%\left(
%\begin{array}{c c c c c }
%	A_{q\times q} & \bigzero & \bigzero & \dots & \bigzero \\
%	\bigzero & A_{q\times q} & \bigzero & \dots & \bigzero \\
%	\dots & \dots & \dots & \dots & \dots \\
%	\bigzero & \dots & \dots & \dots & A_{q\times q} \\
%\end{array}
%\right),
%\end{align*}
%and take the element at column $i$, row $j$.
%Note that it is $B_{i,j} I_{q\times q} A_{q\times q}$.
%Take then
%\begin{align*}
%	&(I_{p\times p} \otimes A_{q\times q}) (B_{p\times p} \otimes I_{q\times q}) =\\
%	&\left(
%	\begin{array}{c c c c c }
%		A_{q\times q} & \bigzero & \bigzero & \dots & \bigzero \\
%		\bigzero & A_{q\times q} & \bigzero & \dots & \bigzero \\
%		\dots & \dots & \dots & \dots & \dots \\
%		\bigzero & \dots & \dots & \dots & A_{q\times q} \\
%	\end{array}
%	\right)
%	\left(\begin{array}{c c c}
%		B_{1,1} I_{q\times q} & \dots & B_{1,p} I_{q\times q}\\
%		\dots & & \dots\\
%		B_{p,1} I_{q\times q} & \dots & B_{p,p} I_{q\times q}\\
%	\end{array}\right),
%\end{align*}
%and take the element at column $i$, row $j$.
%Note that it is $A_{q\times q} B_{i,j} I_{q\times q}$.
%The thesis follow by linearity and commutativity of the identity.

%\begin{align*}
%&\left(
%\begin{array}{c c c c }
%	B_{1,1} I_{q\times q} A_{q\times q} & B_{1,2} I_{q\times q} A_{q\times q} & \dots & B_{1,p} I_{q\times q} A_{q\times q} \\
%	B_{2,1} I_{q\times q} A_{q\times q} & B_{2,2} I_{q\times q} A_{q\times q} & \dots & B_{2,p} I_{q\times q} A_{q\times q} \\
%	\dots & \dots & \dots & \dots\\
%	B_{p,1} I_{q\times q} A_{q\times q} & B_{p,2} I_{q\times q} A_{q\times q} & \dots & B_{p,p} I_{q\times q} A_{q\times q} \\
%\end{array}
%\right) = \\
%&\left(
%\begin{array}{c c c c }
%	A_{q\times q} B_{1,1} I_{q\times q} & A_{q\times q} B_{1,2} I_{q\times q} & \dots & A_{q\times q} B_{1,p} I_{q\times q} \\
%	A_{q\times q} B_{2,1} I_{q\times q} & A_{q\times q} B_{2,2} I_{q\times q} & \dots & A_{q\times q} B_{2,p} I_{q\times q} \\
%	\dots & \dots & \dots & \dots\\
%	A_{q\times q} B_{p,1} I_{q\times q} & A_{q\times q} B_{p,2} I_{q\times q} & \dots & A_{q\times q} B_{p,p} I_{q\times q} \\
%\end{array}
%\right) = \\
%\end{align*}

The proof for rule {\scshape SemQMeas} is the same, considering every $m \in \{0, \dots, 2^{\widetilde{x}}\}$ separately.
\end{proof}


\begin{theorem}
%	For any pair of configurations $\conf, \conf'$ well-typed under $\Gamma; \Sigma$, 
%	$\conf \approx_{SPB} \conf'$ implies that for any $\sop \in TSO(\hilbert_{\overline{\Sigma}})$, $\sop (\conf)\ \approx_{SPB}\ \sop (\conf')$.
	For any pair of configurations $\conf, \conf'$ well-typed under $\Gamma; \Sigma$, 
	$\conf \approx_{SPB} \conf'$ implies
	\begin{enumerate}
		{\item for any $\sop \in TSO(\hilbert_{\overline{\Sigma}})$, $\sop (\conf)\ \approx_{SPB}\ \sop (\conf')$; and \label{point:thmchinese1}}
		{\item $tr_{\Sigma}(\conf) = tr_{\Sigma}(\conf')$. \label{point:thmchinese2}}
	\end{enumerate}
\end{theorem}
\begin{proof}
	The proof of point \ref{point:thmchinese1} follows by transitivity and Lemma~\ref{lemma:sop}.
	\[
	 \sop (\conf)\ \approx_{SPB}\ \conf\ \approx_{SPB}\ \conf'\ \approx_{SPB}\ \sop (\conf') 
	\]
	For point \ref{point:thmchinese2} we proceed by refutation.
	Assume $\conf \approx_{SPB} \conf'$ and $tr_{\Sigma}(\conf) \neq tr_{\Sigma}(\conf')$.
	Take $env(\conf), env(\conf') \in \hilbert_{\{ q \} \otimes \Sigma}$ with $env(\conf) = \ket{0} \otimes \ket{\phi}$ and $env(\conf') = \ket{1} \otimes \ket{\phi'}$.
	Consider the context $B[\_] = M[q \triangleright x] . \textbf{ if } x = 0 \textbf{ then } c!q \textbf{ else } free(q) \parallel [\_]$.
	It is clear that $B[\conf] \not\approx_{SPB} B[\conf']$. Contradiction.
\end{proof}

\begin{example}
	Consider the following, wrong definition of quantum teleportation:
	\begin{align*}
		\proc{A} &\Coloneqq \text{in}_a?x.\text{CNOT}(q_0, x).\text{H}(q_0).M(x,q_0 \rhd n).(\text{m}_a!n \parallel out_{a_1}! q_0 \parallel out_{a_2}! x )\\
    \proc{B} &\Coloneqq \text{in}_b?x.\text{m}_a?n.
       \\ & \ite{n = 0}{\sigma_0(x).\text{out}_b!x\\&\quad}
      {\ite{n = 1}{\sigma_1(x).\text{out}_b!x\\&\qquad}
          {\ite{n = 2}{\sigma_2(x).\text{out}_b!x}{\sigma_3(x).\text{out}_b!x}}
      } \\
		\proc{S} &\Coloneqq \text{H}(q_1).\text{CNOT}(q_1, q_2).(\text{in}_a!q_1 \parallel \text{in}_b!q_2) \\
		\proc{Tel} &\Coloneqq (A \parallel B \parallel S) \setminus \Set{\text{in}_a, \text{in}_b, \text{m}_a } \\
		\proc{TelSpec} &\Coloneqq \text{SWAP}(q_0,q_2).(\text{out}_b!q_2 \parallel out_{a_1}! q_0 \parallel out_{a_2}! q_1)
	\end{align*}
	We have that $\proc{Tel} \not\approx_{SPB} \proc{TelSpec}$.
	Consider indeed a context:
  \[ B[\blank] = out_{a_1} ? x . M[x \rhd y] . \ite{y = 1}{c!y}{nil}. \]
\end{example}

An established notion of behavioural equivalence between quantum processes is the Deng-Feng bisimilarity, here we focus on the strong version of it.
\begin{definition}[Deng-Feng Bisimilarity]
	A symmetric relation $\rel \subseteq \conf \times \conf$ is \emph{Deng-Feng bisimulation} if $\conf\ \rel\ \conf'$ implies that $\conf, \conf'$ are well-typed under a typing context $\Gamma; \Sigma$, and for any context $C[\blank] \in Context_{\Gamma, \Sigma}$
	\begin{itemize}
		\item $tr_{\Sigma}(\conf) = tr_{\Sigma}(\conf')$;
		\item if $C[\sop(\conf)] \downarrow_{c}$ then $C[\sop(\conf')] \downarrow_{c}$; and
		\item whenever $C[\sop(\conf)] \xrightarrow{\tau} \Delta$, there exists $\Delta'$ such that $C[\sop(\conf')] \xrightarrow{\tau} \Delta'$ and $\Delta \slift{\rel} \Delta'$
	\end{itemize}
	Let \emph{Deng-Feng bisimilarity} $\approx_{DF}$ be the largest saturated probabilistic barbed bisimulation.
\end{definition}

\begin{theorem}
	For any pair of configurations $\conf, \conf'$, $\conf \approx_{DF} \conf'$ iff $\conf \approx_{SPB} \conf'$.
\end{theorem}


From Davidson we expect that $\confw{\beta, M_{0,1}[q_0 \triangleright x] . c ! q_0 \parallel discard(q_1)} \approx_{SPB} \confw{\beta, M_{0,1}[q_0 \triangleright x] . c ! q_0 \parallel discard(q_1)}$.

{\color{red} se li invio entrambi?}
%The converse does not hold.
%This is because $tr_{\Sigma}(\conf) = tr_{\Sigma}(\conf')$ is strictly stronger than our closure on contexts, given that it requires the state of qbits to be equivalent

%\begin{definition}[FD-barbed congruence]
%	A probabilistic barbed bisimulation $\rel \subseteq \conf \times \conf$ is a \emph{FD-barbed congruence} if $\conf \rel \conf'$ implies that 
%	\begin{itemize}
%		\item $(\conf \parallel P) \rel (\conf' \parallel P)$ for any process $P$ with $qv(P)$ disjoint from $qv(\conf) \cup qv(\conf')$ 
%		\item $\sop (\conf) \rel \sop (\conf')$ for any $\sop \in TSO(\hilbert_{\overline{qv(\conf)}})$
%	\end{itemize}
%\end{definition}
%
%The following congruence was proposed in~\cite{Deng:2012} by Feng and Deng.
%\begin{definition}
%	Let FD-congruence is the largest relation $\approx_{FD} \subseteq \conf \times \conf$ which is barb preserving, reduction closed, compositional and such that if $\conf \rel \conf'$ implies $qv(\conf) = qv(\conf')$ and $env(\conf) = env(\conf')$.
%\end{definition}
%
%\begin{definition}[open FD-simulation and bisimulation]
%	A relation $\rel \subseteq \conf \times \conf$ is an \emph{open FD-simulation} if $\conf \rel \conf'$ implies that $qv(\conf) = qv(\conf')$, $env(\conf) = env(\conf')$ and, for all $\sop \in TSO(\hilbert_{\overline{qv(\conf)}})$,
%	\begin{itemize}
%		\item whenever $\sop(\conf) \xrightarrow{\alpha} \Delta$, there is some $\Delta'$ such that $\sop(\conf') \xrightarrow{\alpha} \Delta'$ and $\Delta \slift{\rel} \Delta'$.
%	\end{itemize}
%
%	A relation $\rel \subseteq \conf \times \conf$ is an \emph{open FD-bisimulation} iff $\rel$ and $\rel^{-1}$ are open FD-simulations.
%	
%	Let FD-open bisimilarity $\approx_{oFD}$ be the largest FD-open bisimulation. 
%\end{definition}
%
%The following theorem is proved in~\cite{Deng:2012}.
%\begin{theorem}
%	$\conf \approx_{FD} \conf'$ iff $\conf \approx_{oFD} \conf'$
%\end{theorem}

%\begin{definition}[Barbed Simulation]
%  A relation $\mathcal{R}_{\Gamma, \Sigma}$ is called a \emph{barbed simulation} if it is closed under the $\boxplus$-operator and for each
%  pair of configurations $\langle \rho, P \rangle$, $\langle \sigma, Q \rangle$, well-typed under the context $\Gamma; \Sigma$, it holds that:
%  \begin{enumerate}
%    \item if $\langle \rho, P \rangle \longrightarrow \boxplus_{i \in I} \langle \rho_i, P_i \rangle$, then
%      $\langle \sigma, Q \rangle \longrightarrow \boxplus_{j \in J} \langle \sigma_j, Q_j \rangle$, and
%      $\forall i \in I \ldotp \exists j \in J \ldotp \langle \rho_i, P_i \rangle \mathcal{R}_{\Gamma, \Sigma} \langle \sigma_j, Q_j \rangle $;
%    \item if $\langle \rho, P \rangle \downarrow_{p, c}, then \langle \sigma, Q \rangle \downarrow_{p, c}$.
%  \end{enumerate}
%\end{definition}
% 
%\begin{definition}[Barbed Bisimulation]
%  A relation $\mathcal{R}_{\Gamma, \Sigma}$ is called a \emph{barbed bisimulation} if it's symmetric, closed under the $\boxplus$-operator and for each
%  pair of configurations $\langle \rho, P \rangle$, $\langle \sigma, Q \rangle$, well-typed under the context $\Gamma, \Sigma$ it holds that:
%  \begin{enumerate}
%    \item if $\langle \sigma, \rho \rangle \longrightarrow \boxplus_{i \in I} \langle \rho_i, P_i \rangle$, then
%      $\langle \sigma, q \rangle \longrightarrow \boxplus_{i \in I} \langle \sigma_i, Q_i \rangle$, and
%      $\forall i \in I \ldotp \langle \rho_i, P_i \rangle \mathcal{R}_{\Gamma, \Sigma} \langle \sigma_i, Q_i \rangle$;
%    \item if $\langle \rho, P \rangle \downarrow_{p, c}, then \langle \sigma, Q \rangle \downarrow_{p, c}$.
%  \end{enumerate}
%\end{definition}


\section{Examples}

\begin{example}
Let $\rho = \frac{1}{2} \ketbra{0}{0}$ and $\rho' = \frac{1}{2} \ketbra{1}{1}$, then
\begin{align*}
	\langle \rho, P \rangle \boxplus \langle \rho', P \rangle \equiv \langle \rho + \rho', P \rangle \sim \langle \frac{1}{2} I, P \rangle
\end{align*}
In particular, the following holds in our system but not in~\cite{Feng:2012, Deng:2012}
\begin{align*}
	Set_{\ketbra{+}{+}}(q).M(q \triangleright x).c!0 \sim Set_{\frac{1}{2} I}(q).\tau.c!0
\end{align*}
\end{example}





%\chapter{chapter 5}
%In cerca di una process algebra minimale
%senza configurazioni	
%con distribuzioni
%
%\section{Quantum Transition System}
%\subsection{Mathematical Preliminaries}

A quantum distribution $\Delta \in S^\hilbert$ over a set $S$ is a function from the finite-dimensional Hilbert space $\hilbert$ of dimension $n$ to probability distributions $D(S)$ over $S$.
In the following, we write $K_{x} : \hilbert \to S$ defined as the function that always returns $x$ for every state $\ket{\phi} \in \hilbert$.

Let $\Set{P_i | \sum_{i = 1}^{n} P_i = I }$ be a set of quantum projectors, and let $\Delta_i$, $1 \leq i \leq n$, be a collection of quantum distributions.
We use $\sum_{i = 1}^{n} P_i \Delta_i$ to denote the distribution determined by 
\[
\left(\sum_{i = 1}^{n} P_i \Delta_i\right) (\ket{\phi}) = \sum_{i = 1}^{n} p_i(\ket{\phi}) \Delta_i \left(\frac{P_i \ket{\phi}}{p_i(\ket{\phi})}\right)
\]
where $p_i(\ket{\phi}) = \bra{\phi} P_i \ket{\phi}$.

Note that for any $\ket{\phi}$, ($\sum_{i = 1}^{n} P_i \Delta_i) (\ket{\phi})$ is a legal distribution since $\Delta_i(\ket{\psi})$ is always a distribution and
\[
\sum_{i = 1}^{n} p_i (\ket{\phi}) = \sum_{i = 1}^{n} \bra{\phi} P_i \ket{\phi} = \bra{\phi} I \ket{\phi} = \braket{\phi | \phi} = 1.
\]

When $n = 2$, $P_2$ is derivable as $I - P_1$, thus we write $\Delta_1 \qsum{P_1} \Delta_2$ for $\sum_{i = 1}^{n} P_i \Delta_i$.
The operator $\blank \qsum{P} \blank$ can be defined from $\blank \tensor[_p]{\oplus}{} \blank$ as follows:
\[
  (\Delta \qsum{P_1} \Theta) (\ket{\phi}) = \Delta \left(\frac{P_1 \ket{\phi}}{p_1(\ket{\phi})}\right) \tensor[_{p_1(\ket{\phi})}]{\oplus}{} \Theta \left(\frac{P_2 \ket{\phi}}{p_2(\ket{\phi})}\right)
\]

Given a relation $\rel \subseteq D(S) \times D(S)$, we call $\text{ext}(\rel)$ its \emph{point-wise extension} such that $\Delta\,\text{ext}(\rel)\,\Theta$ if and only if $\forall \ket{\phi}\ldotp \Delta(\ket{\phi})\,\rel\,\Theta(\ket{\phi})$.
We say that a relation $\rel \subseteq D(S)^\hilbert \times D(S)^\hilbert$ is \emph{point-wise} whenever a relation $\rel_0 \subseteq D(S) \times D(S)$ exists such that $\rel = ext(\rel_0)$.  

We say that a relation $\rel \subseteq D(S)^\hilbert \times D(S)^\hilbert$ is \emph{linear} over $\blank \boxplus \blank$ if $\Delta_i\,\rel\,\Theta_i$, $i = 1,2$, implies $(\Delta_1 \qsum{P} \Delta_2)\,\rel\,(\Theta_1 \qsum{P} \Theta_2)$ for any $P$.

\begin{proposition}
  Given any relation $\rel \subseteq D(S) \times D(S)$, $\text{ext}(\rel)$ is \emph{linear} over $\blank \boxplus \blank$ if and only if $\rel$ is linear over $\blank \oplus \blank$.
\end{proposition}

\begin{definition}
	Let $R$ be a relation in $S \times S$, we define its quantum lifting $\qlift{R}$ as a relation in $D(S)^\hilbert \times D(S)^\hilbert$ as the minimal relation such that
	\begin{enumerate}
		\item $s R s'$ implies $K_{\overline{s}} \qlift{R} K_{\overline{s'}}$; and
		\item $\Delta_i \qlift{R} \Theta_i$, i = $1, 2$, implies $\Delta_1 \qsum{P} \Delta_2 \qlift{\rel} \Theta_1 \qsum{P} \Theta_2$ for any projector $P$.
	\end{enumerate}
\end{definition}

\paragraph{Physical Consistency}

Not all quantum distributions are physically implementable, as a trivial example, take any $\Delta$ such that $\Delta(\ket{\phi}) \neq \Delta(-\ket{\phi})$.
Clearly, since all the normalized vectors $\ket{\phi}, -\ket{\phi}, i \ket{\phi}, -i \ket{\phi}$ represent the same quantum state, $\Delta$ is not consistent with physical law.

%A quantum boh $\delta \in d_q(S)$ over a set $S$ is a function from the Hilbert space $\hilbert$ to $S$ such that:
%\begin{itemize}
%	\item for each $s \in S$, the constant function $\delta(\phi) = s$ is in $d_q(S)$;
%	\item for each unitary transformation $U$, if $\delta \in d_q(S)$ then $\delta'$ is also in $d_q(S)$ with $\delta'(\ket{\phi}) = U \delta(\ket{\phi})$;
%	\item for any projector $P$, if $\delta \in d_q(S)$ then $\delta'$ is also in $d_q(S)$ with $\delta'(\ket{\phi}) = \delta(\frac{P \ket{\phi}}{p(\ket{\phi}})$;
%\end{itemize}
%
%A quantum distribution $\Delta \in D_q(S)$ over a set $S$ is \emph{legal} if and only if:



\subsection{Quantum Labeled Transition Systems}

A quantum labeled transition system QLTS on an Hilbert space $\hilbert$ is a triple $(S, Act_\tau, \rightarrow)$ where
\begin{itemize}
	\item $S$ is a set of states $s, s_1, \dots$;
	\item $Act_\tau$ is a set of transition labels with $\tau$ a distinguished element;
	\item $\rightarrow\;\subseteq S \times Act_\tau \times D(S)^\hilbert$ is the transition relation. 
%	such that for each $\delta \xrightarrow{\mu} \Delta$ either:
%	\begin{enumerate}
%		\item $\Delta = \bar{\delta'}$ for some $\delta'$ and a unitary matrix $U$ exists such that $\delta'(U \ket{\phi}) = \delta(U \ket{\phi})$ for any $\ket{\phi}$;	
%	\end{enumerate}
\end{itemize}

We define a minimal quantum process algebra (mQPA) for describing quantum processes.
A quantum process $Q$ is defined as
\[
Q ::= \nil \mid \mu.Q \mid Q + Q \mid U \circ Q \mid Q \qsum{P} Q
\]
where $U$ is a unitary transformation over $\hilbert$.

We give the semantics of mQPA in terms of QLTS.
Some terms are taken as states $s \in S$, in particular the ones where unitary operators and $\blank \boxplus \blank$ are guarded.
\[
s ::= \nil \mid \mu.Q \mid s + s
\]
%We write $\sema{s}$ for the function $S^\hilbert$ they stand for, i.e., 
%\begin{align*}
%	&\sema{s}(\ket{\phi}) = (\ket{\phi}, s)
%\end{align*}

The interpretation of an arbitrary term $Q$ as quantum distribution $\sem{Q}$ over S is given by the function $\sem{\blank} : Q \to D(S)$:
\begin{align*}
	&\sem{\nil}(\ket{\phi}) = \overline{\nil}\\
	&\sem{\mu.Q}(\ket{\phi}) = \overline{\mu.Q}\\
	&\sem{Q_1 + Q_2}(\ket{\phi})(s) = 
	\begin{cases}
		\sem{Q_1}(\ket{\phi})(s_1) \cdot \sem{Q_2}(\ket{\phi})(s_2) & \text{if } s = s_1 + s_2\\
		0 & \text{otherwise}
	\end{cases}\\
	&\sem{U \circ Q}(\ket{\phi}) = \sem{Q}(U \ket{\phi})\\
	&\sem{Q_1 \qsum{P} Q_2} = \sem{Q_1} \qsum{P} \sem{Q_2}
\end{align*}

\begin{proposition}
	For any $s \in S$, $\sem{s} = K_{\overline{s}}$.
\end{proposition}

The transition relation $\to$ is defined as follows, with $s \xrightarrow{\mu} \Delta$ as notation for $(s, \mu, \Delta) \in\;\to$.
\begin{gather*}
  \infer[\mbox{\footnotesize\scshape Action}]{\mu.Q \xrightarrow{\mu} \sem{Q}}{} \qquad 
  \infer[\mbox{\footnotesize\scshape Ext.L}]{Q_1 + Q_2 \xrightarrow{\mu} \Delta}{Q_1 \xrightarrow{\mu} \Delta} \qquad
  \infer[\mbox{\footnotesize\scshape Ext.R}]{Q_1 + Q_2 \xrightarrow{\mu} \Delta}{Q_2 \xrightarrow{\mu} \Delta}
\end{gather*}

\begin{definition}
	A relation $\rel : S \times S$ is called a QLTS \emph{bisimulation} if it's symmetric and for each pair of states $s, s' \in S$ such that $s \rel s'$,
	if $s \xrightarrow{\mu} \Delta$ then $s' \xrightarrow{\mu} \Delta'$ and $\Delta\,\qlift{\rel}\,\Delta'$, for some quantum distributions $\Delta, \Delta' \in D(S)$.
\end{definition}


\subsubsection{Alternative}
We define an alternative characterization of a quantum transition system that is more in-line with preexisting quantum transition systems like~\cite{Feng:2012, Deng:2012}.
A quantum labeled transition system qLTS on an Hilbert space $\hilbert$ is a triple $(S, Act_\tau, \hookrightarrow)$ where
\begin{itemize}
	\item $S$ is a set of states $s, s_1, \dots$;
	\item $Act_\tau$ is a set of transition labels with $\tau$ a distinguished element;
  \item $\hookrightarrow\;\subseteq (\hilbert \times S) \times Act_\tau \times D(\hilbert \times S)$ is the transition relation. 
%	such that for each $\delta \xrightarrow{\mu} \Delta$ either:
%	\begin{enumerate}
%		\item $\Delta = \bar{\delta'}$ for some $\delta'$ and a unitary matrix $U$ exists such that $\delta'(U \ket{\phi}) = \delta(U \ket{\phi})$ for any $\ket{\phi}$;	
%	\end{enumerate}
\end{itemize}

To simplify our presentation, we reduce to $Q$ and $S$ of a specific form, namely where the branches of non-deterministic choices are always guarded by a transition, like in $(\tau \circ U. \nil) + (\alpha \circ U'. \nil)$.
Note that this is consistent with the behaviour of preexisting quantum process algebras like~\cite{Feng:2012, Deng:2012}.
We stress that a term is in this specific form by writing $\hat{Q}$ or $\hat{S}$.

We define the interpretation of a pair $(\ket{\phi}, Q)$ as a distribution as given by the function $\sema{\blank} : (\hilbert \times S) \to D(\hilbert \times S)$:
\begin{align*}
	&\sema{\ket{\phi}, s} = \overline{(\ket{\phi}, s)} \\
	&\sema{\ket{\phi}, U \circ Q} = \sema{U\ket{\phi}, Q} \\
	&\sema{\ket{\phi}, Q_1 \qsum{P} Q_2 } = \sema{P \ket{\phi}, Q_1} \psum{p(P, \ket{\phi})} \sema{P^{\bot}\ket{\phi}, Q_2} 
\end{align*}

The transition relation $\hookrightarrow$ is defined as follows, with $s \xhookrightarrow{\mu} \Delta$ as notation for $(s, \mu, \Delta) \in\;\hookrightarrow$.
\begin{gather*}
  \infer[\mbox{\footnotesize\scshape Action}]{(\ket{\phi}, \mu.Q) \xhookrightarrow{\mu} \sema{\ket{\phi}, Q}}{} \qquad 
  \infer[\mbox{\footnotesize\scshape Ext.L}]{(\ket{\phi}, Q_1 + Q_2) \xhookrightarrow{\mu} C}{(\ket{\phi}, Q_1) \xhookrightarrow{\mu} C} \qquad
  \infer[\mbox{\footnotesize\scshape Ext.R}]{(\ket{\phi}, Q_1 + Q_2) \xhookrightarrow{\mu} C}{(\ket{\phi}, Q_2) \xhookrightarrow{\mu} C}
\end{gather*}

%\subsubsection{Bisimulation}

\begin{definition}
	A relation $\rel : (\hilbert \times S) \times (\hilbert \times S)$ is called a qLTS \emph{bisimulation} if it's symmetric and for each pair $(\ket{\phi}, s), (\ket{\phi'}, s)$ such that $(\ket{\phi}, s) \rel (\ket{\phi'}, s)$,
	if $(\ket{\phi}, s) \xrightarrow{\mu} C$ then $(\ket{\phi}', s') \xrightarrow{\mu} C'$ and $C \slift{\rel} C'$, for some distributions $C, C' \in D(\hilbert \times S)$.
\end{definition}

We now establish a common behaviour result between QLTS and qLTS.
\begin{theorem}
  $\forall s, s' \in S \ldotp s \sim s'$ iff $\forall \ket{\phi} \ldotp \sema{\ket{\phi}, s} \sim \sema{\ket{\phi}, s'}$.
\end{theorem}
\begin{proof}
  ($\Longleftarrow$) Let $\rel \subseteq S \times S$ be a relation such that $s\,\rel\,s'$ iff $\forall \ket{\phi} \ldotp (\ket{\phi}, s)\,\rel_{\ket{\phi}}\,(\ket{\phi}, s')$,
  for some set of relations $\rel_{\ket{\phi}} \subseteq (H \times S) \times (H \times S)$.
  Assume $s \xrightarrow{\mu} \Delta$ then $s \equiv \mu.Q + Q'$, without loss of generality we can assume $s = \mu.Q$,
  thus $\forall \ket{\phi} \ldotp (\ket{\phi}, s) \xrightarrow{\mu} \sema{\ket{\phi}, Q}$ and $\Delta = \sem{Q} \ldots$.
\end{proof}


\chapter{Conclusions}
\printbibliography[
heading=bibintoc,
title={Bibliography}
]
\end{document}
