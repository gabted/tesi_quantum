\subsection{QPAlg/CQP}\label{lalire_bisimulation}

In \cite{lalireRelationsQuantumProcesses2006}, the authors propose a notion of bisimilarity for QPAlg, that was adapted in \cite{davidsonFormalVerificationTechniques2012} for the labelled semantic of CQP. This bisimilarity is based on a probabilistic branching bisimulation, and deals with quantum communication equating the reduced density matrix of sent qubits. In other words, when a process $P$ performs an output transition $c!v$, for a classical value $v$, a bisimilar process $Q$ should perform an output transition on the same channel with the same value. When a process $P$ performs an output transition $c!x$, for a \textit{quantum name} $x$, $Q$ is not required to output the same \textit{name} $x$, it should instead send a qubit with the same \textit{state} of $x$. The chosen way to define the state of qubit $q$ in the configuration with (global) state $q, q_1 \ldots q_n = \rho$ is the reduced density matrix of $q$, i.e. $tr_{q_1 \ldots q_n}(\rho)$.

%\note{non so se metterla
%\begin{definition}{QPAlg bisimulation}
%A symmetric relation $\rel$ between configurations $\confw{\widetilde{q} = \rho, P}$, $\confw{\widetilde{q} = \sigma, Q}$ is a probabilistic branching bisimulaton if \begin{itemize}
%\item $\confw{\widetilde{q} = \rho, P} \xrightarrow{c!v} \Delta$ for a classical value $v$, then $\confw{\widetilde{q} = \sigma, Q} \Rightarrow \confw{\widetilde{q} = \sigma', Q'} \xrightarrow{c!v} \Theta$, with $\confw{\widetilde{q} = \rho, P} \rel \confw{\widetilde{q} = \sigma', Q'}$ and $\Delta \lrel \Theta$.  
%\item $\confw{\widetilde{q} = \rho, P} \xrightarrow{c!x} \Delta$ for a quantum name $x$, then $\confw{\widetilde{q} = \sigma, Q} \Rightarrow \confw{\widetilde{q} = \sigma', Q'} \xrightarrow{c!y} \Theta$, with  $tr_{\widetilde{q} - x}(\rho) = tr_{\widetilde{q} - y}(\sigma')$, $\confw{\widetilde{q} = \rho, P} \rel \confw{\widetilde{q} = \sigma', Q'}$ and $\Delta \lrel \Theta$.
%\item The same holds also for classical input and quantum input.
%\end{itemize}
%\end{definition}
%}

In QPAlg, as in all the other quantum calculi, the bisimilarity is given as a relation between configuration, and then lifted to a relation between processes. Two processes $P$ and $Q$ are considered bisimilar, written $P \sim Q$, when $\confw{\widetilde{q} = \rho, P}$ is bisimilar to $\confw{\widetilde{q} = \rho, Q}$ for any $\rho$.

This bisimilarity consider the state of a qubit $Q$ an observable only when it is sent to the external environment. This means that the processes 
\[ P = X[q].\nil \qquad Q = Z[q].\nil\]
are indeed bisimilar, as the qubit $q$ ends up with a different quantum state, but the difference is never observed, because the qubit is never sent.

This bisimilarity is not a congruence with respect to summation and parallel composition. The former is expected with weak or branching bisimulation, and is due to the usual not quantum-related problem 
\begin{align*}
c!v.\nil &\sim \tau.c!v.\nil \\
c!v.\nil + d!v &\not\sim \tau.c!v.\nil + d!v
\end{align*}

The problem with parallel composition is instead purely quantum-related, as it is caused by the side effect of measurements and entanglement. Cnsider the following example:

\begin{example}
The two processes $P$ and $Q$ are bisimilar, but $P\parallel c!q_2$ is not bisimilar to $Q \parallel c!q2$:
\begin{align*}
 P &= M_{01}[q_1].\nil &\sim & & Q &= M_\pm[q_1].\nil \\ 
 P \parallel c!q_2 &=  M_{01}[q_1].\nil \parallel c!q_2 &\not\sim & & Q \parallel c!q_2 &=  M_\pm[q_1].\nil \parallel c!q_2 
\end{align*} 

$P$ and $Q$ are clearly bisimilar, as they do not perform any visible action. But when $P\parallel c!q_2$ operates on an entangled state, like the Bell state $\beta = \oost\ket{00} + \oost\ket{11}$, the measurement that happens on $q_1$ has effect on $q_2$, and $q_2$ will decay in state $\kz$ or $\ko$. The same happens when $Q\parallel c!q_2$ operates on the same Bell state, and $q_2$ will decay in state $\kpl$ or $\km$. When $q_2$ is sent, this difference becomes observable, and the two processes are not bisimilar.
\end{example}
\subsection{qCCS}

The bisimilarity relations proposed for qCCS differ from the one for QPAlg in a number of "classical" details, not related to the quantum properties. First of all, for qCCS are defined strong and weak bisimilarity, not branching bisimilarity. QPAlg addresses probabilistic behaviour with a Larsen-Skou bisimulation, qCCS with a combined-moves bisimulation.

Besides these "non-quantum" differences in the treatment of silent transitions and probabilities, one key difference telling the two relations apart is the notion of "observable" quantum state. Where in QPAlg the state of a qubit could be observed only when it is sent, in qCCS the state of a qubit can be observed as soon as the computation has "ended", i.e. when there are no more transformation to be applied to that qubit.

In  \cite{fengProbabilisticBisimulationsQuantum2007}, two configuration $\confw{\widetilde{q} = \rho, P}$ and $\confw{\widetilde{q} = \sigma, Q}$ being bisimilar requires that if $\confw{\widetilde{q} = \rho, P} \not\rightarrow$, then $\confw{\widetilde{q} = \sigma, Q} \not\rightarrow$ and $\rho$ is equal to $\sigma$, up to a permutation of the quantum names in $\widetilde{q}$. 

An equivalent condition was present in \cite{yingAlgebraQuantumProcesses2010}, where the quantum actions $U[q]$, $M[q]$ and $\sop[q]$ were considered observable actions, and used as labels of the LTS. Since two bisimilar processes will perform the same observable action, when startig from the same quantum state $\rho$ they will necessarily end up in the same quantum state $\rho'$.

The condition presented in \cite{fengProbabilisticBisimulationsQuantum2007} was too strict, and didn't work well with non-terminating recursive processes. So in \cite{fengBisimulationQuantumProcesses2012} a more general condition was proposed, requiring the equality not of the whole quantum state, but only of the qubits that will not be further modified. For example, in a configuration $\confw{q_1, q_2 = \rho, H[q_2].\nil}$, the bit $q_1$ should be regarded as observable, as it does not appear as a free quantum variable in $H[q_2].\nil$. The bit $q_2$, instead, should not be considered when checking for bisimilarity, as it is not in its final state.

More formally, \cite{fengBisimulationQuantumProcesses2012} defines the notion of \textit{environment} of a configuration, namely the reduced density operator of the values of the qubits not used by the process.
\[env(\confw{\widetilde{q} = \rho, P}) = tr_{qv(P)}(\rho)\]


The bisimilarity of $\confw{\widetilde{q} = \rho, P} $ and $\confw{\widetilde{q} = 	\sigma, Q}$ then requires that 
$env(\confw{\widetilde{q} = \rho, P}) = env(\confw{\widetilde{q} = \sigma, P})$. Thanks to this condition, the bisimilarity proposed in \cite{fengBisimulationQuantumProcesses2012} is a congruence, as it is preserved by parallel composition (contrary to QPAlg).


Consider the example seen before 
\[P = M_{01}[q_1].\nil \qquad Q = M_\pm[q_1].\nil \]
%\[ P = X[q].\nil \qquad Q = Z[q].\nil\]

$P$ and $Q$ are not bisimilar for \cite{fengProbabilisticBisimulationsQuantum2007, fengBisimulationQuantumProcesses2012, dengOpenBisimulationQuantum2012} , as they end in different quantum states. $\confw{q = \proj{0}, P}$ will evolve in a (point) distiribution $\overline{\confw{q = \proj{0}, \nil}}$, while $\confw{q = \proj{0}, Q}$ will evolve in a distribution $\confw{q = \proj{+}, \nil} \psum{\frac{1}{2}} \confw{q = \proj{+}, \nil}$. Since the state of $q$ is observable, the two distributions are not bisimilar in the sense of probabilistic bisimulation, as they are composed by not-bisimilar states.

In \cite{dengOpenBisimulationQuantum2012}, a new, definitive notion of bisimilarity for qCCS is defined, under the name of open bisimilarity. This bisimilarity is similar to the previous one \cite{fengBisimulationQuantumProcesses2012}, but it is \textit{closed under superoperator application}. That is, two configuration $\conf$ and $conf'$ being bisimilar requires that also $\sop(\conf)$ and $\sop(\conf')$ are bisimilar, for any trace-preserving superoperator $\sop$ that acts on the qubits not used by $\conf$ and $\conf'$. The intuition behind this condition is that, if two configuration are bisimilar, they must still be bisimilar also when the external environment performs any kind of transformation on the unused qubits of the system. As this bisimilarity incorporates and extends all the ideas of \cite{fengBisimulationQuantumProcesses2012, yingAlgebraQuantumProcesses2010, fengBisimulationQuantumProcesses2012} we will use this open bisimilarity as a touchstone for the novel behavioural equavelences defined in the next chapter.


In \cite{fengSymbolicBisimulationQuantum2014} the authors propose a symbolic semantic for qCCS, that abstracts away from the actual quantum state in the initial configuration. Instead of probabilistic distribution, the concept of \textit{superoperator-valued distribution} is used, i.e. functions that associate to each process a trace non-increasing superoperator. In this context, a superoperator valued distribution represent both the probability that a process is reached, and also the transformation that were applied to the quantum state to reach that process. The associated symbolic bisimilarity is proven to be equivalent to open bisimilarity of \cite{dengOpenBisimulationQuantum2012}, but has the advantage of considering smaller transition systems, and so be better suited for model checking.  


\subsection{Mixed Configuration CQP}


In \cite{davidsonFormalVerificationTechniques2012}, the author provides a labelled semantic and a bisimilarity notion of CQP, but that bisimilarity is not a congruence with respect to parallel composition, as seen in section \ref{lalire_bisimulation}. In \cite{davidsonFormalVerificationTechniques2012}, the author defines the mixed configuration semantic for CQP, and on that system describes a bisimilarity that is also a congruence. This bisimilarity, similar to the one in QPAlg, is a probabilistic branching bisimilarity, that observes the quantum values only when they are sent. Differently from the previous proposal, it also requires the equality of the "environment" of a configuration, like in qCCS.


Recall that in this semantics a configuration that performs a quantum  measurement evolves in a "mixed configuration", i.e. a sort of probabilistic distribution of configurations, in which the probabilities are not immediately observable.  


Both mixed configuration and distributions of configurations describe all the various outcomes with their relative probabilities. The difference is that a distribution exhibits immediately this probabilistic behaviour, choosing only  only one of the possible configuration with a probabilistic transition. A mixed configuration, instead, can perform all the usual labelled action, remaining in the state of a mixed configuration, without picking a single outcome and dropping the others.


Reguarding bisimilairity, the consequence is that the configuration $\conf$, bisimilar to the configuration $\conf'$, must exhibit the same the probabilistic moves of $\conf'$ only when the outcome is sent, and not immediately after the measurement.

Also the environment of $\conf$ must be equal to the environment of $\conf'$ only when $\conf$ and $\conf'$ perform an output transition. Note that, since mixed configurations must be treated as a single state, the notion of environment must be extended accordingly. If $\confw{\widetilde{q} = \rho, P} \psum{p} \confw{\widetilde{q} = \sigma, Q}$ is a mixed configuration, the environment is defined as:
\[env(\confw{\widetilde{q} = \rho, P} \psum{p} \confw{\widetilde{q} = \sigma, Q}) = p*env(\confw{\widetilde{q} = \rho, P}) + (1-p)*env(\confw{\widetilde{q} = \sigma, Q})\]

and the environment of a configuration is defined as the reduced density operator of the qubits not used owned by the process.

This bisimulation is coarser than the one in qCCS, and arguably captures more appropriately the expected behevioural equivalence of quantum processes. For this bisimulation, for example, it holds that 
\begin{align*}
 P &= M_{01}[q_1].\nil &\sim & & Q &= M_\pm[q_1].\nil \\ 
 P \parallel c!q_2 &=  M_{01}[q_1].\nil \parallel c!q_2 &\sim & & Q \parallel c!q_2 &=  M_\pm[q_1].\nil \parallel c!q_2 
\end{align*} 

In fact, since the qubit $q_1$ is lost, an external observer can rely only on $q_2$ to distinguish $P$ and $Q$, which is not possible because of quantum properties of mixed states. 

As a drawback, this result is obtained through a considerably not-standard approach, diverging from the usual rules of probabilistic calculi. Recovering this result in the usual framework of pLTS remains an interesting open problem.
