



There is a number of proposals of quantum process calculi in the literature, often with different syntax, semantics and behavioural equivalences, even if they all model the same systems and the same protocols. 
The are three main line of research that developed in recent years. The first, started with QPAlg and then developed with CQP, is inspired by the $\pi$-calculus. 
The second approach, developed simultaneously but independently, is centered around qCCS, that is a quantum extension of value-passing CCS. 
This thesis will focus on analyzing similarities and differences of these two calculi, CQP and qCCS. 
The third proposal, exploring the quantum process algebra qACP, is less directly related and comparable with the first two, in the same way as its classical counterpart ACP is designed in a different fashion with respect to CCS/$\pi$-calculus.

\section{LTS and quantum states} 

\subsection{QPAlg}

Since from the first works by Lalire and Jorrand [qpalg2004], it became evident that the operational semantic of a \textit{quantum} process calculus could not of consist only of syntactic elements, like the transitions $P \rightarrow P'$ of a classical process algebra. In all the calculi proposed so far, processes manipulating and communicating quantum data are always be coupled with a state vector, describing the current state of the quantum system. The operational semantic of such calculi produces a LTS is composed of \textit{configurations}, i.e. states of the form 
\[
	\langle q_0, \ldots, q_n = \kp, P\rangle
\] 
in which $P$ is a process containing $q_0 \ldots q_n$ as free variables, and $\kp \in \Hto{n}$  describes the state of the qubits manipulated by $P$. 

This approach leads to a imperative, stateful style of computation.
In QPAlg, processes can be contructed with the usual actions of vaule passing calculi, $c?x$, $c!v$, but also with \textit{quantum action}, like $X, Z, H, M_{01}$, that correspond to applying unitaries or measurement to the underlying quantum state: \[ \langle q = \kp, H[q].P \rangle \xrightarrow{\tau} \langle q = H\kp, P \rangle\]

Quantum actions are labelled as $\tau$ in the LTS, since are internal action of a process changing its state. Despite that, quantum action cannot always be considered local to the process, as becomes evident when considering parallel processes with entangled qubits.

If two processes $P$ and $Q$ act on separable qubits, it would possible to consider them together in a compositional way, like in classical process algebra
\[ \infer{\langle p,q=\kp\otimes\kf, P\parallel Q\rangle \xrightarrow{\alpha} \langle p,q=\kp'\otimes\kf, P'\parallel Q\rangle}{\langle p=\kp, P\rangle \xrightarrow{\alpha} \langle p=\kp', P'\rangle}\]

But due to entanglement, qubits are not always separable, and the quantum action of $P$ can change the state of $Q$ (even thou the quantum action are labelled as a silent transition). So in QPAlg a more general rule is used:
\[\infer{\langle p,q=\nu, P\parallel Q\rangle \xrightarrow{\alpha} \langle p,q=\nu', P'\parallel Q\rangle}{\langle p, q=\nu, P\rangle \xrightarrow{\alpha} \langle p, q=\nu', P'\rangle}\]
where quantum actions of $P$ must be intended as transformations on the whole Hilbert space $\calH_p \otimes \calH_q$.

Separating the syntactic, control part of the configuration (the process $P$) from the underlying quantum data (the statevector $\kp$) allows also for a "pass-by-reference" way of communication, opposed to the "pass-by-value" of classical value passing process algebras.
The usual rule for syncrhonization, in fact, would be
	\[ \infer{c!v.P \parallel c?x.Q \xrightarrow{\tau} P\parallel Q[v/x]}
	{ c!v.P \xrightarrow{c!v} P & c?x.Q \xrightarrow{c?v} Q[v/x]}
	\] 
In this way, if $Q$ contains two occurrences of $x$, each of them gets instantiated with a different, independent copy of the value $v$. But if the value $v$ was a state vector $\kp$, this would require duplicating the quantum information, wich is impossible due to the no-cloning theorem.
So in QPAlg (and in all other quantum calculi), quantum data is passed in a "pass-by-reference" manner:
\[ \infer{ \langle q=\kp, c!q.P \parallel c?x.Q \rangle \xrightarrow{\tau}  \langle q=\kp, P\parallel Q[q/x] \rangle}
{  \langle q=\kp, c!q.P \rangle \xrightarrow{c!q}  \langle q=\kp, P \rangle & \langle q=\kp,  c?x.Q \rangle \xrightarrow{c?q} \langle q=\kp,  Q[q/x] \rangle}
\] 
where $q$ is just a pointer to the quantum variable stored in a configuration.



Another key feature present in QPAlg and in all other calculi is the coexistence of \textit{nondeterminism}, arising from sums and parallel composition, and probabilistic behaviour, arising from the probabilistic nature of quantum measurements. 
So in all quantum process calculi, a process can be defined by a pLTS $\langle Conf, Act, \rightarrow \rangle$, where $Conf$ is the set of all possible configurations. 
QPAlg and CQP follow the "probabilistic-transition" approach, while qCCS follows the $probabilistic-state$ approach.



\note{qpalg2005 different rules for comunication (with the environment) and synronization. }

\subsection{CQP}

in [CQP2005], Gay and Nagarajan presented their calculus Communicating Quantum Process. CQP makes use of an (affine) type system to restrict the set of possible processes of the algebra to the "admissible" ones, the ones respecting the no-cloning theorem. Under the assumption that Alice, Bob and Charlie are in three different physical location, the process \[Alice = b!q.c!q.nil\] should not be well typed, because Bob could read from the $b$ channel, Charlie from the $c$ channel, and there will be duplication of quantum information.


Variables and expressions in CQP can have types \textbf{Int}, \textbf{Qbit} and \textbf{Unit}, and channels have the correspogint types $\widehat{\ }\textbf{Int}$, $\widehat{\ }\textbf{Qbit}$, $\widehat{\ }\textbf{Unit}$.
The typing judgements in CQP have the form \[\Gamma \vdash P\] meaning that $P$ is well typed under the context $\Gamma$. $\Gamma$ contains both classical variables and quantum variables: the former are treated following the usual typing rules, the latter are subject to affine typing rules. Affine rules guarantee that each quantum variable will be sent at most once, thanks to how the typing contexts $\Gamma$ are constructed, lacking a contraction rules for quantum variables. \note{approfondisco una spiegazione sulle regole strutturali e il type system affine?}.


From the practical point of view, this means that, \begin{itemize} 
\item if $c!q.P$ is well typed, where $q$ is a quantum variable, then $P$ cannot contain any other occurence $q$
\item if $P \parallel Q$ is well typed, then $P$ and $Q$ cannot have occurrences of the same quantum variables
\end{itemize}
\note{
CQP2005 is a probabilistic-transtition, pi calculus like reduction system, MUST REPRESENTS CLOSED SYSTEMS, WITH REDUCTION RULES, BECAUSE GLOBAL STATE INFROMATION IS REQUIRED, otherwise we can't represent input and output of entangled state. cfr with qpalg2005 and qCCS 2006.
}

\note{
QPAlg2005 introduces a probabilistic branching bisimulation


Davidson introduces a labelled semantics  to CQP. ANd two probabilistic branching bisimulation, the second is a congruence


Puthoor develops the equational theory of davidson's bisimulation, and extends CQP to LOQC
}


 


\note{note:}
\note{\textbf{Inglesi}}
\begin{itemize}
\item \textbf{QPALg2004} configurazioni e probabilismo. complex variablescoping, with a stack in the configuration, sending shrinks rho. 

\item \textbf{QPAlg2006} probabilistic branching bisimilarity, non congruenza perchè entabglement e larsen skou. sending does not shrink rho, reception enlarges rho, but syncronization doesn't use any of the two rules. 
\[ \langle q = \rho, c?x.P \rangle \xrightarrow{c?x} \langle x,q = \nu\otimes\rho, c?x.P \rangle
\]
for any $\nu$, infinite branching


\item \textbf{CQP gay nagarajan popl 05}: pi-calculus like, measurements are expressions, no probabilistic sum (can be implemented with parallel syncronization), reduction semantic con congruenza, typesystem affine per garantire il no cloning. probability-on-transitions approach: a reduction relation $\rightarrow \subseteq S \times \distr(S)$ and a probabilistic choice transition $\rightsquigarrow \subseteq \distr(S)\times [0, 1] \times S$. Configurations of the form (quantum state, channel names, P).

\item \textbf{Thesis Davidson 2011}:

Gives a labelled transition semantic $\langle \sigma, \omega, P\rangle \xrightarrow{alpha} \distr(\sigma, \omega, P)$ and a probabilistic transition  $\rightsquigarrow$ as before, where $\sigma$ contains the quantum state, $\omega$ the used qubits, and $P$ the process. Quantum input doesn't extend rho (here called sigma).

semantics: out removes $q$ from $\omega$, in and qbit add $q$ to $\omega$ 
typing: measure and ops don't add $q$ to $\Sigma$, but expression does. out removes, qbit adds, in should add. 
Sigma is a subset of omega

THe chinese approach equates the quantum names, and require the same final state.
the french-english approach doesn't equates the quantum names, but the (partial trace) of the state in the moment of communication.
Our congurence doesn't equates quantum names, it could if we add a more specific barb $\downarrow_{c!q}$, or a name-matching construct in our contexts. 
Ora come ora, nel nostro sistema, 
\[ P = H(q_1, q_2).c!q_1 \parallel d!q_2 \quad H(q_1, q_2).d!q_1 \parallel c!q_2\]
sono bisimili.

The example 3.2 in page 74 shows two processes that are bisimilar but not congruent. There are to solution to this problem: provide a finer bisimilation, that distinguishes P and Q, confronting the enviroment ($tr_\Sigma(\rho)$) in a larsen skou way, or a coarser bisimulation, that doen't distinguis C[P] and C[Q], confronting the environment of distribution (called in davidson mixed configuration.)


\item \textbf{thesis Puthoor 2015}:
provides a correct set of equational axioms, to define behavioural equivalence axiomatically. Extends CQP to Linear optical quantum computing
\end{itemize}


\note{\textbf{Cinesi}}
\begin{itemize}
\item Feng duan 2006, probabilistic bisimulation for quantum:

Probabilities-on-state approach, strong and week bisimilarity, deadlock quantum state equivalence, different inputs rules for correlated and uncorrelated qubits. Uncorrelated input extends rho. Introduces conbined transitions, i.e. convex closure transitions. bisimilarities not preserved by parallel composition, and restriction 
P = U1[q].c!0.U2[q].nil, Q = V1[q].c!0.V2[q].nil. are bisimilar, but not $P\setminus c$ and $Q\setminus c$
\item Ying feng 2009, an algebra of quantum, no classical comunxication. Input and output don't change rho. superoperators as visible transitions, reduction (i.e. independent superoperators) bisimilarity, approximate bisimulation based on diamond distance between superoperators

\item Feng duan ying bisimulation for quantum, is a congruence, requires equality of the environment, not of the total state. 
\item Open bisimulation for quantum
\end{itemize}