
\section{LTS and quantum states} 

There is a number of proposals of quantum process calculi in the literature, often with different syntax, semantics and behavioural equivalences, even if they all model the same systems and the same protocols. 
The are three main line of research that developed in recent years. The first, started with QPAlg and then developed with CQP, is inspired by the $\pi$-calculus. 
The second approach, developed simultaneously but independently, is centered around qCCS, that is a quantum extension of value-passing CCS. 
This thesis will focus on analyzing similarities and differences of these two calculi, CQP and qCCS. 
The third proposal, exploring the quantum process algebra qACP, is less directly related and comparable with the first two, in the same way as its classical counterpart ACP is designed in a different fashion with respect to CCS/$\pi$-calculus.


Since from the first works by Lalire and Jorrand [qpalg2004], it became evident that the operational semantic of a \textit{quantum} process calculus could not of consist only of syntactic elements, like the transitions $P \rightarrow P'$ of a classical process algebra. A process manipulating and communicating quantum data should always be coupled with a state vector, describing the current state of the quantum system. In all the quantum process calculi, the LTS is always composed of \textit{configurations}, i.e. states of the form 
\[
	\langle q_0, \ldots, q_n = \kp, P\rangle
\] 
in which $P$ is a process containing $q_0 \ldots q_n$ as free variables, and $\kp \in \Hto{n}$  describes the state of the qubits manipulated by $P$. 
This approach solves two crucial problems arising from the peculiarity of quantum computation:\begin{itemize}
\item \textbf{No cloning}: Since quantum information cannot be copied, variable instantiation cannot be performed in a "pass-by-value" fashion like classical process algebras:
	\[ c?x.P \xrightarrow{c?v} P[v/x]
	\] 
In this way, if $P$ contains two occurences of $x$, each of them gets instantiated with a different, independent copy of the value $v$. But if the value $v$ was a state vector $\kp$, this would require duplicating the quantum information.
So in QPAlg and in all other quantum calculi, information is manipulated and passed in a imperative, "pass-by-reference" manner:
\[ \langle q=\kp, c?x.P \rangle \xrightarrow{c?q} \langle q=\kp, P[q/x]\rangle
\]
where $q$ is just a pointer to the quantum variable stored in a configuration.
\item \textbf{Entanglement}: Since a composite quantum system is not always separable, the semantic of two parallel processes $P$ and $Q$ cannot always be described separately, and then simply interleaved with the parallel operator. If manipulating an entangled state the semantic of process $P$ depends by the behaviour of process $Q$, and so the two must be described together, in a global configuration $\langle q_1, q_2 = \beta, P \parallel Q$ (where $\beta$ is the bell state $\oost\ket{00} + \oost\ket{11}$. 
\end{itemize}

Another key feature present in QPAlg and in all other calculi is the coexistence of \textit{nondeterminism}, arising from sums and parallel composition, and probabilistic behaviour, arising from the probabilistic nature of quantum measurements. 
So in all quantum process calculi, a process can be defined by a pLTS $\langle Conf, Act, \rightarrow \rangle$, where $Conf$ is the set of all possible configurations. 
QPAlg and CQP follow the "probabilistic-transition" approach, while qCCS follows the $probabilistic-state$ approach.


in [CQP2005], Gay and Nagarajan presented their calculus Communicating Quantum Process. CQP makes use of an (affine) type system to restrict the set of possible processes of the algebra to the "admissible" ones, the ones respecting the no-cloning theorem. Under the assumption that Alice, Bob and Charlie are in three different physical location, the process \[Alice = b!q.c!q.nil\] should not be well typed, because Bob could read from the $b$ channel, Charlie from the $c$ channel, and there will be duplication of quantum information.


Variables and expressions in CQP can have types \textbf{Int}, \textbf{Qbit} and \textbf{Unit}, and channels have the correspogint types $\widehat{\ }\textbf{Int}$, $\widehat{\ }\textbf{Qbit}$, $\widehat{\ }\textbf{Unit}$.
The typing judgements in CQP have the form \[\Gamma \vdash P\] meaning that $P$ is well typed under the context $\Gamma$. $\Gamma$ contains both classical variables and quantum variables: the former are treated following the usual typing rules, the latter are subject to affine typing rules. Affine rules guarantee that each quantum variable will be sent at most once, thanks to how the typing contexts $\Gamma$ are constructed, lacking a contraction rules for quantum variables. \note{approfondisco una spiegazione sulle regole strutturali e il type system affine?}.


From the practical point of view, this means that, \begin{itemize} 
\item if $c!q.P$ is well typed, where $q$ is a quantum variable, then $P$ cannot contain any other occurence $q$
\item if $P \parallel Q$ is well typed, then $P$ and $Q$ cannot have occurrences of the same quantum variables
\end{itemize}
\note{
CQP2005 is a probabilistic-transtition, pi calculus like reduction system, MUST REPRESENTS CLOSED SYSTEMS, WITH REDUCTION RULES, BECAUSE GLOBAL STATE INFROMATION IS REQUIRED, otherwise we can't represent input and output of entangled state. cfr with qpalg2005 and qCCS 2006.
}

\note{
QPAlg2005 introduces a probabilistic branching bisimulation


Davidson introduces a labelled semantics  to CQP. ANd two probabilistic branching bisimulation, the second is a congruence


Puthoor develops the equational theory of davidson's bisimulation, and extends CQP to LOQC
}


 


\note{note:}
\note{\textbf{Inglesi}}
\begin{itemize}
\item \textbf{Lalire} configurazioni e probabilismo. complex variablescoping, with a stack in the configuration, reception extends rho. Non congruenza perchè entabglement e larsen skou.


\item \textbf{CQP gay nagarajan popl 05}: pi-calculus like, measurements are expressions, no probabilistic sum (can be implemented with parallel syncronization), reduction semantic con congruenza, typesystem affine per garantire il no cloning. probability-on-transitions approach: a reduction relation $\rightarrow \subseteq S \times \distr(S)$ and a probabilistic choice transition $\rightsquigarrow \subseteq \distr(S)\times [0, 1] \times S$. Configurations of the form (quantum state, channel names, P).

\item \textbf{Thesis Davidson 2011}:

Gives a labelled transition semantic $\langle \sigma, \omega, P\rangle \xrightarrow{alpha} \distr(\sigma, \omega, P)$ and a probabilistic transition  $\rightsquigarrow$ as before, where $\sigma$ contains the quantum state, $\omega$ the used qubits, and $P$ the process. Quantum input doesn't extend rho (here called sigma).

semantics: out removes $q$ from $\omega$, in and qbit add $q$ to $\omega$ 
typing: measure and ops don't add $q$ to $\Sigma$, but expression does. out removes, qbit adds, in should add. 
Sigma is a subset of omega

THe chinese approach equates the quantum names, and require the same final state.
the french-english approach doesn't equates the quantum names, but the (partial trace) of the state in the moment of communication.
Our congurence doesn't equates quantum names, it could if we add a more specific barb $\downarrow_{c!q}$, or a name-matching construct in our contexts. 
Ora come ora, nel nostro sistema, 
\[ P = H(q_1, q_2).c!q_1 \parallel d!q_2 \quad H(q_1, q_2).d!q_1 \parallel c!q_2\]
sono bisimili.

The example 3.2 in page 74 shows two processes that are bisimilar but not congruent. There are to solution to this problem: provide a finer bisimilation, that distinguishes P and Q, confronting the enviroment ($tr_\Sigma(\rho)$) in a larsen skou way, or a coarser bisimulation, that doen't distinguis C[P] and C[Q], confronting the environment of distribution (called in davidson mixed configuration.)


\item \textbf{thesis Puthoor 2015}:
provides a correct set of equational axioms, to define behavioural equivalence axiomatically. Extends CQP to Linear optical quantum computing
\end{itemize}


\note{\textbf{Cinesi}}
\begin{itemize}
\item Feng duan 2006, probabilistic bisimulation for quantum:

Probabilities-on-state approach, strong and week bisimilarity, deadlock quantum state equivalence, different inputs rules for correlated and uncorrelated qubits. Uncorrelated input extends rho. Introduces conbined transitions, i.e. convex closure transitions. bisimilarities not preserved by parallel composition, and restriction 
P = U1[q].c!0.U2[q].nil, Q = V1[q].c!0.V2[q].nil. are bisimilar, but not $P\setminus c$ and $Q\setminus c$
\item Ying feng 2009, an algebra of quantum, no classical comunxication. Input and output don't change rho. superoperators as visible transitions, reduction (i.e. independent superoperators) bisimilarity, approximate bisimulation based on diamond distance between superoperators

\item Feng duan ying bisimulation for quantum, is a congruence, requires equality of the environment, not of the total state. 
\item Open bisimulation for quantum
\end{itemize}