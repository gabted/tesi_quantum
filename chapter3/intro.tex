There is a number of proposals of quantum process calculi in the literature, often with different syntax, semantics and behavioural equivalences, even if they all model the same systems and the same protocols. 
The are three main line of research that developed in recent years. The first, started with QPAlg and then developed with CQP, is inspired by the $\pi$-calculus. 
The second approach, developed simultaneously but independently, is centered around qCCS, that is a quantum extension of value-passing CCS. 
This thesis will focus on analyzing similarities and differences of these two calculi, CQP and qCCS. 
The third proposal, exploring the quantum process algebra qACP, is less directly related and comparable with the first two, in the same way as its classical counterpart ACP is designed in a different fashion with respect to CCS/$\pi$-calculus.
