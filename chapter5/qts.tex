\newcommand{\quantumdst}{\mathfrak{D}}
\newcommand{\quantumdsta}{\mathfrak{T}}

Given the difficulties of finding a good notion of behavioural equivalence in quantum process algebras, we consider to address the problem on a minimal setting, only focusing basic constructs.

\section{Mathematical Preliminaries}

A quantum distribution $\quantumdst \in D(S)^\hilbert$ over a set $S$ is a function from the finite-dimensional Hilbert space $\hilbert$ of dimension $n$ to probability distributions $\Delta$ over $S$.
In the following, we write $K_{\Delta}$ defined as the function that always returns $\Delta$ for every state $\ket{\phi} \in \hilbert$.

Let $\Set{P_i | \sum_{i = 1}^{n} P_i = I }$ be a set of quantum projectors, and let $\quantumdst_i$, $1 \leq i \leq n$, be a collection of quantum distributions.
We use $\sum_{i = 1}^{n} P_i \quantumdst_i$ to denote the distribution determined by 
\[
\left(\sum_{i = 1}^{n} P_i \quantumdst_i\right) (\ket{\phi}) = \sum_{i = 1}^{n} p_i(\ket{\phi}) \quantumdst_i \left(\frac{P_i \ket{\phi}}{p_i(\ket{\phi})}\right)
\]
where $p_i(\ket{\phi}) = \bra{\phi} P_i \ket{\phi}$.

Note that for any $\ket{\phi}$, ($\sum_{i = 1}^{n} P_i \quantumdst_i) (\ket{\phi})$ is a legal probability distribution since $\quantumdst_i(\ket{\psi})$ is a probability distribution and
\[
\sum_{i = 1}^{n} p_i (\ket{\phi}) = \sum_{i = 1}^{n} \bra{\phi} P_i \ket{\phi} = \bra{\phi} I \ket{\phi} = \braket{\phi | \phi} = 1.
\]

When $n = 2$, $P_2$ is derivable as $I - P_1$, thus we write $\quantumdst_1 \qsum{P_1} \quantumdst_2$ for $\sum_{i = 1}^{n} P_i \quantumdst_i$.
The operator $\blank \qsum{P} \blank$ can be defined from $\blank \tensor[_p]{\oplus}{} \blank$ as follows:
\[
  (\quantumdst \qsum{P_1} \quantumdsta) (\ket{\phi}) = \quantumdst \left(\frac{P_1 \ket{\phi}}{p_1(\ket{\phi})}\right) \tensor[_{p_1(\ket{\phi})}]{\oplus}{} \quantumdsta \left(\frac{P_2 \ket{\phi}}{p_2(\ket{\phi})}\right)
\]

We define the common notions of linearity and decomposability as usual.
\begin{definition}
We say that a relation $\rel \subseteq D(S)^\hilbert \times D(S)^\hilbert$ is \emph{linear} over $\blank \boxplus \blank$ if $\quantumdst_i\,\rel\,\quantumdsta_i$, $i = 1,2$, implies $(\quantumdst_1 \qsum{P} \quantumdst_2)\,\rel\,(\quantumdsta_1 \qsum{P} \quantumdsta_2)$ for any $P$.
We say that a relation $\rel \subseteq D(S)^\hilbert \times D(S)^\hilbert$ is \emph{left-decomposable} over $\blank \boxplus \blank$ if $(\quantumdst_1 \qsum{P} \quantumdst_2)\ \rel\ \quantumdsta$ implies $\quantumdsta = (\quantumdsta_1 \qsum{P} \quantumdsta_2)$ where $\quantumdst_i\ \rel\ \quantumdsta_i$, for $i = 1, 2$.
\emph{Right-decomposable} relations are defined as expected, and a relation is \emph{decomposable} if it is both left- and right-decomposable. 
\end{definition}

The quantum lifting of a relation is defined as follows.
\begin{definition}
	Let $\rel$ be a relation in $S \times S$, we define its quantum lifting $\qlift{\rel}$ as a relation in $D(S)^\hilbert \times D(S)^\hilbert$ as the minimal relation such that
	\begin{enumerate}
		\item $s\ \rel\ s'$ implies $K_{\overline{s}} \qlift{\rel} K_{\overline{s'}}$; and
		\item $\quantumdst_i \qlift{\rel} \quantumdsta_i$, i = $1, 2$, implies $\quantumdst_1 \qsum{P} \quantumdst_2 \qlift{\rel} \quantumdsta_1 \qsum{P} \quantumdsta_2$ for any projector $P$.
	\end{enumerate}
\end{definition}

%The quantum lifting of a function is linear by definition.
%We prove it is also decomposable.
%\begin{proposition}
%	Let $\rel$ be a relation in $S \times S$, its quantum lifting $\qlift{\rel}$ is decomposable.
%\end{proposition}
%\begin{proof}
%	We proceed by induction on the derivation of $\quantumdst \qlift{\rel} \quantumdsta$.
%	Case \textit{1.} is trivial, $K_{\overline{s}}$ can be decomposed only as $K_{\overline{s}} \qsum{P} K_{\overline{s}}$, or as $K_{\overline{s}} \qsum{I} \quantumdsta'$ for any $\quantumdsta'$, and the same for $K_{\overline{s'}}$.
%	In case \textit{2.}, let 
%	Assume $(\quantumdst_1 \qsum{P} \quantumdst_2) \qlift{\rel} \quantumdsta$, 
%\end{proof}
%
%\begin{proposition}
%	if $\rel$ is an equivalence relation, then $\qlift{\rel}$ is an equivalence relation.
%\end{proposition}
%\begin{proof}
%	Symmetry and reflexivity are trivial.
%	We prove transitivity by induction on the derivations of $\quantumdst \qlift{\rel} \quantumdsta$.
%	The only axiom is $K_{\overline{s}} \qlift{\rel} K_{\overline{s'}}$ if $s \rel s'$.
%	Assume 	
%\end{proof}

\begin{proposition}
	Let $\rel$ be a relation in $S \times S$, and let $\quantumdst, \quantumdsta$ be quantum distributions such that $\quantumdst \qlift{\rel} \quantumdsta$.
	Then, for any $\ket{\phi}$, $\quantumdst(\ket{\phi}) \slift{\rel} \quantumdsta(\ket{\phi})$.
\end{proposition}
\begin{proof}
	We proceed by induction on the derivation of $\quantumdst \qlift{\rel} \quantumdsta$.
	Case \textit{1.} is trivial, for any $\ket{\phi}, K_{\overline{s}}(\ket{\phi}) = \overline{s}$, and $K_{\overline{s'}}(\ket{\phi}) = \overline{s'}$, and $s\ \rel\ s'$ implies $\overline{s} \slift{\rel} \overline{s'}$.
	For case \textit{2.}, we have that
	\begin{align*}
	(\quantumdst_1 \qsum{P_1} \quantumdst_2) (\ket{\phi}) = \quantumdst_1 \left(\frac{P_1 \ket{\phi}}{p_1(\ket{\phi})}\right) \tensor[_{p_1(\ket{\phi})}]{\oplus}{} \quantumdst_2 \left(\frac{P_2 \ket{\phi}}{p_2(\ket{\phi})}\right)\\
	(\quantumdsta_1 \qsum{P_1} \quantumdsta_2) (\ket{\phi}) = \quantumdsta_1 \left(\frac{P_1 \ket{\phi}}{p_1(\ket{\phi})}\right) \tensor[_{p_1(\ket{\phi})}]{\oplus}{} \quantumdsta_2 \left(\frac{P_2 \ket{\phi}}{p_2(\ket{\phi})}\right)
	\end{align*}
	By induction hypothesis we know that $\quantumdst_i \left(\frac{P_i \ket{\phi}}{p_i(\ket{\phi})}\right) \slift{\rel} \quantumdsta_i \left(\frac{P_i \ket{\phi}}{p_i(\ket{\phi})}\right)$, for $i = 1,2$, and the thesis holds by linearity.
\end{proof}


%Given a relation $\rel \subseteq D(S) \times D(S)$, we call $\text{ext}(\rel)$ its \emph{point-wise extension} such that $\quantumdst\,\text{ext}(\rel)\,\quantumdsta$ if and only if $\forall \ket{\phi}\ldotp \quantumdst(\ket{\phi})\,\rel\,\quantumdsta(\ket{\phi})$.
%We say that a relation $\rel \subseteq D(S)^\hilbert \times D(S)^\hilbert$ is \emph{point-wise} whenever a relation $\rel_0 \subseteq D(S) \times D(S)$ exists such that $\rel = ext(\rel_0)$.  
%
%\begin{proposition}
%	Given any relation $\rel \subseteq D(S) \times D(S)$, $\text{ext}(\rel)$ is \emph{linear} over $\blank \boxplus \blank$ if and only if $\rel$ is linear over $\blank \oplus \blank$.
%\end{proposition}

%\paragraph{Physical Consistency}
%
%Not all quantum distributions are physically implementable, as a trivial example, take any $\Delta$ such that $\Delta(\ket{\phi}) \neq \Delta(-\ket{\phi})$.
%Clearly, since all the normalized vectors $\ket{\phi}, -\ket{\phi}, i \ket{\phi}, -i \ket{\phi}$ represent the same quantum state, $\Delta$ is not consistent with physical law.

%A quantum boh $\delta \in d_q(S)$ over a set $S$ is a function from the Hilbert space $\hilbert$ to $S$ such that:
%\begin{itemize}
%	\item for each $s \in S$, the constant function $\delta(\phi) = s$ is in $d_q(S)$;
%	\item for each unitary transformation $U$, if $\delta \in d_q(S)$ then $\delta'$ is also in $d_q(S)$ with $\delta'(\ket{\phi}) = U \delta(\ket{\phi})$;
%	\item for any projector $P$, if $\delta \in d_q(S)$ then $\delta'$ is also in $d_q(S)$ with $\delta'(\ket{\phi}) = \delta(\frac{P \ket{\phi}}{p(\ket{\phi}})$;
%\end{itemize}
%
%A quantum distribution $\Delta \in D_q(S)$ over a set $S$ is \emph{legal} if and only if:



\section{Quantum Labeled Transition Systems}

A quantum labeled transition system QLTS on an Hilbert space $\hilbert$ is a triple $(S, Act_\tau, \rightarrow)$ where
\begin{itemize}
	\item $S$ is a set of states $s, s_1, \dots$;
	\item $Act_\tau$ is a set of transition labels with $\tau$ a distinguished element;
	\item $\rightarrow\;\subseteq S \times Act_\tau \times D(S)^\hilbert$ is the transition relation. 
%	such that for each $\delta \xrightarrow{\mu} \Delta$ either:
%	\begin{enumerate}
%		\item $\Delta = \bar{\delta'}$ for some $\delta'$ and a unitary matrix $U$ exists such that $\delta'(U \ket{\phi}) = \delta(U \ket{\phi})$ for any $\ket{\phi}$;	
%	\end{enumerate}
\end{itemize}

We define a minimal quantum process algebra (mQPA) for describing quantum processes.
A quantum process $Q$ is defined as
\[
Q ::= \nil \mid \mu.Q \mid Q + Q \mid U \circ Q \mid Q \qsum{P} Q
\]
where $U$ is a unitary transformation over $\hilbert$.

We give the semantics of mQPA in terms of QLTS.
Some terms are taken as states $s \in S$, in particular the ones where unitary operators and $\blank \boxplus \blank$ are guarded.
\[
s ::= \nil \mid \mu.Q \mid s + s
\]
%We write $\sema{s}$ for the function $S^\hilbert$ they stand for, i.e., 
%\begin{align*}
%	&\sema{s}(\ket{\phi}) = (\ket{\phi}, s)
%\end{align*}

The interpretation of an arbitrary term $Q$ as quantum distribution $\sem{Q}$ over S is given by the function $\sem{\blank}$:
\begin{align*}
	&\sem{\nil}(\ket{\phi}) = \overline{\nil}\\
	&\sem{\mu.Q}(\ket{\phi}) = \overline{\mu.Q}\\
	&\sem{Q_1 + Q_2}(\ket{\phi})(s) = 
	\begin{cases}
		\sem{Q_1}(\ket{\phi})(s_1) \cdot \sem{Q_2}(\ket{\phi})(s_2) & \text{if } s = s_1 + s_2\\
		0 & \text{otherwise}
	\end{cases}\\
	&\sem{U \circ Q}(\ket{\phi}) = \sem{Q}(U \ket{\phi})\\
	&\sem{Q_1 \qsum{P} Q_2} = \sem{Q_1} \qsum{P} \sem{Q_2}
\end{align*}

The following proposition is trivially derivable by definition.
\begin{proposition}
	For any $s \in S$, $\sem{s} = K_{\overline{s}}$.
\end{proposition}

The transition relation $\to$ is defined as follows, with $s \xrightarrow{\mu} \Delta$ as notation for $(s, \mu, \Delta) \in\;\to$.
\begin{gather*}
  \infer[\mbox{\footnotesize\scshape Action}]{\mu.Q \xrightarrow{\mu} \sem{Q}}{} \qquad 
  \infer[\mbox{\footnotesize\scshape Ext.L}]{Q_1 + Q_2 \xrightarrow{\mu} \Delta}{Q_1 \xrightarrow{\mu} \Delta} \qquad
  \infer[\mbox{\footnotesize\scshape Ext.R}]{Q_1 + Q_2 \xrightarrow{\mu} \Delta}{Q_2 \xrightarrow{\mu} \Delta}
\end{gather*}

We define bisimularity on QLTS as usual.
\begin{definition}
	A symmetric relation $\rel : S \times S$ is called a QLTS \emph{bisimulation} if it's symmetric and for each pair of states $s, s' \in S$ such that $s \rel s'$,
	if $s \xrightarrow{\mu} \Delta$ then $s' \xrightarrow{\mu} \Delta'$ and $\Delta\,\qlift{\rel}\,\Delta'$, for some quantum distributions $\Delta' \in D(S)$.
	\emph{Q-bisimilarity} $\sim_Q$ is the largest QLTS bisimulation.
\end{definition}


\subsection{Testing Bisimilarity over problematic Cases}

We take the example 6, 7, 8, and show that they behave as expected in mQPA.

For \note{examples 6 and 7} consider the following.
\begin{example}
	A qbit which is in a mixed state of $\ket{0}$ and $\ket{1}$ with equal probability behaves exactly as a qbit which is in a mixed state of $\ket{+}$ and $\ket{-}$.
	Consider the two following processes.
	\begin{align*}
		Q &= (s \qsum{\ketbra{0}{0}} s') \qsum{\ketbra{+}{+}} (s \qsum{\ketbra{0}{0}} s')\\
		Q' &= (s \qsum{\ketbra{+}{+}} s') \qsum{\ketbra{0}{0}} (s \qsum{\ketbra{+}{+}} s')
	\end{align*}
	Note that in $Q$ (in $Q'$ resp.), after the nested measure, the qbit is in state $\ket{0}$ or $\ket{1}$ ($\ket{+}$ or $\ket{-}$ resp.) with equal probability.
	Indeed, $Q$ and $Q'$ stand for the same quantum distribution: $\sem{Q} = \sem{Q'} = K_{\bar{s}}$.
\end{example}

Example \note{8} is addressed as follows.
\begin{example}
	Consider the bell pair $\ket{\Phi^+} = 1/\sqrt{2}(\ket{00} + \ket{11})$.
	Let $Q$, $Q'$, $Q_0$ and $Q_+$ be as follows
	\begin{align*}
		Q &= z \qsum{P_{\_0}} u\\
		Q' &= p \qsum{P_{\_+}} m\\
		Q_0 &= ((Q + Q') \qsum{P_{\_0}} (Q + Q')) \qsum{P_{\ketbra{\Phi^+}{\Phi^+}}} \nil\\
		Q_+ &= ((Q + Q') \qsum{P_{\_+}} (Q + Q'))  \qsum{P_{\ketbra{\Phi^+}{\Phi^+}}} \nil
	\end{align*}
	Where $P_{\_0}$ and $P_{\_+}$ are defined as $\ketbra{00}{00} + \ketbra{10}{10}$ and $\ketbra{0+}{0+} + \ketbra{1+}{1+}$ respectively.
	Note that $\sem{P_0} \qlift{\sim} \sem{P_+}$ iff they behave the same on $\ket{\Phi^+}$, and they should, because the mixed states obtained by measuring a single qbit on the computational basis, and on the Hadamard basis are the same.

	Indeed $\sem{P_0} = \sem{P_+}$.
	Take any $\ket{\phi}$,
		\begin{align*}
		\sem{P_0}(\ket{\phi}) &= 
		\sem{((Q + Q') \qsum{P_{\_0}} (Q + Q'))}(\ket{\Phi^+}) \psum{p} \bar{\nil}\\
		&= (\sem{Q + Q'}(\ket{00}) \psum{1/2} \sem{Q + Q'}(\ket{11})) \psum{p} \bar{\nil}\\
		&= (((\bar{z} + \bar{p}) \psum{1/2} (\bar{z} + \bar{m})) \psum{1/2} ((\bar{u} + \bar{p}) \psum{1/2} (\bar{u} + \bar{m}))) \psum{p} \bar{\nil}\\
		&= (((\bar{z} + \bar{p}) \psum{1/2} (\bar{u} + \bar{p})) \psum{1/2} ((\bar{z} + \bar{m}) \psum{1/2} (\bar{u} + \bar{m}))) \psum{p} \bar{\nil}\\		
		&= (\sem{Q + Q'}(\ket{++}) \psum{1/2} \sem{Q + Q'}(\ket{--})) \psum{p} \bar{\nil}\\
		&= \sem{((Q + Q') \qsum{P_{\_+}} (Q + Q'))}(\ket{\Phi^+}) \psum{p} \bar{\nil} = \sem{P_+}(\ket{\phi}).
		\end{align*}
\end{example}


\subsection{Alternative}
We define an alternative characterization of a quantum transition system that is more in-line with preexisting quantum transition systems like~\cite{fengBisimulationQuantumProcesses2012, dengOpenBisimulationQuantum2012}.
A quantum labeled transition system qLTS on an Hilbert space $\hilbert$ is a triple $(S, Act_\tau, \hookrightarrow)$ where
\begin{itemize}
	\item $S$ is a set of states $s, s_1, \dots$;
	\item $Act_\tau$ is a set of transition labels with $\tau$ a distinguished element;
  \item $\hookrightarrow\;\subseteq (\hilbert \times S) \times Act_\tau \times D(\hilbert \times S)$ is the transition relation. 
%	such that for each $\delta \xrightarrow{\mu} \Delta$ either:
%	\begin{enumerate}
%		\item $\Delta = \bar{\delta'}$ for some $\delta'$ and a unitary matrix $U$ exists such that $\delta'(U \ket{\phi}) = \delta(U \ket{\phi})$ for any $\ket{\phi}$;	
%	\end{enumerate}
\end{itemize}

To simplify our presentation, we reduce to $Q$ and $S$ of a specific form, namely where the branches of non-deterministic choices are always guarded by a transition, like in $(\tau. U \circ \nil) + (\alpha . U' \circ \nil)$.
Note that this is consistent with the behaviour of preexisting quantum process algebras like qCCS.
We stress that a term is in this specific form by writing $\hat{Q}$ or $\hat{S}$.

We define the interpretation of a pair $(\ket{\phi}, \hat{Q})$ as a distribution as given by the function $\sema{\blank} : (\hilbert \times S) \to D(\hilbert \times S)$:
\begin{align*}
	&\sema{\ket{\phi}, s} = \overline{(\ket{\phi}, s)} \\
	&\sema{\ket{\phi}, U \circ \hat{Q}} = \sema{U\ket{\phi}, \hat{Q}} \\
	&\sema{\ket{\phi}, \hat{Q_1} \qsum{P} \hat{Q_2} } = \sema{P \ket{\phi}, \hat{Q_1}} \psum{p(P, \ket{\phi})} \sema{P^{\bot}\ket{\phi}, \hat{Q_2}} 
\end{align*}

The transition relation $\hookrightarrow$ is defined as follows, with $s \xhookrightarrow{\mu} \Delta$ as notation for $(s, \mu, \Delta) \in\;\hookrightarrow$.
\begin{gather*}
  \infer[\mbox{\footnotesize\scshape Action}]{(\ket{\phi}, \mu.Q) \xhookrightarrow{\mu} \sema{\ket{\phi}, Q}}{} \\[0.3cm]
  \infer[\mbox{\footnotesize\scshape Ext.L}]{(\ket{\phi}, Q_1 + Q_2) \xhookrightarrow{\mu} \Delta}{(\ket{\phi}, Q_1) \xhookrightarrow{\mu} \Delta} \qquad
  \infer[\mbox{\footnotesize\scshape Ext.R}]{(\ket{\phi}, Q_1 + Q_2) \xhookrightarrow{\mu} \Delta}{(\ket{\phi}, Q_2) \xhookrightarrow{\mu} \Delta}
\end{gather*}

%\subsubsection{Bisimulation}

\begin{definition}
	A symmetric relation $\rel : (\hilbert \times S) \times (\hilbert \times S)$ is a qLTS \emph{bisimulation} if for each pair $(\ket{\phi}, s), (\ket{\phi'}, s)$ such that $(\ket{\phi}, s) \rel (\ket{\phi'}, s)$,
	if $(\ket{\phi}, s) \xrightarrow{\mu} \Delta$ then $(\ket{\phi}', s') \xrightarrow{\mu} C'$ and $C \slift{\rel} \Delta'$.
	\emph{q-bisimilarity} $\sim_q$ is the largest qLTS bisimulation.
\end{definition}
