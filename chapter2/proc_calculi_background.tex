\subsection{Process Calculi}


Process Calculus, also known as Process Algebra, is an algebraic approach to model concurrent computation, often focus on the communication between different agents. Each agent is formalized as a \textit{process}, a syntactic element that describes its capabilties. Processes can be composed in different ways, and so form an \textit{algebra of processes}, with various operators for parallel composition, sequential compositions, probabilistic and non-deterministic sum, and so on.


All the foundational and most successful process calculi, [ccs, acp, csp, pi-calc] have a number of key features in common \begin{itemize}
\item \textbf{Nil Process} Each calculus has a \textit{constant process}, usually denoted as $nil$ or $\nil$, that is in a terminal state, can not perform any action.
\item \textbf{Visible actions} Each calculus has \textit{action prefixes}, usually denotes as $\alpha, \beta, \ldots, \overline{\alpha}, \overline{\beta}, \ldots$. If $P$ is a process, $\alpha.P$ is a process that can perform an action $\alpha$, and then behaves as $P$. The two actions $\alpha$ and $\overline{\alpha}$ are called \textit{coactions}, and denotes two dual vies of the same comunication event. If $\alpha$ is the action of sending a qubit through a channel, $\overline{\alpha}$ is the dual action of recieving that qubit. 
\item \textbf{Internal Actions} Each calculus defines also a $\tau$ action, called \textit{internal action} or silent action. Process calculi are designed to model the interaction between system, and abstract away from the low level details. A process $\tau.P$ can then performs a silent action, indicating internal operations, not known to an external observer, outside the scope of the system being modelled. 
%For example, when describing a TCP handshake protocol, only the IP packets sent on the net are represented as visible actions, while the actual construction of the IP packet in an internal action.
\item \textbf{Non-determinism} Each calculus has a \textit{non-detirministic sum} of processes, denoted as $P + Q$. Such a process can non deterministically "choose" to behave like $P$ or like $Q$. 
\item \textbf{Parallel Execution} Each calculus has \textit{parallel composition} of processes, denoted as $P \parallel Q$. Such a process represent the concurrent execution of both $P$ and $Q$, and can perform all the actions of $P$ as well as all the action of $Q$, in a interleaving manner. Differently from the nondeterministic sum, after $P\parallel Q$ performs an action of $P$, it can still perfoms the actions of $Q$.
\end{itemize}

\subsection{Labelled Transition System}
\note{lts}

A \textit{Labelled Transition System} (LTS) is a triple $\langle S , Act, \rightarrow \rangle$ where \begin{itemize}
\item $S$ is a set of states
\item $Act$ is a set of transition labels
\item $\rightarrow 	\subseteq S\times Act_\tau \times S$ is the transition relation, with $Act_\tau = Act \cup \{\tau\}$
\end{itemize} 

An element $(s, \alpha, t) \in \rightarrow$ is called a \textit{transition}, and is often written as $s \xrightarrow{\alpha} t$. We denote with $\Rightarrow$ the reflexive and transitive closure of $\xrightarrow{\tau}$, and use $s \xRightarrow{\alpha} t$ as an abbreviation for $s \Rightarrow s' \xrightarrow{\alpha} t' \Rightarrow t$ for some $s', t' \in S$.

\subsection{Bisimulation}

Let $\langle S , Act, \rightarrow \rangle$ be a LTS. Then a symmetric relation $\rel \subseteq S \times S$ is a \textit{strong bisimulation} if and only if, whenever $s \rel t$, then 
\begin{center}
if $s \xrightarrow{\alpha} s'$ then $t \xrightarrow{\alpha} t'$ for some $t'$ such that $s' \rel t'$
\end{center}
Two states $s, t \in S$ are said to be \textit{strongly bisimilar}, written $s \sim t$, if exists a strong bisimulation $\rel$ such that $s \rel t$.


Let $\langle S , Act, \rightarrow \rangle$ be a LTS. Then a symmetric relation $\rel \subseteq S \times S$ is a \textit{weak bisimulation} if and only if, whenever $s \rel t$, then 
\begin{center}
if $s \xrightarrow{\alpha} s'$ then $t \xRightarrow{\alpha} t'$ for some $t'$ such that $s' \rel t'$
\end{center}
Two states $s, t \in S$ are said to be \textit{weakly bisimilar}, written $s \approx t$, if a weak bisimulation $\rel$ exists such that $s \rel t$. 


Let $\langle S , Act, \rightarrow \rangle$ be a LTS. Then a symmetric relation $\rel \subseteq S \times S$ is a \textit{branching bisimulation} if and only if, whenever $s \rel t$, then 
\begin{center}
if $s \xrightarrow{\alpha} s'$ then $t \xRightarrow t' \xrightarrow{\alpha} t''$ for some $t', t''$ such that $s \rel t'$ and $s' \rel t''$
\end{center}
Two states $s, t \in S$ are said to be \textit{branching bisimilar}, written $s \simeq t$, if a branching bisimulation $\rel$ exists such that $s \rel t$.

\subsection{Probabilistic LTS} \label{pLTS}

Given a set $S$, a (discrete) \textit{probability distribution on $S$} is a mapping $\Delta: S \rightarrow [0, 1]$ such that $\sum_{s\in S} \Delta(s) = 1$. We indicate with $\distr(S)$ the set of all probability distribution on $S$.
The \textit{support} of $\Delta \in \distr(S)$ is defined as $\lceil\Delta\rceil = \{s \in S \mid \Delta(s) > 0\}$. We use $\overline{s}$ to denote the point distribution on $s$ (also known as Dirac distribution, in the continuous case):
\[
	\overline{s}(t) = 
	\begin{cases} 1 \text{ if }t = s \\
	0 \text{ if } t\neq s
	\end{cases}
\]

Given a set $\{p_i\}$ of probabilities (i.e. $\sum_i p_i = 1$ and $p_i > 0$ for each $i$), we define the \textit{convex combination} of distributions as the distribution obtained as
\[
\left(\sum_i p_i \Delta_i\right)(s) = \sum_i p_i \Delta_i(s)
\]

We often abbreviate $p \Delta + (1-p) \Theta$ as $\Delta \psum{p} \Theta$.


A relation $\mathcal{R} \subseteq \distr(S) \times \distr(S)$ is said to be \textit{linear} if $\Delta_i \rel \Theta_i$ and $\Delta_j \rel \Theta_j$ implies $(\Delta_i \psum{p} \Delta_j) \rel (\Theta_i \psum{p} \Theta_j)$ of any $0 \leq p \leq 1$.


A \textit{Probabilistic Labelled Transition System} (pLTS) is a triple $\langle S , Act, \rightarrow \rangle$ where \begin{itemize}
\item $S$ is a set of states
\item $Act$ is a set of transition labels
\item $\rightarrow 	\subseteq S\times Act_\tau \times \mathcal{D}(S)$ is the transition relation, with $Act_\tau = Act \cup \{\tau\}$
\end{itemize} 

two possible way to transform a pLTS in a LTS: \begin{itemize}
\item Probabilities on transitions
\item Probabilities on states
\end{itemize}


Given a relation $\rel \subseteq S \times S$, we define its \textit{lifting} $\mathring{\rel} \subseteq \distr(S) \times \distr(S)$ as the smaller linear relation such that $s \rel t$ implies $\overline{s} \mathring{\rel} \overline{t}$.

%With abuse of notation, we denote with the same symbol also the lifting of relations $\rel \subseteq S \times \distr(S)$. Given a relation $\rel \subseteq S \times \distr(S)$, we define its \textit{lifting} $\mathring{\rel} \subseteq \distr(S) \times \distr(S)$ as the smaller linear relation such that $s \rel \Delta$ implies $\overline{s} \mathring{\rel} \Delta$.


This lifted relation enjoys two useful properties. Interestingly, both this property are equivalent to the given definition, and are indeed used as the definition in various works on probabilistic bisimulations.

Given $\rel \subseteq S \times S$, then $\Delta \lrel \Theta$ if ans only if: \begin{enumerate}
\item $\Delta = \sum_{i \in I} p_i \overline{s_i}$, where $I$ is a finite index set and $\sum_{i \in I}p_i = 1$
\item For each $i \in I$ there is a state $t_i$ such that $s_i \rel t_i$
\item $\Theta = \sum_{i\in I}p_i\overline{t_i}$ 
\end{enumerate}

Given an equivalence $\rel \subseteq S \times S$, then $\Delta \lrel \Theta$ if and only if, for all equivalence classes $C \in S/R$
\[\sum_{s\in C} \Delta(s) = \sum_{s\in C} \Theta(s)\]


Let $\langle S , Act, \rightarrow \rangle$ be a pLTS. Then a symmetric relation $\rel \subseteq S \times S$ is a \textit{probabilistic bisimultion} if and only if, whenever $s \rel t$, then 
\begin{center}
if $s \xrightarrow{\alpha} \Delta$ then $t \xrightarrow{\alpha} \Theta$ for some $\Theta$ such that $\Delta \lrel \Theta$
\end{center}

\subsection*{Reduction systems}

A \textit{Reduction System} (RS) is a couple $\langle S,  \rightarrow \rangle$ where \begin{itemize}
\item $S$ is a set of states
\item $\rightarrow 	\subseteq S\times S$ is the transition relation.

We call \textit{barb} a predicate on states, often used to capture a certain notion of "observable property".
Given a barb $b$, we write $s\downarrow_b$ to say that $s$ statisfies the predicate $b$, i.e. expresses that property.

Let $\langle S , \rightarrow \rangle$ be a RS, and $B$ a set of barbs. Then a symmetric relation $\rel \subseteq S \times S$ is a \textit{barbed bisimulation} if and only if, whenever $s \rel t$, then 
\begin{itemize}
\item If $s \downarrow_b$ for some barb $b \in B$, then $t \downarrow_b$ \\
\item If $s \rightarrow s'$ then $t \rightarrow t'$ for some $t'$ such that $s' \rel t'$
\end{itemize}
Two states $s, t \in S$ are said to be \textit{barbed bisimilar}, written $s \sim_b t$, if a barbed bisimulation $\rel$ exists such that $s \rel t$.

Given a set of contexts, two states $s, t \in S$ are said to be \textit{barbed congruent} if for any context $C[]$, it holds that $C[s] \sim_b C[t]$. 

\note{Sarebbe meglio definire i contesti su processi, non su stati, probabilmente ridefinirò tutte le bisimulazioni come relazioni su processi non su stati.}
\end{itemize} 