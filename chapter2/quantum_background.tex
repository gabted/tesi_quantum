
The laws on Quantum Mechanics, as we understand them, are elegantly formalized in a mathematical framework, built upon simple linear algebra. This framework is based on a few \textit{postulates} that describe the nature and evolution of quantum systems. Since quantum computing is just the technique of manipulating quantum systems to perform some computation, it will necessarily follows the same postulates. Before presenting each postulate, we will recall the necessary basic definition from linear algebra, formulated in the Dirac's "bra-ket" notation. For further reading, the standard textbook on the subject is \cite{nielsen_chuang_2010}.


\subsection{State space}

A \textit{column vector} in a complex vector space is written $\kp$, and it's called a "ket",
\[ \kp = \begin{pmatrix}
		\alpha_1\\
		\vdots\\
		\alpha_n
\end{pmatrix}
\]
where $\alpha_1, \ldots,  \alpha_n \in \mathbb{C}$. Its \textit{conjugate transpose} is written $\bra{\psi}$, and its called a "bra".
	\[
		\bra{\psi} = \kp^\dagger = (\alpha_1^* \ldots \alpha_n^*)
	\]

A (finite-dimensional) \textit{Hilbert space}, often denoted as $\cal{H}$, is a complex inner product space, i.e. a complex vector space equipped with a binary operator $\braket{  \,\_\, |\, \_\, }: \calH \times \calH \rightarrow \mathbb{C}$ called \textit{inner product}, dot product, or simply "braket".
\[
	\braket{\psi | \phi} = 
	\begin{pmatrix}
	\alpha_1^* \ldots \alpha_n^*
	\end{pmatrix}\begin{pmatrix}
	\beta_1 \\
	\vdots \\
	\beta_n
	\end{pmatrix} = 
	\sum_i \alpha_i^*\beta_i
\]

The inner product satisfies the following properties:
\begin{align*}
\text{Conjugate symmetry} & &  \braket{\psi | \phi} &= \braket{\phi | \psi}^* \\
\text{Linearity \quad\quad} & & \bra{\psi} (\alpha\ket{\phi} + \beta\ket{\varphi}) &= \alpha\braket{\psi | \phi} + \beta\braket{\psi | \varphi}\\
\text{Positive definiteness} & & \braket{\psi | \psi} &\geq 0
\end{align*}

Notice that $\braket{\psi|\psi} = 0$ if and only if $\kp$ is the $\mathbf{0}$ vector. Besides, thanks to conjugate symmetry, we have $\braket{\psi | \psi} = \braket{\psi | \psi} ^ *$, so $\braket{\psi | \psi}$ it's always a real, non-negative number, when $\kp \neq \mathbf{0}$.

Two vectors $\kp$ and $\kf$ are \textit{orthogonal} if
 \[\braket{\psi | \phi} = 0\]


The \textit{norm} of $\kp$ is defined as: 
\[ \|\kp\| = \sqrt{\braket{\psi|\psi}}\]


A \textit{unit vector} is a vector $\kp$ such that \[\|\kp\| = 1\].

A set of vectors $\{\kp_i\}_i$ is an \textit{orthonormal basis} of $\calH$ if \begin{itemize}
	\item each vector $\kf \in \calH$ can be expressed as a \textit{linear combination} of the vector in the basis, $\kf = \sum_i \alpha_i\kp_i$.
	\item All the vector in the basis are orthogonal
	\item All the vector in the basis are unit vector 
\end{itemize}

We're now ready to present the postulates of Quantum Mechanics, in the form more convenient for quantum computing.

\begin{quote} 
\textbf{Postulate I}: The state of an isolated physical system is represented, at a fixed time $t$, by a unit vector $\kp$, called the \textit{state vector}, belonging to a Hilbert space $\calH$, called the \textit{state space}. 
\end{quote}

 When describing the state oh a quantum system, we ignore the \textit{global phase factor}\footnote{An equivalent formulation, in fact, says that a quantum system is described not by a vector but by a \textit{ray}, a one-dimensional subspace of $\calH$}, i.e. 
 \[
 	\kp = \begin{pmatrix}
	\alpha \\
	\beta
	\end{pmatrix} = - \begin{pmatrix}
	\alpha \\
	\beta
	\end{pmatrix} = \lambda \begin{pmatrix}
	\alpha \\
	\beta
	\end{pmatrix} 
	\text{ for each $\lambda \in \mathbb{C}$  such that $|\lambda| = 1$}
 \]

The simplest, prototypical example of a quantum physical system is a \textit{qubit}: a qubit is a physical system with associated a two-dimensional Hilbert Space $\calH^2$. Such systems comprehend an electron in the ground or excited state, a vertically or horizontally polarized photon, or a spin up or spin down particle.

Taken for example a photon, we could say that the photon is in state $\kz$ when vertically polarized, and in state $\ko$ when is horizontally polarized, where $\kz$ and $\ko$ are the two unit vector of the Hilbert space defined as
\[
	\kz = \begin{pmatrix}
	1 \\
	0
	\end{pmatrix} \qquad
	\ko = \begin{pmatrix}
	0 \\
	1
	\end{pmatrix}
\]

The vectors $\{\kz, \ko\}$ form an orthonormal basis of $\ $, called the \textit{computational basis}. Since they form a basis, each vector $\kp \in \calH^2$ can be expressed as 
	\[\kp = \begin{pmatrix}
	\alpha \\
	\beta
	\end{pmatrix} = 
	\alpha\kz + \beta\ko
	\]
where $\alpha, \beta \in \mathbb{C}$ and $|\alpha|^2 + |\beta|^2 = 1$.

So, the state of any qubit can mathematically be described as $\kp = \alpha\kz +\beta\ko$, a linear combination of $\kz$ and $\ko$. From the physical point of view, this means that the qubit is in a \textit{quantum superposition} of state $\kz$ and $\ko$, like a photon being diagonally-polarized, or an electron being at the same time in the excited and in the ground state.

Other important vectors in the $\calHt$ state space are $\kpl$ and $\km$,
\begin{align*}
	\kpl = \oost &\begin{pmatrix}
	1 \\
	1
	\end{pmatrix}  = \oost\kz + \oost\ko   \\
	\km  = \oost &\begin{pmatrix}
	1 \\
	-1
	\end{pmatrix}  = \oost\kz - \oost\ko 
\end{align*}
that form the so called \textit{hadamard basis} of $\calHt$. As we will see, $\kpl$ and $\km$ are both an equal superposition of $\kz$ and $\ko$. What differs in the two is the \textit{relative phase}, i.e. the phase between the $\kz$ and $\ko$ component. The states $\kz + \ko$, $-\kz -\ko$, $i\kz + i\ko$ are all equal to $\kpl$ times a certain global phase, and are considered the same state. The states $\kz + \ko$, $\kz - \ko$, $\kz +i\ko$, instead, all differs for a relative phase factor, and have a different behaviors when applied to the same computation. 
\subsection{Unitary Transformations}

For each linear operator $A$ acting on a Hilbert space $\calH$, we denote as $A^\dagger$ the \textit{adjoint} of $A$, i.e. the unique linear operator such that
\[
	\braket{\psi|A\phi} = \braket{A^\dagger\psi|\phi}
\]

A linear operator $A$ acting on a $n$-dimensional Hilbert space $\calH^n$ can be represented as a $n\times n$ matrix, and its adjoint is calculated as 
\[
	A^\dagger = (A^*)^T
\] the conjugate transpose of the matrix $A$.


If it holds that $A = A^\dagger$, we say that $A$ is self-adjoint, or \textit{Hermitian}.

A linear operator $U$ is said to be \textit{unitary} when $U^\dagger = U^{-1}$, which implies 
\[
	UU^\dagger = U^\dagger U = I
\]

Unitary matrices enjoy many useful properties, first of all that they have a spectral decomposition. An other defining characteristic is that they preserve the inner product, $\braket{\psi | \phi} = \braket{U\psi | U\phi}$
\[
	\braket{U\psi | U\phi} = \bra{\psi}U^\dagger U\ket{\psi} = \bra{\psi}I\ket{\psi} = \braket{\psi|\phi}
\]

A corollary of this property is that applying a unitary operator to a unit vector gives a unit vector
\[
	\braket{U\psi | U\psi} = \braket{\psi|\psi} = 1
\]

The following postulate makes obvious why we are interested in unitary transformation.
\begin{quote}
\textbf{Postulate II}: The evolution of a closed quantum system is described by a unitary transformation. That is, the state $\kp$ of the system at time $t_0$ is related to the state $\ket{\psi^\prime}$ of the system at time $t_2$ by a unitary operator $U$ which depends only on the times $t_0$ and $t_1$.
\[
	\ket{\psi^\prime} = U\kp
\]
\end{quote}

According to what we said before, if a physical system starts in a unit state, it will always remain in a unit state.

In quantum computing, the the programmer can manipulate the state of a qubit, applying unitary transformations to it. Some of the most frequent transformation, implemented in every quantum computer, are:
\begin{gather*}
	X = \sigma_X = \begin{bmatrix}
	0 & 1 \\
	1 & 0
	\end{bmatrix} \qquad
	Y = \sigma_Y = \begin{bmatrix}
	0 & -i \\
	i & 0
	\end{bmatrix} \qquad
	Z = \sigma_Z = \begin{bmatrix}
	1 & 0 \\
	0 & -1
	\end{bmatrix} \\ 
	H = \oost\begin{bmatrix}
	1 & 1 \\
	1 & -1
	\end{bmatrix} \qquad
 	S = \begin{bmatrix}
	1 & 0 \\
	0 & i
	\end{bmatrix}\qquad
	T = \begin{bmatrix}
	1 & 0 \\
	0 & e^{i\frac{\pi}{4}}
	\end{bmatrix} 
\end{gather*}

For example, the $H$ operator, called the Hadamard operator, or Hadamard gate, is used to \textit{create superposition}, as it transforms a vector from the computational basis to the Hadamard basis:
\begin{gather*}
	H\kz = \kpl \qquad H\kpl = \kz  \\
	H\ko = \km \qquad H\km = \ko \\	 
\end{gather*}
\subsection{Measurement}

The second postulate describes only the evolution of isolated systems. Such system do not exchange energy nor information with the environment, and all their computations are always reversible (and are in fact formalized with invertible, unitary matrices). To extract classical information from the system, to \textit{measure} the output of a quantum computation, it is needed an interaction between the quantum system and the environment. As we will see, this measurement operation is first of all non-invertible, as different state could produce the same outcome when measured, but is also fundamentally probabilistic: a generic state $\kp$ could produce different measurement outcomes $m_1, m_2,\ldots$, each with a certain probability that depends on $\kp$.

From a physical point of view, if a system is in a superposition of states, measuring it can  cause the wavefunction to collapse to a single state, in a purely probabilistic way. This means that, even if we compute a state that contains the desired information, this information is often difficult to recover, because directly measuring  it can destroy the information and produce a trivial outcome.

\begin{quote} \textbf{Postulate III}: Quantum measurements are described by a set $\{M_m\}_m$ of measurement operators, where the index $m$ refers to the measurement outcomes that may occur in the experiment. The set of measurement operators must be \textit{complete}, i.e.:
\[\sum_m M_m^\dagger M_m = I\]
If the state of the quantum system is $\kp$ before the measurement, then the probability that result $m$ occurs is 
\[p_m = \bra{\psi}M_m^\dagger M_m\kp\]
and the state after the measurement will be \[\frac{1}{\sqrt{p_m}}M_m\kp\]
\end{quote}

The most common class of quantum measurements is composed of \textit{projective measurements}. Such measurements are described by a set of \textit{orthogonal projectors}, i.e. Hermitian operators such that \[ M_mM_{m'} = \begin{cases}\mathbf{0}\quad\text{if } m \neq m' \\ M_m\quad\text{if }m = m'\end{cases}\]


The simplest example of (projective) measurement is simply measuring a state in the computational basis, i.e. projecting it in its $0$-$1$ component. The measurement in the computational basis is defined as \[ M_0 = \begin{pmatrix} 1 & 0 \\ 0 & 0\end{pmatrix} \qquad
M_1 = \begin{pmatrix}0 & 0 \\ 0 & 1\end{pmatrix} \]
And its effect of the state $\kp = \begin{pmatrix}\alpha \\ \beta \end{pmatrix}$ is:
\begin{align*}
\frac{1}{\sqrt{p_0}}M_0\kp = \frac{1}{|\alpha|}\begin{pmatrix} 1 & 0 \\ 0 & 0\end{pmatrix}
\begin{pmatrix}\alpha \\ \beta \end{pmatrix} = \begin{pmatrix} 1 \\ 0 \end{pmatrix} = \kz 
\qquad \text{with probability } \bra{\psi}M_0^\dagger M_0\kp = |\alpha^2| \\
\\
\frac{1}{\sqrt{p_1}}M_1\kp = \frac{1}{|\beta|}\begin{pmatrix} 0 & 0 \\ 0 & 1\end{pmatrix}
\begin{pmatrix}\alpha \\ \beta \end{pmatrix} = \begin{pmatrix} 0 \\ 1 \end{pmatrix} = \ko 
\qquad \text{with probability } \bra{\psi}M_0^\dagger M_1\kp = |\beta^2|
\end{align*}

Notice that, when measuring the $\kz$ state in the computational basis, the outcome will always be $\kz$ with probability $|\alpha|^2 = 1$, in a completely deterministic behavior. 
When instead measuring the $\kpl$ or the $\km$ state, we get either $\kz$ or $\ko$, with equal probability $|\alpha|^2 = |\beta|^2 = \left(\oost\right)^2 = \frac{1}{2}$.
\subsection{Composite quantum systems}
In the previous sections we characterized 1-qubit systems and how they evolve, describing a computation as a series of unitaries and measurements acting on a 2-dimensional Hilbert space. What said before applies easily to larger quantum systems and higher dimensional Hilbert spaces, once again thanks to an elegant mathematical formulation.

If we have two photons, each described by a (2-dimensional) Hilbert space $\calH$, the most natural way to describe the system composed of both photons is as the \textit{tensor product} of Hilbert spaces.	

If $\calH_n$ is a $n$-dimensional Hilbert space, and $\calH_m$ is a $m$ dimensional Hilbert space, their tensor product $\calH_n \otimes \calH_m$ is a $nm$ Hilbert space. If $\{\ket{\psi_1}, \ldots \ket{\psi_n}\}$ is a basis of $\calH_n$, and $\{\ket{\phi_1}, \ldots\ket{\phi_m}\}$ is a base of $\calH_m$, then a basis of $\calH_n \otimes \calH_m$ is\[\big\{\ket{\psi_i}\otimes\ket{\phi_j} \mid i \in [1, \ldots n], j \in [1, \ldots m] \big\}\]
 where $\kp \otimes \kf$ denotes the Kronecker product. We will often omit the tensor symbol, writing $\ket{\psi}\ket{\phi}$ or also $\ket{\psi\phi}$ instead of $\kp\otimes\kf$. We can now state the last postulate we need:

\begin{quote}
\textbf{Postulate IV}: The state space of a composite physical system is the tensor product of the state spaces of the component physical systems. 
\end{quote}

If a single qubit is described by a $2$-dimensional space $\calH$, we will write $\calH^{\otimes n}$ to intend the otimes product $n$ copies of $\calH$, of dimension $2^n$. So, a compound system composed of two qubits has a state space $\calH^{\otimes 2}$, its canonical basis is 
\[\{\ket{00}, \ket{01}, \ket{10}, \ket{11}\}\]
and all its vector can be expressed as a linear combination:
\[\kp \in \Hto{2} = \begin{pmatrix}
\alpha \\ \beta \\ \gamma \\ \delta
\end{pmatrix} = \alpha\ket{00} + \beta\ket{01} + \gamma\ket{10} + \delta\ket{11}\]

A quantum state in $\calH_1 \otimes \calH_2$ is said \textit{separable} when can be expressed as the product of two vectors, one in $\calH_1$ and the other in $\calH_2$. From the definition of the Kronecker product, all separable states of $\Hto{2}$ are of the form:
\[
 \kp = \begin{pmatrix}
 \alpha\gamma \\ \alpha\delta \\ \beta\gamma \\ \beta\delta
 \end{pmatrix} = 
 \begin{pmatrix}
 \alpha \\ \beta
 \end{pmatrix} \otimes 
 \begin{pmatrix}
 \gamma \\ \delta
 \end{pmatrix}
\]

One of the defining characteristics of quantum systems is that not all states in $\Hto{2}$ are separable. The existence of such states, called \textit{entangled} states, implies that a composite system can not always be described as simply the juxtaposition of two smaller states. When a qubit $q_1$ is entangled with an other qubit $q_2$, its evolution depends not only on the transformations applied to $q_1$, but also on the transformations applied on $q_2$, that could be even light-years away. This surprising result does not allow faster then light communication, as we will see in the next section.

The classical example of an entangled state is the so called $\ket{\Phi^+}$ \textit{Bell state}:
\[\ket{\Phi^+} = \oost\ket{00} + \oost\ket{11} = \oost\begin{pmatrix}
1 \\ 0 \\ 0\\ 1
\end{pmatrix}
\]

The fourth postulate tells us that the state space of a composite system is simply the tensor product of the state spaces of the smaller systems. The tensor product of Hilbert spaces is still an Hilbert space, the composition of unit vector is still a unit vector, and the composition of unitary transformation is still unitary. For example, the (Kronecker) composition of $H$ and $I$ matrices is defined as a block matrix:
\[	H\otimes I = \oost\begin{pmatrix}
	I & I \\ I & -I
	\end{pmatrix}\]
And when applied to a two-qubit system, it applies $H$ on the first qubit, and leaves the second one unaltered:
\[ (H\otimes I) \ket{00} = H\kz \otimes I\kz = \oost \begin{pmatrix}
	I & I \\ I & -I
	\end{pmatrix} \begin{pmatrix}
	1 \\ 0 \\0 \\0
	\end{pmatrix} = \oost \begin{pmatrix}
	1 \\ 0 \\ 1 \\ 0
	\end{pmatrix} = \kpl\otimes\kz
\]

One of the most common two-qubit unitary that is not just the composition of one-qubit transformations is the CNOT matrix:
\[\text{CNOT} = 
\begin{pmatrix}
1 & 0 & 0 & 0 \\
0 & 1 & 0 & 0 \\
0 & 0 & 0 & 1 \\
0 & 0 & 1 & 0 \\
\end{pmatrix}
\]

This matrix applies a $X$ transformation on the second qubit only if the first one is $\ko$. This means that if the first bit is in a superposition of $\kz$ and $\ko$, after applying this transformation the whole system will be in a superposition: or the two bits are left equal, or the second one has been flipped. As an example, we can show how the CNOT transformation is used to create entanglement.
\begin{gather*}
\cnot \ket{00} = \ket{00} \qquad \cnot \ket{10} = \ket{11} \\
\cnot \ket{+0} = \oost\big(\cnot\ket{00} + \cnot\ket{10}\big) = \oost
\big(\ket{00} + \ket{11}\big) = \ket{\Phi^+}\\
\cnot \ket{-0} = \oost\big(\cnot\ket{00} - \cnot\ket{10}\big) = \oost
\big(\ket{00} - \ket{11}\big) = \ket{\Phi^-}
\end{gather*}

.\subsection{Density operator formalism}

The formalism presented so far describes quantum system in terms of unit vectors and unitary transformations. There is an alternative, more general formulation, the \textit{density operator formalism}, in which states are represented as positive operators, and transformations as linear maps from operators to operators, i.e. superoperators.

The main advantage of this formulation is that represents also \textit{partial information} about a quantum system. When describing open systems, that are systems which interact with the external environment, it is often impossible to have complete knowledge on the state of our systems. Instead, one could know that the open system is either in state $\kp$, with a certain probability $p$, or in state $\kf$, with probability $1-p$. In other words, we know that the system is in a \textit{probabilistic mixture of states}, called an \textit{ensemble} of states, or also a \textit{mixed state}.

In general, given an $n$-dimensional Hilbert space $\calH$, an \textit{ensemble} of quantum states is a set:
\[\{(\ket{\psi_i}, p_i)\} \]
of quantum states in $\calH$, each with a different probability, such that $\forall i \, p_i > 0$ and $\sum_i p_i \leq 1$. Notice that when $\sum_i p_i = 1$, we have a probability distribution of states, when $\sum_i p_i < 1$, we have a so called subprobability distribution.

Each ensemble defines a density operator, that is a matrix in $\mathbb{C}^{n \times n}$, i.e. an operator $\calH \rightarrow \calH$. The ensemble $\{(\ket{\psi_i}, p_i)\}$, with $\ket{\psi_i} \in \calH$ defines the density operator: 
\[
	\rho = \sum_i p_i \proj{\psi_i}
\]
where $\ketbra{\psi}{\phi}$ denotes the matrix multiplication between the column vector $\kp$ and the row vector $\bra{\phi}$, known as the \textit{outer product}. Notice that this construction is not injective, as there are different ensembles that correspond to the same density operator. We indicate with $\calDH$ the set of density operators of $\calH$.

Density operators enjoy two useful properties (see \cite{nielsen_chuang_2010}): \begin{enumerate}
\item The trace of $\rho$ is the sum of the probabilities of the ensemble $tr(\rho) = \sum_i p_i \leq 1$.
\item $\rho$ is \textit{positive semidefinite}, i.e. 
\[\forall \kp \in \calH \ \bra{\psi}\rho\kp \geq 0\]
\end{enumerate}

Positive semidefinite operators are always diagonalizable with eigenvalues real and positives. So, each positive semidefinite operator with trace $\leq 1$ represents at least one ensemble, with the eigenvectors as states and the corresponding eigenvalues as probabilities.


One of the main application of density operators is to describe the state of a subsystem of a composite quantum system.

Suppose a composite system, made of two subsystem $A$ and $B$, with state space  $\calH = \calH_A \otimes \calH_B$.  Given a generic $\rho^{AB} \in \calH$, that describes the state of the whole system, the operator $\rho^A$ that describes the subsystem $A$ is obtained as 
\[
	\rho^A = tr_B(\rho^{AB})
\]
where the $tr_B$ is called the \textit{partial trace over $B$}, and is defined by 
\[
 tr_B\big(\proj{\psi} \otimes \proj{\phi}\big) = \proj{\psi}tr\big(\proj{\phi}\big)
\]
together with linearity. $\rho^A$ is called the \textit{reduced density operator} of system $A$.

When applied to a separable state $\rho^A \otimes \rho^B$, the partial trace $tr_B$ leaves $\rho^A$ unaltered. When applied to an entangled state, instead, it produces a probabilistic mixture of states, because "forgetting" the information on the state of subsystem $B$ leaves us with only partial information on subsystem $A$. The canonical example is the bell state $\rho = \frac{1}{2} \proj{00} + \frac{1}{2}\proj{11} \in \calH_A \otimes \calH_B$: 

\begin{align*}
tr_B(\rho) &= \frac{1}{2} tr_B\big(\proj{00}\big) + \frac{1}{2}tr_B\big(\proj{11}\big)  \\
	&= \frac{1}{2} tr_B\big(\proj{0}\otimes\proj{0}\big) + \frac{1}{2}tr_B\big(\proj{1}\otimes\proj{1}\big) \\
	&= \frac{1}{2} \proj{0}tr\big(\proj{0}\big) + \frac{1}{2}\proj{1}tr\big(\proj{1}\big) \\
	&= \frac{1}{2} \big(\proj{0} + \proj{1}\big) = \frac{1}{2}I
\end{align*}

\note{superoperators definitions}

Let $\calH_A$ and $\calH_B$ be Hilbert spaces. Given $\mathcal{E}: \mathcal{D}(\calH_A) \rightarrow \mathcal{D}(\calH_A)$ and $\mathcal{F}: \mathcal{D}(\calH_B) \rightarrow \mathcal{D}(\calH_B)$,  their tensor product $\sop\otimes \mathcal{F}: \mathcal{D}(\calH_A \otimes \calH_B)\rightarrow \mathcal{D}(\calH_A \otimes \calH_B)$ is defined as:

\[
	(\sop \otimes \mathcal{F})(\rho_A \otimes \rho_B) = \sop(\rho_A)\otimes \mathcal{F}(\rho_B)
\]

The superoperator $\mathcal{E} \otimes \mathcal{F}$ can be applied to arbitrary density operators $\rho$, expressed as sum of separable states, by linearity.


Superoperators on $\calH$: all the maps $\sop: \calDH \rightarrow \calDH$ satifying:
\begin{itemize}
\item $\sop$ is \textit{convex linear}: for any set of probabilities	$\{p_i\}_i$,
\[\sop\left(\sum_i p_i \rho_i \right) = \sum_i p_i \sop(\rho_i)\]
\item $\sop$ is \textit{completely positive}: for any extra Hilbert space $\calH_R$, for any positive $\rho \in \calH_R \otimes \calH$, $(\mathcal{I}_R\otimes \sop)(\rho)$ is also positive, where $\mathcal{I}_R$ is the identity operator on $\mathcal{D}(\calH_R)$.
\item $\sop$ is \textit{trace non-increasing}: for any density operator $\rho \in \calDH$, 
\[tr\big(\sop(\rho)\big) \leq tr(\rho) \leq 1\].
\end{itemize}

We call $\calSH$ the set of all superoperators on $\calH$.

 	
\textit{Löwner order}: we define the partial order $\sqsubseteq$ on $\calDH$ as :
\[ \rho \sqsubseteq \sigma \text{ if and only if } \sigma-\rho \text{ is positive}\]

\textit{Kraus operator sum}: For any superoperator $\sop \in \calSH$, there exists a (non-unique) finite set of operators $\{E_i\}$, with $1 \leq i \leq dim(\calH)$ such that
\begin{gather*}
\sop(\rho) = \sum_i E_i\rho E_i^\dagger \\
\sum_i E_i^\dagger E_i \sqsubseteq I_\calH
\end{gather*}

Although there are possibly multiple Kraus form for any superoperator $\sop \in \calSH$, they are related by a unitary transformation.
Formally, for any two set of decompositions $\{C_i\}_i$ and $\{B_j\}_j$, there is a unitary matrix $U = \{u_{ij}\}_{ij}$ such that $C_i = \sum_j u_{ij} B_j$.

Given $\mathcal{E}: \mathcal{D}(\calH_A) \rightarrow \mathcal{D}(\calH_A)$ and $\mathcal{F}: \mathcal{D}(\calH_B) \rightarrow \mathcal{D}(\calH_B)$,
respectively expressed in Kraus form as $\mathcal{E}(\rho) = \sum_i E_i \rho E_i^\dag$ and $\mathcal{F}(\rho) = \sum_j F_j \rho E_j^\dag$, the thensor product $\mathcal{E} \otimes \mathcal{F}$
can be expressed in Kraus form as 

\[
  (\mathcal{E} \otimes \mathcal{F})(\rho) = \sum_i \sum_j (E_i \otimes F_j)\,\rho\:(E_i \otimes F_j)^\dag
\]

Superoperators can describe both unitary transformations and measurements. Given unitary operator $U$ on $\calH$, we can define  the (trace preserving) superoperator $\sop_U \in \calSH$ as:
\[ \sop_U(\rho) = U \rho U^\dagger\]
Given a measurement $\{M_m\}_m$, we can define the (trace non-increasing) superoperator $\sop_m \in \calSH$ as 
\[\sop_m(\rho) = M_m\rho M_m^\dagger\]
Notice that $\sop_m(\rho)$ is equal to $p_m\rho_m$, where $p_m$ is the probability of the outcome $m$ when measuring the state $\rho$, and $\rho_m$ is the state after outcome $m$ has occurred. 
